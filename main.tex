\documentclass[a4paper, 10pt, oneside, DIV=9, chapterprefix=true, numbers=enddot,bibliography=totoc]{scrbook}

\usepackage{style}
\usepackage{shortcuts}
\usepackage[normalem]{ulem}
\RedeclareSectionCommand[tocdynnumwidth]{chapter}
\RedeclareSectionCommands[tocdynindent]{section,subsection}
\usepackage[outline]{contour}
\contourlength{2.25pt}

% \bibliographystyle{plain}
\bibliography{res}

\newcommand{\embrace}[1]{\textup{(}#1\textup{)}}
\newlength{\LETTERheight}
\AtBeginDocument{\settoheight{\LETTERheight}{I}}
\newcommand*{\longrightsquigarrow}[1]{\ \raisebox{0.24\LETTERheight}{\tikz \draw [-to,
		line join=round, line cap=round,
		decorate, decoration={
			zigzag,
			segment length=4,
			amplitude=.9,
			post=lineto,
			post length=0.42ex
		}] (0,0) -- (#1,0);}\ }
	
\newlength{\HeightOfTextstyleOne}
\settoheight{\HeightOfTextstyleOne}{$\mathbf{1}$}
\newlength{\HeightOfScriptstyleOne}
\settoheight{\HeightOfScriptstyleOne}{$\scriptstyle\mathbf{1}$}
\newlength{\HeightOfScriptscriptstyleOne}
\settoheight{\HeightOfScriptscriptstyleOne}{$\scriptscriptstyle\mathbf{1}$}
\newcommand{\FancyOne}[1]{{\tikz[line cap=round,line join=round,line width=0.35*#1/\HeightOfTextstyleOne,scale=#1/\HeightOfTextstyleOne]{\draw (-0.0225,0.205) to (-0.0225,0.02) to[out=270,in=0] (-0.0425,0) to (-0.071,0) to (0.071,0) to (0.0425,0) to[out=180,in=270] (0.0225,0.02) to (0.0225,0.235) to (0.0175,0.235) to[out=210,in=0] (-0.075,0.201);}}}
\newcommand{\IOne}{\mathchoice%
	{\FancyOne{\HeightOfTextstyleOne}}%
	{\FancyOne{\HeightOfTextstyleOne}}%
	{\FancyOne{\HeightOfScriptstyleOne}}%
	{\FancyOne{\HeightOfScriptscriptstyleOne}}%
}
\newcommand{\IDigamma}{\tikz[line cap=round,line join=round,line width=0.35]{\draw (0.06,0.2286) to[out=0,in=90] (0.1353,0.172) to (0.1353,0.2286) to (-0.0525,0.2286) to (-0.0415,0.2286) to[out=0,in=90] (-0.0215,0.2086) to (-0.0215,0.02) to[out=270,in=0] (-0.0415,0) to (-0.0525,0) to (0.0605,0) to (0.0415,0) to[out=180,in=270] (0.0215,0.02) to (0.0215,0.2086) to[out=90,in=180] (0.0415,0.2286);\draw (0.025,0.1335) to[out=0,in=90] (0.0968,0.0769) to (0.0968,0.1335) to cycle;}\hspace{0.1ex}\vphantom{\IF}}


\DeclareFontFamily{U}{min}{}
\DeclareFontShape{U}{min}{m}{n}{<-> udmj30}{}

	
\makeatletter
\renewcommand{\@pnumwidth}{2em} 
\renewcommand{\@tocrmarg}{3em}
\makeatother
%\RedeclareSectionCommand[tocindent+=0.5em]{section}
%\RedeclareSectionCommand[tocindent+=0.5em]{subsection}






\subject{Notes on}
\title{Derived Categories}
\author{ }
\date{{\normalsize Typed by}\\
	Filippo Papallo}
\publishers{\today \\
	Pisa}

\usepackage{bookmark}


\makenomenclature







\begin{document}

\setlength{\parindent}{0pt}
\setlength{\parskip}{4pt}

\frontmatter
\KOMAoption{chapterprefix}{false}
\renewcommand{\thedummy}{\arabic{dummy}}
\maketitle
My original intention is to collect here some 
material about derived categories, with applications to
algebraic geometry, which will hopefully turn out to be 
useful for my thesis.
I will mainly follow the first chapters of the book
``\textit{Fourier-Mukai transforms in algebraic geometry}''
by Daniel Huybrects \parencite[]{huybrechts}. 

\hrulefill

Last update: \today
	
%Some additions have been made by the author. To distinguish them from the lecture's actual contents, they are labelled with an asterisk. So any \emph{Proof}* or \emph{Lemma}* etc.\ that the reader might encounter are wholly the author's responsibility.
%\\[\thmsep]Please report errors, typos etc.\ through the \href{https://github.com/}{\emph{Issues}} feature of GitHub, or just tell me before or after the lecture.
	


\tableofcontents

	\listoftodos

\listoftoc{lol}
\setcounter{llecture}{0}
\mainmatter\KOMAoption{chapterprefix}{true}
\renewcommand{\thedummy}{\thechapter.\arabic{dummy}}
%\renewcommand{\thechapter}{\arabic{chapter}}
\renewcommand{\thechapter}{\Roman{chapter}}

%%%%%%%%%%%%%%%%%%%%%%%%%%%%%%%%%%%%%%%%%%%%%%%%%%%%%%%%%%%
%%%%%%%%%%%%%%%%%%%%%%%%%%%%%%%%%%%%%%%%%%%%%%%%%%%%%%%%%%%
%%%%%%%%%%%%%%%%%%%%%%%%%%%%%%%%%%%%%%%%%%%%%%%%%%%%%%%%%%%


\nomenclature{$\Aa, \Bb, \Cc, \Dd \dots$}{Calligraphic capital letters are used to denote categories.}%

\nomenclature{$\Hom_{\Aa}(-,-)$}{Set of morphisms in the category $\Aa$.}%

\nomenclature{$\cat{1}_{A}$}{The identity morphism of the object $A$.}%

\nomenclature{$\cat{Fun}_{+}(\Aa,\Bb)$}{Additive functors between two preadditive categories $\Aa$ and $\Bb$.}%

\nomenclature{$C^{\bullet}(\Aa)$}{The category of cochain complexes in the abelian category $\Aa$. Objects are written as $A^{\bullet}$.}

\printnomenclature


%%%%%%%%%%%%%%%%%%%%%%%%%%%%%%%%%%%%%%%%%%%%%%%%%%%%%%%%%%%
%%%%%%%%%%%%%%%%%%%%%%%%%%%%%%%%%%%%%%%%%%%%%%%%%%%%%%%%%%%
%%%%%%%%%%%%%%%%%%%%%%%%%%%%%%%%%%%%%%%%%%%%%%%%%%%%%%%%%%%

\chapter{Structure of the thesis}

	%%% Plan of the thesis

The idea is to go through the article 
\emph{Braid groups actions on derived categories of coherent sheaves},
by Siedel and Thomas.

The article starts with a sections containing 
motivations on the theory: the authors give a survey on 
the main fields where the derived category $\cat{D}^{b}(X)$
appears. In particular, symplectic geometry
and mirror symmetry are the main interest of the paper.

Section 2 introduces braid group actions on a category.
For a detailed description, check \textbf{Deligne}'s article
``\emph{Action du group de tresses sur une catégorie}''.
In Section 2.b spherical objects and twist functors are
introduced in a pure functorial way, i.e. using only
monoidal abelian categories.
By introducing enough hypothesis on our category,
we can study braid relations in a pure abstract way:
Section 2.c is devoted to the contruction of braid relations
between twist functors induced by spherical objects.
Here, \textbf{Proposition 2.12} and \textbf{Proposition 2.13}
show that there exists a braid group action an suitable
category $\Kk$. The main result is stated in \textbf{Theorem 2.18}.

\begin{thm**}
	Suppose that $n \ge 2$. Then the homomorphism 
	\begin{equation*}
	 	\rho : B_{m+1} \longrightarrow \textrm{Autoeq}(\cat{D}^{b}(X))
	 \end{equation*} 
	is injective, and in fact the following stronger statement holds: 
	if $g \in B_{m+1}$ is not the identity element, 
	then $\rho(g)(E_{i}) \ne E_{i}$ for some $i \in \{1,...,m\}$.
\end{thm**}

The proof of the faithfulness of this action
is the whole Section 4.
The machinery involved is mainly based on
the theory of \textbf{differential graded algebras} (dgas).
A detailed survey of dgas can be found in
\textbf{Bernstein} and \textbf{Lunts}'s
``\emph{Equivariant sheaves and functors}''.

In the context of the derived category of a dga,
standard twist functors are introduced.
These twists commute with equivalences induced 
by qis of dgas.
\textbf{Lemma 4.3} relates the standard twist functors $t$
of dgas with the twist functors $T_{E}$ in our favorite triangulated category.

We then try to study the twist functors $T_{E_{i}}$
by analizing the dga $end(E)$. We will focus
on a particular class of dgas, 
namely \textbf{intrinsically formal} graded algebras:
these objects are determined by their cohomology.
This property can be studied in terms
of the vanishing of some \textbf{Hochschild cohomology} modules,
as stated in \textbf{Theorem 4.7}.
A simple intoduction to this topic can be found
in the ``Istituzioni di Algebra'' notes by \textbf{Tamas}.

The proof of this fact is by no mean trivial.
The authors develop a theory of \textbf{$A_{\infty}$-morphisms},
because they induce morphisms of dgas.

In Section 4.c the algebras $A_{m,n}$ are introduced
as the path algebras of a suitable \textbf{quiver}.
I should check for a little survey on this topic.
This section aims to explain how to translate the
problem on twist funtors with the language of 
dgas, indeed $A_{m,n}$ are isomorphic
to some $end(E)$.

Standard twist functors $t_{i}:\cat{D}(A_{m,n}) \to \cat{D}(A_{m,n})$
are exact equivalences (\textbf{Lemma 4.11}) that
satisfy the braid relations.

Finally, the faithfullness of the braid action
is proved in Section 4.d 
for the derived categories of the algebras $A_{m,n}$.
To achieve this, one studies the twist functor
associated to the generator of the center of $B_{m+1}$.

\textbf{The authors conjecture that $(t_{1}t_{2} \dots t_{m})^{m+1}$
is isomorphic to the translation $[2m-(m+1)n]$.}

The proof of the faithfulness relies on topology:
in particular, \textbf{mapping class groups} 
of the punctured disc are involved. It is well known
their link with the braid groups,
and \textbf{Lemma 4.16} explains the link
between some diffeomorphisms
and the generator of the center of $B_{m+1}$.


\textbf{Geometric intersection numbers}
are defined; one can check
``\emph{Quivers, Floer cohomology, and braid group actions}''
by \textbf{Khovanov} and \textbf{Seidel}
for an insight. 
In fact, we can find braid group actions 
already in this paper;
Siedel and Thomas make a comparison
with their definition $A_{m,n}$ .

Strictly speaking, Section 4.d is a comparison
with the article by Khovanov and Seidel,
by the end of which the faithfulness
is proved in the category of some dga.

The final section is dedicated
to the computation of Hochschild cohomology
of $A_{m,n}$, so that its intrinsic formality is proved.
This fact is essential in the proof of \textbf{Theorem 2.18},
together with the faithfulness described above.


\section{Study plan}

	I'm new to the following topics:
	\begin{itemize}
		\item derived category of a scheme;
		\item Fourier-Moukai Transforms;
		\item group actions on a category;
		\item DG-algebras;
		\item Hochschild cohomology and its applications;
		\item quivers and representations;
		\item mapping class group.
	\end{itemize}
	Thus, I'll need some study!
	
\section{Structure idea}

	Here I write a possible structure of the strategy:
	\begin{itemize}
		\item introduction, which will be written at the very end,
		possibly with motivations;
		
		\item recollection of some properties of derived categories;
		
		\item a survey about DG-algebras;
		
		\item something about quivers and their path algebras;
		
		\item we start to set the main tools:
		definition of spherical objects in a triangulated category,
		and definition of twists;
		
		\item definition of group action over a category.
		Braids acting on a category and faithfulness: discussion and proof,
		assuming \textbf{Theorems 4.13} e \textbf{4.21}, whose proofs
		are postponed;
		
		\item technicalities about dgas, intrinsic formality 
		and geometric intersections;
		
		\item applications to the algebro-geometric context
		(definition of FMT is needed) and examples;
		recent develpments.
	\end{itemize}





















% I comment from here
%%%%%%%%%%%%%%%%%%%%%%%%%%%%%%%%%%%%%%%%%%%%%%%%%%%%%%%%%%%%
%%%%%%%%%%%%%%%%%%%%%%%%%%%%%%%%%%%%%%%%%%%%%%%%%%%%%%%%%%%%
%%%%%%%%%%%%%%%%%%%%%%%%%%%%%%%%%%%%%%%%%%%%%%%%%%%%%%%%%%%%
%
%
%\chapter{Preliminaries}
%
%This chapter is dedicated to a
%recollection of some important definitions and
%results about general category theory,
%so that we can fix some notation.
%Our main goal is to understand 
%adjunctions and equivalences of categories, 
%with some criteria for equivalence,
%because it will be fundamental when
%studying the derived category associated
%to a noetherian scheme.
%
%%    
\section{Equivalences}

\begin{df}
    Given two categories $\Aa$ and $\Bb$ and
    a functor $F : \Aa \to \Bb$.
    For any two objects $A, A' \in \Aa$ 
    consider the map
    \begin{equation*}
        F_{A,A'} : \Hom_{\Aa}(A,A') \longrightarrow \Hom_{\Bb}(F(A),F(A'))\,.
    \end{equation*}
    The functor $F$ is called:
    \begin{itemize}
        \item \textbf{full} if $F_{A,A'}$ is surjective
        for any $A, A' \in \Aa$;

        \item \textbf{faithful} if $F_{A,A'}$ is injective
        for any $A, A' \in \Aa$;

        \item \textbf{fully faithful} if $F$ is both full and faithful.
    \end{itemize}
\end{df}

Recall that a \emph{morphism of functors},
or \emph{natural transformation},
$\phi : F \to F'$ between two functors
$F,F' : \Aa \to \Bb$ is a collection of arrows
$\phi_{A} \in \Hom_{\Bb}(F(A), F'(A))$, for every $A \in \Aa$,
which is functorial in $A$, i.e. given $f : A \to B$ in $\Aa$,
the diagram
\begin{center}
    \begin{tikzcd}
        F(A) \ar[r, "\phi_{A}"] \ar[d, "F(f)"'] 
        & F'(A) \ar[d, "F'(f)"] \\
        F(B) \ar[r, "\phi_{B}"] & F'(B)
    \end{tikzcd}
\end{center}
commutes. Thus, once two categories $\Aa$ and $\Bb$
are fixed, one may verify that functors from $\Aa$ to $\Bb$
and natural transformations form a category 
$\cat{Fun}(\Aa, \Bb)$. In here, two functors $F,F'$
are \textbf{isomorphic} if and only if
there exists a natural transformation $\phi:F \to F'$
such that $\phi_{A}$ is an isomorphism in $\Bb$, 
for every $A \in \Aa$.

\begin{df!}\label{cat-equivalence}
    A functor $F : \Aa \to \Bb$ is called 
    an \textbf{equivalence} if there exists a
    functor $F^{-1} : \Bb \to \Aa$ such that
    $F^{-1} \circ F \simeq \cat{1}_{\Aa}$ in $\cat{Fun}(\Aa, \Aa)$
    and
    $F \circ F^{-1} \simeq \cat{1}_{\Bb}$ in $\cat{Fun}(\Bb, \Bb)$,
    where $\cat{1}_{\Aa}$ and $\cat{1}_{\Bb}$
    denote the identity functors. The functor $F^{-1}$ is called a \textbf{quasi-inverse} of $F$.

    Two categories $\Aa$ and $\Bb$ are called \textbf{equivalent} if there exists an
    equivalence $F:\Aa \to \Bb$.
\end{df!}

In particular, notice that 
for any $A,A' \in \Aa$,
%for any arrow $f: A \to A'$ in $\Aa$,
%there exists a commutative diagram
%\begin{center}
%    \begin{tikzcd}
%        F^{-1} \circ F(A) \ar[rr, "F^{-1} \circ F(f)"] \ar[d, "\simeq"']
%        & & F^{-1} \circ F(A') \ar[d, "\simeq"] \\
%        A \ar[rr, "f"] & & A'\,,
%    \end{tikzcd}
%\end{center}
the map $F_{A,A'} : \Hom_{\Aa}(A, A') \to \Hom_{\Bb}(F(A),F(A'))$
is bijective, whose inverse is the composition
\begin{equation*}
    \Hom_{\Bb}(F(A),F(A')) \longrightarrow
    \Hom_{\Aa}(F^{-1} \circ F(A), F^{-1} \circ F(A')) \simeq
    \Hom_{\Aa}(A, A')\,.
\end{equation*}
Thus, an equivalence is fully faithful. It holds a partial converse:

\begin{prop}
    A fully faithful functor $F : \Aa \to \Bb$ is an equivalence
    if and only if every $B \in \Bb$ is isomorphic
    to an object of the form $F(A)$, for some $A \in \Aa$.
    \begin{proof}
        If $F$ is an equivalence, then by definition there exists
        a quasi-inverse $F^{-1}$ and for every $B \in \Bb$ we have
        $B \simeq F(F^{-1}(B))$.

        Conversely, we build a quasi-inverse $F^{-1}$ 
        in the following way: for every $B \in \Bb$,
        let $\phi_{B} : B \simeq F(A_{B})$ a fixed isomorphism,
        with $A_{B} \in \Aa$. Since $F$ is fully faithful,
        for any two $B, B' \in \Bb$ the map $F_{A_{B}, A_{B'}}$ 
        is bijective, hence we can define $F^{-1}$ to be
        \begin{equation*}
            \Big( f : B \to B' \Big)
            \longmapsto
            \left( \left(F_{A_{B},A_{B'}}\right)^{-1}(\phi_{B'} \circ f \circ \phi_{B}^{-1}) : A_B \to A_B' \right)\,,
        \end{equation*}
        and then one verifies that $F^{-1}$ works.
    \end{proof}
\end{prop}

\begin{cor}
    Any fully faithful functor $F:\Aa \to \Bb$
    defines and equivalence between $\Aa$ and
    the full subcategory $\Bb_{\Aa} \subset \Bb$
    of all objects of $\Bb$ isomorphic to $F(A)$,
    for some $A \in \Aa$.
\end{cor}

The following result is fundamental to understand
when a functor $F$ can be represented by an object,
i.e. the functor is nothing but the morphisms hitting
some \emph{universal object}.
Let $\cat{Fun}(\Aa^{op})$ be the category of all
contravariant functors from $\Aa$ to $\CSet$,
whose objects are $F : \Aa^{op} \to \CSet$.
There exists a natural functor
\begin{equation*}
    h : \Aa \longrightarrow \cat{Fun}(\Aa^{op})\,,
    \quad A \longmapsto \Hom_{\Aa}(-,A)\,,
\end{equation*}
called the \textbf{Yoneda embedding}.

\begin{thm}[Yoneda's Lemma]\label{yoneda}
    For any fixed $A \in \Aa$, the map 
    \begin{equation*}
        \Hom_{\cat{Fun}(\Aa^{op})}\left( h(A), F \right)
        \longrightarrow F(A)\,, \quad
        \phi \longmapsto \phi_{A}(\cat{1}_{A})
    \end{equation*}
    is bijective and it is functorial
    both in $A$ and in $F$.
    It follows that the Yoneda embedding $h$ defines 
    an equivalence between $\Aa$ and the full subcategory 
    of \textbf{representable functors}.

    \begin{proof}
        See \parencite{riehl}.
    \end{proof}
\end{thm}

As a consequence, it follows that an object $A \in \Aa$ is
determined by all the morphisms that hit it: more
precisely, whenever there exist $A,A' \in \Aa$ such that
for every $T \in \Aa$ there are natural isomorphisms
\begin{equation*}
    \Hom_{\Aa}(T,A) \simeq \Hom_{\Aa}(T,A')\,,
\end{equation*}
then we conclude that $A \simeq A'$ in $\Aa$.
%%    
\section{Adjunctions}

\begin{df}
    Given $F:\Aa \to \Bb$, a functor $H : \Bb \to \Aa$
    is \textbf{right adjoint} to $F$, and one writes $F \dashv H$,
    if for any objects $A \in \Aa, B \in \Bb$ there exist 
    isomorphisms
    \begin{equation}\label{adj-morphism}
        \Hom_{\Bb}(F(A),B) \simeq \Hom_{\Aa}(A, H(B))\,,
    \end{equation}
    which are functorial both in $A$ and $B$.

    Similarly, a functor $G : \Bb \to \Aa$ is \textbf{left adjoint}
    to $F$ (witten $G \dashv F$) if there exist natural isomorphisms 
    \begin{equation}\label{l-adj-morphism}
        \Hom_{\Bb}(B, F(A)) \simeq \Hom_{\Aa}(G(B), A)\,,
    \end{equation}
    for any $A \in \Aa$ and $B \in \Bb$.
\end{df}

\begin{rmk}
        Suppose $F \dashv H$. By adjunction, the identity
        $\cat{1}_{F(A)}$ induces a morphism 
        $\eta_{A} : A \to H(F(A))$.
        Since in the definition of adjunction
        the isomorphisms are natural,
        it follows that all these morphisms define
        a natural transformation $\eta : \cat{1}_{\Aa} \to H \circ F$
        called the \textbf{unit} of the adjunction.
        Similarly, changing roles of $F$ and $H$, we see there exists
        a natural transformation $\epsilon : F \circ H \to \cat{1}_{\Bb}$ called 
        the \textbf{counit} of the adjunction.
\end{rmk}

\begin{prop}
    Let $F : \Aa \to \Bb$ be a functor. Whenever they exist,
    right and left adjoint functors are unique up to isomorphism.
    \begin{proof}
        Suppose there exist $H, H':\Bb \to \Aa$ such that 
        $F \dashv H$ and $F \dashv H'$.
        Then for every $A \in \Aa$ and $B \in \Bb$ we have 
        natural isomorphisms
        \begin{equation*}
            \Hom_{\Aa}(A,H(B)) \simeq 
            \Hom_{\Bb}(F(A),B) \simeq 
            \Hom_{\Aa}(A,H'(B))\,.
        \end{equation*}
        In particular, for any $B \in \Bb$, the functors
        $\Hom_{\Aa}(-,H(B))$ and $\Hom_{\Aa}(-,H'(B))$
        are isomorphic in $\cat{Fun}(\Aa^{op})$,
        so by the \hyperref[yoneda]{Yoneda's Lemma~\ref*{yoneda}}
        it follows that there exists a natural isomorphism
        $H(B) \simeq H'(B)$.
        As this holds for every $B$, these define
        an isomorphism $H \simeq H'$.
    \end{proof}
\end{prop}

\begin{exercise!}\label{unit-counit-identity}
    Suppose $F \dashv H$. Show that, for the unit $\eta$ 
    and the counit $\epsilon$ of the adjunction,
    the composition
    \begin{equation*}
        H \xrightarrow[]{\eta_{H(-)}} H \circ F \circ H \xrightarrow[]{H(\epsilon)} H
    \end{equation*}
    is the identity.
    \begin{proof}[Solution]
        %Since the adjunction morphism~\eqref{adj-morphism}
        %is natural in $B$, if we apply it to $B=F(A)$ and
        %to the morphism $F\eta_{A} : F(A) \to FHF(A)$,
        %then we get a commutative square
        %\begin{center}
        %    \begin{tikzcd}
        %        \Hom_{\Bb}(FHF(A),F(A)) \ar[r, "\simeq"] \ar[d, "- \circ F\eta_{A}"']
        %        & \Hom_{\Aa}(HF(A),HF(A)) \ar[d, "- \circ \eta_{A}"]
        %        & & \epsilon_{F(A)}  \ar[d, mapsto] 
        %        & \cat{1}_{HF(A)} \ar[d,mapsto] \ar[l, mapsto] \\
        %        \Hom_{\Bb}(F(A),F(A)) \ar[r, "\simeq"] 
        %        & \Hom_{\Aa}(A,HF(A))\,,
        %        & & \cat{1}_{F(A)} 
        %        & \eta_{A} \ar[l,mapsto]  \,,
        %    \end{tikzcd}
        %\end{center}
        %which witnesses the equation $\cat{1}_{F(A)} = \epsilon_{F(A)} \circ %F\eta_{A}$.
        
        Since the adjunction morphism~\eqref{adj-morphism}
        is natural in $A$, if we apply it to $A=H(B)$ and
        to the morphism $H\epsilon_{B} : HFH(B) \to H(B)$,
        then we get a commutative square
        \begin{center}
            \begin{tikzcd}[column sep=small]
                \Hom_{\Bb}(FH(B),FH(B)) \ar[r, "\simeq"] \ar[d, "\epsilon_{B} \circ - "']
                & \Hom_{\Aa}(H(B),HFH(B)) \ar[d, "H\epsilon_{B} \circ - "]
                & & \cat{1}_{FH(B)}  \ar[d, mapsto] \ar[r, mapsto]
                & \eta_{H(B)} \ar[d,mapsto]  \\
                \Hom_{\Bb}(FH(B),B) \ar[r, "\simeq"] 
                & \Hom_{\Aa}(H(B),H(B))\,,
                & & \epsilon_{B} \ar[r,mapsto]
                & \cat{1}_{H(B)}   \,,
            \end{tikzcd}
        \end{center}
        which witnesses the equation $\cat{1}_{H(B)} = H\epsilon_{B} \circ \eta_{H(B)}$, for any $B \in \Bb$.

        One can show the same holds true for the composition $\epsilon_{F(-)} \circ F\eta = F$ and, in fact, these triangle identities
        give an equivalent definition of adjunction: one may say $F \dashv H$
        whenever these two identities hold (see \parencite[Proposition 4.2.6]{riehl}).
    \end{proof}
\end{exercise!}

\begin{exercise}
    Suppose $F \dashv H$. Show that
    \begin{equation*}
        f \longmapsto \Big( A \xrightarrow[]{\eta_{A}} H(F(A))  \xrightarrow{H(f)} H(B) \Big)
    \end{equation*}
    describes the adjunction morphism 
    $\Phi_{A,B} : \Hom_{\Bb}(F(A),B) \simeq \Hom_{\Aa}(A,H(B))$.
    \begin{proof}[Solution]
        First, notice that $\Psi_{A,F(A)}\left(\cat{1}_{F(A)}\right) = \eta_{A}$
        gives the unit of the adjunction, which is a good sign. 
        By following the construction of $\Phi$, 
        one defines a map 
        $\Psi_{A,B} : \Hom_{\Aa}(A,H(B)) \to \Hom_{\Bb}(F(A),B)$
        by setting
        \begin{equation*}
        g \longmapsto \Big( F(A) \xrightarrow[]{F(g)} F(H(B))  \xrightarrow{\epsilon_{B}} B \Big)\,. 
    \end{equation*}
    We now show that $\Psi_{A,B}$ is the inverse of $\Phi_{A,B}$: 
    given $f \in \Hom_{\Bb}(F(A),B)$, we have a commutative square
    \begin{center}
        \begin{tikzcd}[column sep=large]%, row sep=large]
            FHF(A) \ar[r, "FH(f)"] \ar[d, "\epsilon_{F(A)}"']
            & FH(B) \ar[d, "\epsilon_{B}"] \\
            F(A) \ar[r, "f"] 
            & B\,,
        \end{tikzcd}
    \end{center}
    thus it holds
    \begin{align*}
        \Psi_{A,B} \circ \Phi_{A,B}(f) 
        &= \Psi_{A,B} \left( H(f) \circ \eta_{A} \right) \\
        &= \epsilon_{B} \circ F\left( H(f) \circ \eta_{A} \right) \\
        &= f \circ \epsilon_{F(A)} \circ F\eta_{A} = f\,,
    \end{align*}
    where the last \textbf{Exercise} was applied to the last equality. 
    Similarly, one shows that $\Phi_{A,B} \circ \Psi_{A,B}$ is the identity,
    so we conclude these morphisms are, in fact, isomorphisms.

    It remains to prove naturality:
    given $a:A \to A'$ and $f':F(A') \to B$, 
    we want the square
    \begin{center}
        \begin{tikzcd}
            \Hom_{\Bb}(F(A'),B) \ar[r, "\Phi_{A',B}"] \ar[d, "- \circ Fa"']
            & \Hom_{\Aa}(A',H(B)) \ar[d, "- \circ a"] \\
            \Hom_{\Bb}(F(A),B) \ar[r, "\Phi_{A,B}"]
            & \Hom_{\Aa}(A,H(B)) 
        \end{tikzcd}
    \end{center}
    to be commutative; it follows from the commutativity 
    of the diagram
    \begin{center}
        \begin{tikzcd}[row sep=small]
            A \ar[dd, "a"', dashed] \ar[r, "\eta_{A}", dotted] 
            & HF(A) \ar[dd, "HFa"'] \ar[rd, "H(f' \circ Fa)", dotted] 
            & \\
            & & H(B) \\
            A' \ar[r, "\eta_{A'}",dashed] 
            & HF(A') \ar[ur, "Hf'"',dashed] 
            & \,,
        \end{tikzcd}
    \end{center}
    where the dotted arrows witness the composition 
    $\Phi_{A,B} \circ (-\circ Fa)$, 
    while the dashed arrows
    $(- \circ a) \circ \Phi_{A',B}$.
    \end{proof}
\end{exercise}

As a consequence of the previous \textbf{Exercises},
we get the following

\begin{lemma}\label{adj-triangle}
    Let $F:\Aa \to \Bb$ be a functor and $F \dashv H$.
    For any $A, A' \in \Aa$, the unit $\eta:\cat{1}_{\Aa} \to H \circ F$ 
    induces the commutative triangle
    \begin{center}
        \begin{tikzcd}
            \Hom_{\Aa}(A,A') \ar[rr, "\eta_{A'} \circ - "] \ar[dr, "F"']
            & & \Hom_{\Aa}(A,HF(A')) \ar[dl, "\simeq"', "\Psi_{A,F(A')}"] \\
            & \Hom_{\Bb}(F(A),F(A')) & \,.
        \end{tikzcd}
    \end{center}
    Similarly, for any $B,B' \in \Bb$,
    the counit $\epsilon: FH \to \cat{1}_{\Bb}$
    induces a commutative triangle
    \begin{center}
        \begin{tikzcd}
            \Hom_{\Bb}(B,B') \ar[rr, "- \circ \eta_{B'}"] \ar[dr, "F"']
            & & \Hom_{\Bb}(FH(B),B') \ar[dl, "\simeq"', "\Phi_{H(B),B'}"] \\
            & \Hom_{\Aa}(H(B),H(B')) & \,.
        \end{tikzcd}
    \end{center}
\end{lemma}

\begin{cor}\label{ff-adj}
    If a fully faithful functor $F:\Aa \to \Bb$ admits a right adjoint
    $F \dashv H$, then the unit
    \begin{equation*}
        \eta : \cat{1}_{\Aa} \xrightarrow[]{\sim} H \circ F
    \end{equation*}
    is an isomorphism.
    Similarly, if $F$ admits a left adjoint $G \dashv F$, then the
    counit $\epsilon : F \circ G \simeq \cat{1}_{\Bb}$ is an isomorphism\,.
    \begin{proof}
        Assume $G \dashv F$.
        By assumption, for every $B,B' \in \Bb$ we have
        a bijection $G : \Hom_{\Bb}(B,B') \simeq \Hom_{\Aa}(G(B),G(B'))$,
        hence by \hyperref[adj-triangle]{Lemma~\ref*{adj-triangle}}
        there is a natural bijection
        \begin{equation*}
            \Hom_{\Bb}(B,B') \simeq \Hom_{\Bb}(F \circ G(B),B')\,.
        \end{equation*}
        By the \hyperref[yoneda]{Yoneda's Lemma~\ref*{yoneda}} we conclude
        that $B \simeq FG(B)$ for any $B \in \Bb$, thus the functors $\cat{1}_{\Bb}$
        and $F \circ G$ are isomorphic.

        In the case $F \dashv H$, the proof must be modified
        by applying the covariant version of the 
        \hyperref[yoneda]{Yoneda's Lemma}.
    \end{proof}
\end{cor}

\begin{exercise}
    Let $F : \Aa \to \Bb$ be a fully faithful functor 
    and assume $G \dashv F \dashv H$.
    Construct a canonical homomorphism $H \to G$.
    \begin{proof}[Solution]
        Notice that, in fact, the two functors $G$ and $H$
        turn out to be isomorphic: by the previous
        \hyperref[ff-adj]{Corollary~\ref*{ff-adj}},
        it holds
        \begin{equation*}
            H = H \circ \cat{1}_{\Bb} 
            \simeq H \circ (F \circ G)
            \simeq (H \circ F) \circ G
            \simeq \cat{1}_{\Aa} \circ G = G\,.
        \end{equation*}
        To construct this homomorphism explicitly,
        we use the fact that both unit and counit are isomorphisms;
        thus, for any $B, B' \in \Bb$ we have
        \begin{center}
            \begin{tikzcd}[column sep=large]
                \Hom_{\Bb}(B,B') \ar[r, "\epsilon_{B'}^{-1} \circ -"',"\sim"]
                & \Hom_{\Bb}(B,FG(B')) \ar[d, "H"] & \\
                & \Hom_{\Aa}\big(H(B),HF(G(B'))\big) 
                \ar[r, "\eta_{G(B')}^{-1} \circ -"',"\sim"]
                & \Hom_{\Aa}(H(B),G(B'))\,.
            \end{tikzcd}
        \end{center}
        By plugging $B'=B$, the image of the identity $\cat{1}_{B}$ gives
        a natural transformation $H \to G$ given %, for every $B \in \Bb$,
        by %$\eta_{G(B)}^{-1} \circ H(\epsilon_{B}^{-1}) =
        $ \left( \eta_{G(-)} \circ H(\epsilon) \right)^{-1}$.
    \end{proof}
\end{exercise}

In many cases, adjoint functors exist. The case that interests us most is the case of equivalences. Here, the existence of left and right adjoints is granted by the following general result.

\begin{prop}
    Let $F:\Aa \to \Bb$ be an equivalence of categories.
    Then $F$ admits a right adjoint and a left adjoint,
    which are given by its quasi-inverse $F^{-1}$,
    i.e. $F \dashv F^{-1} \dashv F$.
    \begin{proof}
        For any $A \in \Aa$ and $B \in \Bb$, we have functorial bijections
        \begin{equation*}
            \Hom_{\Bb}(F(A),B) 
            \simeq \Hom_{\Aa}(F^{-1}(F(A)),F^{-1}(B))
            \simeq \Hom_{\Aa}(A,F^{-1}(B))\,,
        \end{equation*}
        where we use $F^{-1}(F(A)) \simeq A$.
    \end{proof}
\end{prop}
%
%
%%%%%%%%%%%%%%%%%%%%%%%%%%%%%%%%%%%%%%%%%%%%%%%%%%%%%%%%%%%%
%%%%%%%%%%%%%%%%%%%%%%%%%%%%%%%%%%%%%%%%%%%%%%%%%%%%%%%%%%%%
%%%%%%%%%%%%%%%%%%%%%%%%%%%%%%%%%%%%%%%%%%%%%%%%%%%%%%%%%%%%
%
%
%\chapter{Abelian categories}\label{AbelianCategories}
%    In many areas of mathematics, especially in geometry 
%    and topology, the introduction of algebraic tools
%    turned out to be a really powerful weapon: for instance,
%    homology is a very important homotopy invariant for topological
%    spaces; derived functors $\Tor$ and $\Ext$ have many applications
%    in commutative algebra (e.g. \emph{Serre's Theorem} 
%    for regular local rings); sheaf cohomology 
%    appears in the study of algebraic varieties,
%    for example \emph{Riemann-Roch Theorem} and so on...
%    \textbf{Abelian categories} are the most general setting
%    in which one can develop homological algebra.
%    The idea is to generalize the properties of the categories
%    of modules over a ring, in such a way that the notions
%    of \emph{(co)chain complexes} and \emph{exact sequences}
%    make sense.
%
%%    
\section{Equalizers}

\begin{df}
    Let $\Aa$ be a category and $s,t : A \to B$ two morphisms.
    The \textbf{equalizer} of $s$ and $t$ is a pair $(E,e)$
    of an object $E$ together with a morphism $e : E \to A$ such that
    $se=te$, which satisfies the following universal property:
    for every morfism $f:T \to A$ such that $sf =tf$, there
    exists a unique arrow $f':T \to E$ which makes 
    the following triangle commute:
    \begin{center} \begin{tikzcd}
        T \ar[d, dotted,"f'"'] \ar[dr, "f"] & & \\
        E \ar[r, "e"] & 
        A \ar[r, "s", black!50, bend left] \ar[r, "t"', black!50, bend right] 
        & \textcolor{black!50}{B} \,.
    \end{tikzcd} \end{center}
\end{df}

Equivalently, one may define equalizers in $\Aa$ as limits of diagrams $\Ee \to \Aa$,
where $\Ee$ is the small category with two objects and four arrows, depicted as
\begin{center} \begin{tikzcd}
    \bullet \ar[r, bend left] \ar[r, bend right] \ar[loop left] & \bullet \ar[loop right] &\,.
\end{tikzcd} \end{center}
Dually, one defines \textbf{coequalizers} in $\Aa$ as colimits of diagrams of shape $\Ee$.


\begin{prop}
    Equalizers are monomorphisms. Coequalizers are epimorphisms.
    \begin{proof}
        By duality, we only prove the statement for equalizers.
        Consider the equalizer diagram
        \begin{equation*}
        \begin{tikzcd}
        E \ar[r, "e"]
        & A \ar[r, shift left=0.7ex, "s"] \ar[r, shift right=0.7ex, "t"']
        & B\,
        \end{tikzcd}
        \end{equation*}
        and let $\alpha, \beta: T \to E$ be two morphisms such that
        $e\alpha = e \beta$. By the universal property of $(E,e)$, 
        we then obtain a diagram
        \begin{equation*}
        \begin{tikzcd}
        E \ar[r, "e"]
        & A \ar[r, shift left=0.7ex, "s"] \ar[r, shift right=0.7ex, "t"']
        & B \\
        T \ar[ur, "e\alpha=e\beta"'] 
        \ar[u, dashed, "\exists !"] & & \,,
        \end{tikzcd}
        \end{equation*}
        from which we deduce that $\alpha = \beta$.
    \end{proof}
\end{prop}
%%    \section{Zero objects and kernels}

\begin{df}
    By a \textbf{zero object} in a category $\Aa$
    we mean an object $\cat{0}$ which is both 
    an initial and a terminal object.
\end{df}

\begin{ex}
    In the category of abelian groups, or any category of modules over a ring, both notions coincide and correspond to the group (respectively the module) reduced to {0}.
\end{ex}

\begin{df}
    Consider a category $\Aa$ with a zero object $\cat{0}$.
    A morphism $f : A \to B$ is a \textbf{zero morphism}
    when it factors through the zero object, 
    which means that the following triangle commutes
    \begin{equation*}
        \begin{tikzcd}
            A \ar[rr, "f"] \ar[dr] & & B \\
            & \cat{0} \ar[ur] & \,.
        \end{tikzcd}
    \end{equation*}
\end{df}

Notice that the factorization is unique by definition
zero object; in particular, for each pair of objects
$A$ and $B$ in $\Aa$, there always exists a unique
zero morphism $0_{AB} \in \Hom_{\Aa}(A,B)$.
We will soon drop the indices, 
whenever it is clear from the context.

Notice that the notion of having ``zero morphisms'' 
allows us to talk about \emph{complexes}, which are
sequences of objects and morphisms whose composite 
is zero. If we want to talk about \emph{exactness},
we need \emph{kernels} and \emph{images}.

If we consider a morphism $f : G \to H$
in the category of abelian groups, 
we know that the kernel of $f$ is a subgroup of $G$, 
which is defined by
\begin{equation*}
    \ker f = \Set{g \in G | f(g) = 0 = 0_{GH}(g)}\,.
\end{equation*}
In particular, we see that the diagram
\begin{equation*}
    \begin{tikzcd}
        \ker f \ar[r, "\subset"]
        & G \ar[r, shift left=0.7ex, "f"] \ar[r, shift right=0.7ex, "0_{GH}"']
        & H\,
    \end{tikzcd}
\end{equation*}
is an equalizer.

\begin{df}
    Let $\Aa$ be a category with a zero object $\cat{0}$.
    The \textbf{kernel} of a morphism $f : A \to B$
    is the equalizer of $f$ and the zero morphism $0_{AB}$,
    whenever it exists. As equalizers are unique up to isomorphisms,
    we will denote the kernel of $f$ by $\ker f$:
    \begin{equation*}
    \begin{tikzcd}
        \ker f \ar[r]
        & A \ar[r, shift left=0.7ex, "f"] \ar[r, shift right=0.7ex, "0_{AB}"']
        & B\,.
    \end{tikzcd}
\end{equation*}
\end{df}

\begin{rmk}
    Since it is an equalizer, the map $\ker f \to A$ is a 
    monomorphism; in the example of $\Aa = \cat{Ab}$ it
    is indeed an inclusion of a subgroup. Hence, this agrees
    with the intuition that the kernel is something
    ``contained'' in the domain $A$. 
    The converse does not hold true: there exist
    monomorphisms that are not kernels. For instance,
    if $\Aa = \cat{Grp}$ is the category of groups,
    then any inclusion of a subgroup $H \subset G$ is
    a monomorphism in $\Aa$, but if $H$ is not normal,
    then it cannot be a kernel.
\end{rmk}

We define the dual notion of a kernel.

\begin{df}
    Let $\Aa$ be a category with a zero object $\cat{0}$.
    The \textbf{cokernel} of a morphism $f : A \to B$
    is the coequalizer of $f$ and the zero morphism $0_{AB}$,
    whenever it exists. As coequalizers are unique up to isomorphisms,
    we will denote the cokernel of $f$ by $\Coker f$:
    \begin{equation*}
    \begin{tikzcd}
        A \ar[r, shift left=0.7ex, "f"] \ar[r, shift right=0.7ex, "0_{AB}"']
        & B \ar[r] & \Coker f\,.
    \end{tikzcd}
\end{equation*}
\end{df}

We have almost everything we need to develop
the homological algebra language. Indeed, in order
to talk about ``\emph{exactness}'' of
composable arrows, 
we define \textbf{images} and \textbf{coimages}.

\begin{df}
    Let $\Aa$ be a category with a zero object $\cat{0}$,
    and assume that all equalizers and coequalizers exist.
    Given a morphism $f:A \to B$, we call \textbf{image} of $f$
    the kernel of $B \to \Coker f$ and we denote it by $\im f$.
    Thus, we have a factorization
    \begin{equation*}
        \begin{tikzcd}
            A \ar[r, "f"] \ar[rd, dashed] & B \ar[r, "c"] & \Coker f\\
            & \im f := \ker c \ar[u, hook] & \,.
        \end{tikzcd}
    \end{equation*}
    Dually, we define the \textbf{coimage} of $f$ as the cokernel of the kernel of $f$,
    thus we have the factorization
    \begin{equation*}
        \begin{tikzcd}
            \ker f \ar[r, "k"] & A \ar[r, "f"] \ar[d,two heads] & B \\
            & \operatorname{coim} f := \Coker k \ar[ur, dashed] & \,.
        \end{tikzcd}
    \end{equation*}
\end{df}

\begin{ex}
    Given a morphism $f : A \to B$ in $\Aa = \cat{Ab}$, then $\im f$ coincides
    with the set theoretic image of the map, that is
    \begin{equation*}
        \im f = \Set{f(a) | a \in A}\,;
    \end{equation*}
    this forms a subgroup of $B$ and we may identify the cokernel of $f$
    with the quotient
    \begin{equation*}
        \Coker f = B/\im f\,.
    \end{equation*}
    In particular, if we consider the inclusion $\ker f \subset A$,
    its cokernel is
    \begin{equation*}
        \operatorname{coim} f = A/\ker f\,.
    \end{equation*}
\end{ex}

\begin{prop}\label{coim-im}
    For any $f : A \to B$, there
    exists a unique map $\operatorname{coim}f \to \im f$ that
    makes the following square commute:
        \begin{equation*}
            \begin{tikzcd}
                A \ar[r, "f"] \ar[d] & B \\
                \operatorname{coim} f \ar[r] & \im f \ar[u] \,.
            \end{tikzcd}
        \end{equation*}
    \begin{proof}
        By definition of kernel, the composition $\ker f \to A \overset{f}{\to} B$
        is the zero map, hence we get the factorization
        \begin{equation*}
        \begin{tikzcd}
            \ker f \ar[r] & A \ar[r, "f"] \ar[d,two heads, "\pi"'] & B \\
            & \operatorname{coim} f \ar[ur, dashed, "f'"'] & \,.
        \end{tikzcd}
        \end{equation*}
        Now we notice that the diagram
        \begin{equation*}
        \begin{tikzcd}
            A \ar[r,two heads, "\pi"] 
            & \operatorname{coim} f  \ar[r, "0"] \ar[dr, "f'"'] 
            & \Coker f \\
            & & B \ar[u, "c"'] 
        \end{tikzcd}
        \end{equation*}
        commutes, and since $\pi$ is an epimorphism, then $cf' = 0$.
        Since the image is the kernel of $c$, there exists a unique map $\ol{f}$
        which makes the following diagram commute
        \begin{equation*}
        \begin{tikzcd}
            \ker f \ar[r] & A \ar[r, "f"] \ar[d,two heads, "\pi"'] 
            & B \ar[r, "c"] & \Coker f \\
            & \operatorname{coim} f \ar[ur, "f'"'] \ar[r, dashed, "\ol{f}"'] 
            & \im f \ar[u, hook] & \,.
        \end{tikzcd}
        \end{equation*}
    \end{proof}
\end{prop}

\begin{rmk}
    In abstract categorical nonsense, this statement
    may look at first a bit counter intuitive,
    but it is in fact a very familiar statement in algebra:
    if we consider the case $\Aa = \cat{Ab}$, then
    the proposition tells that every homomorphism $f:A \to B$
    factors as
    \begin{equation*}
        \begin{tikzcd}
             A \ar[r, "f"] \ar[d,two heads, "\pi"'] 
            & B \\
            A/\ker f \ar[r, "\ol{f}"', "\simeq"] 
            & \im f \ar[u, hook] \,,
        \end{tikzcd}
        \end{equation*}
    where the map $\ol{f}$ is an isomorphism.
    This is the well-known 
    \textbf{First Isomorphism Theorem} (\textbf{for abelian groups})!
    Be careful that, in general, the map $\ol{f}$ needs \textbf{not}
    be an iso, and this property is what characterizes 
    \textbf{abelian categories}.
\end{rmk}
%%    \section{Additive categories}\label{AdditiveCategories}

\begin{df}
    A \textbf{preadditive category} is a category $\Aa$ 
    in which each set  of morphisms $\Hom_{\Aa}(A,B)$ has an operation $+_{AB}$
    that gives it an abelian group structure, 
    in such a way that the composition maps
    \begin{equation}\label{compatibility}
    \begin{split}
        \circ_{ABC} : \Hom_{\Aa}(B,C) \times \Hom_{\Aa}(A,B) \to \Hom_{\Aa}(A,C)\,, \\
        \quad (g,f) \longmapsto  g \circ f := \circ_{ABC}(g,f)\,
    \end{split}
    \end{equation}
    are group homomorphisms in each variable, that is
    \begin{equation*}
        g \circ (f +_{AB} f') = g \circ f +_{AC} g \circ f'\,,
        \quad (g +_{BC} g') \circ f = g \circ f +_{AC} g \circ f'\,.
    \end{equation*}
\end{df}

\begin{rmk}
    If one is familiar with the concept of 
    \textbf{enriched category},
    then they can equivalently define 
    a preadditive category $\Aa$ as an enriched 
    $\cat{Ab}$-category, %with a zero object, 
    where $\cat{Ab}$ is the category
    of abelian groups with group homomorphisms;
    notice that  $(\cat{Ab},\otimes_{\Z},\Z)$ 
    is a monoidal category, 
    so we may rewrite the composition morphism \eqref{compatibility}
    as a homomorphism of abelian groups
    \begin{equation*}
        \Hom_{\Ab}(B,C) \otimes_{\Z} \Hom_{\Ab}(A,B) \to \Hom_{\Ab}(A,C)\,, 
        \quad g \otimes f \longmapsto g \circ f\,;
    \end{equation*}
    indeed, the universal property the tensor products 
    tells that it ``converts'' bilinear maps
    into homomorphisms.
\end{rmk}

\begin{ex}
    Let $\Aa = \cat{Ab}$. We already noticed that the category
    of abelian groups has the zero element $\cat{0} = \set{0}$,
    so for every pair of groups $G,H$, we have the zero morphism
    $0_{GH}:G \to \cat{0} \to H$.
    Then $\cat{Ab}$ is a \textbf{preadditive category}:
    if $(G, +_{G})$ and $(H,+_{H})$ are abelian groups, then
    \begin{align*}
        +_{GH} : \Hom_{\Ab}(G,H) \times \Hom_{\Ab}(G,H) &\to \Hom_{\Ab}(G,H)\,,\\
        (\phi,\psi) &\longmapsto [\phi +_{GH} \psi : g \mapsto \phi(g) +_{H} \psi(g)]
    \end{align*}
    defines an associative operation, whose neutral element is $0_{GH}$;
    moreover, for every homomorphism $\phi:G \to H$, its inverse is 
    given by the pointwise inverse $-\phi : g \longmapsto -\phi(g)$. 
    Taken any $\phi,\phi' \in \Hom_{\Ab}(G,H)$
    and $\psi, \psi' \in \Hom_{\Ab}(H,K)$, then for every $g \in G$ it holds
    \begin{align*}
        \psi \circ (\phi +_{GH} \phi')(g) &= \psi(\phi(g) +_{H} \phi'(g)) \\
        &= \psi(\phi(g)) +_{K} \psi(\phi'(g)) \\
        &= (\psi \circ \phi +_{GK} \psi \circ \phi')(g)\,,
    \end{align*}
    so we deduce that $\psi \circ (\phi +_{GH} \phi') = \psi \circ \phi +_{GK} \psi \circ \phi'$, and similarly $(\psi +_{HK} \psi') \circ \phi = (\psi \circ \phi) +_{GK} (\psi' \circ \phi)$, which means that the composition of
    homomorphisms is a bilinear map.
\end{ex}

\begin{ex}
    Let $R$ be a ring with unity. Then the category $\Aa = \Mod_{R}$ 
    of left $R$-modules is preadditive, and the proof
    of this is similar to one for $\cat{Ab}$.
\end{ex}

\begin{ex!}\label{ring-cat}
    A ring $R$ with unit can be made into a preadditive
    category $\cat{P}(R)$ in the following way:
    let $\cat{P}(R)$ consist of just one object $\ast$,
    whose arrows $\Hom_{\cat{P}(R)}(\ast, \ast) = R$
    are the elements of the ring. Given $r,s \in R$,
    define the composition $r \circ s := rs$ to be the product
    in the ring. Then by the definition of ring we see
    that $\cat{P}(R)$ has a group structure,
    and the composition of morphisms distributes
    over the sum.
\end{ex!}

\begin{ex}
    Let $\Aa = \cat{Fld}$ be the category whose objects are fields and
    whose morphisms are unital ring homomorphisms 
    (i.e. $k \to K$ sends $1_{k} \mapsto 1_{K}$).
    Then $\cat{Fld}$ is \textbf{not} preadditive because
    $\Hom_{\cat{Fld}}(k,K) = \emptyset$ whenever 
    $\operatorname{char}k \ne \operatorname{char} K$,
    so it cannot have a group structure.
\end{ex}

\begin{prop}
    In a preadditive category $\Aa$, the following are equivalent:
    \begin{enumerate}
        \item $\Aa$ has an initial object;
        \item $\Aa$ has a terminal object;
        \item $\Aa$ has a zero object.
    \end{enumerate}
    \begin{proof}
        By definition of zero object, 
        \emph{(3)} implies both \emph{(1)} and \emph{(2)}.
        Now assume $\Aa$ has an initial object $\cat{0}$. 
        We prove it is also final. 
        Since $\cat{0}$ is initial, the set 
        $\Hom_{\Aa}(\cat{0}, \cat{0}) = \set{1_{\cat{0}}}$ is the trivial
        group; moreover, we know that for each object $C$,
        the set $\Hom_{\Aa}(C,\cat{0})$ has at least one element 
        $f : C \to \cat{0}$. By the computation
        \begin{equation*}
            f = f \circ 1_{\cat{0}} 
            = f \circ (1_{\cat{0}} + 1_{\cat{0}})
            = (f \circ 1_{\cat{0}}) + (f \circ 1_{\cat{0}})
            = f + f
        \end{equation*}
        we deduce that $f = 0_{C\cat{0}}$ is the only element
        of $\Hom_{\Aa}(C, \cat{0})$, hence \emph{(3)} follows.
        Similarly one proves that \emph{(2)} implies \emph{(3)}.
    \end{proof}
\end{prop} 

\begin{prop}\label{biproduct}
    Given two objects $A,B$ in a preadditive category $\Aa$,
    the following are equivalent:
    \begin{enumerate}
        \item the product $(P,\pi_{A}, \pi_{B})$ of $A,B$ exists;
        \item the coproduct $(P,j_{A},j_{B})$ of $A,B$ exists;
        \item there exists an object $P$ with morphisms
        \begin{equation*}
            \pi_{A} : P \to A\,, \quad \pi_{B} : P \to B\,,
            \quad j_{A} : A \to P\,, \quad j_{B} : B \to P\,,
        \end{equation*}
        with the properties
        \begin{align*}
            \pi_{A} \circ j_{A} &= 1_{A}\,, \quad
            \pi_{B} \circ j_{B} = 1_{B}\,, \\
            \pi_{B} \circ j_{A} &= 0_{AB}\,, \quad
            \pi_{A} \circ j_{B} = 0_{BA}\,, \\
            (j_{A} \circ &\pi_{A}) +_{PP} (j_{B} \circ \pi_{B})
            = 1_{P}\,.
        \end{align*}
    \end{enumerate}
    \begin{proof}
        By duality, it is enough to show that \emph{(1)} and \emph{(3)}
        are equivalent.
        
        Assume \emph{(1)} holds. By the universal property of the
        product, there exist a unique morphisms $j_{A}$ and $j_{B}$ that
        make the diagrams commute:
        \begin{equation*}
            \begin{tikzcd}
                    & A \ar[d,"j_{A}", dashed] \ar[dr, "0_{AB}", bend left] \ar[dl, "1_{A}"', bend right] & 
                    & & B \ar[d,"j_{B}", dashed] \ar[dr, "1_{B}", bend left] \ar[dl,"0_{BA}"', bend right] &  \\
                    A & P \ar[l, "\pi_{A}"] \ar[r, "\pi_{B}"'] & B 
                    &  A & P \ar[l, "\pi_{A}"] \ar[r, "\pi_{B}"'] & B\,.
            \end{tikzcd}
        \end{equation*}
        The computation
        \begin{align*}
            &\pi_{A} \circ (j_{A} \circ \pi_{A} + j_{B} \circ \pi_{B}) = \pi_{A} + 0_{BA} = \pi_{A}\,, \\
            & \pi_{B} \circ (j_{A} \circ \pi_{A} + j_{B} \circ \pi_{B}) = 0_{AB} + \pi_{B} = \pi_{B}\,,
        \end{align*}
        shows that the diagram
        \begin{equation*}
            \begin{tikzcd}
                    & &P \ar[d,"j_{A}\pi_{A} + j_{B}\pi_{B}", dashed] \ar[drr, "\pi_{B}", bend left] \ar[dll, "\pi_{A}"', bend right] & &\\
                    A & & P \ar[ll, "\pi_{A}"] \ar[rr, "\pi_{B}"'] & &  B \,
            \end{tikzcd}
        \end{equation*}
        commutes, hence $j_{A} \circ \pi_{A} + j_{B} \circ \pi_{B} = 1_{P}$.
        
        Conversely, if we have morphisms $f:C \to A$ and $g: C \to B$, 
        then they factor through $h := j_{A} \circ f + j_{B} \circ g$, indeed
        \begin{align*}
            \pi_{A} \circ h 
            = (\pi_{A} \circ j_{A}) \circ f + (\pi_{A} \circ j_{B}) \circ g
            = 1_{A} \circ f + 0_{BA} \circ g = f\,,\\
            \pi_{B} \circ h 
            = (\pi_{B} \circ j_{A}) \circ f + (\pi_{B} \circ j_{B}) \circ g
            = 0_{AB} \circ f + 1_{B} \circ g = g\,.\\
        \end{align*}
        Moreover, given $h':C \to P$ with the same property, then
        \begin{align*}
            h' = 1_{P} \circ h' &= (j_{A} \circ \pi_{A} + j_{B} \circ \pi_{B}) \circ h'\\
            &= j_{A} \circ (\pi_{A} \circ h') + j_{B} \circ (\pi_{B} \circ h')\\
            &= j_{A} \circ f + j_{B} \circ g = h\,,
        \end{align*}
        so we deduce that $(P,\pi_{A},\pi_{B})$ is the product of $A$ and $B$.
    \end{proof}
\end{prop}

\begin{df}
    Given two objects $A,B$ in a preadditive category, 
    a quintuple 
    $(P,\pi_{A}, \pi_{B}, j_{A}, j_{B})$ as in 
    \hyperref[biproduct]{\textbf{Proposition~\ref*{biproduct}}~\emph{.3}}
    is called a \textbf{biproduct}, or \textbf{direct sum},
    and we write $A \oplus B$.
\end{df}

\begin{df}
    An \textbf{additive category} is a preadditive category with
    a zero object and binary biproducts.
\end{df}

\begin{ex}
    The category $\Aa = \cat{Ab}$ of abelian groups is an additive
    category. Given a ring $R$ with unit, the category $\Aa = {}_{R}\Mod$ 
    of left $R$-modules is additive.
\end{ex}

\begin{ex}
    Given a ring $R \ne 0$ with unit, the category
    $\cat{P}(R)$ is \textbf{not} additive, 
    for it has one object only and it is not a zero.
\end{ex}

\begin{ex}
    The category $\Aa = \cat{Grp}$ of groups is \textbf{not} additive:
    indeed, given two groups $A$ and $B$, their product is the
    direct product $A \times B$, while their coproduct is the
    free product $A \ast B$, and these two objects
    are not isomorphic in general.
\end{ex}

\begin{ex}
    The category $\Aa = \cat{CRing}$ 
    of unital commutative rings
    has both products and coproducts: 
    given $A,B$ two rings,
    then $A \prod B = A \times B$ is the direct product,
    while $A \coprod B = A \otimes_{\Z} B$ is the tensor
    product, and these two are \textbf{not} isomorphic in general.
    Thus, $\cat{CRing}$ is \textbf{not} additive.
    The same holds true for the category of unital rings,
    but the construction of the coproduct for
    the non-commutative case is a little bit trickier
    (see \parencite{coprod-rings} for an idea).
\end{ex}

\begin{df}
    Given two preadditive categories $\Aa$ and $\Bb$,
    a functor $F: \Aa \to \Bb$ is called \textbf{additive} 
    if, for every pair of objects $A,A' \in \Aa$,
    the map
    \begin{equation*}
       F_{AA'} : \Hom_{\Aa}(A,A') \longrightarrow \Hom_{\Bb}(F(A),F(A'))\,,
       \quad f \longmapsto F(f)\,,
    \end{equation*}
    is a group homomorphism. 
    Additive functors from $\Aa$ to $\Bb$ and
    natural transformations between them
    form a category $\cat{Fun}_{+}(\Aa,\Bb)$.
\end{df}

\begin{ex}
    Given a preadditive category $\Aa$ and an object $B$,
    the functor
    \begin{equation*}
        \Hom_{\Aa(-,B)} : \Aa^{op} \longrightarrow \cat{Ab}\,,
        \quad A \longmapsto \Hom_{\Aa}(A,B)
    \end{equation*}
    is additive. Indeed, given any two objects $A,A' \in \Aa$,
    we verify that
    \begin{equation*}
        \Hom_{\Aa}(A,A') \longrightarrow 
        \Hom_{\Ab}\big(\Hom_{\Aa}(A',B),\Hom_{\Aa}(A,B)\big)\,,
        \quad
        f \longmapsto [g \longmapsto g \circ f]
    \end{equation*}
    is a homomorphism: given any two $f,\tilde{f} : A \to A'$,
    then for every $g : A' \to B$ it holds
    \begin{equation*}
        \Hom_{\Aa}(A,f+\tilde{f})(g) = g \circ (f+\tilde{f}) 
        = g \circ f + g \circ \tilde{f}
        = \Hom_{\Aa}(A,f)(g) + \Hom_{\Aa}(A,\tilde{f})(g)\,,
    \end{equation*}
    where $\circ$ distributes over the sum by definition
    of a preadditive category.
\end{ex}

The previous example points out that
a contravariant representable functor from a
preadditive category is always additive.
Hence, we get the following version
for preadditive categories
of the \hyperref[yoneda]{\textbf{Yoneda's Lemma}~\ref*{yoneda}}.

\begin{thm}[\textbf{Additive Yoneda's Lemma}]\label{additive-yoneda}
    If $\Aa$ is a preadditive category,
    $A$ an object in $\Aa$ and $F : \Aa^{op} \to \cat{Ab}$
    an additive functor,
    then there exists an isomorphism of
    abelian groups
    \begin{equation*}
        \cat{Fun}_{+}(\Aa^{op},\Ab)\big(\Hom_{\Aa}(-,A),F\big) \simeq F(A)
    \end{equation*}
    which is natural both in $A$ and in $F$.
    \begin{proof}
        The proof goes through as in the ``classic''
        \textbf{Yoneda's Lemma}: on one side, one sends a
        transformation $\alpha : \Hom_{\Aa}(-,A) \to F$
        to the element $\alpha_{A}(1_{A}) \in F(A)$
        and conversely, for any $x \in F(A)$,
        one defines $\alpha^{x}$ to be the natural transformation
        \begin{equation*}
            \alpha_{B}^{x} : \Hom_{\Aa}(B,A) \longrightarrow F(B)\,,
            \quad f \longmapsto F(f)(x)\,.
        \end{equation*}
        These two constructions are one the inverse of the other
        and moreover they define group \emph{homomorphisms},
        indeed $\alpha_{A}: \Hom_{\Aa}(A,A) \to F(A)$ 
        is an arrow in $\cat{Ab}$
        and $\alpha^{x}$ is a homomorphism by additivity of $F$.
    \end{proof}
\end{thm}
%    %
\section{\texorpdfstring{$k$}{k}-linear categories}
% FOR MATH SYMBOLS IN TITLES 
% https://latex.org/forum/viewtopic.php?t=21256


As the categories we will eventally be interested in
have a geometric origin, i.e. are defined in terms
of certain variety over some base field $k$,
we usually deal with the following special kind of categories.

\begin{df}
    Let $k$ be an arbitrary field. 
    A \textbf{$k$-linear category} is an additive category
    $\Aa$ such that the groups $\Hom_{\Aa}(A,A')$
    are $k$-linear vector spaces, for any $A,A' \in \Aa$.
    Moreover, we require the compositions to be $k$-bilinear.
\end{df}

In short, a $k$-linear category $\Aa$ is a $\cat{Vect}_{k}$-enriched category,
where we require $\Aa(A,A') = \Hom_{\Aa}(A,A')$.

\begin{ex}
    Since $\Vv = \Vect{k}$ endowed with $\otimes_{k}$
    is a symmetric monoidal closed category,
    then it is enriched over itself,
    i.e. it is $k$-linear.
\end{ex}

\begin{df}
    Given $\Aa,\Bb$ two $k$-linear categories,
    an additive functor $F : \Aa \to \Bb$ 
    is called \textbf{$k$-linear}
    if the map $F_{A,A'} : \Hom_{\Aa}(A,A') \to \Hom_{\Bb}(F(A),F(A'))$ 
    is a $k$-linear map for any two
    $A,A' \in \Aa$. As in the additive case, 
    $k$-linear functors from $\Aa$ to $\Bb$ 
    and natural transformations
    form a category $\cat{Fun}_{k}(\Aa, \Bb)$.
\end{df}


As special case of the 
\hyperref[enriched-yoneda]{\textbf{Theorem~\ref*{enriched-yoneda}}},
we will adapt the \hyperref[yoneda]{\textbf{Yoneda's Lemma}~\ref*{yoneda}} 
to the $k$-linear setting:
first, we recall that a \textbf{$k$-linear} equivalence
$F : \Aa \to \Bb$ between two $k$-linear categories
is an equivalence which is a $k$-linear functor,
whose quasi-inverse $F^{-1}$ is again $k$-linear.

%\begin{thm}[Additive Yoneda]
%    For an additive category $\Aa$, the Yoneda embedding
%    \begin{equation*}
%        \Aa \longrightarrow \cat{Fun}_{+}(\Aa^{op},\Ab)\,, \quad
%        A \longmapsto \Hom_{\Aa}(-,A)
%    \end{equation*}
%    defines an equivalence between $\Aa$ and
%    the category of contravariant \emph{additive} functors
%    between $\Aa$ and the category of abelian groups.
%\end{thm}

%Similarly, it holds:

\begin{thm}[\textbf{Linear Yoneda's Lemma}]\label{linear-yoneda}
    Let $k$ be a field. For a $k$-linear category $\Aa$, 
    the Yoneda embedding
    \begin{equation*}
        \Aa \longrightarrow \cat{Fun}_{k}(\Aa^{*},\cat{Vect}_{k})\,, \quad
        A \longmapsto \Hom_{\Aa}(-,A)
    \end{equation*}
    defines an equivalence between $\Aa$ and
    the category of contravariant \emph{$k$-linear} functors
    between $\Aa$ and the category of vector spaces over $k$.
\end{thm}

\begin{lemma}\label{tensor-w-zero}
    Any $k$-linear category $\Dd$ is tensored over
    finite-dimensional vector spaces, i.e.
    given $n \in \NN$ and $A \in \Dd$,
    then
    \begin{equation*}
        k^{n} \otimes_{\Dd} A = A^{\oplus n}\,.
    \end{equation*}
    In particular, for every $A \in \Dd$, one has
    $0 \otimes_{\Dd} A \simeq \cat{0}$.
    \begin{proof}
        Given any object $B \in \Dd$, one notices that
        \begin{align*}
            \Hom_{\Dd}(k^{n} \otimes_{\Dd} A, B)
            \simeq \Hom_{k}(k^{n}, \Hom_{\Dd}(A,B)) 
            \simeq \Hom_{\Dd}(A,B)^{\oplus n}
            \simeq \Hom_{\Dd}\left( A^{\oplus n}, B \right)\,,
        \end{align*}
        thus, $k^{n} \otimes_{\Dd} A = A^{\oplus n}$ 
        by the covariant version of the \hyperref[linear-yoneda]{Linear Yoneda's Lemma}.
    \end{proof}
\end{lemma}

\begin{df}
    Let $\Aa$ be a $k$-linear category.
    A \textbf{Serre functor} is a $k$-linear equivalence
    $S : \Aa \to \Aa$ such that, for any two objects
    $A,A' \in \Aa$, there exists an isomorphism
    of $k$-vector spaces
    \begin{equation*}
        \sigma_{A,A'} : \Hom_{\Aa}(A,A') 
        \xrightarrow[]{\sim} \Hom_{\Aa}(A',S(A))^*\,,
    \end{equation*}
    which is functorial both in $A$ and $A'$.
\end{df}

A Serre functor induces a pairing
\begin{equation*}
    \Hom_{\Aa}(A',S(A)) \times \Hom_{\Aa}(A,A') \longrightarrow k\,,
    \quad (f,g) \longmapsto \langle f | g \rangle 
    := \sigma_{A,A'}(g)(f)\,.
\end{equation*}

In order to avoid any trouble with the dual, 
one usually assumes that all Hom’s in A are \emph{finite-dimensional}. 
Under this hypothesis it is easy to see that a Serre functor, 
if it exists, is unique up to isomorphism. 
More generally one has the following

\begin{lemma}\label{serre-eq-comm}
    Let $\Aa$ and $\Bb$ be $k$-linear categories with
    finite dimensional $\Hom$'s. Assume $\Aa$, resp. $\Bb$,
    is endowed with a Serre functor $S_{\Aa}$, resp. $S_{\Bb}$.
    Then any $k$-linear equivalence $F:\Aa \to \Bb$ commutes
    with Serre duality, i.e. there exists an isomorphism
    \begin{equation*}
        F \circ S_{\Aa} \simeq S_{\Bb} \circ F\,.
    \end{equation*}
    \begin{proof}
        We show the two functors are isomorphic
        by applying the \hyperref[yoneda]{Yoneda Lemma~\ref*{yoneda}}:
        given any two $A, A' \in \Aa$, 
        we compute the Serre duality in $\Aa$
        and we use that $F$
        is fully faithful to get
        \begin{align*}
            \Hom_{\Aa}(A,A') \simeq \Hom_{\Aa}(A',S_{\Aa}(A))^* 
            \simeq \Hom_{\Bb}\big(F(A'),F(S_{\Aa}(A))\big)^*\,.
        \end{align*}
        On the other hand, if we first apply $F$ and then Serre duality
        we obtain
        \begin{align*}
            \Hom_{\Aa}(A,A') 
            \simeq \Hom_{\Bb}(F(A),F(A'))
            \simeq \Hom_{\Bb}\big(F(A'),S_{\Bb}(F(A))\big)^*\,.
        \end{align*}
        Since $F$ is essentially surjective, 
        for every $B \in \Bb$ we have isomorphisms
        \begin{align*}
            \Hom_{\Bb}\big(F(A'),S_{\Bb}(F(A))\big)^*
            \simeq \Hom_{\Bb}\big(F(A'),S_{\Bb}(F(A))\big)^*\,,
        \end{align*}
        thus, by passing to the double dual we conclude.
    \end{proof}
\end{lemma}

\begin{rmk}
    Let $\Aa$ and $\Bb$ be $k$-linear categories 
    as in the hypothesis of \hyperref[serre-eq-comm]{Lemma~\ref*{serre-eq-comm}}.
    If $F: \Aa \to \Bb$ is a functor such that $G \dashv F$,
    then Serre duality gives us a way to build a right adjoint for $F$:
    indeed, we have $F \dashv S_{\Aa} \circ  G \circ S_{\Bb}^{-1}$.
    To see this,  we have natural isomorphisms
    \begin{align*}
        \Hom_{\Aa}(A, S_{\Aa} \circ  G \circ S_{\Bb}^{-1}(B)) 
        &\simeq \Hom_{\Bb}(G \circ S_{\Bb}^{-1}(B),A)^{*} \\
        &\simeq \Hom_{\Bb}(S_{\Bb}^{-1}(B),F(A))^{*} \\
        &\simeq \Hom_{\Bb} \big(F(A), S_{\Bb} \circ S_{\Bb}^{-1}(B)\big)^{**} \\
        &\simeq \Hom_{\Bb}(F(A),B)\,.
    \end{align*}
\end{rmk}

 A similar argument allows the construction of 
 a left adjoint if a right adjoint 
 $F \dashv H$ is given. 
 In particular, for functors between categories with 
 Serre functors the existence of the left or 
 the right adjoint implies the existence of the other one.

%%    
\section{Abelian categories}

\begin{df!}\label{df-abel}
    A category $\Aa$ is \textbf{abelian}
    when it satisfies the following properties:
    \begin{itemize}
        \item[(\textbf{A1})] $\Aa$ has a \textbf{zero object} $\cat{0}$;
        \item[(\textbf{A2})] every pair of objects of $\Aa$ has a product and a coproduct;
        \item[(\textbf{A3})] every morphism in $\Aa$ has both \textbf{kernel} and a \textbf{cokernel};
        \item[(\textbf{A4})] for every $f \in \Hom_{\Aa}(A,B)$,
        the canonical map $\operatorname{coim}f \to \im f$
        defined in \hyperref[coim-im]{\textbf{Proposition~\ref*{coim-im}}}
        is an isomorphism.
    \end{itemize}
\end{df!}

\begin{rmk}
    Axiom (\textbf{A4}) does not follow by the previous ones,
    as shown in the following examples;
    a category that satisfies (\textbf{A1}), (\textbf{A2}) and (\textbf{A3}) is usually called \textbf{pre-abelian}.
    As already noticed, (\textbf{A4}) tells us
    that in $\Aa$ it holds a sort of \textbf{First Isomorphism Theorem},
    but many authors (for instance \parencite[]{borceux})
    use an equivalent definition, where the last axiom 
    is substituted by:
    \begin{itemize}
         \item[(\textbf{A4}*)] every monomorphism of $\Aa$ is a kernel
        and any epimorphism of $\Aa$ is a cokernel.
    \end{itemize}
    An explanation for this equivalence may be found in \parencite[]{ab-eq-def}.
\end{rmk}

\begin{ex}
    Let $k$ be a field and $\cat{Vect}_{k}$ be the
    category of $k$-vector spaces.
    We define the category $\cat{Fil}(\cat{Vect}_{k})$ 
    of \textbf{filtered vector spaces}
    to be the category whose objects are decreasing filtrations
    \begin{equation*}
        V_{\bullet} := V  \supset \dots \supset V_{n} 
        \supset V_{n+1} \supset \dots \supset 0\,,
    \end{equation*}
    indexed on $\Z$, and morphisms 
    $\phi_{\bullet} : V_{\bullet} \to W_{\bullet}$ 
    are given by $k$-linear maps $\phi : V \to W$
    such that $\phi(V_{n}) \subset W_{n}$ for every $n \in \Z$.
    Then $\cat{Fil}(\cat{Vect}_{k})$ has the zero object
    $0_{\bullet}$ given by the trivial
    filtration $0 = 0 = \dots = 0$,
    and every pair $V_{\bullet}, W_{\bullet}$ has biproduct
    \begin{equation*}
        V_{\bullet} \oplus W_{\bullet} 
        = (V \oplus W)_{\bullet} 
        = V_{} \oplus W_{} \supset \dots \supset V_{n} \oplus W_{n}
        \supset V_{n+1} \oplus W_{n+1} \supset \dots \supset 0\,.
    \end{equation*}
    Moreover, this category satisfies the axiom (\textbf{A3}):
    for any morphism $\phi_{\bullet} : V_{\bullet} \to W_{\bullet}$,
    we have
    \begin{align*}
        \ker \phi_{\bullet} &= \ker \phi \supset \dots \supset (\ker \phi \cap V_{n}) 
        \supset (\ker \phi \cap V_{n+1}) \supset \dots 0\,,\\
        \Coker \phi_{\bullet} &= W/\im f \supset \dots \supset W_{n}/\im\left(f\vert_{V_{n}}\right)
        \supset W_{n+1}/\im \left(f\vert_{V_{n+1}} \right) \supset \dots \supset 0\,.
    \end{align*}
    Nevertheless, $\cat{Fil}(\cat{Vect}_{k})$ is \textbf{not}
    an abelian category, for axiom (\textbf{A4}) does not hold:
    consider $V_{\bullet}, W_{\bullet}$ to be the two filtrations
    \begin{equation*}
        V_{n} := \begin{cases}
            k\,, \quad &\text{if } n>0\,;\\
            0\,, \quad &\text{if } n \le 0\,;
        \end{cases}
        \qquad
        W_{n} := \begin{cases}
            k\,, \quad &\text{if } n\ge0\,;\\
            0\,, \quad &\text{if } n < 0\,,
        \end{cases}
    \end{equation*}
    and $\phi_{\bullet} : V_{\bullet} \to W_{\bullet}$ to be the morphism
    induced by the identity $1_{k} : k \to k$. 
    Then both $\ker \phi_{\bullet}$ and $\Coker \phi_{\bullet}$
    are the zero filtration, hence
    \begin{equation*}
        \operatorname{coim}\phi_{\bullet} = V_{\bullet}\,, \quad
        \im \phi_{\bullet} = W_{\bullet}\,,
    \end{equation*}
    but the two objects are not isomorphic, for any arrow
    $\psi_{\bullet}:W_{\bullet} \to V_{\bullet}$ has to vanish on $W_{0}=k$,
    so it is necessarily induced by the zero map $k \to 0$.
\end{ex}

\begin{ex}
    Let $\Aa = \cat{TopAb}$ be the category of topological abelian groups,
    that is abelian groups endowed with a topology in such a way that
    their operation $+$ and the inverse $x \mapsto -x$ are continuous maps;
    morphisms in $\cat{TopAb}$ are continuous homomorphisms.
    As for the category $\cat{Ab}$, one sees that $\cat{TopAb}$
    is an additive category; moreover, kernels and cokernels always exist,
    and for every arrow $f:G \to H$, one might check that
    $\ker f$ is the group theoretic kernel of $f$,
    endowed with the subspace topology $\ker f \subset G$,
    while
    \begin{equation*}
        \Coker f = \left. H \middle/ \ol{f(G)} \right. {\,,}
    \end{equation*}
    where $\ol{f(G)}$ denotes the closure in $H$.
    Thus, $\cat{TopAb}$ is a preabelian category, which is \textbf{not} abelian:
    indeed, consider the additive groups $\Q$ and $\R$ with the euclidean topology.
    The inclusion $\iota : \Q \subset \R$ is injective, hence $\ker \iota = 0$,
    from which we deduce that $\operatorname{coim} \iota = \Q/0 = \Q$.
    As $\Q$ is dense in $\R$, we see that 
    $\Coker \iota = \left. \R \middle/ \ol{\Q}\right. = 0$,
    so $\im \iota = \ker(\R \to 0) = \R$, thus $\operatorname{coim} \iota$
    and $\im \iota$ are not isomorphic.
\end{ex}

\begin{ex!}\label{equiv-modr}
    Let $R$ be a commutative ring with unit and $\cat{P}(R)$
    be the preadditive category defined in 
    \hyperref[ring-cat]{\emph{Example~\ref*{ring-cat}}}.
    The category $\cat{Add}(\cat{P}(R),\cat{Ab})$ of additive functors
    and natural tranformations is abelian, indeed it is equivalent
    to $\Mod_{R}$. 
    
    Given an additive functor $F:\cat{P}(R) \to \cat{Ab}$,
    the image $F(\ast)$ is an abelian group, endowed with a scalar multiplication
    \begin{equation*}
        R \times F(\ast) \longrightarrow F(\ast)\,, \quad
        (r,g) \longmapsto F(r)(g)\,.
    \end{equation*}
    Since $F$ is additive, one checks that 
    $$(r+r')g = F(r+r')(g) = F(r)(g)+F(r')(g) = rg +r'g$$
    and since it is a functor, is follows $1_{R}g = g$ and $r(g+g') = rg+rg'$,
    so $F(\ast)$ is an $R$-module. A natural transformation
    $\phi: F \to G$ consists of a group homomorphism
    $\phi' : F(\ast) \to G(\ast)$ such that for every $r \in R$
    the following square commutes:
    \begin{equation}\label{r-linear}
        \begin{tikzcd}
            F(\ast) \ar[r, "\phi'"] \ar[d, "r \cdot "] & G(\ast) \ar[d, "r\cdot"] \\
            F(\ast) \ar[r, "\phi'"] & G(\ast)\,,
        \end{tikzcd}
    \end{equation}
    that is $\phi'(rg) = r\phi'(g)$, hence $\phi'$ is $R$-linear.
    
    Conversely, given any $R$-module $M$, one defines an
    additive functor $F:\cat{P}(R) \to \cat{Ab}$ by setting $F(\ast) = M$
    and, for every $r \in R$, the morphisms $F(r) : M \to M$
    are given by the multiplication $F(r)(m):=r\cdot m$.
    Any $R$-linear map $\phi':M \to N$ in $\Mod_{R}$
    describes the same diagram as in \eqref{r-linear},
    so it induces a natural transformation between
    additive functors $F \to G$, whose images are respectively 
    $M$ and $N$. These two constructions are one the inverse of the other,
    thus the equivalence is proved.
\end{ex!}

%%% COMPARE WITH DEFINITION IN https://stacks.math.columbia.edu/tag/00ZX

The properties listed in 
\hyperref[df-abel]{\textbf{Definition~\ref*{df-abel}}}
give us the appropriate language to talk about exact sequences,
and hence abelian categories seem the appropriate setting
we were hoping to build to develop a notion of homology;
but after a closer inspection, one might notice
that something is missing... the algebra!
It is not clear from \hyperref[df-abel]{\textbf{Definition~\ref*{df-abel}}} 
if there exists of any algebraic operation in $\Aa$ or
on its morphism, so that we can actually talk about ``groups''.
Quite surprisingly, it turns out that any abelian
category is $\cat{Ab}$-enriched!

\begin{thm}
    Every abelian category is an additive category.
    \begin{proof}[Idea of proof]
    Let $\Aa$ be an abelian category.
    The main steps of the proof are the following:
        \begin{enumerate}
            \item for each object $A$ in $\Aa$, 
            let $\Delta_{A}:A \to A \times A$ be the diagonal morphism
            and $q_{A} : A \times A \to \Coker \Delta_{A}$ its cokernel.
            One checks that $\Coker \Delta_{A}$ is isomorphic to $A$
            via the composition $\operatorname{pr}_{1} \circ \langle 1_{A},0 \rangle$,
            where $\operatorname{pr}_{1}$ is the projection on
            the first component of the product;
            
            \item we define a map from the product to $A$, by composing
            \begin{equation*}
                \sigma_{A} : A \times A \overset{q}{\longrightarrow} \Coker \Delta_{A} 
                \longrightarrow A\,,
            \end{equation*}
            which maybe though of as the ``\emph{subtraction}'' in $A$:
            for any two arrows $f,g : B \to A$, we define $f-g$ to be the composite
            \begin{equation*}
            \begin{tikzcd}
                f-g : B \ar[r,"{\langle f, g \rangle}"] 
                & A \times A \ar[r, "\sigma_{A}"] & A\,.
            \end{tikzcd}
            \end{equation*}
            \item For every pair of objects $A,B$ in $\Aa$, define 
            \begin{equation*}
                + : \Hom_{\Aa}(B,A) \times \Hom_{\Aa}(B,A) \longrightarrow \Hom_{\Aa}(B,A)\,,
                \quad (f,g) \longmapsto f + g := f - (0_{BA} - g)\,.
            \end{equation*}
            One has to verify that this operation is associative, 
            with neutral element $0_{BA}$ and commutative, so that
            every set of morphisms $\Hom_{\Aa}(B,A)$ is an abelian group.
            It follows that $\Aa$ is a preadditive category,
            hence additive because it has a zero and both products
            and coproducts (which are isomorphic by \hyperref[biproduct]{\textbf{Proposition~\ref*{biproduct}}}).
        \end{enumerate}
        For a full and detailed proof, check \parencite[1.6~Additivity of abelian categories]{borceux}.
    \end{proof}
\end{thm}

As a consequence of \hyperref[coim-im]{\textbf{Proposition~\ref*{coim-im}}},
one has the following result.

\begin{thm}[\textbf{Image factorization}]
    Let $\Aa$ be an abelian category. Any morphism $f$ in $\Aa$
    can be factored uniquely as the composition $f = i \circ p$,
    where $i$ is a monomorphism and $p$ is a epimorphism.
    \begin{proof}
        We can apply \hyperref[coim-im]{\textbf{Proposition~\ref*{coim-im}}} 
        and take $i : \im f \hookrightarrow B$ and $p = \ol{f} \pi$,
        where $\pi : A \to \operatorname{coim}f$.
        All these maps are unique up to isomorphisms.
    \end{proof}
\end{thm}


%%    
\section{Homology functors}

As noted already, the study of an abstract abelian category is highly motivated by the
category of $R$-modules. There, the reader has probably seen that we have functors 
$H_{n}$ from chain complexes of $R$-modules back to $R$-modules. 
These functors are useful in many ways,
such as giving algebraic invariants like singular homology in algebraic topology. 


Let $\Aa$ be an abelian category.

\begin{df}
    A sequence of objects and composable morphisms in $\Aa$ indexed on $\Z$
    \begin{equation*}
        \begin{tikzcd}
            A_{\bullet} := \dots \ar[r]
            & A_{n+1} \ar[r, "d_{n+1}"]
            & A_{n} \ar[r, "d_{n}"]
            & A_{n-1} \ar[r]
            & \dots
        \end{tikzcd}
    \end{equation*}
    is a \textbf{chain complex} if $d_{n}d_{n+1}=0$ for every $n \in \Z$,
    and the $d_{n}$ are called \textbf{boundary maps}.
    Dually, if the indexing increases
    \begin{equation*}
        \begin{tikzcd}
            A^{\bullet} := \dots \ar[r]
            & A^{n-1} \ar[r, "d^{n-1}"]
            & A^{n} \ar[r, "d^{n}"]
            & A^{n+1} \ar[r]
            & \dots
        \end{tikzcd}
    \end{equation*}
    and $d^{n}d^{n-1}=0$ for every $n \in \Z$,
    we call $A^{\bullet}$ a \textbf{cochain complex},
    and the morphisms are the \textbf{coboundary maps}.
\end{df}

\begin{df}
    Given two chain complexes $A_{\bullet}, B_{\bullet}$ in $\Aa$,
    a \textbf{chain map} $\phi_{\bullet} : A_{\bullet} \to B_{\bullet}$
    is a sequence of morphisms $\phi_{n} \in \Aa(A_{n}, B_{n})$,
    so that the following diagram commutes:
    \begin{equation*}
        \begin{tikzcd}
            \dots \ar[r] 
            & A_{n+1} \ar[d, "\phi_{n+1}"] \ar[r, "d_{n+1}^{A}"]
            & A_{n} \ar[d, "\phi_{n}"] \ar[r, "d_{n}^{A}"]
            & A_{n-1} \ar[d, "\phi_{n-1}"] \ar[r]
            & \dots \\
            \dots \ar[r] 
            & B_{n+1} \ar[r, "d_{n+1}^{B}"']
            & B_{n} \ar[r, "d_{n}^{B}"']
            & B_{n-1} \ar[r]
            & \dots
        \end{tikzcd}
    \end{equation*}
    In a similar fashion one defines \textbf{cochain maps}.
\end{df}

The composition of two chain maps $\psi_{\bullet} \circ \phi_{\bullet}$
is given by the sequence of compositions 
$(\psi_{n} \circ \phi_{n})_{n \in \Z}$
and it is easy to see that it is associative;
moreover, each chain $A_{\bullet}$ has the identity morphism 
$\cat{1}_{A_{\bullet}} = (\cat{1}_{A_{n}})_{n \in \Z}$,
hence chain complexes and chain maps in $\Aa$ form a category
which will be denoted by $C_{\bullet}(\Aa)$.
Analogously, cochain complexes in $\Aa$ and cochain morphisms
form a category we will denote by $C^{\bullet}(\Aa)$.

\begin{prop}
    Given any abelian category $\Aa$, then $C_{\bullet}(\Aa)$
    is again abelian.
    \begin{proof}
        We just point out what are the objects
        that satisfy axioms (\textbf{A1}) - (\textbf{A4}),
        without checking the details.
        If $\cat{0}$ be the zero object of $\Aa$, 
        then the zero complex
        \begin{equation*}
            \cat{0}_{\bullet} := 
            \dots \longrightarrow \cat{0} \longrightarrow \cat{0} \longrightarrow \cat{0} \longrightarrow \dots
        \end{equation*}
        is the zero object in $C_{\bullet}(\Aa)$.
        For any two complexes $A_{\bullet}, B_{\bullet}$,
        we know that for every $n \in \Z$ there exists the
        biproduct $A_{n} \oplus B_{n}$ in $\Aa$,
        so one easily checks that the complex
        \begin{equation*}
            \begin{tikzcd}[column sep=large]
                \dots \ar[r] 
                & A_{n+1} \oplus B_{n+1} \ar[r, "d^{A}_{n+1} \oplus d^{B}_{n+1}"]
                & A_{n} \oplus B_{n} \ar[r, "d^{A}_{n} \oplus d^{B}_{n}"]
                & A_{n-1} \oplus B_{n-1} \ar[r] & \dots
            \end{tikzcd}
        \end{equation*}
        is exactly the biproduct $A_{\bullet} \oplus B_{\bullet}$.
        Similarly, also kernels and cokernels are built layer by layer:
        given a chain map $\phi_{\bullet} : A_{\bullet} \to B_{\bullet}$,
        then $\ker \phi_{\bullet}$ is the chain whose objects are 
        $(\ker \phi_{n})_{n \in \Z}$, and the boundary maps
        are given by the universal property of the kernel
        \begin{equation*}
            \begin{tikzcd}
                & \dots \ar[d] & \dots \ar[d] \\
                \ker \phi_{n} \ar[r, hook] \ar[d, "\exists !"', dashed] \ar[dr, "h"]
                & A_{n} \ar[r, "\phi_{n}"] \ar[d, "d_{n}^{A}"]
                & B_{n} \ar[d, "d_{n}^{B}"] \\
                \ker \phi_{n-1} \ar[r, hook] 
                & A_{n-1} \ar[r, "\phi_{n-1}"] \ar[d]
                & B_{n-1} \ar[d] \\
                & \dots & \dots\,,
            \end{tikzcd}
        \end{equation*}
        because it holds $\phi_{n-1}h = 0$ by the commutativity of the square.
        Dually, one sees that $\Coker \phi_{\bullet}$ is given
        by the sequence $(\Coker \phi_{n})_{n \in \Z}$ and boundary maps
        obtained by the universal property. Finally, by abelianity of $\Aa$,
        we have isomorphisms $\operatorname{coim} \phi_{n} \simeq \im \phi_{n}$
        for every $n \in \Z$, and hence they induce an isomorphism
        of complexes $\operatorname{coim} \phi_{\bullet} \simeq \im \phi_{\bullet}$,
        thus we conclude that $C_{\bullet}(\Aa)$ is an abelian category.
    \end{proof}
\end{prop}

\missingfigure{Define the inclusion of $\Aa$ into the category $C^{\bullet}(\Aa)$. }

We now want to generalize the homology functors to an arbitrary abelian category.
We wish to construct functors $H_n : C_{\bullet}(\Aa) \to \Aa$. 
In the category of $R$-modules, 
this is greatly simplified because we can conclude that 
$\im d_{n+1} \subset \ker d_n$ using only set-theoretic notions, 
but this becomes more complicated if $\Aa$ is a general abelian category.
In the case of $R$-modules, the $n$-th homology of a chain complex $A_{\bullet}$
is the quotient $\ker d_{n}/\im d_{n+1}$, which is exactly 
$\Coker(\im d_{n+1} \subset \ker d_n)$. In a more general setting,
set-theoretical inclusions may not have sense;
nevertheless, we can always find a morphism $\im d_{n+1} \to \ker d_{n}$,
and its cokernel always exists in $\Aa$! Consider the diagram
\begin{equation*}
\begin{tikzcd}
    A_{n+1} \ar[r, "d_{n+1}"] \ar[d, two heads] 
    & A_{n} \ar[r, "d_{n}"] & A_{n-1} \\
    \im d_{n+1} \ar[ur, hook, "i"] \ar[r, dashed, "\exists !"] 
    & \ker d_{n} \ar[u, hook] & \,.
\end{tikzcd}
\end{equation*}
By the image factorization, we see that $d_{n}i=0$, 
so $i$ must factor through $\ker d_{n}$;
notice that $\im d_{n+1} \to \ker d_{n}$ is a monomorphism
because their composition is a monomorphism.

\begin{df}
    Define the \textbf{$n$-th homology object} of $A_{\bullet}$ to be
    \begin{equation*}
        H_{n}(A_{\bullet}) := \Coker\left(\im d_{n+1} \to \ker d_{n} \right)\,.
    \end{equation*}
\end{df}

\begin{rmk!}\label{homology-defs}
    The homology object can be defined in three different ways:
    a sequence $A \overset{f}{\to} B \overset{g}{\to} C$ such that
    $gf=0$ gives raise to the following diagram
    \begin{equation*}
        \begin{tikzcd}
            & \im f \ar[dr, hook] \ar[rr, dashed, "\phi"] & & \ker g \ar[dl, hook] \ar[dr, black!50, "0"] & \\
            A \ar[dr, black!50, "0"'] \ar[rr, "f"] \ar[ur, two heads] & & B \ar[rr, "g"] \ar[dl, two heads] \ar[dr, two heads]
            & & C \\
            & \Coker f \ar[rr, dashed, "\psi"] & & \im g \ar[ur, hook] & \,.
        \end{tikzcd}
    \end{equation*}
    One can show that, in any abelian category, the following three
    objects are canonically isomorphic:
    \begin{equation*}
        \Coker \phi \simeq \ker \psi \simeq \im(\ker g \to \Coker f)\,.
    \end{equation*}
    To get an idea why this is true, check \parencite[]{hom-eq-def}.
\end{rmk!}

We would like the construction $H_{n}$ to be functorial,
i.e. given a chain map $\phi_{\bullet} : A_{\bullet} \to B_{\bullet}$,
we can define a map $H_{n}\phi_{\bullet} : H_{n}(A_{\bullet}) \to H_{n}(B_{\bullet})$.
Consider the diagram
\begin{equation*}
    \begin{tikzcd}
        \im d_{n+1}^{A} \ar[r, hook, "j"] 
        & \ker d_{n}^{A} \ar[r] \ar[d, "i_{A}", hook] & H_{n}(A_{\bullet}) \\
        A_{n+1} \ar[u, two heads] \ar[r, "d_{n+1}^{A}"] \ar[d, "\phi_{n+1}"]
        & A_{n} \ar[r, "d_{n}^{A}"] \ar[d, "\phi_{n}"] & A_{n-1} \ar[d, "\phi_{n-1}"]\\
        B_{n+1}  \ar[r, "d_{n+1}^{B}"] \ar[d, two heads]
        & B_{n}  \ar[r, "d_{n}^{B}"] & B_{n-1} \\
        \im d_{n+1}^{B} \ar[r, hook] 
        & \ker d_{n}^{B} \ar[u, hook, "i_{B}"] \ar[r, "l"]
        & H_{n}(B_{\bullet})\,.
    \end{tikzcd}
\end{equation*}
Since $d_{n+1}^{B} \circ \phi_{n} \circ i_{A} = \phi_{n-1} \circ d^{A}_{n} \circ i_{A} = 0$,
then there exists a map $k: \ker d^A_{n} \to \ker d^{B}_n$.
If we denote by $q : B_{n} \to \Coker d_{n+1}^{B}$, 
then by the fact that
\begin{equation*}
    0 = q \circ \phi_{n} \circ d_{n+1}^{A} = q \circ \phi_{n}\circ i_{A} \circ j\,,
\end{equation*}
we deduce there exists a unique map $\im d^{A}_{n+1} \to \im d^{B}_{n+1}$
because $\im d^{B}_{n+1} = \ker q$. Thus, we end up with the commutative diagram
\begin{equation*}
    \begin{tikzcd}
        \im d_{n+1}^{A} \ar[d] \ar[r, hook, "j"] 
        & \ker d_{n}^{A} \ar[r] \ar[d, "k"] & H_{n}(A_{\bullet}) \\
        \im d_{n+1}^{B} \ar[r, hook] 
        & \ker d_{n}^{B} \ar[r, "l"]
        & H_{n}(B_{\bullet})\,,
    \end{tikzcd}
\end{equation*}
and we see that the composition $l \circ k \circ j = 0$, 
so by the universal property of $\Coker j$ there exists a unique map
$H_{n}\phi_{\bullet}$ that makes the following diagram commute:
\begin{equation*}
    \begin{tikzcd}
        & & H_{n}(A_{\bullet}) \ar[d, dashed, "\exists !"', "H_{n}\phi_{\bullet}"]\\
        \im d_{n+1}^{A} \ar[r, hook, "j", shift left=0.7ex] \ar[r, shift right=0.7ex, "0"']
        & \ker d_{n}^{A} \ar[r, "l \circ k"] \ar[ur]
        & H_{n}(B_{\bullet})\,.
    \end{tikzcd}
\end{equation*}
This construction shows how the $n$-th homology acts on morphisms.

\begin{prop}
    The construction above defines an additive covariant functor
    $H_{n} : C_{\bullet}(\Aa) \to \Aa$, called the
    \textbf{$n$-th homology functor}.
    \begin{proof}[Idea of proof]
        One has to check that $H_{n}(\cat{1}_{A_{\bullet}}) 
        = \cat{1}_{H_{n}(A_{\bullet})}$
        and that $H_{n}$ is compatible both with the composition
        and the sum of morphisms, essentially by exploiting
        the universal properties of kernels and cokernels.
        Everything works well because $\Aa$ has an additive structure,
        but we will not go through the details here.
    \end{proof}
\end{prop}

\begin{df}
    A composable pair of morphisms in $\Aa$
    \begin{equation*}
        \begin{tikzcd}
            A \ar[r, "f"] & B \ar[r, "g"] & C\,
        \end{tikzcd}
    \end{equation*}
    is called an \textbf{exact sequence} if
    the image of $f$ coincides with the kernel of $g$;
    more precisely, if we consider the 
    \emph{image factorization} of $f$
    \begin{equation*}
        \begin{tikzcd}
            A \ar[rr, "f"] \ar[dr, two heads] 
            & & B \ar[r, "g"] & C \\
            & \im f \ar[ur, "i", hook] & & \,,
        \end{tikzcd}
    \end{equation*}
    then $(\im f,i) \simeq \ker g$. A finite or infinite
    sequence of morphisms
    \begin{equation*}
        \dots \longrightarrow A_{n-1} \overset{f_{n-1}}{\longrightarrow}
        A_{n} \overset{f_{n}}{\longrightarrow} A_{n+1} \longrightarrow \dots
    \end{equation*}
    is \textbf{exact} if every pair of consecutive
    arrows is an exact sequence as in the previous sense.
    A \textbf{short exact sequence} is an exact sequence
    in $\Aa$ of the form
    \begin{equation*}
        \begin{tikzcd}
            \cat{0} \ar[r] & A \ar[r, "f"] & B \ar[r, "g"] & C \ar[r] & \cat{0}\,.
        \end{tikzcd}
    \end{equation*}
\end{df}

As promised, we now have all the tools we need
to develop homological algebra, and thus 
we can generalize many well-known results
to any abelian category, for instance:
\begin{thm}[\textbf{The Snake's Lemma}]\label{snake-lemma}
    In an abelian category $\Aa$, 
    any commutative diagram of the form
    \begin{equation*}
        \begin{tikzcd}
            %\cat{0} \ar[r]  
            & A \ar[r] \ar[d, "f"] & B \ar[r] \ar[d, "g"] & C \ar[r] \ar[d, "h"] & \cat{0}\\
            \cat{0} \ar[r] & A' \ar[r] & B \ar[r] & C &
            %\ar[r] & \cat{0}
        \end{tikzcd}
    \end{equation*}
    with exact rows induces an exact sequence
    \begin{equation*}
    \begin{tikzcd}[column sep=small]
        %\cat{0} \ar[r] 
        & \ker f \ar[r] & \ker g \ar[r] & \ker h \ar[out=0, in=180, looseness=2, overlay]{dll} & \\
        & \Coker f \ar[r] & \Coker g \ar[r] & \Coker h \ar[r] & \,.
        %\cat{0}\,.
    \end{tikzcd}
    \end{equation*}
\end{thm}

A very important consequence of this theorem is
the following result, which relates the homology
of different chain complexes.

\begin{thm}[\textbf{Long Homology Sequence}]\label{LHS}
    Let $\Aa$ be an abelian category. 
    Any short exact sequence in $C_{\bullet}(\Aa)$
    \begin{equation*}
        \begin{tikzcd}
            \cat{0} \ar[r] & A_{\bullet} \ar[r] & B_{\bullet} \ar[r] & C_{\bullet} \ar[r] & \cat{0}
        \end{tikzcd}
    \end{equation*}
    induces a long exact sequence in $\Aa$ of the form
    \begin{equation*}
    \begin{tikzcd}[column sep=small]
        \dots \ar[r] & H_{n}(A_{\bullet}) \ar[r] & H_{n}(B_{\bullet}) \ar[r] & H_{n}(C_{\bullet}) \ar[out=0, in=180, looseness=2, overlay]{dll} & \\
        & H_{n-1}(A_{\bullet}) \ar[r] & H_{n-1}(B_{\bullet}) \ar[r] & H_{n-1}(C_{\bullet}) \ar[r] & \dots\,.
    \end{tikzcd}
    \end{equation*}
    \begin{proof}
        For every $n \in \Z$, 
        apply the \hyperref[snake-lemma]{\textbf{Snake's Lemma~\ref*{snake-lemma}}}
        to every diagram
        \begin{equation*}
        \begin{tikzcd}
            %\cat{0} \ar[r]  
            & \Coker d_{n+1}^{A} \ar[r] \ar[d] & \Coker d_{n+1}^{B} \ar[r] \ar[d] & \Coker d_{n+1}^{C} \ar[r] \ar[d] & \cat{0}\\
            \cat{0} \ar[r] & \ker d_{n-1}^{A} \ar[r] & \ker d^{B}_{n-1} \ar[r] & \ker d^{C}_{n-1} &
            %\ar[r] & \cat{0}
        \end{tikzcd}
    \end{equation*}
    and use \hyperref[homology-defs]{\emph{Remark~\ref*{homology-defs}}}. 
    %where the vertical maps are
    %obtained thanks to the universal property:
    %\begin{equation*}
    %    \begin{tikzcd}
    %        A_{n} \ar[dr, "0"'] \ar[rr, "d^{A}_{n}"] &&  A_{n-1} \ar[dl, two heads] \ar[rr, "d^{A}_{n-1}"] && A_{n-2} \\
    %        & \Coker d_{n}^{A} \ar[urrr, dashed, shorten=3ex] \ar[rr, dashed] & & \ker d_{n-1}^{A}  \ar[ul, hook] \ar[ur, "0"'] & \,.
    %    \end{tikzcd}
    %\end{equation*}
    \end{proof}
\end{thm}

%As one can imagine, to prove these diagram lemmas in great generality
%turns out to be a bit convoluted because one must keep track of
%many big commutative diagrams, use extensively the
%epi-mono factorizations and the universal properties of
%kernels and cokernels. Those students who
%followed some basic course in Algebraic Topology
%may be familiar in the case of $\Aa = \cat{Ab}$
%or $\Aa=\Mod_{R}$, for some commutative unital ring $R$:
%the objects in these category are \emph{sets} with
%an additional algebraic structure, so the proofs
%of such lemmas may are based on \emph{chasing elements}
%around the diagrams; that is the reason why many authors
%refer to this technique as ``\textbf{diagram chasing}''.
%In \parencite[1.9 Diagram chasing]{borceux},
%the author makes sense of the ``\emph{set-theoretic notation}''
%of the diagram chasing applied to an arbitrary abelian category, 
%and this choice is actually justified a fortiori
%by the following

%\begin{thm}[]
%    Every small abelian category $\Aa$ has a full, faithful
%    and exact\footnote{An additive functor between abelian categories 
%    is said to be \textbf{exact} if it sends 
%    short exact sequences to short exact sequences.}
%    embedding in a category $\Mod_{R}$ of modules over a ring $R$
%    (not necessarily commutative).
%\end{thm}

%\begin{ex}
%    We gave an explicit construction in \emph{Example~\ref{equiv-modr}},
%    but the general case is way more difficult!
%\end{ex}

\missingfigure{Insert definition of exact functor, and left/right exact too. Add examples.}
%
%%%%%%%%%%%%%%%%%%%%%%%%%%%%%%%%%%%%%%%%%%%%%%%%%%%%%%%%%%%%
%%%%%%%%%%%%%%%%%%%%%%%%%%%%%%%%%%%%%%%%%%%%%%%%%%%%%%%%%%%%
%%%%%%%%%%%%%%%%%%%%%%%%%%%%%%%%%%%%%%%%%%%%%%%%%%%%%%%%%%%%
%
%\chapter{Enriched categories}
%
%In the section of \hyperref[AdditiveCategories]{Additive categories}
%we dealt with some special kind of categories
%endowed with an additional algebraic structure: 
%indeed, morphisms between two objects in a (pre)additive category
%form an abelian group, which is himself an object in $\Ab$;
%thus, we say that it is \emph{$\Ab$-enriched}.
%
%In the context of algebraic geometry, 
%where one deals with varieties, schemes over a field $k$,
%sheaves, it often turns out that morphisms 
%between objects give rise, in fact, to vector spaces:
%thus, one led to talk about $\cat{Vect}_{k}$-enriched
%categories. In this chapter, we give the
%precise definition of what an \emph{enriched category} is,
%and we formalize the concept of a \emph{tensor} in such a setting.
%
%%Vedi Riehl cap. 3 e Borceaux cap.6
%%Questo serve solo a capire i prodotti tensori 
%%nelle $k$-linear categories.
%
%%   
\section{Monoidal categories}

%\emph{The following is taken from \parencite[Chapter~VII]{mclane}.}

%\vspace{6pt}

We want to give the definition of a category $\Vv$ 
endowed with a product $\otimes$ that behaves similarly
to a product in a \textbf{monoid}, that is a bifunctor 
$\otimes : \Vv \times \Vv \to \Vv$ which is \emph{associative}
\begin{equation*}
    X \otimes ( Y \otimes Z) = (X \otimes Y) \otimes Z
\end{equation*}
and there is an object $e$ which is the ``\emph{neutral element}''
with respect to this product, i.e.
\begin{equation*}
    e \otimes X = X = X \otimes e\,,
\end{equation*}
for all objects $X,Y,Z$ in $\Vv$.
As we know, requiring equalities in the above equations
may be too strict in the categorical setting,
hence we will relax this condition accordingly
by requiring that these properties hold \emph{up to natural isomorphism};
by experience, we know for instance that the category of vector spaces
over a field $k$ has this exact behaviour with the tensor product:
indeed, we have a canonical isomorphism
\begin{equation*}
    U \otimes_{k} ( V \otimes_{k} W) = (U \otimes_{k} V) \otimes_{k} W
\end{equation*}
for all $k$-vector spaces $U,V,W$, and the $1$-dimensional line
$k$ is the neutral element. 
Closer analysis shows that more care is requisite in this identification 
- one must use the right isomorphism, 
and one must verify that the resulting identification of multiple 
products can be made in a "coherent" way.

\begin{df}
    A \textbf{monoidal category} $\Vv = (\Vv, \otimes, e, \alpha, l, r)$
    is a category $\Vv$, a bifunctor $\otimes:\Vv \times \Vv \to \Vv$,
    an object $e$ and three natural isomorphisms:
    \begin{itemize}
        \item \textbf{associator:} 
        $\alpha : - \otimes ( - \otimes - ) \simeq (- \otimes -) \otimes -$;
        \item  \textbf{left unit law:} $l: e \otimes - \simeq \cat{1}_{\Vv} $;
        \item  \textbf{right unit law:} $r: - \otimes e \simeq \cat{1}_{\Vv} $;
    \end{itemize}
    such that they satisfy the following properties:
    \begin{itemize}
        \item[(\textbf{M5})]\label{M5} \textbf{pentagon equation:} for every four objects $X,Y,Z$ and $W$ 
        in $\Vv$, the following penthagon containing all possible dispositions
        of brackets among $4$ letters commutes:
        \begin{center}
        \begin{tikzcd}[column sep=0.1in]
             & (X \otimes Y) \otimes (Z \otimes W) \ar[rd, "\alpha"] &  \\
            X \otimes \big(  Y \otimes ( Z \otimes W ) \big) 
            \ar[ru, "\alpha"] \ar[d, "\cat{1}_{X} \otimes \alpha"']
            && \big( ( X \otimes Y )  \otimes Z \big) \otimes W\\
              X \otimes \big( (Y \otimes Z) \otimes W \big) \ar[rr, "\alpha"]
            & & \big( X \otimes ( Y \otimes Z) \big) \otimes W \ar[u, "\alpha \otimes \cat{1}_{W}"] 
        \end{tikzcd}
        \end{center}
        \item[(\textbf{M3})]\label{M3} \textbf{triangle equations:} 
        for every two objects $X, Y$ in $\Vv$, we want the neutral element $e$
        to be compatible with the associativity in the sense that the following triangle
        \begin{center}
        \begin{tikzcd}
            X \otimes (e \otimes Y) \ar[rr, "\alpha"] \ar[dr, "\cat{1}_{X} \otimes l"'] 
            & & (X \otimes e) \otimes Y \ar[dl, "r \otimes \cat{1}_{Y}"] \\
            & X \otimes Y & 
        \end{tikzcd}
        \end{center}
        is commutative.
    \end{itemize}
\end{df}

\begin{ex!}\label{finite-prod}
    Every category $\Vv$ with finite products is \textbf{monoidal}:
    for every $X, Y$ in $\Vv$, we define $X \otimes Y := X \times Y$ any chosen product,
    and the neutral element $e$ to be a terminal object. Then $l$ is given by composing 
    with the projection on the second factor, while $r$ by projecting on the first;
    $\alpha$ is the unique isomorphism given by the universal property of products.
    Then one checks that both property (\textbf{M5}) and (\textbf{M3}) 
    hold by noticing that all maps
    commute with the projections of the four (resp. three) fold products.
\end{ex!}

\begin{df}
    Let $\Vv$ be a monoidal category with product $\otimes$ and denote by
    $\tau : \Vv \times \Vv \to \Vv \times \Vv$ the ``switch'' functor
    that permutes the two components, i.e. for any two objects
    $X,Y \in \Vv$ it is defined as
    \begin{equation*}
        \tau(X,Y) := (Y,X)\,.
    \end{equation*}
    Then $\Vv$ is
    said to be \textbf{symmetric} if there is a natural isomorphism
    $\gamma : (-\otimes-) \simeq (-\otimes-) \circ \tau$, i.e.
    for every pair of object $X,Y$, there is an isomorphism
    $$\gamma_{X,Y} : X \otimes Y \simeq Y \otimes X\,,$$
    natural in both $X$ and $Y$; 
    moreover, it satisfies the following coherence
    conditions:
    \begin{itemize}
        \item[(\textbf{S1})]\label{S1}%
        the transformation $\gamma$ is involutive in the sense that
        \begin{equation*}
            \gamma_{X,Y} \circ \gamma_{Y,X} = \cat{1}_{X \otimes Y}\,;
        \end{equation*}
        \item[(\textbf{S2})]\label{S2}%
        it is compatible with the unital laws:
        \begin{equation*}
            r_{X} = l_{X} \circ \gamma_{X,e}\,;
        \end{equation*} 
        \item[(\textbf{S3})]\label{S3}%
        \textbf{exagon equation:} 
        for all $X,Y,Z$ in $\Vv$, the following diagram commutes:
        \begin{center}
        \begin{tikzcd}
            X \otimes (Y \otimes Z) \ar[r, "\alpha"] \ar[d, "\cat{1}_{X} \otimes \gamma"']%_{Y,Z}"']
            & (X \otimes Y) \otimes Z \ar[r, "\gamma"]
            & Z \otimes (X \otimes Y) \ar[d, "\alpha"] \\
            X \otimes (Z \otimes Y) \ar[r, "\alpha"]%_{X,Z \otimes Y}"] 
            & (X \otimes Z) \otimes Y \ar[r, "\gamma \otimes \cat{1}_{Y}"] 
            & (Z \otimes X) \otimes Y \,.
        \end{tikzcd}
        \end{center}
    \end{itemize}
\end{df}

\begin{ex}
    In fact, categories with finite products in \hyperref[finite-prod]{Example~\ref*{finite-prod}}
    are symmetric monoidal categories.
\end{ex}

\begin{ex}
    For each commutative unital ring $R$, the category $\Vv=\Mod_{R}$
    of modules over $R$ is monoidal with respect to the tensor product $\otimes_{R}$
    is a symmetric monoidal category. The neutral element is $e = R$.
    Our main interest is the case of $R=k$ a field 
    and vector spaces $\cat{Vect}_{k}$ over it.
\end{ex}

\begin{ex}
    The category $\Vv=\cat{CGHaus}_{\ast}$ of pointed compactly generated
    Hausdorff spaces is a symmetric monoidal category, 
    whose product $\otimes = \wedge$ is the \textbf{smash product}
    and neutral element given by $e = (S^{0}, +1)$ the discrete set with two points.
\end{ex}

\begin{ex}
    Remember that a poset $(P, \le)$ gives rise to a category $\cat{P}_{\le}$
    whose objects are elements of $P$, and arrows given by the ordering, i.e.
    for any $x,y \in P$, then there exists $x \to y$ if and only if $x \le y$.
    Assume $P$ has a top element $1$. If $P$ is a meet-semilattice,
    i.e. for any two elements $x,y \in P$ there exists their greatest lower bound
    $x \wedge y$, then the category $\Vv=\cat{P}_{\le}$ endowed with the operation
    $\wedge$ is a symmetric monoidal category, whose neutral element is $e=1$.
\end{ex}

\begin{ex!}\label{tensor-complex}
    Let $\Aa=\Mod_{R}$ and 
    consider any two cochain complexes $X^{\bullet}, Y^{\bullet} 
    \in \cat{C}^{\bullet}(\Aa)$.
    We define the \textbf{tensor product} $X^{\bullet} \otimes Y^{\bullet}$
    to be the complex whose $n$-th level is given by
        \begin{equation*}
            \left( X^{\bullet} \otimes Y^{\bullet} \right)^{n}
            := \bigoplus_{p+q=n} X^{p} \otimes_{R} Y^{q}\,,
        \end{equation*}
    and the coboundary map $d^{n}:\left( X^{\bullet} \otimes Y^{\bullet} \right)^{n}
    \to \left( X^{\bullet} \otimes Y^{\bullet} \right)^{n+1}$
    is defined on simple tensors as
    \begin{equation*}
        d^{n}(x \otimes y) := d_{X}^{p}(x) \otimes y + (-1)^{p}x \otimes d_{Y}^{q}(y)\,,
        \quad \text{with } x \in X^{p}, y \in Y^{q}\,.
    \end{equation*}
    One verifies that $(\Cc^{\bullet}(\Aa), \otimes)$ is a monoidal category,
    with unital element $e=R$, seen as a complex concentrated in degree zero.
    Moreover, $\Cc^{\bullet}(\Aa)$ is also symmetric:
    indeed, the isomorphisms $\gamma_{X^{\bullet},Y^{\bullet}}$
    are given by graded commutativity.
    
    One can generalize this example to any abelian category $\Aa$
    which is, in addition, a symmetrical monoidal category.
\end{ex!}

\begin{df}
    A symmetric monoidal category $(\Vv, \otimes)$ is \textbf{closed} if,
    for each object $B \in \Vv$, the functor
    \begin{equation*}
        - \otimes B: \Vv \longrightarrow \Vv
    \end{equation*}
    has a right adjoint, written $[B,-] : \Vv \to \Vv$.
\end{df}

\begin{rmk}
    The notion of monoidal closed category can be seen as
    a generalized setting in which it holds the ``$\otimes - \Hom$''
    adjunction that is familiar in the category of $R$-modules,
    as recalled in the next example.
\end{rmk}

\begin{ex}
    Given a commutative unital ring $R$, 
    the symmetric monoidal category $\Vv = \Mod_{R}$
    is closed: indeed, given any three $R$-modules $M,N$ and $P$, 
    it is well known that there exists a natural isomorphism
    \begin{equation*}
        \Hom_{R}(M \otimes_{R} N, P) \simeq \Hom_{R}\left(M, \Hom_{R}(N,P)\right)\,,
    \end{equation*}
    thus we deduce that $\Hom_{R}(N,-)$ is a right adjoint to $-\otimes_{R} N$.
\end{ex}


\begin{ex}
    By \hyperref[finite-prod]{Example~\ref*{finite-prod}},
    the category $\Vv = \cat{Set}$ has finite products, 
    hence it is a monoidal category, endowed with
    $\otimes = \times$ the cartesian product.
    Clearly, it is symmetric and, moreover,
    the \textbf{Exponential law} shows that $\cat{Set}$
    is monoidal closed: precisely, for any $Y$ a fixed set,
    the exponentiation functor
    \begin{equation*}
        (-)^{Y}:\cat{Set} \longrightarrow \cat{Set}\,,
        \quad Z \longmapsto Z^{Y} := \Hom_{\cat{Set}}(Y,Z)
    \end{equation*}
    is a left adjoint to $- \times Y$,
    for we have natural bijections
    \begin{equation*}
        Z^{X \times Y} \simeq \left(Z^{Y}\right)^{X}
    \end{equation*}
\end{ex}

\begin{rmk!}\label{v-morphisms}
    Taking the previous example into account, for every $X,Y \in \Vv$,
    we might think of $[X,Y] \in \Vv$ as the object in $\Vv$
    that classifies morphisms between $X$ and $Y$.
    In particular, notice that by definition one has
    \begin{equation*}
        \Hom_{\Vv}(X,X) \simeq \Hom_{\Vv}(e \otimes X,X) \simeq \Hom_{\Vv}(e,[X,X])\,,
    \end{equation*}
    which yield a \textbf{unit morphism} $\cat{u}_{X}:e \to [X,X]$, 
    corresponding with the identity $\cat{1}_{B}$.
    On the other hand, from the identifications
    \begin{equation*}
        \Hom_{\Vv}(Y,Y) \simeq \Hom_{\Vv}(Y \otimes e, Y) \simeq \Hom_{\Vv}(Y,[e,Y]),
    \end{equation*}
    one gets the isomorphism $i_{Y}:Y \to [e,Y]$, 
    which allows us to think of $[e,Y]$ as ``\emph{points of $Y$}'':
    note the similarity with $Y \simeq \Hom_{\cat{Set}}(\ast,Y)$ 
    in the category of sets, or even $G \simeq \Hom_{\Z}(\Z,G)$
    in $\Ab$. Finally, there is a stronger hint which validates
    this interpretation: for every $X,Y \in \Vv$, the adjunction
    \begin{equation*}
        \Hom_{\Vv}([X,Y], [X,Y]) \simeq \Hom_{\Vv}([X,Y] \otimes X, Y)
    \end{equation*}
    sends the identity $\cat{1}_{[X,Y]}$ 
    to the \textbf{evaluation morphism}
    \begin{equation*}
        \cat{ev}_{XY} : [X,Y] \otimes X \longrightarrow Y\,.
    \end{equation*}
    Thanks to this, for every triple $X,Y,Z \in \Vv$
    we can define \textbf{composition morphisms}
    \begin{equation*}
        \cat{c}_{XYZ} : [X,Y] \otimes [Y,Z] \longrightarrow [X,Z]
    \end{equation*}
    as the morphism corresponding via adjunction
    to the composite
    \begin{equation*}
            {[X,Y] \otimes [Y,Z] \otimes X} 
            \simeq {[X,Y] \otimes X \otimes [Y,Z]} 
            \to {Y \otimes [Y,Z]}
            \simeq {[Y,Z] \otimes Y}
            \to Z\,.
    \end{equation*}
\end{rmk!}

\begin{prop}
    In a symmetric monoidal closed category $\Vv$:
    \begin{rmnumerate}
        \item the unit morphisms $\cat{u}_{X}:e \to [X,X]$ are natural in $X$;
        \item the morphisms $i_{Y} : Y \to [e,Y]$ are isomorphisms, 
        and they are natural in $Y$.
    \end{rmnumerate}
    \begin{proof}
        The inverse of $i_{Y}$ is given by the composite
        \begin{equation*}
            \begin{tikzcd}
                {[e,Y]} \ar[r, "r_{[e,Y]}^{-1}"]
                & {[e,Y] \otimes e} \ar[r, "\cat{ev}_{eY}"]
                & Y\,.
            \end{tikzcd}
        \end{equation*}
        We omit the routine computations.
    \end{proof}
\end{prop}
%%    
\section{Enriched categories}

When a symmetric monoidal category $(\Vv, \otimes)$ is fixed, 
we show how to enrich over it the notions of category, 
functor and natural transformation.

\begin{df}
    A \textbf{$\Vv$-enriched category} $\Cc$
    (or simply \textbf{$\Vv$-category})
    is given by the following data:
    \begin{itemize}
        \item a class $\Cc$ of \textbf{objects};

        \item for any pair $A,B \in \Cc$ of objects,
        there exists an \textbf{arrows} object $\Cc(A,B) \in \Vv$;

        \item for any triple $A,B,C \in \Cc$,
        there exists a \textbf{composition morphism} in $\Vv$:
            \begin{equation*}
                \cat{c}_{ABC} : \Cc(A,B) \otimes \Cc(B,C) \longrightarrow \Cc(A,C)\,;
            \end{equation*}

        \item any object $A \in \Cc$ has a \textbf{unit morphism}
        in $\Vv$, that is
            \begin{equation*}
                \cat{u}_{A} : e \longrightarrow \Cc(A,A)\,,
            \end{equation*}
        where $e$ is the unit object of $\Vv$.
    \end{itemize}
    Moreover, the previous data satisfy the following
    ``coherence'' axioms:
    \begin{itemize}
        \item[(\textbf{E1})]\label{E1} \textbf{associativity}:
        the composition is associative, i.e. 
        for any five objects $A,B,C,D,E$ in $\Cc$,
        the following diagram commutes:
        \begin{center}
            \begin{tikzcd}[column sep=tiny, row sep=large]
             & \left( \Cc(A,B) \otimes \Cc(B,C) \right) \otimes \Cc(C,D) 
             \ar[rd, "\cat{c}_{ABC} \otimes \cat{u}_{\Cc(C,D)}"] 
             \ar[dl, "\alpha"'] &  \\
            \Cc(A,B) \otimes \left( \Cc(B,C) \otimes \Cc(C,D) \right)\ar[d, "\cat{u}_{\Cc(A,B)} \otimes \cat{c}_{BCD}"']
            && \Cc(A,C) \otimes \Cc(C,D) \ar[d, "\cat{c}_{ACD}"] \\
              \Cc(A,B) \otimes \Cc(B,D) \ar[rr, "\cat{c}_{ABD}"]
            & & \Cc(A,D)  
            \end{tikzcd}
        \end{center}

        \item[(\textbf{E2})]\label{E2} \textbf{unital}:
        the unit morphism is the neutral element for the composition,
        i.e. for any pair $A,B \in \Cc$ we have a commutative diagram
            \begin{equation*}
                \begin{tikzcd}
                    e \otimes \Cc(A,B) 
                    \ar[r, "l_{\Cc(A,B)}"] \ar[d, "\cat{u}_{A} \otimes \cat{1}_{\Cc(A,B)}"]
                    & \Cc(A,B) \ar[d, equals] 
                    & \Cc(A,B) \otimes e 
                    \ar[l, "r_{\Cc(A,B)}"'] \ar[d, "\cat{1}_{\Cc(A,B)} \otimes \cat{u}_{B}"] \\
                    \Cc(A,A) \otimes \Cc(A,B) \ar[r, "\cat{c}_{AAB}"']
                    & \Cc(A,B)
                    & \Cc(A,B) \otimes \Cc(B,B) \,. \ar[l, "\cat{c}_{ABB}"]
                \end{tikzcd}
            \end{equation*}
    \end{itemize}
\end{df}

\begin{ex}
    A preadditive category is $\Ab$-enriched.
\end{ex}

\begin{ex!}\label{one-obj-ab}
    A one object $\Ab$-category $R$ is a unitary ring: indeed, 
    it consists of an object $\ast$ and an abelian group
    \begin{equation*}
        R := R(\ast,\ast) \in \Ab\,,
    \end{equation*}
    with a multiplication
    \begin{equation*}
        \cdot := \cat{c}_{\ast \ast \ast} : R \otimes_{\Z} R \longrightarrow R
    \end{equation*}
    given by the composition in $R$,
    which is associative by \hyperref[E1]{(\textbf{E1})}.
    Moreover, the unit morphism
    \begin{equation*}
        \cat{u} : \Z \longrightarrow R\,,
        \quad 1 \longmapsto u
    \end{equation*}
    detects the neutral element for the multiplication,
    for
    \begin{equation*}
        u \cdot x = \cat{c}(\cat{u}(1) \otimes x) 
        = x 
        = \cat{c}(x \otimes \cat{u}(1)) = x \cdot u\,,
    \end{equation*}
    by \hyperref[E2]{(\textbf{E2})}.
\end{ex!}

\begin{ex}
    Any locally small category $\Aa$ is enriched over $\Vv=\cat{Set}$:
    indeed, morphisms between two objects form sets and, for each $A \in \Aa$,
    the unit morphism is simply
    \begin{equation*}
        \cat{u}_{A}: \Set{\ast} \longrightarrow \Hom_{\Aa}(A,A)\,,
        \quad \ast \longmapsto \cat{1}_{A}\,.
    \end{equation*}
\end{ex}

\begin{ex!}\label{smcc-enriched}
    A symmetric monoidal closed category $\Vv$ is enriched over itself:
    for every pair $A,B \in \Vv$, one sets
    \begin{equation*}
        \Vv(A,B) := [A,B] \in \Vv\,,
    \end{equation*}
    and takes units and compositions as defined in 
    \hyperref[v-morphisms]{Remark~\ref*{v-morphisms}}.
    One can check that both (\textbf{E1}) and (\textbf{E2}) hold.
\end{ex!}

\begin{ex}[Dual category]
    If the monoidal category $\Vv$ is symmetric,
    a $\Vv$-category $\Cc$ gives rise to a 
    \textbf{dual} $\Vv$-category $\Cc^{*}$,
    whose objects are the same as $\Cc$,
    and for each pair $A,B \in \Cc^{*}$
    one sets
    \begin{equation*}
        \Cc^*(A,B) := \Cc(B,A)\,.
    \end{equation*}
    Compositions are then defined by
    \begin{equation*}
        \cat{c}^*_{ABC} : \Cc^*(A,B) \otimes \Cc^*(B,C)
        \xrightarrow[]{\tau} \Cc(C,B) \otimes \Cc(B,A)
        \xrightarrow[]{\cat{c}_{CBA}} \Cc(C,A) = \Cc^*(A,C)\,,
    \end{equation*}
    thus the unit is $\cat{u}_{A}^* = \cat{u}_{A}$, for every $A \in \Cc^*$;
    one checks that $\Cc^*$ is indeed a $\Vv$-category.
    
\end{ex}

Now we adapt the notion of a functor between enriched categories.

\begin{df}
    Let $(\Vv, \otimes)$ be a monoidal category 
    and $\Aa,\Bb$ two $\Vv$-categories.
    A \textbf{$\Vv$-functor} $F:\Aa \to \Bb$ 
    is defined by the following data:
    \begin{itemize}
        \item an object $F(A) \in \Bb$, for each $A \in \Aa$;
        \item for every pair of objects $A,A' \in \Aa$,
        there exists a morphism
        \begin{equation*}
            F_{AA'} : \Aa(A,A') \longrightarrow \Bb(F(A),F(A'))\,
        \end{equation*}
        in $\Vv$ such that the following axioms are satisfied:
        \begin{rmnumerate}
            \item \textbf{composition:} 
            for every three objects $A,A',A'' \in \Aa$,
            the following diagram commutes
            \begin{equation*}
                \begin{tikzcd}[column sep=huge]
                    \Aa(A,A') \otimes \Aa(A',A'') 
                    \ar[r, "\cat{c}_{AA'A''}"] \ar[d, "F_{AA'} \otimes F_{A'A''}"']
                    & \Aa(A,A'') \ar[d, "F_{AA''}"] \\
                    \Bb(F(A),F(A')) \otimes \Bb(F(A'),F(A''))
                    \ar[r, "\cat{c}_{F(A)F(A')F(A'')}"]
                    & \Bb(F(A),F(A''))\,;
                \end{tikzcd}
            \end{equation*}

            \item \textbf{unit to unit:}
            for each object $A \in \Aa$, 
            there is a commutative diagram
            \begin{equation*}
                \begin{tikzcd}
                     e \ar[r, "\cat{u}_{A}"] \ar[dr, "\cat{u}_{F(A)}"']
                     & \Aa(A,A) \ar[d, "F_{AA}"] \\
                     & \Bb(F(A),F(A))\,.
                \end{tikzcd}
            \end{equation*}
        \end{rmnumerate}
    \end{itemize}
\end{df}

\begin{ex!}\label{left-R-mod}
    Consider a one object $\Ab$-category $R$. 
    A covariant $\Ab$-functor
    \begin{equation*}
        M : R \longrightarrow \Ab\,,
        \quad \left( \ast \xrightarrow[]{r} \ast \right)
        \longmapsto \left( M(\ast) \xrightarrow[]{r \times -} M(\ast) \right)
    \end{equation*}
    describes a left $R$-module: indeed,
    we have seen in \hyperref[one-obj-ab]{Example~\ref*{one-obj-ab}}
    that $R$ is a unitary ring, 
    so $M(\ast)$ is an abelian group
    endowed with the external multiplication
    \begin{equation*}
        r \times m := M(r)(m)\,, \quad \text{for every } m \in M(\ast)\,.
    \end{equation*}
\end{ex!}

Finally, we get to the notion of natural transformation,
which is a bit more involved to state.

\begin{df}
    Let $F,G : \Aa \to \Bb$ be two $\Vv$-functors between two $\Vv$-categories.
    A \textbf{$\Vv$-natural transformation} $\phi:F \implies G$
    consists in giving, for every object $A \in \Aa$,
    a morphism 
    \begin{equation*}
        \phi_{A} : e \longrightarrow \Bb(F(A),G(A))
    \end{equation*}
    in $\Vv$ such that the following diagram commutes,
    for each pair $A,A''$:
    \begin{equation*}
        \begin{tikzcd}[column sep=small]
            & \Aa(A,A') 
            \ar[dr, "r_{A}^{-1}"] \ar[dl, "l_{A}^{-1}"'] & \\
            e \otimes \Aa(A,A') \ar[d, "\phi_{A} \otimes G_{AA'}"]
            && \Aa(A,A') \otimes e \ar[d, "F_{AA} \otimes \phi_{A'}"] \\
            \Bb(F(A),G(A)) \otimes \Bb(G(A),G(A'))
            \ar[dr, "\cat{c}_{F(A)G(A)G(A')}"']
            && \Bb(F(A),F(A')) \otimes \Bb(F(A'),G(A'))
            \ar[dl, "\cat{c}_{F(A)F(A')G(A')}"] \\
            & \Bb(F(A),G(A')) & \,.
        \end{tikzcd}
    \end{equation*}
\end{df}

\begin{ex}
    Consider a one object $\Ab$-category $R$, 
    as in \hyperref[one-obj-ab]{Example~\ref*{one-obj-ab}},
    and two $\Ab$-functors $M,N : R \to \Ab$,
    that is two left $R$-modules by 
    \hyperref[left-R-mod]{Example~\ref*{left-R-mod}}.
    Then an $\Ab$-natural transformation $f : M \implies N$
    is a group homomorphism $f : M(\ast) \to M(\ast)$
    such that, for every $r \in R$, the following square
    \begin{equation*}
        \begin{tikzcd}
            M(\ast) \ar[r, "f"] \ar[d, "r \times -"']
            & N(\ast) \ar[d, "r\times -"] \\
            M(\ast) \ar[r, "f"] & N(\ast)
        \end{tikzcd}
    \end{equation*}
    commutes, that is
    \begin{equation*}
        r \times f(m) = f(r \times m)\,, \quad m \in M(\ast).
    \end{equation*}
    Thus, $f$ is nothing but a left $R$-module homomorphism!
    In fact, one can notice that the category 
    of $\Ab$-functors from $R$ to $\Ab$,
    with $\Ab$-natural transformations, is
    \begin{equation*}
        \cat{Fun}_{\Ab}(R,\Ab) \simeq \Mod_{R}\,.
    \end{equation*}
\end{ex}

In the section on \hyperref[AdditiveCategories]{Additive categories}
in the previous chapter,
we came across to the 
\hyperref[additive-yoneda]{Additive Yoneda's Lemma},
which is an adaptation of the classical
\hyperref[yoneda]{Yoneda's Lemma} to the setting
of (pre)additive categories.
In fact, it is nothing but a special case 
of an \textbf{enriched} version of this result;
by skipping some more technical details,
we are now ready to state it 
in its full generality:

\begin{thm}[\textbf{Enriched Yoneda's Lemma}]\label{enriched-yoneda}
    Let $\Vv$ be a symmetric monoidal closed category
    and $\Aa$ a small\footnote{A $\Vv$-category $\Aa$ is \textbf{small} if objects form a set $|\Aa|$.}
    $\Vv$-category.
    For every object $A \in \Aa$ and 
    every $\Vv$-functor $F:\Aa^* \to \Vv$,
    the $\Vv$-natural transformations 
    from $\Aa(-,A)$ to $F$ form an object
    \begin{equation*}
        \Vv-\cat{Nat}(\Aa(-,A),F) \in \Vv \,;
        \footnote{In fact, one can prove that $\Vv$-functors between $\Vv$-categories
        $\Cc$ and $\Dd$, endowed with $\Vv$-natural transformations,
        give rise to a $\Vv$-category $\cat{Fun}_{\Vv}(\Cc,\Dd)$.}
    \end{equation*}
    moreover, there exists an isomorphism
        \begin{equation*}
            \Vv-\cat{Nat}(\Aa(-,A),F) \xrightarrow[]{\sim} F(A)
        \end{equation*}
    in $\Vv$, which is $\Vv$-natural in both $A$ and $F$.
\end{thm}
%%    
\section{Tensors}

\begin{df}
    Let $\Vv$ be a symmetric monoidal \emph{closed} category,
    $\Aa$ a $\Vv$-category and consider two objects
    $A \in \Aa$ and $V \in \Vv$.
    The \textbf{tensor} of $V$ and $A$ (\emph{if it exists})
    is an object $V \otimes_{\Aa} A \in \Aa$, 
    together with isomorphisms
    \begin{equation*}
        \Aa(V \otimes_{\Aa} A,B) \simeq [V,\Aa(A,B)]
    \end{equation*}
    in $\Vv$, for every $B \in \Aa$, which are $\Vv$-natural in $B$.
    We say that $\Aa$ is \textbf{tensored} when 
    the tensor $V \otimes_{\Aa} A$ exists, 
    for every $V \in \Vv$ and $A \in \Aa$.
\end{df}

\begin{ex}
    We have seen that $\Vv = \cat{Set}$ is a 
    symmetric monoidal closed category.
    Any locally small category $\Aa$ with products
    is a tensored $\cat{Set}$-category: once a set $X$ is fixed,
    for any pair $A,B \in \Aa$ one sees that
    \begin{equation*}
        \left( \Hom_{\Aa}(A,B) \right)^{X} 
        \simeq \prod_{\ast \in X} \Hom_{\Aa}(A,B)^{\Set{\ast}}
        \simeq \prod_{\ast \in X} \Hom_{\Aa}(A,B)
        \simeq \Hom_{\Aa}\left(\prod_{\ast \in X}A,B\right)\,,
    \end{equation*}
    where $\prod_{\ast \in X}A$ is the product of many copies of $A$,
    indexed over $X$: in other words, it is the limit over 
    the diagram given by the
    discrete category consisting of points of $X$.
    Thus, $\Aa$ is tensored with
    \begin{equation*}
        X \otimes_{\Aa} A := \prod_{\ast \in X}A\,.
    \end{equation*}
\end{ex}

\begin{ex}
    As we have seen in \hyperref[smcc-enriched]{Example~\ref*{smcc-enriched}},
    every symmetric monoidal closed category $(\Vv,\otimes)$ is a tensored category:
    indeed, for every $V,A \in \Vv$ one has
    \begin{equation*}
        V \otimes_{\Vv} A = V \otimes A\,.
    \end{equation*}
    To check this,
    for any three $A,B,V,X \in \Vv$
    one has natural isomorphisms
    \begin{align*}
        \Hom_{\Vv}(X,[V \otimes A,B])
        &\simeq \Hom_{\Vv}(X \otimes V \otimes A,B) \\
        &\simeq \Hom_{\Vv}(X \otimes V, [A,B]) %\\
        \simeq \Hom_{\Vv}(X, [V,[A,B]])\,,
    \end{align*}
    thus we conclude that
    \begin{equation*}
        [V \otimes A,B] \simeq [V,[A,B]]
    \end{equation*}
    by the \hyperref[yoneda]{Yoneda's Lemma}.
\end{ex}

\begin{rmk}
    Let $\Aa$ be a tensored $\Vv$-category.
    For any $X,Y \in \Vv$ and $A,B \in \Aa$ it holds
    \begin{align*}
        \Aa\big( Y \otimes_{\Aa} ( X \otimes_{\Aa} A) , B \big)
        &\simeq [Y,\Aa(X \otimes_{\Aa} A, B)] \\
        &\simeq [Y,[X,\Aa(A,B)]] \\
        &\simeq [Y \otimes X, \Aa(A,B)] \\
        &\simeq \Aa\big((Y \otimes X) \otimes_{\Aa} A, B\big)\,,
    \end{align*}
    hence, to resume we have
    \begin{equation*}
        \Aa\big( Y \otimes_{\Aa} ( X \otimes_{\Aa} A) , B \big)
        \simeq \Aa\big((Y \otimes X) \otimes_{\Aa} A, B\big)\,.
    \end{equation*}
\end{rmk}

\begin{prop}
    Let $\Vv$ be a symmetric monoidal closed category and fix $V \in \Vv$. 
    If $\Aa$ is a tensored $\Vv$-category, the correspondence
    \begin{equation*}
        V \otimes_{\Aa} - : \Aa \longrightarrow \Aa\,,
        \quad A \longmapsto V \otimes_{\Aa} A
    \end{equation*}
    induces a $\Vv$-functor.
    \begin{proof}
        For $A,B \in \Aa$, by the definition of tensor in $\Aa$
        we get an ``\textbf{evaluation morphism}'' in $\Vv$ as
        \begin{equation*}
            \begin{tikzcd}
                e \ar[r, "\cat{u}_{\Aa(A,B)}"]
                & {[\Aa(A,B), \Aa(A,B)]} \ar[r, "\simeq"]
                & \Aa(\Aa(A,B) \otimes_{\Vv} A,B)\,.
            \end{tikzcd}
        \end{equation*}
        Then, consider the composite
        \begin{equation*}
            \begin{tikzcd}
                e \ar[d, "\cat{u}_{V \otimes_{\Aa} B}"] \\%, "\cat{u}_{V \otimes_{\Aa} B}"] \\
                \Aa(V \otimes_{\Aa} B, V \otimes_{\Aa} B) 
                \ar[d, "f"] \\
                \Aa\big(V \otimes_{\Aa} (\Aa(A,B) \otimes_{\Aa} A), V \otimes_{\Aa} B \big) 
                \ar[d, "\simeq"] \\
                \Aa\big(\Aa(A,B) \otimes_{\Aa} ( V \otimes_{\Aa} A), V \otimes_{\Aa} B \big) 
                \ar[d, "\simeq"] \\
                {[\Aa(A,B), \Aa(V \otimes_{\Aa} A, V \otimes_{\Aa} B)]}\,,
            \end{tikzcd}
        \end{equation*}
        where $f$ is the image of $\cat{1}_{V} \otimes_{\Aa} \cat{ev}_{AB}$
        under the functor $\Aa(-,V\otimes_{A}B)$. 
        Thus, by using the adjunction
        \begin{align*}
            \Hom_{\Vv}\big(e,{[\Aa(A,B), \Aa(V \otimes_{\Aa} A, V \otimes_{\Aa} B)]}\big)
            &\simeq \Hom_{\Vv}\big(e \otimes \Aa(A,B), \Aa(V \otimes_{\Aa} A, V \otimes_{\Aa} B)\big)\\
            &\simeq \Hom_{\Vv}\big(\Aa(A,B), \Aa(V \otimes_{\Aa} A, V \otimes_{\Aa} B)\big)\,,
        \end{align*}
        we obtain a map
        \begin{equation*}
            \Aa(A,B) \longrightarrow \Aa(V \otimes_{\Aa} A, V \otimes_{\Aa} B)
        \end{equation*}
        in $\Vv$ which is $\Vv$-functorial by construction.
    \end{proof}
\end{prop}
%%    
\section{\texorpdfstring{$k$}{k}-linear categories}
% FOR MATH SYMBOLS IN TITLES 
% https://latex.org/forum/viewtopic.php?t=21256


As the categories we will eventally be interested in
have a geometric origin, i.e. are defined in terms
of certain variety over some base field $k$,
we usually deal with the following special kind of categories.

\begin{df}
    Let $k$ be an arbitrary field. 
    A \textbf{$k$-linear category} is an additive category
    $\Aa$ such that the groups $\Hom_{\Aa}(A,A')$
    are $k$-linear vector spaces, for any $A,A' \in \Aa$.
    Moreover, we require the compositions to be $k$-bilinear.
\end{df}

In short, a $k$-linear category $\Aa$ is a $\cat{Vect}_{k}$-enriched category,
where we require $\Aa(A,A') = \Hom_{\Aa}(A,A')$.

\begin{ex}
    Since $\Vv = \Vect{k}$ endowed with $\otimes_{k}$
    is a symmetric monoidal closed category,
    then it is enriched over itself,
    i.e. it is $k$-linear.
\end{ex}

\begin{df}
    Given $\Aa,\Bb$ two $k$-linear categories,
    an additive functor $F : \Aa \to \Bb$ 
    is called \textbf{$k$-linear}
    if the map $F_{A,A'} : \Hom_{\Aa}(A,A') \to \Hom_{\Bb}(F(A),F(A'))$ 
    is a $k$-linear map for any two
    $A,A' \in \Aa$. As in the additive case, 
    $k$-linear functors from $\Aa$ to $\Bb$ 
    and natural transformations
    form a category $\cat{Fun}_{k}(\Aa, \Bb)$.
\end{df}


As special case of the 
\hyperref[enriched-yoneda]{\textbf{Theorem~\ref*{enriched-yoneda}}},
we will adapt the \hyperref[yoneda]{\textbf{Yoneda's Lemma}~\ref*{yoneda}} 
to the $k$-linear setting:
first, we recall that a \textbf{$k$-linear} equivalence
$F : \Aa \to \Bb$ between two $k$-linear categories
is an equivalence which is a $k$-linear functor,
whose quasi-inverse $F^{-1}$ is again $k$-linear.

%\begin{thm}[Additive Yoneda]
%    For an additive category $\Aa$, the Yoneda embedding
%    \begin{equation*}
%        \Aa \longrightarrow \cat{Fun}_{+}(\Aa^{op},\Ab)\,, \quad
%        A \longmapsto \Hom_{\Aa}(-,A)
%    \end{equation*}
%    defines an equivalence between $\Aa$ and
%    the category of contravariant \emph{additive} functors
%    between $\Aa$ and the category of abelian groups.
%\end{thm}

%Similarly, it holds:

\begin{thm}[\textbf{Linear Yoneda's Lemma}]\label{linear-yoneda}
    Let $k$ be a field. For a $k$-linear category $\Aa$, 
    the Yoneda embedding
    \begin{equation*}
        \Aa \longrightarrow \cat{Fun}_{k}(\Aa^{*},\cat{Vect}_{k})\,, \quad
        A \longmapsto \Hom_{\Aa}(-,A)
    \end{equation*}
    defines an equivalence between $\Aa$ and
    the category of contravariant \emph{$k$-linear} functors
    between $\Aa$ and the category of vector spaces over $k$.
\end{thm}

\begin{lemma}\label{tensor-w-zero}
    Any $k$-linear category $\Dd$ is tensored over
    finite-dimensional vector spaces, i.e.
    given $n \in \NN$ and $A \in \Dd$,
    then
    \begin{equation*}
        k^{n} \otimes_{\Dd} A = A^{\oplus n}\,.
    \end{equation*}
    In particular, for every $A \in \Dd$, one has
    $0 \otimes_{\Dd} A \simeq \cat{0}$.
    \begin{proof}
        Given any object $B \in \Dd$, one notices that
        \begin{align*}
            \Hom_{\Dd}(k^{n} \otimes_{\Dd} A, B)
            \simeq \Hom_{k}(k^{n}, \Hom_{\Dd}(A,B)) 
            \simeq \Hom_{\Dd}(A,B)^{\oplus n}
            \simeq \Hom_{\Dd}\left( A^{\oplus n}, B \right)\,,
        \end{align*}
        thus, $k^{n} \otimes_{\Dd} A = A^{\oplus n}$ 
        by the covariant version of the \hyperref[linear-yoneda]{Linear Yoneda's Lemma}.
    \end{proof}
\end{lemma}

\begin{df}
    Let $\Aa$ be a $k$-linear category.
    A \textbf{Serre functor} is a $k$-linear equivalence
    $S : \Aa \to \Aa$ such that, for any two objects
    $A,A' \in \Aa$, there exists an isomorphism
    of $k$-vector spaces
    \begin{equation*}
        \sigma_{A,A'} : \Hom_{\Aa}(A,A') 
        \xrightarrow[]{\sim} \Hom_{\Aa}(A',S(A))^*\,,
    \end{equation*}
    which is functorial both in $A$ and $A'$.
\end{df}

A Serre functor induces a pairing
\begin{equation*}
    \Hom_{\Aa}(A',S(A)) \times \Hom_{\Aa}(A,A') \longrightarrow k\,,
    \quad (f,g) \longmapsto \langle f | g \rangle 
    := \sigma_{A,A'}(g)(f)\,.
\end{equation*}

In order to avoid any trouble with the dual, 
one usually assumes that all Hom’s in A are \emph{finite-dimensional}. 
Under this hypothesis it is easy to see that a Serre functor, 
if it exists, is unique up to isomorphism. 
More generally one has the following

\begin{lemma}\label{serre-eq-comm}
    Let $\Aa$ and $\Bb$ be $k$-linear categories with
    finite dimensional $\Hom$'s. Assume $\Aa$, resp. $\Bb$,
    is endowed with a Serre functor $S_{\Aa}$, resp. $S_{\Bb}$.
    Then any $k$-linear equivalence $F:\Aa \to \Bb$ commutes
    with Serre duality, i.e. there exists an isomorphism
    \begin{equation*}
        F \circ S_{\Aa} \simeq S_{\Bb} \circ F\,.
    \end{equation*}
    \begin{proof}
        We show the two functors are isomorphic
        by applying the \hyperref[yoneda]{Yoneda Lemma~\ref*{yoneda}}:
        given any two $A, A' \in \Aa$, 
        we compute the Serre duality in $\Aa$
        and we use that $F$
        is fully faithful to get
        \begin{align*}
            \Hom_{\Aa}(A,A') \simeq \Hom_{\Aa}(A',S_{\Aa}(A))^* 
            \simeq \Hom_{\Bb}\big(F(A'),F(S_{\Aa}(A))\big)^*\,.
        \end{align*}
        On the other hand, if we first apply $F$ and then Serre duality
        we obtain
        \begin{align*}
            \Hom_{\Aa}(A,A') 
            \simeq \Hom_{\Bb}(F(A),F(A'))
            \simeq \Hom_{\Bb}\big(F(A'),S_{\Bb}(F(A))\big)^*\,.
        \end{align*}
        Since $F$ is essentially surjective, 
        for every $B \in \Bb$ we have isomorphisms
        \begin{align*}
            \Hom_{\Bb}\big(F(A'),S_{\Bb}(F(A))\big)^*
            \simeq \Hom_{\Bb}\big(F(A'),S_{\Bb}(F(A))\big)^*\,,
        \end{align*}
        thus, by passing to the double dual we conclude.
    \end{proof}
\end{lemma}

\begin{rmk}
    Let $\Aa$ and $\Bb$ be $k$-linear categories 
    as in the hypothesis of \hyperref[serre-eq-comm]{Lemma~\ref*{serre-eq-comm}}.
    If $F: \Aa \to \Bb$ is a functor such that $G \dashv F$,
    then Serre duality gives us a way to build a right adjoint for $F$:
    indeed, we have $F \dashv S_{\Aa} \circ  G \circ S_{\Bb}^{-1}$.
    To see this,  we have natural isomorphisms
    \begin{align*}
        \Hom_{\Aa}(A, S_{\Aa} \circ  G \circ S_{\Bb}^{-1}(B)) 
        &\simeq \Hom_{\Bb}(G \circ S_{\Bb}^{-1}(B),A)^{*} \\
        &\simeq \Hom_{\Bb}(S_{\Bb}^{-1}(B),F(A))^{*} \\
        &\simeq \Hom_{\Bb} \big(F(A), S_{\Bb} \circ S_{\Bb}^{-1}(B)\big)^{**} \\
        &\simeq \Hom_{\Bb}(F(A),B)\,.
    \end{align*}
\end{rmk}

 A similar argument allows the construction of 
 a left adjoint if a right adjoint 
 $F \dashv H$ is given. 
 In particular, for functors between categories with 
 Serre functors the existence of the left or 
 the right adjoint implies the existence of the other one.

%
%
%%%%%%%%%%%%%%%%%%%%%%%%%%%%%%%%%%%%%%%%%%%%%%%%%%%%%%%%%%%%
%%%%%%%%%%%%%%%%%%%%%%%%%%%%%%%%%%%%%%%%%%%%%%%%%%%%%%%%%%%%
%%%%%%%%%%%%%%%%%%%%%%%%%%%%%%%%%%%%%%%%%%%%%%%%%%%%%%%%%%%%
%
%
%
%\chapter{Triangulated categories}
%
%While studying algebraic topology and geometry,
%abelian categories arise as a natural
%setting to develop homological algebra.
%In many cases, we come across long exact sequences 
%in homology, e.g. the Mayer-Vietoris sequence for
%a topological space, which are a powerful tool
%for the study of our object of interest,
%starting from some sort of decomposition.
%Triangulated categories are a special kind of
%abelian categories, endowed with a \emph{shift functor}
%which allows us to build long exact sequences of objects.
%Thanks to this, we will develop
%new techniques to study (co)homology objects.
%
%%    
\section{Chain homotopy}

The ideas in this section and the next are motivated 
by homotopy theory in topology. 

Let $\Aa$ be an abelian category and consider 
two cochain complexes $C^{\bullet}, D^{\bullet}$ in $\Aa$. 
Consider a degree $-1$ map $s : C^{\bullet} \to D^{\bullet}$,
that is a collection of morphisms
\begin{equation*}
    s^{n} : C^{n} \longrightarrow D^{n-1}\,,
    \quad n \in \Z\,.
\end{equation*}
For every $n \in \Z$, set $f^{n} := d_{D}^{n-1}s^{n} + s^{n+1}d_{C}^{n}$
to get a morphism $f^{n}:C^{n} \to D^{n}$:
\begin{center}
    \begin{tikzcd}
        C^{n-1} \ar[r]
        & C^{n} \ar[r, "d"] \ar[d, "f^{n}"] \ar[dl, "s"']
        & C^{n+1} \ar[dl,"s"] \\
        D^{n-1} \ar[r, "d"]
        & D^{n} \ar[r]
        & D^{n+1} \,.
    \end{tikzcd}
\end{center}

Then the collection of $f^{n}$ defines
a cochain map $f^{\bullet} : C^{\bullet} \to D^{\bullet}$;
indeed, dropping the superscripts for clarity, we compute
\begin{align*}
    d f = d(ds + sd) = dsd = (ds + sd) d = fd\,.
\end{align*}

\begin{df}
    A chain map $f^{\bullet}:C^{\bullet} \to D^{\bullet}$
    is \textbf{null homotopic} if it is of the form
    $f^{\bullet} = ds + sd$, for some map $s$ of degree $-1$.
    We will call $s$ a \textbf{cochain contraction} of $f^{\bullet}$.
\end{df}

\begin{lemma}
    If $f^{\bullet}:C^{\bullet} \to D^{\bullet}$
    is \textbf{null homotopic}, then every induced map
    in cohomology 
    $f^{*}:H^{n}(C^{\bullet}) \to H^{n}(D^{\bullet})$
    is zero.
    \begin{proof}
        If we focus on the $n$-th level, 
        we have a commutative diagram
        \begin{equation*}
            \begin{tikzcd}
                & \ker d^{n}_{C} \ar[d, hook] & \\
                C^{n-1} \ar[r, "d^{n-1}_{C}"] \ar[d]
                & C^{n} \ar[r, "d^{n}_{C}"] \ar[d, "f^{n}"'] \ar[dl, "s"']
                & C^{n+1} \ar[d] \ar[dl, "s"'] \\
                D^{n-1} \ar[r, "d^{n-1}_{D}"'] \ar[d, two heads]
                & D^{n} \ar[r, "d^{n}_{D}"']
                & D^{n+1} \\
                \im d^{n-1}_{D} \ar[ru, hook] & &
            \end{tikzcd}
        \end{equation*}
        which shows that $\ker d^{n}_{C} \hookrightarrow \im d^{n-1}_{D}$;
        thus, by passing to the cokernel, 
        the map $\ker d^{n}_{C} \to H^{n}(D^{\bullet})$ is zero.
        Since this holds for every $n \in \Z$, we conclude that $f^{*} = 0$.
    \end{proof}
\end{lemma}

Notice that this contraction construction gives
us a way to ``proliferate'' cochain maps:
indeed, given any morphism $g^{\bullet}: C^{\bullet} \to D^{\bullet}$,
then $g^{\bullet} + (ds+sd)$ is again a cochain map.
By the previous \textbf{Lemma}, 
it turns out that $g^{\bullet} + (ds+sd)$ is in fact
not very different from $g^{\bullet}$,
because they induce the same maps in cohomology!

\begin{df}
    Two maps $f^{\bullet}, g^{\bullet}: C^{\bullet} \to D^{\bullet}$
    are \textbf{cochain homotopic}, 
    and write $f^{\bullet} \simeq g^{\bullet}$,
    if their difference is null homotopic, that is
    \begin{equation*}
        f^{\bullet} - g^{\bullet} = ds + sd\,,
    \end{equation*}
    for some degree $-1$ map $s$. In this case,
    we call $s$ a (\textbf{cochain}) \textbf{homotopy}
    from $f^{\bullet}$ to $g^{\bullet}$.

    Finally, $f^{\bullet}: C^{\bullet} \to D^{\bullet}$
    is a (\textbf{cochain}) \textbf{homotopy equivalence}
    if there exists $g^{\bullet} : D^{\bullet} \to C^{\bullet}$
    such that
    \begin{equation*}
        g^{\bullet}f^{\bullet} \simeq \cat{1}_{C^{\bullet}}\,,
        \quad f^{\bullet}g^{\bullet} \simeq \cat{1}_{D^{\bullet}}\,.
    \end{equation*}
\end{df}

\begin{cor}
    If $f^{\bullet}, g^{\bullet}: C^{\bullet} \to D^{\bullet}$
    are \textbf{cochain homotopic}, then $f^* = g^*$.
    In particular, if $f^{\bullet}$ is a homotopy equivalence,
    then the two complexes have the same cohomology: 
    $H^{n}(C^{\bullet}) = H^{n}(D^{\bullet})$.
\end{cor}

\begin{df}
    A morphism of complexes
    $f^{\bullet} : A^{\bullet} \to B^{\bullet}$
    is a \textbf{quasi-isomorphism} (shortened \textbf{qis})
    if, for all $n \in \NN$, the induced map
    \begin{equation*}
        H^{n}(f^{\bullet}) : H^{n}(A^{\bullet}) \xrightarrow[]{\sim} H^{n}(B^{\bullet})
    \end{equation*}
    is an isomorphism.
\end{df}

Thus, the previous corollary may be stated as 
``\emph{homotopy equivalences are quasi-isomorphisms}''.

%%    
\section{Mapping cones and cylinders}\todo{Check sign conventions and computations of the differentials.}

Let $\Aa$ be an abelian category, 
and consider the category $C^{\bullet}(\Aa)$
of cochain complexes in $\Aa$.

\begin{df}
    Let $f^{\bullet} : A^{\bullet} \to B^{\bullet}$
    be a map of cochain complexes.
    The \textbf{mapping cone} of $f^{\bullet}$
    is the cochain complex $\cat{C}(f^{\bullet})$
    whose degree $n$ part is
    \begin{equation*}
        \cat{C}(f^{\bullet})^{n} := A^{n+1} \oplus B^{n}\,,
    \end{equation*}
    and the coboundary maps 
    $d_{f}^{n} : \cat{C}(f^{\bullet})^{n} \longrightarrow \cat{C}(f^{\bullet})^{n+1}$
    are given by the matrices
    \begin{equation}\label{cone-coboundary}
        % d^{n} : \cat{C}(f^{\bullet})^{n} \longrightarrow \cat{C}(f^{\bullet})^{n+1}\,,
        %\quad 
%        \begin{pmatrix} a \\ b\end{pmatrix}
%        \longmapsto
        d^{n}_{f} := \begin{pmatrix} -d^{n+1}_{A} & 0 \\ f^{n+1} & d^{n}_{B} \end{pmatrix}
%        \begin{pmatrix} a \\ b\end{pmatrix}
%        = \begin{pmatrix} -d^{n+1}_{A}(a) \\ d^{n}_{B}(b) + f^{n+1}(a) \end{pmatrix}
    \end{equation}
\end{df}

\begin{df}
    Let $f^{\bullet} : A^{\bullet} \to B^{\bullet}$
    be a map of cochain complexes.
    The \textbf{mapping cylinder} of $f^{\bullet}$
    is the cochain complex $\cat{cyl}(f^{\bullet})$
    whose degree $n$ part is
    \begin{equation*}
        \cat{cyl}(f^{\bullet})^{n}
        := A^{n} \oplus A^{n+1} \oplus B^{n}.
    \end{equation*}
    The coboundary map $d_{\cat{cyl}} : \cat{cyl}(f^{\bullet}) 
    \to \cat{cyl}(f^{\bullet})$ is given 
    on the $n$-th level by the matrix
    \begin{equation*}
        d_{\cat{cyl}}^n :=
        \begin{pmatrix}
            d_{A}^{n} & \cat{1}_{A^{n+1}} & 0 \\
            0 & -d_{A}^{n+1} & 0 \\
            0 & -f^{n+1} & d^{n}_{B}
        \end{pmatrix}\,.
    \end{equation*}
\end{df}

\begin{exercise!}\label{cyl-criterion}
    Show that two cochain maps 
    $f^{\bullet}, g^{\bullet} : A^{\bullet} \to B^{\bullet}$
    are cochain homotopic if and only if
    they extend to a map $(f,s,g) : \cat{cyl}(\cat{1}_{A^{\bullet}}) \to B^{\bullet}$.
    \begin{proof}[Solution]
        Given a homotopy $s$ from $f^{\bullet}$ to $g^{\bullet}$,
        we may define the map
        \begin{equation*}
            \Psi^{\bullet} : \cat{cyl}(\cat{1}_{A^{\bullet}}) \longrightarrow
            B^{\bullet}\,, \quad
            \Psi^{n}(a, a', a'') = f^{n}(a) + s^{n+1}(a') + g^{n}(a'')\,.
        \end{equation*}
        Then $\Psi^{\bullet}$ is a cochain map:
        indeed, if we drop the indices for simplicity,
        we compute
        \begin{align*}
            \Psi^{\bullet} d
            = \Psi^{\bullet}
            \begin{pmatrix}
            d & \cat{1} & 0 \\
            0 & -d & 0 \\
            0 & -\cat{1} & d
            \end{pmatrix}
            &= \Psi^{\bullet}(d + \cat{1}, -d, -\cat{1}+d) \\
            &= f^{\bullet}d + f^{\bullet} - sd - g^{\bullet} + g^{\bullet} d \\
            &= f^{\bullet}d + ds + g^{\bullet} d \\
            &= df^{\bullet} + ds + dg^{\bullet} 
            =d \Psi^{\bullet}\,.
        \end{align*}
        The equations $\Psi^{n}(a, 0, 0) = f^{n}(a)$ and
        $\Psi^{n}(0, 0, a'') = g^{n}(a'')$ show that $\Psi^{\bullet}$
        extends both $f^{\bullet}$ and $g^{\bullet}$.

        Conversely, suppose an extension $\Psi^{\bullet} : \cat{cyl}(\cat{1}_{A^{\bullet}}) \to B^{\bullet}$ is given and write
        $$j_{2} = (0, \cat{1},0) : A^{\bullet+1} \to \cat{cyl}(\cat{1}_{A^{\bullet}})$$
        for the inclusion. Then $s := \Psi j_{2}$ defines a homotopy
        between $f^{\bullet}$ and $g^{\bullet}$, indeed
        \begin{align*}
            ds = d\Psi^{\bullet}j_{2} 
            = \Psi^{\bullet} d j_{2}
            = \Psi^{\bullet} 
            \begin{pmatrix}
            d & \cat{1} & 0 \\
            0 & -d & 0 \\
            0 & -\cat{1} & d
            \end{pmatrix}
            \begin{pmatrix}
            0 \\ \cat{1} \\ 0 
            \end{pmatrix}
            = \Psi^{\bullet}(\cat{1}, -d, -\cat{1})
            = f^{\bullet} - sd - g^{\bullet}\,.
        \end{align*}
    \end{proof}
\end{exercise!}

\begin{lemma}
    The inclusion map 
    $$\iota := 0 \oplus 0 \oplus \cat{1}_{B^{\bullet}} : B^{\bullet}
    \to \cat{cyl}(f^{\bullet})$$
    is a quasi-isomorphism.
    \begin{proof}
        The above inclusion fits in the following
        short exact sequence:
        \begin{center}
            \begin{tikzcd}
                \cat{0} \ar[r]
                & B^{\bullet} \ar[r, "\iota"]
                & \cat{cyl}(f^{\bullet}) \ar[r, "\pi"]
                & \cat{C}(\cat{1}_{A^{\bullet}}) \ar[r]
                & \cat{0}\,,
            \end{tikzcd}
        \end{center}
        where $\pi$ is the projection on the first two components, switched:
        indeed, the inclusion $\iota$ is clearly a cochain map,
        and so is the projection because for every $n \in \Z$
        it holds
        \begin{align*}
            \pi \circ d_{\cat{cyl}}^n (x)
            &= 
            \begin{pmatrix}
                0 & 1 & 0 \\
                1 & 0 & 0
            \end{pmatrix}
            \begin{pmatrix}
            d_{A}^{n} & \cat{1}_{A^{n+1}} & 0 \\
            0 & -d^{n+1}_{A} & 0 \\
            0 & -f^{n+1} & d
            \end{pmatrix}
            x
            \begin{pmatrix}
                0 & 1 \\ 1 & 0 \\ 0 & 0
            \end{pmatrix} \\
            &= 
            \begin{pmatrix}
                -d_{A}^{n+1} & 0 \\ \cat{1}_{A^{n+1}} & d_{A}^{n}
            \end{pmatrix}
            \begin{pmatrix}
                0 & 1 & 0 \\
                1 & 0 & 0
            \end{pmatrix}
            x
            \begin{pmatrix}
                0 & 1 \\ 1 & 0 \\ 0 & 0
            \end{pmatrix}
            = d^{n}_{\cat{1}} \circ \pi(x)
        \end{align*}
        Thus, by the cohomological version of 
        \hyperref[LHS]{Theorem~\ref*{LHS}} we get 
        the long exact sequence
        \begin{center}
            \begin{tikzcd}[column sep=small]
                \dots \ar[r] 
                & H^{n-1}\big(\cat{C}(\cat{1}_{A^{\bullet}})\big) \ar[r]
                & H^{n}(A^{\bullet}) \ar[r]
                & H^{n}\big(\cat{cyl}(f^{\bullet})\big) \ar[r]
                & H^{n}\big(\cat{C}(\cat{1}_{A^{\bullet}})\big) \ar[r]
                & \dots
            \end{tikzcd}
        \end{center}
        If we show that the cone has trivial cohomology,
        then it follows that $H^{*}(B^{\bullet}) \simeq  
        H^{*}\big(\cat{cyl}(f^{\bullet})\big)$.
        By the definition of the coboundary maps~\eqref{cone-coboundary},
        % the cocycles $Z^{\bullet}$ and the coboundaries $B^{\bullet}$ 
        % for the cone complex are
        % \begin{equation*}
        %    Z^{n}\big(\cat{C}(-\cat{1}_{A^{\bullet}})\big)
        %    = B^{n+1}(A^{\bullet}) \oplus A^{n}\,,
        % \end{equation*}
        any $n$-cocycle is of the form $(-d_{A}a,a)$, for some $a \in A^{n}$,
        and in fact it is also a coboundary:
        \begin{equation*}
            \begin{pmatrix}
                -d^{n}_{A}(a) \\ a
            \end{pmatrix}
            =
            \begin{pmatrix}
                -d^{n}_{A} & 0 \\ \cat{1}_{A^n} & d^{n-1}_{A}
            \end{pmatrix}
            \begin{pmatrix}
                a \\ 0
            \end{pmatrix}
            = d^{n} \begin{pmatrix}
                a \\ 0
            \end{pmatrix}\,,
        \end{equation*}
        hence we conclude $H^{*}\big(\cat{C}(\cat{1}_{A^{\bullet}})\big) \simeq 0$.
    \end{proof}
\end{lemma}

\begin{exercise!}\label{cyl-equivalence}
    Show that the map $\beta^{\bullet} : \cat{cyl}(f^{\bullet}) \to B^{\bullet}$,
    defined at the $n$-th level by
    \begin{equation*}
        \beta^{n}(a,a',b) = f^{n}(a) + b\,,
    \end{equation*}
    defines a cochain map such that $\beta^{\bullet} \iota = \cat{1}_{B^{\bullet}}$.
    Then show that the formula
    \begin{equation*}
        s(a,a',b) = (0,a,0)
    \end{equation*}
    defines a cochain homotopy from the identity of $\cat{cyl}(f^{\bullet})$
    to $\iota \beta^{\bullet}$.
    Conclude that $\iota$ is in fact a cochain homotopy equivalence.
    \begin{proof}
        It is clear that $\beta^{\bullet}\iota = \cat{1}_{B^{\bullet}}$;
        we show it is a map of complexes:
        \begin{equation*}
            \beta^{\bullet} d_{\cat{cyl}} 
            = \beta^{\bullet} 
            \begin{pmatrix}
                d_{A} & \cat{1}_{A^{\bullet + 1}} & 0 \\
                0 & -d_{A} & 0 \\
                0 & -f^{\bullet} & d_{B}
            \end{pmatrix}
            =f^{\bullet}d_{A} + f^{\bullet} - f^{\bullet} + d_{B}
            = d_{B}(f^{\bullet} + \cat{1})
            = d_{B} \beta^{\bullet}\,.
        \end{equation*}
        To conclude $\iota$ is a homotopy equivalence,
        we show $\cat{1}_{\cat{cyl}} \simeq \iota\beta^{\bullet}$:
        \begin{align*}
            sd + ds
            &= (0, d_{A} + \cat{1}_{A^{\bullet+1}}, 0)
            + (\cat{1}_{A}, -d_{A}, -f^{\bullet})\\
            &= \left(\cat{1}_{A}, \cat{1}_{A^{\bullet+1}}, \cat{1}_{B} -(f^{\bullet}+\cat{1}_{B})\right)
            = \cat{1}_{\cat{cyl}} - \iota\beta^{\bullet}\,.
        \end{align*}
    \end{proof}
\end{exercise!}
%%    

\section{Bicomplexes}

\todo{Guarda Rotman per definizioni e esempi.}
%%    
\section{The homotopy category of complexes} %\texorpdfstring{$\cat{K}(\Aa)$}{K(A)}}

There are many formal similarities between
homological algebra and algebraic topology. 
The Dold-Kan correspondence, for example, 
provides a dictionary between 
positive complexes and simplicial theory. 
The algebraic notions of chain homotopy, mapping cones, 
and mapping cylinders have their historical 
origins in simplicial topology.

Let $\Aa$ be an abelian category and consider the
category $C^{\bullet}(\Aa)$ of cochain complexes in $\Aa$.
We now define the \textbf{homotopy category} $\cat{K}(\Aa)$ of $C^{\bullet}(\Aa)$
as follows: we take as objects the same of $C^{\bullet}(\Aa)$,
i.e. cochain complexes, 
and the morphisms of $\cat{K}(\Aa)$ to be
\emph{chain homotopy equivalence classes of maps} in $C^{\bullet}(\Aa)$:
for any two cochains $A^{\bullet}, B^{\bullet}$, it holds
\begin{equation*}
    \Hom_{\cat{K}(\Aa)}(A^{\bullet}, B^{\bullet}) :=
    \left. \Hom_{C^{\bullet}(\Aa)}(A^{\bullet}, B^{\bullet}) \middle/ \sim \right. \,,
\end{equation*}
where the relation $\sim$ is given by homotopy equivalence.
Notice that the quotient naturally inherits
the sum: given $f^{\bullet} \sim \tilde{f}^{\bullet}$ 
and $g^{\bullet} \sim \tilde{g}^{\bullet}$,
if $s$ is a homotopy between $f^{\bullet}$ and $\tilde{f}^{\bullet}$
and $t$ is a homotopy between $g^{\bullet}$ and $\tilde{g}^{\bullet}$,
then $s+t$ is a homotopy between $f^{\bullet} + g^{\bullet}$ 
and $\tilde{f}^{\bullet} + \tilde{g}^{\bullet}$, indeed
\begin{align*}
    (s+t)d_{A}^{\bullet} + d_{B}^{\bullet}(s+t)
    &= (sd_{A}^{\bullet} + d_{B}^{\bullet}s)
    + (td_{A}^{\bullet} + d_{B}^{\bullet}t) \\
    &= (f^{\bullet} - \tilde{f}^{\bullet}) + (g^{\bullet} - \tilde{g}^{\bullet}) \\
    &= (f^{\bullet} + g^{\bullet}) - (\tilde{f}^{\bullet} + \tilde{g}^{\bullet})\,, 
\end{align*}
thus $f^{\bullet} + g^{\bullet} \sim \tilde{f}^{\bullet} + \tilde{g}^{\bullet}$.
It follows that $\cat{K}(\Aa)$ is an additive category and the quotient functor
\begin{equation*}
    C^{\bullet}(\Aa) \longrightarrow \cat{K}(\Aa)
\end{equation*}
is an additive functor. 

Sometimes it is useful to consider 
categories of complexes having special properties: 
if $\Cc$ is any \emph{full} subcategory of $C^{\bullet}(\Aa)$, 
let $\cat{K}(\Cc)$ denote the \emph{full} subcategory of $\cat{K}(\Aa)$
whose objects are the cochain complexes in $\Cc$. 
Then $\cat{K}(\Aa)$ is a quotient category of $\Cc$;
moreover, if $\Cc$ contains the zero object and it is closed
under direct sum $\oplus$, then both 
$\Cc$ and $\cat{K}(\Cc)$ are additive categories
and the quotient $\Cc \to \cat{K}(\Cc)$ is an additive functors.

\begin{df}
    We write $\cat{K}^{\flat}(\Aa), \cat{K}^{-}(\Aa)$ and $\cat{K}^{+}(\Aa)$ 
    for the full subcategories of $\cat{K}(\Aa)$ 
    corresponding to the full subcategories 
    $C^{\flat}(\Aa), C^{-}(\Aa)$ and $C^{+}(\Aa)$ 
    of bounded, bounded above, 
    and bounded below cochain complexes respectively.
\end{df}

Having introduced the cast of categories, we turn to their properties.

\begin{lemma}
    For every $n \in \Z$, the $n$-th cohomology $H^{n}$
    is a well defined functor on the quotient category:
    \begin{equation*}
        H^{n} : \cat{K}(\Aa) \longrightarrow \Aa\,.
    \end{equation*}
    \begin{proof}
        We know that, if $f^{\bullet} \sim g^{\bullet}$,
        then $H^{n}(f^{\bullet}) = H^{n}(f^{\bullet})$,
        thus cohomology descends to the quotient.
    \end{proof}
\end{lemma}

The homotopy category $\cat{K}(\Aa)$ is characterized by the following
\begin{prop}[\textbf{Universal property}]
    Let $\Dd$ be a category and $F : C^{\bullet}(\Aa) \to \Dd$
    be any functor that sends chain homotopy equivalences
    to isomorphisms. Then $F$ factors uniquely through $\cat{K}(\Aa)$,
    that is
    \begin{center}
        \begin{tikzcd}
            C^{\bullet}(\Aa) \ar[r] \ar[d, "F"'] 
            & \cat{K}(\Aa) \ar[dl, dashed, "\exists !"]\\
            \Dd\,.
        \end{tikzcd}
    \end{center}
    \begin{proof}
        Let $A^{\bullet}$ be any cochain complex in $\Aa$.
        As we have seen in 
        \hyperref[cyl-equivalence]{Exercise~\ref*{cyl-equivalence}},
        the inclusion $\iota : A^{\bullet} \to \cat{cyl}(\cat{1}_{A^{\bullet}})$
        is a homotopy equivalence, hence by assumption
        $F\iota$ is an isomorphism with inverse $F\beta^{\bullet}$,
        where $\beta(a,a',a'') =  a'$. Moreover, notice that also the map
        \begin{equation*}
            j: A^{\bullet} \longrightarrow \cat{cyl}(\cat{1}_{A^{\bullet}})\,,
            \quad a \longmapsto (a,0,0)
        \end{equation*}
        is such that $\beta^{\bullet}j=\cat{1}_{B^{\bullet}}$,
        so in particular it holds
        \begin{equation*}
            Fj = F(\iota\,\beta^{\bullet})Fj
            = F\iota\,F(\beta^{\bullet}j) = F\iota\,.
        \end{equation*}

        Suppose now there is a cochain homotopy
        $s$ between $f^{\bullet}, g^{\bullet} : A^{\bullet} \to B^{\bullet}$;
        by \hyperref[cyl-criterion]{Exercise~\ref*{cyl-criterion}},
        is extends to a map $\gamma^{\bullet}:\cat{cyl}(\cat{1}_{A^{\bullet}}) 
        \to B^{\bullet}$ such that
        \begin{equation*}
            \gamma^{\bullet} j = f\,, \quad \gamma^{\bullet} \iota = g\,,
        \end{equation*}
        thus in $\Dd$ we have
        \begin{equation*}
            Ff = F\gamma^{\bullet}\,,Fj = F\gamma^{\bullet}\,,F\iota = Fg\,.
        \end{equation*}
        We conclude that $F$ factors through the quotient $\cat{K}(\Aa)$.
    \end{proof}
\end{prop}

We introduce now a useful operation we can perform on (co)chain complexes:
translating indices. This concept of ``shift'' functor will be 
taken to define triangulated categories later.

\begin{df}
    Let $A^{\bullet} \in C^{\bullet}(\Aa)$ be a cochain complex. 
    For any $p \in \Z$, we define the \textbf{$p^{\text{th}}$-translate}
    $A^{\bullet}[p]$ of $A^{\bullet}$ to be the complex whose $n$-th level
    objects and differentials are
    \begin{equation*}
        (A^{\bullet}[p])^{n} := A^{n+p}\,, \quad d_{[p]}^{n} = (-1)^{p}d^{n+p}\,.
    \end{equation*}
    Dually, if $A_{\bullet}$ is a chain complex,
    we define its \textbf{$p^{\text{th}}$-translate} as
    \begin{equation*}
        (A_{\bullet}[p])_{n} := A_{n+p}\,, \quad d^{[p]}_{n} = (-1)^{p}d_{n+p}\,.
    \end{equation*}
\end{df}

By shifting indices on (co)chain maps accordingly,
the $p^{\text{th}}$-translation defines a functor
\begin{equation*}
    [p] : C^{\bullet}(\Aa) \longrightarrow C^{\bullet}(\Aa)
\end{equation*}
which is in fact an equivalence of categories
(one checks that $[-p]$ is a quasi-inverse).
Note that translations shift (co)homology: indeed
\begin{equation}\label{shift-coh}
    H^{n}(A^{\bullet}[p]) = H^{n+p}(A^{\bullet})\,,
    \quad H_{n}(A_{\bullet}[p]) = H_{n-p}(A_{\bullet})\,.
\end{equation}

\begin{df}
    Let $f^{\bullet} : A^{\bullet} \to B^{\bullet}$ be a cochain map.
    The mapping cone of $f^{\bullet}$ fits into a
    short exact sequence
    \begin{center}
        \begin{tikzcd}
            \cat{0} \ar[r]
            & B^{\bullet} \ar[r, "g^{\bullet}"]
            & \cat{C}(f^{\bullet}) \ar[r, "\delta"]
            & A[-1]^{\bullet} \ar[r]
            & \cat{0}\,.
        \end{tikzcd}
    \end{center}
    The \textbf{strict triangle} on $f^{\bullet}$
    is the triple $(f^{\bullet}, g^{\bullet}, \delta)$
    of maps in $\cat{K}(\Aa)$, displayed as
    \begin{center}
        \begin{tikzcd}
            A^{\bullet} \ar[rr, "f^{\bullet}"]
            & & B^{\bullet} \ar[dl, "g^{\bullet}"] \\
            & \cat{C}(f^{\bullet}) \ar[ul, "\delta"] & \,.
        \end{tikzcd}
    \end{center}
    
    Given three cochain complexes $X^{\bullet}, Y^{\bullet}, Z^{\bullet}
    \in \Aa$ and maps $$u^{\bullet} : X^{\bullet} \to Y^{\bullet}, \quad
    v^{\bullet} : Y^{\bullet} \to Z^{\bullet}, \quad
    w^{\bullet} : Z^{\bullet} \to X[-1]^{\bullet}$$ 
    in $\cat{K}(\Aa)$,
    the triple $(u,v,w)$ is an \textbf{exact triangle} 
    on $(X^{\bullet},Y^{\bullet},Z^{\bullet})$
    if it is ``isomorphic'' to a strict triangle of $f^{\bullet}$, 
    for some $f^{\bullet} : A^{\bullet} \to B^{\bullet}$, 
    in the sense that there exists a diagram
    \begin{center}
        \begin{tikzcd}
            X^{\bullet} \ar[r, "u^{\bullet}"] \ar[d, "\alpha^{\bullet}"]
            & Y^{\bullet} \ar[d, "\beta^{\bullet}"] \ar[r, "v^{\bullet}"]
            & Z^{\bullet} \ar[d, "\gamma^{\bullet}"] \ar[r, "w^{\bullet}"]
            & X{[-1]}^{\bullet} \ar[d, "\alpha{[-1]}^{\bullet}"] \\
            A^{\bullet} \ar[r, "f^{\bullet}"]
            & B^{\bullet} \ar[r]
            & \cat{C}(f^{\bullet}) \ar[r]
            & A{[-1]}^{\bullet}
        \end{tikzcd}
    \end{center}
    which commutes in $\cat{K}(\Aa)$, where the vertical arrows
    are isomorphisms (in $\cat{K}(\Aa)$).
\end{df}

\begin{lemma}[\textbf{LECS}]
    Given an exact triangle $(u,v,w)$ on 
    $(A^{\bullet}, B^{\bullet}, C^{\bullet})$,
    the long cohomology sequence
    \begin{center}
        \begin{tikzcd}
            \dots \ar[r]
            & H^{i}(A^{\bullet}) \ar[r, "u^*"]
            & H^{i}(B^{\bullet}) \ar[r, "v^*"]
            & H^{i}(C^{\bullet}) \ar[r, "w^*"]
            & H^{i+1}(A^{\bullet}) \ar[r]
            & \dots
        \end{tikzcd}
    \end{center}
    is exact, where we have identified
    $H^{i}(A^{\bullet}[-1]) = H^{i+1}(A^{\bullet})$
    as in \eqref{shift-coh}.
    \begin{proof}
        For a strict triangle, the result is
        clear because $H^{i}$ is additive 
        and the sequence is split.
        For any other triangle, exactness
        follows because $H^i$ is a functor on $\cat{K}(\Aa)$,
        hence it preserves isomorphisms up to homotopy.
    \end{proof}
\end{lemma}

The following two technical results will be crucial
for the definition of morphisms in the derived category
of $\Aa$.

%\begin{prop}\label{double-cone}
%    Let $f^{\bullet} : A^{\bullet} \to B^{\bullet}$
%    be a cochain map and write 
%    $\iota : B^{\bullet} \to \cat{C}(f)$ for the canonical inclusion
%    and $\pi : \cat{C}(f) \to A^{\bullet}[1]$ for the projection.
%    There exists a homotopy equivalence
%    $g^{\bullet} :A^{\bullet}[1] \to \cat{C}(\iota)$ such that the following
%    diagram
%    \begin{equation*}
%        \begin{tikzcd}
%            B^{\bullet} \ar[r, "\iota"] \ar[d, equals]
%            & \cat{C}(f) \ar[r, "\pi"] \ar[d, equals] 
%            & A^{\bullet}{[1]} \ar[d, "g^{\bullet}", dashed] \ar[r, "-f^{\bullet}"]
%            & B^{\bullet}{[1]} \ar[d, equals] \\
%            B^{\bullet} \ar[r, "\iota"]
%            & \cat{C}(f) \ar[r, "\iota_{\iota}"] 
%            & \cat{C}(\iota) \ar[r, "\pi_{\iota}"]
%            & B^{\bullet}{[1]} 
%        \end{tikzcd}
%    \end{equation*}
%    commutes up to homotopy.
%    \begin{proof}
%        Set $g^{\bullet} = -f^{\bullet}[1] \oplus \cat{1}_{A^{\bullet}[1]} \oplus 0$
%        and notice that it defines a map of complexes: for each $n \in \Z$,
%        it holds
%        \begin{align*}
%            g^{n+1}d_{A}[1]^{n}
%            &= \begin{pmatrix}
%                (-f^{n+2}) (-d_{A}^{n+1}) \\ -d_{A}^{n+1} \\ 0
%            \end{pmatrix}
%            = \begin{pmatrix}
%                f^{n+2} d_{A}^{n+1} \\ -d_{A}^{n+1} \\ 0
%            \end{pmatrix}
%            = \begin{pmatrix}
%                d_{B}^{n+2} f^{n+1}\\ -d_{A}^{n+1} \\ 0
%            \end{pmatrix} \\
%            &= \begin{pmatrix}
%                -d_{B}^{n+2} & & \\ & -d_{A}^{n+1} & \\ \cat{1}_{B^{n+2}} & f^{n+1} & d_{B}^{n+1}
%            \end{pmatrix}
%            \begin{pmatrix}
%                -f^{n+1} \\ \cat{1}_{A^{n+1}} \\ 0
%            \end{pmatrix}
%            = d_{\iota}^{n+1} \, g^{n}\,.
%        \end{align*}
%        One can check that the canonical projection 
%        morphism $q:\cat{C}(\iota) \to A^{\bullet}{[1]}$ defines
%        a homotopy inverse to $g^{\bullet}$.
%        
%        Notice that the right-most square commutes in $C^{\bullet}(\Aa)$,
%        i.e. $\pi_{\iota}g^{\bullet} = -f^{\bullet}$, but in general
%        $g^{\bullet}\pi \ne \iota_{\iota}$; nevertheless,
%        that box commutes up to homotopy: indeed, 
%        from the identity $\pi=q\iota_{\iota}$,
%        we deduce that
%        \begin{equation*}
%            g^{\bullet}\pi = g^{\bullet}q\iota_{\iota} \simeq \iota_{\iota}\,,
%        \end{equation*}
%        which translates to an equality in $\cat{K}(\Aa)$.
%    \end{proof}
%\end{prop}

\begin{prop}[Rooves composition]\label{roof-comp}
    Given two morphisms $f^{\bullet}:A^{\bullet} \to B^{\bullet}$
    and $g^{\bullet}:C^{\bullet} \to B^{\bullet}$,
    there exists a commutative diagram in $\cat{K}(\Aa)$
    \begin{equation*}
        \begin{tikzcd}
            C_{0}^{\bullet} \ar[r, "\Tilde{f}^{\bullet}"] \ar[d, "\Tilde{g}^{\bullet}"']
            & C^{\bullet} \ar[d, "g^{\bullet}"] \\
            A^{\bullet} \ar[r, "f^{\bullet}"]
            & B^{\bullet}\,,
        \end{tikzcd}
    \end{equation*}
    such that:
    \begin{itemize}
        \item if $f^{\bullet}$ is a qis, then $\Tilde{f}^{\bullet}$ is a qis too;
        \item if $g^{\bullet}$ is a qis, then $\Tilde{g}^{\bullet}$ is a qis too.
    \end{itemize}
    \begin{proof}
        Consider the two following strict triangles:
        \begin{equation*}
            \begin{tikzcd}
                A^{\bullet} \ar[rr, "f^{\bullet}"]
                & & B^{\bullet} \ar[dl, "\iota"] 
                &
                & \cat{C}(\iota \circ g^{\bullet}) \ar[rr, "\varpi"]
                & & C^{\bullet} \ar[dl, "\iota \circ g^{\bullet}"] \\
                & \cat{C}(f^{\bullet}) \ar[ul, "\pi"] &
                & \,, & &  \cat{C}(f^{\bullet}) \ar[ul, "j"] & \,,
            \end{tikzcd}
        \end{equation*}
        and define $\gamma^{\bullet} : \cat{C}(\iota \circ g^{\bullet}) \to A^{\bullet}[1]$
        on the $n$-th level as
        \begin{equation*}
            \gamma^{n} := \begin{pmatrix}
                0 & \cat{1}_{A^{n+1}} & 0
            \end{pmatrix}\,;
        \end{equation*}
        one can check that $\gamma^{\bullet}$ actually defines a cochain map
        and, in fact, it makes the following diagram
        \begin{equation*}
            \begin{tikzcd}
                \cat{C}(\iota \circ g^{\bullet}){[-1]} 
                \ar[d, "\gamma^{\bullet}{[-1]}"', dashed] \ar[r, "\varpi"]
                & C^{\bullet} \ar[r] \ar[d, "g^{\bullet}"]
                & \cat{C}(f^{\bullet}) \ar[d, equals] \ar[r, "j"]
                & \cat{C}(\iota \circ g^{\bullet}) \ar[d, "\gamma^{\bullet}"', dashed] \\
                A^{\bullet} \ar[r, "f^{\bullet}"]
                & B^{\bullet} \ar[r, "\iota"]
                & \cat{C}(f^{\bullet}) \ar[r, "\pi"]
                & A^{\bullet}{[1]}\,
            \end{tikzcd}
        \end{equation*}
        commute in $\cat{K}(\Aa)$: in fact, 
        the rightmost box commutes in $C^{\bullet}(\Aa)$ because
        \begin{equation*}
            \gamma^{\bullet} \circ j
            = \begin{pmatrix}
                0 & \cat{1}_{A^{\bullet}{[1]}} & 0
            \end{pmatrix}
            \begin{pmatrix}
                0 & 0 \\ \cat{1}_{A^{\bullet}{[1]}} & 0 \\ 0 & \cat{1}_{B^{\bullet}}
            \end{pmatrix}
            = \begin{pmatrix}
                \cat{1}_{A^{\bullet}{[1]}} & 0
            \end{pmatrix}
            = \pi\,,
        \end{equation*}
        while the map $h:\cat{C}(\iota \circ g^{\bullet}){[-1]} \to B^{\bullet}[-1]$
        given by 
        $h = \begin{pmatrix}
            0 & 0 & \cat{1}_{B^{\bullet}{[-1]}}
        \end{pmatrix}$ 
        is a homotopy between $g^{\bullet} \circ \varpi$
        and $f^{\bullet} \circ \gamma^{\bullet}[-1]$ because
        \begin{align*}
            hd_{-(\iota \circ g)} + d_{B^{\bullet}[-1]}h 
            &= h
            \begin{pmatrix}
                -d_{C} & 0 \\ -(\iota \circ g^{\bullet}) & d_{f}
            \end{pmatrix}
            - d_{B} \begin{pmatrix}
                0 & 0 & \cat{1}_{B^{\bullet}[-1]}
            \end{pmatrix}   \\
            &= \begin{pmatrix}
                0 & 0 & \cat{1}_{B^{\bullet}[-1]}
            \end{pmatrix}
            \begin{pmatrix}
                -d_{C} & 0 & 0 \\ 0 & -d_{A} & 0 \\ -g^{\bullet} & f^{\bullet} & d_{B}
            \end{pmatrix}
            +  \begin{pmatrix}
                0 & 0 & -d_{B}
            \end{pmatrix}   \\
            &= \begin{pmatrix}
                -g^{\bullet} & f^{\bullet} & 0
            \end{pmatrix} 
            = \begin{pmatrix}
                0 & f^{\bullet} & 0
            \end{pmatrix} 
            - \begin{pmatrix}
                g^{\bullet} & 0 & 0
            \end{pmatrix}
            = f^{\bullet} \circ \gamma^{\bullet}[-1] - g^{\bullet} \circ \varpi\,.
        \end{align*}
        This shows that the triple $(\gamma^{\bullet}[-1],g^{\bullet}, \cat{1})$
        defines a morphism of triangles in $\cat{K}(\Aa)$.
        Thus, by setting 
        $$C_{0}^{\bullet} := \cat{C}(\iota \circ g^{\bullet})[-1]\,, \quad
        \Tilde{f}^{\bullet} := \varpi \,, \quad  
        \Tilde{g}^{\bullet} = \gamma^{\bullet}[-1]\,,$$
        we get the desired commutative square in $\cat{K}(\Aa)$.
        
        Moreover, by applying the cohomology functor
        we obtain the following commutative diagram in $\Aa$:
            \begin{equation*}
                \begin{tikzcd}
                    \dots \ar[r]
                    & H^{n-1}(\cat{C}(f^{\bullet})) \ar[r, "\Tilde{f}^*"] \ar[d, equals]
                    & H^{n}(C^{\bullet}_{0}) \ar[r] \ar[d, "\Tilde{g}^*"]
                    & H^{n}(C^{\bullet}) \ar[r] \ar[d, "g^*"]
                    & H^{n}(\cat{C}(f^{\bullet})) \ar[r] \ar[d, equals]
                    & \dots \\
                    \dots \ar[r]
                    & H^{n-1}(\cat{C}(f^{\bullet})) \ar[r]
                    & H^{n}(A^{\bullet}) \ar[r, "f^*"]
                    & H^{n}(B^{\bullet}) \ar[r]
                    & H^{n}(\cat{C}(f^{\bullet})) \ar[r]
                    & \dots
                \end{tikzcd}
            \end{equation*}
        where the rows are long exact sequences.
        Thus, we notice that if $f^{\bullet}$ is a qis,
        then $\cat{C}(f^{\bullet})$
        has trivial cohomology, which implies that 
        $\Tilde{f}^* : H^{*}(C^{\bullet}_{0}) \simeq H^{*}(C^{\bullet})$,
        so $\Tilde{f}^{\bullet}$ is a qis. On the other hand, 
        if $g^{\bullet}$ is a qis, then by the
        \hyperref[5-lemma]{Five Lemma} we deduce that
        also $\Tilde{g}^{\bullet}$ is a qis.
    \end{proof}
\end{prop}

\begin{exercise}
    One might be tempted to define $C_{0}^{\bullet}$
    in the above \hyperref[roof-comp]{Proposition~\ref*{roof-comp}}
    as the fibered product
    \begin{equation*}
        C_{0}^{\bullet} := A^{\bullet} \times_{B^{\bullet}} C^{\bullet}\,.
    \end{equation*}
    Show that, in general, this choice does not work
    and it does not guarantee the nice properties of the
    $C_{0}^{\bullet}$ built above.
    \begin{proof}[An example]
        Let $B^{\bullet}$ be the complex
        \begin{equation*}
            \begin{tikzcd}
                \cat{0} \ar[r]
                & B^{0} \ar[r,"d", two heads] 
                & B^{1} \ar[r]
                & \cat{0}\,,
            \end{tikzcd}
        \end{equation*}
        where $d$ is an epimorphism between non-trivial objets,
        with non trivial kernel $A:=\ker d \ne \cat{0}$.
        Denote by $A^{\bullet}$ the complex with $A$ 
        concentrated in degree $0$: then, there is
        a natural inclusion 
        $\iota:A^{\bullet} \hookrightarrow B^{\bullet}$.
        Set $C^{\bullet} := B^{\bullet}[1]$ and consider the
        cochain map $\Delta:C^{\bullet} \to B^{\bullet}$
        given by the diagram
        \begin{equation*}
            \begin{tikzcd}
                \cat{0} \ar[r]
                & \cat{0} \ar[r] \ar[d]
                & B^{0} \ar[r, "d", two heads] \ar[d, "d", two heads]
                & B^{1} \ar[r] \ar[d]
                & \cat{0} \\
                \cat{0} \ar[r] 
                & B^{0} \ar[r, "d", two heads]
                & B^{1} \ar[r]
                & \cat{0} \ar[r]
                & \cat{0}\,.
            \end{tikzcd}
        \end{equation*}
        One can check that the fibered product $A^{\bullet} \times_{B^{\bullet}} C^{\bullet}$
        is given by the complex
        \begin{equation*}
            \begin{tikzcd}
                \cat{0} \ar[r]
                & A \ar[r, hook]
                & B^{0} \ar[r, "d", two heads]
                & B^{1} \ar[r]
                & \cat{0}\,,
            \end{tikzcd}
        \end{equation*}
        with projection morphisms on $A^{\bullet}$ and $C^{\bullet}$
        being the obvious ones.

        Now consider the $0$-th cohomology of these complexes:
        \begin{equation*}
            H^{0}(B^{\bullet}) 
            = \mathrm{coker}\left( \cat{0} \to A \right)
            \simeq A = H^{0}(A^{\bullet})\,,
        \end{equation*}
        and $H^{0}(C^{\bullet}) \simeq H^{-1}(B^{\bullet}) = \cat{0}$.
        Since the $0$-th cohomology object
        of the fibered product is again $A$,
        by applying the functor $H^{0}$ to the usual
        cartesian square one gets
        \begin{equation*}
            \begin{tikzcd}
                A \ar[r, equals] \ar[d] & A \ar[d, "\iota^*", equals] \\
                \cat{0} \ar[r] & A\,,
            \end{tikzcd}
        \end{equation*}
        which does not commute in $\Aa$. This means that
        \begin{equation*}
            \begin{tikzcd}
                A^{\bullet} \times_{B^{\bullet}} C^{\bullet}
                \ar[r] \ar[d] & A^{\bullet} \ar[d,"\iota"]\\
                C^{\bullet} \ar[r] & B^{\bullet}
            \end{tikzcd}
        \end{equation*}
        does not commute in $\cat{K}(\Aa)$.
    \end{proof}
\end{exercise}
%%    
\section{Triangulated categories}

The notion of triangulated category generalizes the structure 
that exact triangles give to $\cat{K}(\Aa)$. 
One should think of exact triangles as 
substitutes for short exact sequences.

Consider a category $\Dd$ equipped with 
an equivalence $T:\Dd \to \Dd$.
\begin{df}
    A \textbf{triangle} on an ordered triple
    $(A,B,C)$ of objects of $\Dd$ is a 
    triple $(\alpha,\beta,\delta)$ of morphisms
    \begin{center}
        \begin{tikzcd}
            A \ar[r,"\alpha"]
            & B \ar[r, "\beta"]
            & C \ar[r, "\delta"]
            & T(A)\,;
        \end{tikzcd}
    \end{center}
    the triangle $(\alpha,\beta,\delta)$ 
    is usually displayed as follows:
    \begin{center}
        \begin{tikzcd}
            A \ar[rr,"\alpha"]
            & & B \ar[dl, "\beta"] \\
            & C \ar[ul, "\delta"] &\,.
        \end{tikzcd}
    \end{center}
    A \textbf{morphism of triangles} is a triple $(f,g,h)$
    forming the following commutative diagram %in $\Dd$:
    \begin{center}
        \begin{tikzcd}
            A \ar[r] \ar[d, "f"]
            & B \ar[r] \ar[d, "g"]
            & C \ar[r] \ar[d, "h"]
            & T(A) \ar[d, "T(f)"] \\
            A' \ar[r]
            & B' \ar[r]
            & C' \ar[r]
            & T(A')\,.
        \end{tikzcd}
    \end{center}
\end{df}

\begin{notation}
    When dealing with the automorphism $T$,
    it will be convenient to write
    $A[1] := T(A)$ for any object $A \in \Dd$;
    with this idea, if we apply $T$ (or its quasi-inverse)
    many times, we will write
    \begin{equation*}
        A[n] := T^{n}(A)\,, \quad n \in \Z\,.
    \end{equation*}
    Similarly, for any morphism $f \in \Hom_{\Dd}(A,B)$,
    we put $f[n] := T^{n}(f)$.
\end{notation}

\begin{df}
    Let $\Dd$ be an additive category.
    The structure of a \textbf{triangulated category} on $\Dd$
    is given by an additive equivalence 
    $T : \Dd \longrightarrow \Dd$,
    called \textbf{shift functor}, 
    and a set of \textbf{distinguished triangles}
    $(\alpha,\beta,\delta)$
    which are subject to the following four axioms:
    \begin{itemize}
        \item[(\textbf{TR1})]\label{TR1}
        it consists of three points.
        \begin{rmnumerate}
            \item For any $A \in \Dd$, the triangle $(\cat{1}_{A},0,0)$
                %\begin{center}
                %    \begin{tikzcd}
                %        A \ar[rr,"\cat{1}_{A}"]
                %        & & A \ar[dl, "0"] \\
                %        & \cat{0} \ar[ul, "0"] &
                %    \end{tikzcd}
                %\end{center}
                is distinguished.
                
            \item If $(\alpha,\beta,\delta)$ is a distinguished triangle
            and $(\alpha',\beta',\delta')$ is isomorphic to it,
            then $(\alpha',\beta',\delta')$ is distinguished too.
            
            \item Any morphism $\alpha : A \to B$ can be completed
            to a distinguished triangle.
        \end{rmnumerate}
        
        
        \item[(\textbf{TR2})]\label{TR2}
        \emph{\textbf{Rotation}} $-$ 
        If $(\alpha,\beta,\delta)$ is a distinguished triangle on $(A,B,C)$,
        then both its ``\emph{rotates}''
        $(\beta,\delta,-\alpha[1])$ and $(-\delta[-1],\alpha,\beta)$
        are distinguished triangles on $(B,C,A[1])$ and $(C[-1],A,B)$, respectively.
        
        \item[(\textbf{TR3})]\label{TR3}
        \emph{\textbf{Morphisms}} $-$ 
        Suppose there exists a commutative diagram of 
        distinguished triangles with vertical arrows $f$ and $g$:
        \begin{center}
            \begin{tikzcd}
                A \ar[r] \ar[d, "f"]
                & B \ar[r] \ar[d, "g"]
                & C \ar[r] \ar[d, "\exists \, h", dashed]
                & A{[1]} \ar[d, "f{[1]}"] \\
                A' \ar[r]
                & B' \ar[r]
                & C' \ar[r]
                & A'{[1]}\,;
            \end{tikzcd}
        \end{center}
        then there exists a (\textbf{not} necessarily unique) 
        morphism $h:C \to C'$ which 
        completes it to a morphism of triangles.
        
        \item[(\textbf{TR4})]\label{TR4}
        \emph{\textbf{The octahedral axiom}} $-$ 
        Given six objects $A,B,C,A',B',C' \in \Dd$
        and three triangles
        \begin{center}
            \begin{tikzcd}[column sep=small]
                A \ar[rr, "\alpha"] & & B \ar[dl, "i"]
                & & 
                B \ar[rr, "\beta"] & & C \ar[dl, "j"]
                & & 
                A \ar[rr, "\beta \alpha"]& & C \ar[dl, "l"] \\
                & C' \ar[ul,"\de"] & \,, &
                & & A' \ar[ul,"d"] & \,, & 
                & & B' \ar[ul,"\delta"] & \,,
            \end{tikzcd}
        \end{center}
        then there exists a fourth distinguished triangle
        \begin{center}
            \begin{tikzcd}
                C' \ar[rr,"f"]
                & & B' \ar[dl, "g"] \\
                & A' \ar[ul, "i{[1]} \circ d"] &\,,
            \end{tikzcd}
        \end{center}
        such that in the following octahedron 
        \begin{equation}\label{octahedron}
            \begin{tikzcd}[row sep=huge, column sep=huge]
            & B \arrow[dr, squiggly, "\beta"] \arrow[ddl, squiggly, dashed, "i", pos=0.25] \arrow[from=ddr, crossing over, dashed, "\alpha", pos=0.75]  & \\
            A' \arrow[ur, "d"] \arrow[d, "i{[1]} \circ d"'] 
            & & C \arrow[ll, crossing over, "j"'] \\
            C' \arrow[rr, dashed, "\de"] \arrow[dr, squiggly, "f"'] 
            & & A \arrow[u, "\beta \alpha"']  \\
             & B' \arrow[uul, crossing over, "g"', pos=0.6] 
             \arrow[ur, "\delta"'] \arrow[from=uur, squiggly, crossing over, "l"', pos=0.4]  &
            \end{tikzcd}
        \end{equation}
        we have:
        \begin{rmnumerate}
            \item the four distinguished triangles above forming four of the faces;
            
            \item the remaining four faces commute, that boils down to
            \begin{equation*}
                j = g \circ l\,, \quad \de = \delta \circ f\,;
            \end{equation*}
            
            \item the most exterior paths connecting $B$ and $B'$ commute,
            that is
            \begin{equation*}
                f \circ i = l \circ \beta \,, \quad d \circ g = \alpha \circ \delta\,.
            \end{equation*}
        \end{rmnumerate}
    \end{itemize}
\end{df}

\begin{rmk}
    Some comments on axiom 
    \hyperref[TR4]{(\textbf{TR4})} are needed,
    because it is with no doubt 
    the most mysterious
    and confusing one.
    First, we may ``unpack'' the octahedron~\eqref{octahedron}
    to give a nicer visualization:
    notice that the triangle $(f,g,i{[1]}\circ d)$
    completes the commutative plane diagram
    \begin{center}
        \begin{tikzcd}
            A \ar[r, "\alpha"] \ar[d, equals]
            & B \ar[r, "i"] \ar[d, "\beta"]
            & C' \ar[r, "\de"] \ar[d, "\exists"',"f", dashed]
            & A{[1]} \ar[d, equals] \\
            A \ar[r]
            & C \ar[r,"l"] \ar[d, "j"]
            & B' \ar[r, "\delta"] \ar[d, "\exists"',"g", dashed] \ar[d, dashed]
            & A{[1]} \ar[d, "\alpha{[1]}"]\\
            & A' \ar[r, equals] \ar[d, "d"] 
            & A' \ar[r, "d"] \ar[d, "\exists"', dashed]
            & B{[1]} \\
            & B{[1]} \ar[r]
            & C'{[1]}
            & \,.
        \end{tikzcd}
    \end{center}
    
    We may interpret the \hyperref[TR4]{\emph{\textbf{Octahedral Axiom}}}
    as a generalization of the \textbf{Third Isomorphism Theorem} we
    find in abelian categories: indeed, if we think of distinguished
    triangles as an analogous of exact sequences in abelian categories, 
    then the three triangles in \hyperref[TR4]{(\textbf{TR4})} 
    tell us that
    \begin{equation*}
        C' \simeq \left. B \middle/ A \right. \,, \quad
        B' \simeq \left. C \middle/ A \right. \,, \quad
        A' \simeq \left. C \middle/ B \right. \,, 
    \end{equation*}
    thus the \hyperref[TR4]{\emph{\textbf{Octahedral Axiom}}}
    states that $(C/A)/(B/A) \simeq C/B$.
    We may visualize it in the following nice diagram:
    the three lines passing through $A,B$ and $C$ completely
    determine the sequence $C' \to B' \to A'$ on the right:
    \begin{center}
% https://tikzcd.yichuanshen.de/#N4Igdg9gJgpgziAXAbVABwnAlgFyxMJZAJgBoBmAXVJADcBDAGwFcYkQAhEAX1PU1z5CKACykR1Ok1bsAwjz4gM2PASIA2CpIYs2iTgHIF-FUKIAGUuu3S9IAILGlA1cOQAOKzd3t7R3iaCaqKk5t4y+rL+kjBQAObwRKAAZgBOEAC2SOQ0OBBIliCM9ABGMIwACi5m+owwyTggNDoRIAA6bUxoABb0TmmZBbn5iACszbbsWP3pWYg5IHlIAIwBIANzy8NIZFI+tTODiIVLiFt7rR1lOH1rGyvbiGIXdgBWTUWl5VWmwSCpWDi3Uad1mSHGixGuxadmShzmu1Ozxh7DiPEo3CAA
\begin{tikzcd}[column sep=small, row sep=small]
                                         &  &                                           &  & C' \arrow[rrdddd, "f", dashed]                   &  &                       &  &    \\
                                         &&&&&&&&\\
                                         &  &                                           &  &                                         &  &                       &  &    \\
                                         &  &                                           &  &                                         &  &                       &  &    \\
                                         &  & B \arrow[rruuuu, "i"] \arrow[rrd, "\beta"] &  &                                         &  & B' \arrow[rrddd, "g", dashed] &  &    \\
                                         &  &                                           &  & C \arrow[rru, "l"] \arrow[rrrrdd, "j"'] &  &                       &  &    \\
                                         &  &                                           &  &                                         &  &                       &  &    \\
A \arrow[rruuu, "\alpha"] \arrow[rrrruu] &  &                                           &  &                                         &  &                       &  & A'\,.
\end{tikzcd}
    \end{center}
\end{rmk}

\begin{rmk}
    The definition of triangulated category
    may be modified by changing the forth axiom:
    for instance, consider the statement
    \begin{itemize}
        \item[(\textbf{TR4'})]  given a commutative diagram
        \begin{center}
            \begin{tikzcd}
                A \ar[r, "\alpha"] \ar[d, "f"]
                & B \ar[r, "\beta"] \ar[d, "g"]
                & C \ar[r, "\delta"] \ar[d, "\exists"',"h", dashed]
                & A{[1]} \ar[d] \\
                A' \ar[r, "\alpha'"]
                & B' \ar[r, "\beta'"]
                & C' \ar[r, "\delta'"]
                & A'{[1]}
            \end{tikzcd}
        \end{center}
        whose rows are distinguished triangles,
        there exists a morphism $h:C \to C'$
        that makes the diagram commutative and
        makes the mappying cone a distinguished triangle:
        \begin{center}
            \begin{tikzcd}[ampersand replacement=\&, column sep=large]
                B \oplus A' \ar[r, "{\begin{pmatrix}
                    -\beta & 0 \\ g & \alpha'
                \end{pmatrix}}"]
                \& C \oplus B' \ar[r, "{\begin{pmatrix}
                    -\delta & 0 \\ h & \beta'
                \end{pmatrix}}"]
                \& A{[1]} \oplus C' \ar[r, "{\begin{pmatrix}
                    -\alpha{[1]} & 0 \\ f{[1]} & \delta'
                \end{pmatrix}}"]
                \& B{[1]} \oplus A'{[1]}\,.
            \end{tikzcd}
        \end{center}
    \end{itemize}
    Assuming (\textbf{TR1}), (\textbf{TR2}) and (\textbf{TR3}),
    it turns out that (\textbf{TR4'}) is equivalent to (\textbf{TR4}).
    Moreover, it is clear that (\textbf{TR4'}) implies (\textbf{TR3})
    and its strength relies on the fact that it describes the way we can
    complete a morphism of distinguished triangles.
    There are many other different (but equivalent) axioms we can choose
    to modify the \emph{\textbf{Octahedral Axiom}}: for details,
    check \parencite[]{neeman}.
\end{rmk}

\begin{ex}
    Recall that, given a category $\Cc$, its opposite category $\Cc^{op}$
    has the same objects as $\Cc$, but arrows are ``reversed'', i.e.
    $f^{op}: B \to A$ is an morphism in $\Cc^{op}$ if and only if
    there exists $f:A \to B$ in $\Cc$. In fact, this shows that
    we have a bijection
    \begin{equation*}
        (-)^{op} : \Hom_{\Cc}(A,B) \simeq \Hom_{\Cc^{op}}(B,A)\,.
    \end{equation*}
    If $\Dd$ is a triangulated category, the opposite category $\Dd^{op}$
    is triangulated as well: if $T$ is the shift functor in $\Dd$,
    set 
    \begin{equation*}
        t : \Dd^{op} \longrightarrow \Dd^{op}\,,
        \quad
        \left( B \xrightarrow[]{f^{op}} A \right)
        \longmapsto
        \left( T^{-1}B \xrightarrow[]{(T^{-1}f)^{op}} T^{-1}A \right)
    \end{equation*}
    to be the shift functor in $\Dd^{op}$ and say that a sequence
    \begin{center}
        \begin{tikzcd}
            C \ar[r, "g^{op}"]
            & B \ar[r, "f^{op}"]
            & A \ar[r, "h^{op}"]
            & tC
        \end{tikzcd}
    \end{center}
    is a distinguished triangle in $\Dd^{op}$
    if and only if 
    \begin{center}
        \begin{tikzcd}
            C{[-1]} \ar[r, "h"]
            & A \ar[r, "f"]
            & B \ar[r, "g"]
            & C
        \end{tikzcd}
    \end{center}
    is a distinguished triangle in $\Dd$.
    It is easy to check that these two notions define 
    a triangulated structure on $\Dd^{op}$:
    indeed, all the properties \hyperref[TR1]{(\textbf{TR1})-(\textbf{TR4})}
    hold because they are true in $\Dd$, so
    we can build ``opposite'' triangles easily;
    strictly speaking, distinguished triangles
    in $\Dd^{op}$ are the same of $\Dd$,
    but with reversed indexing.
\end{ex}

\begin{ex}
    For any category $\Cc$, let $\Cc^{\Z}$ be the
    category of graded objects in $\Cc$, 
    i.e. families $A_{*} = \Set{A_{n}}_{n \in \Z}$ indexed
    on the integers, a morphism $f: A_* \to B_*$
    being a family of morphisms $f_{n} : A_{n} \to B_{n}$,
    for each $n \in \Z$. This category is naturally
    endowed with a shift functor $T$ given by the
    translation $TA_{*} := A[-1]_*$, that is
    \begin{equation*}
        TA_{n} := A[-1]_n = A_{n-1}\,,
        \quad n \in \Z\,.
    \end{equation*}

    We now define distinguished triangles in $\Cc^{\Z}$
    to be triplets $(\alpha,\beta,\delta)$ such
    that, for every $n \in \Z$, the following
    sequence is exact:
    \begin{center}
        \begin{tikzcd}
            A_{n} \ar[r, "\alpha"]
            & B_{n} \ar[r, "\beta"]
            & C_{n} \ar[r, "\delta"]
            & A_{n-1} \ar[r, "\alpha"]
            & B_{n-1}\,.
        \end{tikzcd}
    \end{center}
    
    Consider now $\Cc = \Ab$. Then $\Ab^{\Z}$ is an abelian category
    (it is essentially $C_{\bullet}(\Ab)$ with no boundary maps),
    and it clearly satisfies both \hyperref[TR1]{(\textbf{TR1})(i)}
    and \hyperref[TR1]{(\textbf{TR1})(ii)}. Note that
    any morphism $\alpha:A_{*} \to B_{*}$ is embedded in a triangle
    of the form
    \begin{center}
        \begin{tikzcd}
            A_* \ar[rr, "\alpha"] 
            & & B_* \ar[ld] \\
            & \Coker(\alpha) \oplus \left(\ker(\alpha){[-1]} \right) \ar[ul] & \,,
        \end{tikzcd}
    \end{center}
    which is given on the $n$-th level by
    \begin{center}
        \begin{tikzcd}
            A_{n} \ar[r, "\alpha_{n}"]
            & B_{n} \ar[r, "q_{n} \oplus 0"]
            & \Coker(\alpha_{n}) \oplus \left(\ker(\alpha_{n-1}) \right) 
            \ar[r, "0 + \iota_{n-1}"]
            & A_{n-1} \ar[r, "\alpha_{n-1}"]
            & B_{n-1}\,,
        \end{tikzcd}
    \end{center}
    where $q$ is the quotient, while $\iota$ the inclusion. 
    Thus, \hyperref[TR1]{(\textbf{TR1})(iii)} holds true.

    If $(\alpha,\beta, \delta)$ is a distinguished triangle,
    then the $n$-th level of
    its rotation $(\beta, \delta, \alpha[-1])$ is given by
    \begin{center}
        \begin{tikzcd}
            B_{n} \ar[r, "\beta"]
            & C_{n} \ar[r, "\delta"]
            & A_{n-1} \ar[r, "\alpha"]
            & B_{n-1} \ar[r, "\beta"]
            & C_{n-1} \,,
        \end{tikzcd}
    \end{center}
    which is an exact sequence because both
    \begin{center}
        \begin{tikzcd}
            A_{n} \ar[r, "\alpha"]
            & B_{n} \ar[r, "\beta"]
            & C_{n} \ar[r, "\delta"]
            & A_{n-1} \ar[r, "\alpha"]
            & B_{n-1}
        \end{tikzcd}
    \end{center}
    and
    \begin{center}
        \begin{tikzcd}
            A_{n-1} \ar[r, "\alpha"]
            & B_{n-1} \ar[r, "\beta"]
            & C_{n-1} \ar[r, "\delta"]
            & A_{n-2} \ar[r, "\alpha"]
            & B_{n-2}
        \end{tikzcd}
    \end{center}
    are exact. The same holds true for the other rotation,
    hence we have \hyperref[TR2]{(\textbf{TR2})}. 
    
    Nevertheless, the axiom \hyperref[TR3]{(\textbf{TR3})}
    is not valid in $\Ab^{\Z}$: consider any abelian group $G$
    to be a graded object concentrated in degree 0; then,
    the diagram
    \begin{center}
        \begin{tikzcd}
            \Z \ar[r, "\cdot 2"] \ar[d, "\cdot 2"]
            & \Z \ar[r] \ar[d, "\Delta"]
            & \Z/2\Z \ar[r]
            & \cat{0} \ar[d, equals] \\
            \Z \ar[r, "\Delta"]
            & \Z^{2} \ar[r, "+"] 
            & \Z \ar[r]
            & \cat{0}\,,
        \end{tikzcd}
    \end{center}
    commutes, but it cannot be completed to a morphism of triangles.
    Thus, the category $\Ab^{\Z}$ is not triangulated.
\end{ex}

\begin{ex!}\label{Z-Vect}\todo{Check Octahedral Axiom.}
    Let $k$ be an arbitrary field, and $\Cc = \Vect{k}$ be the category
    of vector spaces over $k$. Then $\Vect{k}^{\Z}$ is triangulated:
    indeed, axioms \hyperref[TR1]{(\textbf{TR1})} and
    \hyperref[TR2]{(\textbf{TR2})} hold true for the same
    reasons as in the previous example.

    We now verify \hyperref[TR3]{(\textbf{TR3})}:
    given a commutative diagram
    \begin{center}
        \begin{tikzcd}
            A_{n} \ar[r, "\alpha"] \ar[d, "f_n"]
            & B_{n} \ar[r, "\beta"] \ar[d, "g_n"]
            & C_{n} \ar[r, "\delta"] 
            & A_{n-1} \ar[d, "f_{n-1}"] \\
            U_{n} \ar[r, "u"]
            & V_{n} \ar[r, "v"]
            & W_{n} \ar[r, "w"]
            & U_{n-1}
        \end{tikzcd}
    \end{center}
    we can complete it to a morphism of distinguished
    triangles by taking a basis $\Lambda$ of $C_{n}$ 
    and defining $h_{n}:C_{n} \to W_{n}$ via diagram chasing
    as follows:
    \begin{center}
        \begin{tikzcd}
            C_{n} \ar[r, "\delta"] \ar[d, "h_{n}", dashed]
            & A_{n-1} \ar[d, "f_{n-1}"] \ar[r, "\alpha"]
            & B_{n-1} \ar[d, "g_{n-1}"] 
            & & c \in \Lambda \ar[r, mapsto] \ar[d, "h_{n}:="', dashed]
            & \delta c \ar[r, mapsto] \ar[d, mapsto] 
            & 0 \ar[d, mapsto] \\
            W_{n} \ar[r, "w"]
            & U_{n-1} \ar[r, "u"]
            & V_{n-1} \,,
            & & u & f_{n-1}\delta c \ar[r, mapsto] \ar[l, dashed]
            & u\,f_{n-1}\delta c  = 0\,,
        \end{tikzcd}
    \end{center}
    for any basis element $c \in \Lambda$, we note
    that $f\delta c \in \ker u$ by the commutativity
    of the right box; since the bottom row is exact,
    we may lift $f\delta c$ to some $u \in U_{n-1}$,
    so we finally define $h_{n}c := u$. 
    The diagram commutes by construction.

    Finally, we check the \emph{\textbf{Octahedral Axiom}}:
    consider three distinguished triangles
        \begin{center}
            \begin{tikzcd}[column sep=small]
                A_{*} \ar[rr, "\alpha"] & & B_{*} \ar[dl, "i"]
                & & 
                B_{*} \ar[rr, "\beta"] & & C_{*} \ar[dl, "j"]
                & & 
                A_{*} \ar[rr, "\beta \alpha"]& & C_{*} \ar[dl, "l"] \\
                & W_{*} \ar[ul,"\de"] & \,, &
                & & U_{*} \ar[ul,"d"] & \,, & 
                & & V_{*} \ar[ul,"\delta"] & \,,
            \end{tikzcd}
        \end{center}
    and build $f:W_{*} \to V_{*}$ as follows: at the $n$-th level,
    choose a basis $\Lambda$ for $W_{n}$ and,
    for every $c \in \Lambda$, set
    \begin{equation*}
        f_{n} : W_{n} \longrightarrow V_{n}\,, \quad
        f_{n}(c) := 
        \begin{cases}
            v\,, \quad
            &\text{if there exists } v \in \delta_{n}^{-1}(\de_{n}(c))\,; \\
            0\,, \quad &\text{otherwise},
        \end{cases}
    \end{equation*}
    and extend it by linearity; by construction,
    this gives a morphism $f:W_{*} \to V_{*}$ such that
    $\delta f = \de$. Similarly, one builds $g:V_{*} \to U_{*}$
    such that $j=gl$.
\end{ex!}

\begin{exercise}
    If $(\alpha,\beta,\delta)$ is an distinguished triangle,
    show that the compositions $\beta\alpha, \delta\beta$ 
    and $\alpha[1]\,\delta$ are zero in $\Dd$.
    %This is the reason why we call the triangles exact.
    \begin{proof}[Solution]
        By \hyperref[TR1]{(\textbf{TR1})(i)}, 
        the triangle $(\cat{1}_{A},0,0)$ is distinguished,
        thus we may compare
        \begin{center}
            \begin{tikzcd}
                A \ar[r, equals] \ar[d, equals]
                & A \ar[r] \ar[d, "\alpha"]
                & \cat{0} \ar[r] \ar[d]
                & A{[1]} \ar[d, equals] \\
                A \ar[r, "\alpha"]
                & B \ar[r, "\beta"]
                & C \ar[r, "\delta"]
                & A{[1]}\,.
            \end{tikzcd}
        \end{center}
        By the commutativity of the middle box, we deduce $\beta\alpha=0$.
        Similarly, if we use the \emph{Rotation axiom}~\hyperref[TR2]{(\textbf{TR3})}
        we find commutative diagrams
        \begin{center}
            \begin{tikzcd}
                %B \ar[r, equals] \ar[d, equals] &
                B \ar[r] \ar[d, "\beta"]
                & \cat{0} \ar[r] \ar[d]
                & B{[1]} \ar[d, equals] 
                & & C \ar[r] \ar[d, "\delta"]
                & \cat{0} \ar[r] \ar[d]
                & C{[1]} \ar[d, equals]\\
                %B \ar[r, "\beta"] &
                C \ar[r, "\delta"]
                & A{[1]} \ar[r, "\alpha{[1]}"]
                & B{[1]}
                & & A{[1]} \ar[r, "\alpha{[1]}"]
                & B{[1]} \ar[r, "\beta{[1]}"] 
                & C{[1]} \,,
            \end{tikzcd}
        \end{center}
        from which we see that $\delta\beta = 0 = \alpha{[1]}\,\delta$.
    \end{proof}
\end{exercise}

\begin{exercise!}[\textbf{The $5$-lemma}]\label{5lemma}
    Let $(f,g,h)$ be a morphism of distinguished triangles. 
    If two maps are isomorphisms, then the third is an isomorphism as well.
    \begin{proof}[Solution]
        Up to rotating triangles, we may assume
        without loss of generality that both $f$ and $g$ are isomorphism.
        We show that $h$ is an isomorphism too:
        for any object $X \in \Dd$, we apply the functor 
        $\Hom_{\Dd}(-,X)$ to the diagram
        \begin{center}
            \begin{tikzcd}
                A \ar[r] \ar[d, "f", equals]
                & B \ar[r] \ar[d, "g", equals]
                & C \ar[r] \ar[d, "h"]
                & A{[1]} \ar[d, "f{[1]}", equals] \\
                A' \ar[r]
                & B' \ar[r]
                & C' \ar[r]
                & A'{[1]}\,,
            \end{tikzcd}
        \end{center}
        so that we obtain the following commutative diagram in $\Ab$:
        \begin{center}
            \begin{tikzcd}[column sep=small]
                \Hom_{\Dd}(B'{[1]},X) \ar[r] \ar[d, equals]
                & \Hom_{\Dd}(A'{[1]},X) \ar[r] \ar[d, equals]
                & \Hom_{\Dd}(C',X) \ar[r] \ar[d, "- \circ h"]
                & \Hom_{\Dd}(B',X) \ar[r] \ar[d, equals]
                & \Hom_{\Dd}(A',X) \ar[d, equals] \\
                \Hom_{\Dd}(B{[1]},X) \ar[r]
                & \Hom_{\Dd}(A{[1]},X) \ar[r] 
                & \Hom_{\Dd}(C,X) \ar[r] 
                & \Hom_{\Dd}(B,X) \ar[r] 
                & \Hom_{\Dd}(A,X) \,.
            \end{tikzcd}
        \end{center}
        By the classic \textbf{$5$-lemma} for abelian groups,
        we know the central arrow is an isomorphism;
        thus, when $X = C$, we deduce there exists $k:C' \to C$
        such that $kh = \cat{1}_{C}$. Moreover, if we plug $X=C'$,
        we see that $(\cat{1}_{C'} - hk) \circ h = h - h(kh) = 0$,
        so we conclude $\cat{1}_{C'} = hk$ because precomposition
        is an isomorphism.
    \end{proof}
\end{exercise!}

\begin{rmk}
    As a consequence, we see that the completion 
    of any $\alpha: A \to B$ to a triangle,
    as declared in the axiom \hyperref[TR1]{(\textbf{TR1})(iii)},
    is unique up to isomorphism: taken triangles
    $(\alpha,\beta,\delta)$ on $(A,B,C)$ and
    $(\alpha,\beta',\delta')$ on $(A,B,C')$,
    then by \hyperref[TR3]{(\textbf{TR3})} one has
    \begin{center}
            \begin{tikzcd}
                A \ar[r, "\alpha"] \ar[d, equals]
                & B \ar[r, "\beta"] \ar[d, equals]
                & C \ar[r, "\delta"] \ar[d, "\exists"', "h", dashed]
                & A{[1]} \ar[d, equals] \\
                A \ar[r, "\alpha"]
                & B \ar[r, "\beta'"]
                & C' \ar[r, "\delta'"]
                & A{[1]}\,,
            \end{tikzcd}
        \end{center}
    and the \hyperref[5lemma]{$5$-lemma~\ref*{5lemma}} tells us $h$ is an isomorphism.
    This means that every distinguished triangle is determined,
    up to isomorphism, by just one of its maps! In particular,
    the data of the \hyperref[TR4]{\emph{\textbf{Octahedral Axiom}}}
    are completely determined by $A \xrightarrow[]{\alpha} B \xrightarrow[]{\beta} C$.
\end{rmk}

\begin{exercise!}[The Split Lemma]\label{split-lemma}
    Let $A \to B \to C \to A[1]$ be a distinguished triangle 
    in a triangulated category $\Dd$. 
    Suppose that $C \to A[1]$ is trivial. 
    Show that then the triangle is split, 
    i.e. is given by a direct sum 
    decomposition $B \simeq A \oplus C$.
    \begin{proof}[Solution]
        Consider the commutative diagram
        \begin{center}
            \begin{tikzcd}
            C\left[-1\right] \ar[r, "0"]  \ar[d, equals]
            & A \ar[r] \ar[d, equals]
            & B \ar[r] \ar[d, dashed, "\exists"', "h"]
            & C \ar[d, equals] \\
            %& A\left[ 1 \right] \ar[d, equals] \\
            C\left[-1\right] \ar[r, "0"] 
            & A \ar[r, "j_{A}"]
            & A \oplus C \ar[r, "\pi_{C}"]
            & C \,.
            %& A\left[ 1 \right]\,.
            \end{tikzcd}
        \end{center}
        By axiom \hyperref[TR3]{(\textbf{TR3})},
        there exists an arrow $h$ which completes
        the diagram to a morphism of triangles.
        Then by the \hyperref[5lemma]{5-lemma}
        we conclude $h$ must be an isomorphism.
    \end{proof}
\end{exercise!}

\begin{prop}
    Given an abelian category $\Aa$, the category $\cat{K}(\Aa)$ is triangulated.
    \begin{proof}
        We show that a triangulated structure on $\cat{K}(\Aa)$ is given 
        by letting the shift functor $TA^{\bullet} := A^{\bullet}[1]$
        to be the translation and the family of distinguished triangles
        to be given by exact triangles.

        It is easy to check that both axioms \hyperref[TR1]{(\textbf{TR1})}
        and \hyperref[TR2]{(\textbf{TR2})} hold. Without loss of generality,
        it is enough to check \hyperref[TR3]{(\textbf{TR3})} on
        strict triangles: the diagram
        \begin{center}
            \begin{tikzcd}
                A^{\bullet} \ar[r, "u"] \ar[d, "f"]
                & B^{\bullet} \ar[r] \ar[d, "g"]
                & \cat{C}(u) \ar[r] \ar[d, "\exists"', "h", dashed]
                & A^{\bullet}{[1]} \ar[d, "f{[1]}"] \\
                \tilde{A}^{\bullet} \ar[r, "w"]
                & \tilde{B}^{\bullet} \ar[r]
                & \cat{C}(w) \ar[r]
                & \tilde{A}^{\bullet}{[1]}\,
            \end{tikzcd}
        \end{center}
        can be completed by defining the cochain map
        \begin{equation*}
            h := (f[1],g) : \cat{C}(u) \longrightarrow \cat{C}(w)\,.
        \end{equation*}
        It remains to prove the \hyperref[TR4]{\textbf{\emph{Octahedral Axiom}~(\textbf{TR4})}}.
        As before, we may assume the given triangles are strict:
        \begin{center}
            \begin{tikzcd}[column sep=small]
                A^{\bullet} \ar[rr, "\alpha"] & & B^{\bullet} \ar[dl, "j_{B}"]
                & & 
                B^{\bullet} \ar[rr, "\beta"] & & C^{\bullet} \ar[dl, "j_{C}"]
                & & 
                A^{\bullet} \ar[rr, "\beta \alpha"]& & C^{\bullet} \ar[dl, "j_{C}"] \\
                & \cat{C}(\alpha) \ar[ul,"\pi_{A{[1]}}"] & \,, &
                & & \cat{C}(\beta) \ar[ul,"\pi_{B{[1]}}"] & \,, & 
                & & \cat{C}(\beta\alpha) \ar[ul,"\pi_{A{[1]}}"] & \,,
            \end{tikzcd}
        \end{center}
        thus we may define
        \begin{center}
            \begin{tikzcd}
                \cat{C}(\alpha) \ar[rr, "{(\cat{1}_{A[1]},\beta)}"] 
                & & \cat{C}(\beta\alpha) \ar[dl, "{(\alpha[1], \cat{1}_{C})}"]\\
                & \cat{C}(\beta) \ar[ul, "j_{B[-1]}\pi_{B[1]}"] & \,.
            \end{tikzcd}
        \end{center}
        By construction, the octahedron described in \hyperref[TR4]{(\textbf{TR4})}
        commutes, so we conclude if we show that the above triangle is a distinguished triangle.
        Set $f := (\cat{1}_{A[1]},\beta)$ and consider $\cat{C}(f)$;
        on degree $n$ it is
        \begin{equation*}
            \cat{C}(f)^{n} = \cat{C}(\alpha)^{n+1} \oplus \cat{C}(\beta)^{n}
            = (A^{n+2} \oplus B^{n+1}) \oplus (A^{n+1} \oplus C^{n})\,,
        \end{equation*}
        thus we can embed $\cat{C}(\beta)^{n}=B^{n+1} \oplus C^{n}$ in it.
        This gives us a natural inclusion $\iota$ which fits in the commutative diagram
        \begin{center}
            \begin{tikzcd}
                \cat{C}(\alpha) \ar[r, "f"] \ar[d, equals]
                & \cat{C}(\beta\alpha) \ar[r] \ar[d, equals]
                & \cat{C}(\beta) \ar[r] \ar[d, "\iota"]
                & \cat{C}(\alpha){[1]} \ar[d, equals] \\
                \cat{C}(\alpha) \ar[r, "f"]
                & \cat{C}(\beta\alpha) \ar[r]
                & \cat{C}(f) \ar[r]
                & \cat{C}(\alpha){[1]}\,,
            \end{tikzcd}
        \end{center}
        in which $\iota$ is a homotopy equivalence:
        define 
        \begin{equation*}
            \phi^{\bullet}:\cat{C}(f) \longrightarrow \cat{C}(\beta)\,, \quad
            \phi^{n}\big(a',b,a,c\big) = (\alpha^{n+1}(a) + b, c)\,,
        \end{equation*}
        and note that $\phi^{\bullet}\iota(b,c)=\phi(0,b,0,c) = (b,c)$, 
        i.e. it holds $\phi^{\bullet}\iota = \cat{1}_{\cat{C}(\beta)}$.
        On the other hand, the map
        \begin{equation*}
            s : \cat{C}(f)^{n} \longrightarrow \cat{C}(f)^{n-1}\,,
            \quad s(a',b,a,c) = (a,0,0,0)
        \end{equation*}
        defines a homotopy between $\iota\phi^{\bullet}$ and $\cat{1}_{\cat{C}(f)}$:
        the coboundary map of the mappying cone $\cat{C}(f)$ is
        represented by the matrix
        \begin{equation*}
            d_{\cat{C}(f)} 
            = \begin{pmatrix}
                -d_{\cat{C}(\alpha)} & 0 \\ f^{\bullet} & d_{\cat{C}(\beta\alpha)}
            \end{pmatrix}
            = \begin{pmatrix}
                d_{A} & 0 & 0 & 0 \\
                \alpha & -d_{B} & 0 & 0 \\
                \cat{1} & 0 & -d_{A} & 0 \\
                0 & -\beta & \beta\alpha & d_{C}
            \end{pmatrix}\,,
        \end{equation*}
        hence we compute
        \begin{align*}
            ds + sd 
            = \begin{pmatrix}
                d_{A} \\ \alpha \\ -\cat{1}_{A} \\ 0
            \end{pmatrix} 
            + \begin{pmatrix}
                -d_{A} - \cat{1}_{A} \\ 0 \\ 0 \\ 0
            \end{pmatrix}
            = \begin{pmatrix}
                -\cat{1}_{A} \\ \alpha \\ -\cat{1}_{A} \\ 0
            \end{pmatrix} 
            = \begin{pmatrix}
                0 \\ \alpha + \cat{1}_{B} \\ 0 \\ \cat{1}_{C}
            \end{pmatrix} 
            - \begin{pmatrix}
               \cat{1}_{A} \\ \cat{1}_{B} \\ \cat{1}_{A} \\ \cat{1}_{C}
            \end{pmatrix}
            = \iota\phi^{\bullet} - \cat{1}_{\cat{C}(f)}\,.
        \end{align*}
        We conclude that $\iota$ is an isomorphism in $\cat{K}(\Aa)$.
    \end{proof}
\end{prop}

\begin{cor}
    Let $\Cc \subset C^{\bullet}(\Aa)$ be a full subcategory
    and $\cat{K}(\Cc)$ be its corresponding quotient category.
    Suppose that $\Cc$ is an additive category and is closed
    under translations and under contruction of mapping cones.
    Then $\cat{K}(\Cc)$ is a triangulated category.
    In particular, $\cat{K}^{\flat}(\Aa), \cat{K}^{-}(\Aa)$ and
    $\cat{K}^{+}(\Aa)$ are triangulated.
\end{cor}

\begin{df}
    An additive functor $F: \Cc \to \Dd$ 
    between triangulated categories
    with shift functors $T_{\Cc}$, and $T_{\Dd}$
    respectively, is called \textbf{exact} 
    (or \textbf{triangulated}) if it satisfies
    the following:
    \begin{itemize}
        \item[(\textbf{EF1})]\label{EF1} 
        there exists an isomorphism of functors
        \begin{equation*}
            F \circ T_{\Cc} \xrightarrow[]{\sim} T_{\Dd} \circ F\,;
        \end{equation*}

        \item[(\textbf{EF2})]\label{EF2} 
        any distinguished triangle $A \to B \to C \to A[1]$
        in $\Cc$ is mapped to a distinguished triangle
        \begin{center}
            \begin{tikzcd}
                F(A) \ar[r]
                & F(B) \ar[r]
                & F(C) \ar[r]
                & F(A){[1]}\,
            \end{tikzcd}
        \end{center}
        in $\Dd$, where we use the isomorphism $F(A[1]) \simeq F(A)[1]$
        given by \hyperref[EF1]{(\textbf{EF1})}.
    \end{itemize}
\end{df}

\begin{rmk}
    Once again, the notions of a triangulated category 
    and of an exact functor have to be adjusted 
    when one is interested in additive categories over a field $k$.
    In this case, the shift functor should be $k$-linear 
    and one usually considers only $k$-linear exact functors.
\end{rmk}

\begin{ex}
    Recall that in \hyperref[Z-Vect]{Exercise~\ref*{Z-Vect}}
    we proved that $\Vect{k}^{\Z}$ is a triangulated category.
    The total cohomology $H^* : \cat{K}(\Vect{k}) \to \Vect{k}^{\Z}$
    is a morphism of triangulated categories: indeed, 
    $H^{*}$ is additive and for every $n \in \Z$
    and every object $V^{\bullet}$ it holds
    \begin{equation*}
        (H^{*}TV^{\bullet})^{n}
        = H^{n}(V^{\bullet}[1]) 
        = H^{n-1}(V^{\bullet}) 
        = \left(H^{*}(V^{\bullet})[1]\right)^{n}
        = (TH^{*}V^{\bullet})^{n}\,,
    \end{equation*}
    so $H^{*}$ commutes with shifts; 
    axiom \hyperref[EF2]{(\textbf{EF2})}
    is the \hyperref[LHS]{Long Exact Sequence~\ref*{LHS}} induced in cohomology. 
\end{ex}

\begin{df}
    Let $\Dd$ be a triangulated category and
    $\Aa$ an abelian category.
    An additive functor $F : \Dd \to \Aa$ is called
    a \textbf{covariant cohomological functor}
    if, whenever $(\alpha, \beta, \delta)$ is a
    distinguished triangle on $(A,B,C)$,
    the long sequence
    \begin{center}
        \begin{tikzcd}
            \dots \ar[r]
            & H(A{[n]}) \ar[r, "H\alpha"]
            & H(B{[n]}) \ar[r, "H\beta"]
            & H(C{[n]}) \ar[r, "H\delta"]
            & H(A{[n+1]}) \ar[r]
            & \dots
        \end{tikzcd}
    \end{center}
    is exact in $\Aa$.
    We often write $H^{n}(A) := H(A[n])$.

    A \textbf{contravariant cohomological functor}
    on $\Dd$ is a covariant cohomological functor
    $F : \Dd^{op} \to \Aa$ (remember $\Dd^{op}$ is triangulated).
\end{df}

\begin{ex}
    The \textbf{zero-th cohomology} $H^{0} : \cat{K}(\Aa) \to \Aa$
    is a cohomological functor.
\end{ex}

\begin{ex}
    Let $\Dd$ be a triangulated category. For any $X \in \Dd$,
    the functor 
    $$\Hom_{\Dd}(X,-) : \Dd \longrightarrow \Ab$$ 
    is cohomological: indeed,
    given a distinguished triangle $(\alpha, \beta, \delta)$ 
    on $(A,B,C)$, the sequence
    \begin{center}
        \begin{tikzcd}
            \Hom_{\Dd}(X,A) \ar[r, "\alpha_{*}"]
            & \Hom_{\Dd}(X,B) \ar[r, "\beta_{*}"]
            & \Hom_{\Dd}(X,C) 
        \end{tikzcd}
    \end{center}
    is exact because $\beta\alpha=0$ implies $\im \alpha^{*} \subset \ker \beta^{*}$,
    and conversely, whenever $g \in \Hom_{\Dd}(X,B)$ is such that
    $\alpha_{*}g = \alpha \circ g = 0$, by \hyperref[TR3]{(\textbf{TR3})} applied
    to the rotated triangle we have
    \begin{center}
        \begin{tikzcd}
            X{[-1]}\ar[r] \ar[d]
            & \cat{0} \ar[d] \ar[r] 
            & X \ar[r, equals] \ar[d, "\exists"', "f", dashed]
            & X \ar[r] \ar[d, "g"]
            & \cat{0} \ar[d] \\
            B{[-1]} \ar[r]
            & C{[-1]} \ar[r]
            & A \ar[r, "\alpha"]
            & B \ar[r, "\beta"]
            & C\,,
        \end{tikzcd}
    \end{center}
    thus $g = \alpha_{*}f$ and we conclude that $\im \alpha^{*} = \ker \beta^{*}$.
    Finally, by shifting the triangle thanks to \hyperref[TR2]{(\textbf{TR2})},
    it follows that the sequence is exact everywhere else. 
    In a similar fashion, one proves that $\Hom_{\Dd}(-,X)$
    is a contravariant cohomological functor.

    Moreover, if we additionally assume that $\Dd$ is $k$-linear, for some field $k$,
    then for any $X \in \Dd$ yields a functor
    $$\Hom_{\Dd}(X,-) : \Dd \longrightarrow \Vect{k}$$
    which is cohomological and gives rise to a long exact sequence of
    vector spaces over $k$. The same holds true for the
    contravariant functor $\Hom_{\Dd}(-,X)$.
\end{ex}

\begin{rmk}
    Once again, the notions of a triangulated category 
    and of an exact functor have to be adjusted when 
    one is interested in \emph{additive categories 
    over a field} $k$. 
    In this case, the shift functor should be $k$-linear 
    and one usually considers only $k$-linear exact functors.
%Also note that in this case the two 
%long exact cohomology sequences  associated to a distinguished %triangle are long exact sequences of k-vector spaces.
\end{rmk}

% 
\begin{ex}[Verdier]
    Let $\Dd$ be a triangulated category.
    Show that every commutative square
    \begin{center}
        \begin{tikzcd}
            A \ar[r,"i"] \ar[d, "u"]
            & B \ar[d] \\
            A' \ar[r] & B'
        \end{tikzcd}
    \end{center}
    in $\Dd$ can be completed to a diagram
    \begin{center}
        \begin{tikzcd}
            A \ar[r,"i"] \ar[d, "u"]
            & B \ar[r, "j"] \ar[d] 
            & C \ar[r, "k"] \ar[d] 
            & A{[1]} \ar[d, "u{[1]}"]\\
            A' \ar[r] \ar[d, "v"]
            & B' \ar[r] \ar[d] 
            & C' \ar[r] \ar[d] 
            & A'{[1]} \ar[d, "v{[1]}"] \\
            A'' \ar[r] \ar[d, "w"]
            & B'' \ar[r] \ar[d] 
            & C'' \ar[r, squiggly] \ar[d, squiggly] 
            & A''{[1]} \ar[d, "w{[1]}", squiggly] \\
            A{[1]} \ar[r, "i{[1]}"]
            & B{[1]}\ar[r, "j{[1]}"]
            & C{[1]} \ar[r, "k{[1]}", squiggly]
            & A{[2]}\,, \\
        \end{tikzcd}
    \end{center}
    in which all the rows and columns are exact
    and all the squares commute,
    except the squigglyd box on the bottom right,
    which is anticommutative.
    \begin{proof}
        Thanks to axiom~\hyperref[TR1]{(\textbf{TR1})(iii)},
        we may embed every map of the box in some distinguished
        triangle, thus we get a commutative diagram
        \begin{center}
        \begin{tikzcd}
            A \ar[r,"i"] \ar[d, "u"]
            & B \ar[r, "j"] \ar[d] 
            & C \ar[r, "k"]
            & A{[1]} \ar[d, "u{[1]}"]\\
            A' \ar[r] \ar[d, "v"]
            & B' \ar[r] \ar[d] 
            & C' \ar[r] 
            & A'{[1]} \ar[d, "v{[1]}"] \\
            A'' \ar[d, "w"]
            & B'' \ar[d] 
            & 
            & A''{[1]} \ar[d, "w{[1]}", squiggly] \\
            A{[1]} \ar[r, "i{[1]}"]
            & B{[1]}\ar[r, "j{[1]}"]
            & C{[1]} \ar[r, "k{[1]}", squiggly]
            & A{[2]}\,; \\
        \end{tikzcd}
        \end{center}
        since there exists $A'' \to B''$ which completes 
        the left rectangle to
        a morphism of triangles, by appying 
        axiom~\hyperref[TR1]{(\textbf{TR1})(iii)}
        once again we obtain
        \begin{equation}\label{verdier-diag}
        \begin{tikzcd}
            A \ar[r,"i"] \ar[d, "u"]
            & B \ar[r, "j"] \ar[d] 
            & C \ar[r, "k"]
            & A{[1]} \ar[d, "u{[1]}"]\\
            A' \ar[r] \ar[d, "v"]
            & B' \ar[r] \ar[d] 
            & C' \ar[r] 
            & A'{[1]} \ar[d, "v{[1]}"] \\
            A'' \ar[r] \ar[d, "w"]
            & B'' \ar[d] \ar[r]
            & C'' \ar[r]
            & A''{[1]} \ar[d, "w{[1]}", squiggly] \\
            A{[1]} \ar[r, "i{[1]}"]
            & B{[1]}\ar[r, "j{[1]}"]
            & C{[1]} \ar[r, "k{[1]}", squiggly]
            & A{[2]}\,. \\
        \end{tikzcd}
        \end{equation}
        As a first intuition, one would be tempted
        to complete the third column by exploiting
        \hyperref[TR3]{(\textbf{TR3})}: even though
        we would end up with a commutative diagram,
        the problem is that the column obtained this way
        needs not be a distinguished triangle.
        Thus, the key is to build a triangle
        on $(C'',C,C')$ thanks to the first part
        of the \hyperref[TR4]{octahedral axiom~(\textbf{TR4})}.

        First, embed the composition $A \to B'$ into 
        a triangle on $(A,B',D)$, for some $D \in \Dd$,
        to get
        \begin{center}
        \begin{tikzcd}
            A \ar[r,"i"]  \ar[dr] %\ar[d, "u"]
            & B \ar[r, "j"] \ar[d] 
            & C \ar[r, "k"] \ar[dd, dashed, "\exists"']
            & A{[1]} \\
            %A' \ar[r] \ar[d, "v"]
            & B' \ar[dr] \ar[d] 
            & 
            & \\
            %A'' \ar[d, "w"]
            & B'' \ar[d] \ar[dr, equals]
            & D \ar[dr] \ar[d, dashed, "\exists"']
            &\\
            %A{[1]} \ar[r, "i{[1]}"]
            & B{[1]}
            & B'' \ar[d, dashed, "\exists"']
            & A{[1]} \\
            & & C{[1]} & \,,
        \end{tikzcd}
        \end{center}
    thus \hyperref[TR4]{(\textbf{TR4})} ensures
    there is a distinguished triangle on $(C,D,B'')$.
    From the diagram
    \begin{center}
        \begin{tikzcd}
            A \ar[d, "u"] \ar[dr]
            & 
            & 
            & 
            & \\
            A' \ar[r] \ar[d, "v"]
            & B' \ar[r] \ar[dr] 
            & C' \ar[r] \ar[dr, equals]
            & A'{[1]} 
            & \\
            A'' \ar[d, "w"]  \ar[rr, dashed, "\exists"]
            & 
            & D \ar[dr]  \ar[r, dashed, "\exists"]
            & C' \ar[r, dashed, "\exists"]
            & A''{[1]}\\
            A{[1]} 
            & 
            & 
            & A{[1]}
            & \, \\
        \end{tikzcd}
        \end{center}
        we obtain a distinguished triangle on $(A'',D,C')$.
        Finally, we see from
        \begin{center}
        \begin{tikzcd}
            A'' \ar[r]  \ar[dr] %\ar[d, "u"]
            & D \ar[r] \ar[d] 
            & C' \ar[r] \ar[dd, "\exists"', dashed]
            & A{[1]} \\
            %A' \ar[r] \ar[d, "v"]
            & B'' \ar[dr] \ar[d] 
            & 
            & \\
            %A'' \ar[d, "w"]
            & C{[1]} \ar[d] \ar[dr, equals]
            & C'' \ar[dr] \ar[d, dashed, "\exists"']
            &\\
            %A{[1]} \ar[r, "i{[1]}"]
            & D{[1]}
            & C{[1]} \ar[d, dashed, "\exists"']
            & A{[1]} \\
            & & C'{[1]} & \,,
        \end{tikzcd}
        \end{center}
        we find a distinguished triangle on $(C',C'',C)$,
        and by rotating it, we get the desired triangle
        on $(C,C',C'')$ which fits in the diagram~\eqref{verdier-diag}
        by construction.
    \end{proof}
\end{ex}

\begin{prop}\label{adj-exact}
    Let $F:\Dd \to \Dd'$ be an exact functor 
    between triangulated categories.
    If $F \dashv H$, then $H: \Dd' \to \Dd$ is exact.
    Similarly, if $G \dashv F$, then $G: \Dd' \to \Dd$.
    \begin{proof}
        Let $T$, respectively $T'$, denote the shift functor
        of $\Dd$, resp. $\Dd'$. We first show that $H$
        satisfies \hyperref[EF1]{(\textbf{EF1})}: 
        by exactness of $F$, there is a natural isomorphism
        $FT \simeq T'F$, which yields $(T')^{-1}F \simeq FT^{-1}$.
        Hence, for every $A, B \in \Dd$
        it holds
        \begin{align*}
            \Hom_{\Dd}(A,HT'(B))
            &\simeq \Hom_{\Dd'}(F(A),T'(B)) \\
            &\simeq \Hom_{\Dd'}((T')^{-1}F(A),B) \\
            &\simeq \Hom_{\Dd'}(FT^{-1}(A),B) \\
            &\simeq \Hom_{\Dd}(T^{-1}(A),H(B)) \\
            &\simeq \Hom_{\Dd}(A,TH(B))\,,
        \end{align*}
        thus, the \hyperref[yoneda]{Yoneda's Lemma}
        yields an isomorphism $HT' \simeq TH$.

        Consider now a distinguished triangle
        \begin{center}
            \begin{tikzcd}
                A \ar[r]
                & B \ar[r]
                & C \ar[r]
                & A{[1]}
            \end{tikzcd}
        \end{center}
        in $\Dd'$; we need to show that $H$ maps it
        in a distinguished triangle in $\Dd$.
        By (\textbf{EF1}) we know that $H(A[1]) \simeq H(A)[1]$,
        hence by the axiom \hyperref[TR1]{(\textbf{TR1})(iii)} we can
        complete $H(A) \to H(B)$ to a distinguished triangle
        \begin{center}
            \begin{tikzcd}
                H(A) \ar[r]
                & H(B) \ar[r]
                & C_{0} \ar[r]
                & H(A){[1]}\,.
            \end{tikzcd}
        \end{center}
        Usign the counit $\epsilon : FH \to \cat{1}_{\Dd'}$
        and the assumption that $F$ is exact, 
        we get the following commutative diagram,
        whose rows are distinguished triangles:
        \begin{center}
            \begin{tikzcd}
                FH(A) \ar[r] \ar[d, "\epsilon_{A}"]
                & FH(B) \ar[r] \ar[d, "\epsilon_{B}"]
                & F(C_{0}) \ar[d, "h", dashed] \ar[r]
                & FH(A){[1]} \ar[d, "\epsilon_{A}{[1]}"] \\
                A \ar[r]
                & B \ar[r]
                & C \ar[r]
                & A{[1]}\,,
            \end{tikzcd}
        \end{center}
        thus, by \hyperref[TR3]{(\textbf{TR3})} it can be completed
        to a morphism $(\epsilon_{A}, \epsilon_{B}, h)$ of distinguished triangles.
        If we apply $H$ to this diagram, we can ``attach''
        the first triangle to it by using the unit
        $\eta : \cat{1}_{\Dd} \to HF$ in the following way:
        \begin{center}
            \begin{tikzcd}[column sep=large]
                H(A) \ar[r] \ar[d, "\eta_{H(A)}"] \ar[dd, equals, bend right=75]
                & H(B) \ar[r] \ar[d, "\eta_{H(B)}"]
                & C_{0} \ar[r] \ar[d, "\eta_{C_{0}}"] 
                & H(A){[1]} \ar[d] \ar[dd, equals, bend left=75]\\ %, "\eta_{H(A)}{[1]}"] \\
                HFH(A) \ar[r] \ar[d, "\epsilon_{A}"]
                & HFH(B) \ar[r] \ar[d, "\epsilon_{B}"]
                & HF(C_{0}) \ar[d, "H(h)"] \ar[r]
                & HFH(A){[1]} \ar[d] \\ %, "\epsilon_{A}{[1]}"] \\
                H(A) \ar[r]
                & H(B) \ar[r]  \ar[from=uu, equals, bend right=75, crossing over]
                & H(C) \ar[r]
                & H(A){[1]}\,,
            \end{tikzcd}
        \end{center}
        where the compositions $\epsilon \circ \eta_{H(-)}$ are the identities
        by \hyperref[unit-counit-identity]{Exercise~\ref*{unit-counit-identity}}.
        Unfortunately, we cannot conclude by applying 
        the \hyperref[5lemma]{\textbf{5-lemma}}
        because the bottom triangle is not distinguished. Nevertheless,
        remember that the functor $\Hom_{\Dd}(X,-)$ is cohomological,
        for any $X \in \Dd$; thus, by using the adjunction
        we obtain the diagram of abelian groups
        \begin{center}
            \begin{tikzcd}[column sep=small]
                \Hom_{\Dd'}(F(X),A) \ar[r] \ar[d, equals]
                & \Hom_{\Dd'}(F(X),B) \ar[r] \ar[d, equals]
                & \Hom_{\Dd}(X,C_{0}) \ar[d] \ar[r]
                & \Hom_{\Dd'}(F(X),A{[1]}) \ar[d,equals] \\
                \Hom_{\Dd'}(F(X),A) \ar[r] 
                & \Hom_{\Dd'}(F(X),B) \ar[r]
                & \Hom_{\Dd}(X,H(C))  \ar[r]
                & \Hom_{\Dd'}(F(X),A{[1]}) \,,
            \end{tikzcd}
        \end{center}
        so by applying the classical 5 lemma for abelian groups
        we deduce that $\Hom_{\Dd'}(X,C_{0}) \simeq \Hom_{\Dd'}(X,H(C))$,
        thus from the \hyperref[yoneda]{Yoneda's Lemma} it follows
        $C_{0}\simeq H(C)$. Finally, by \hyperref[TR1]{(\textbf{TR1})(ii)}
        we see that $H(A) \to H(B) \to H(C) \to H(A){[1]}$ is distinguished,
        so we conclude that $H$ satisfies \hyperref[EF2]{(\textbf{EF2})}.
    \end{proof}
\end{prop}

\begin{df}
    Given a triangulated category $\Dd$, a subcategory $\Dd' \subset \Dd$
    is a \textbf{triangulated subcategory} if $\Dd'$ admits
    a structure of triangulated category such that the inclusion is exact.
\end{df}

\begin{rmk}
    \begin{rmnumerate}
        \item Notice in particular that the family of distinguished triangles
        of $\Dd'$ must be included in the distinguished triangles of $\Dd$.

        \item If $\Dd' \subset \Dd$ is full, then it is a triangulated subcategory
        if and only if $\Dd'$ is invariant under the shift functor and,
        for every distinguished triangle $A \to B \to C \to A{[1]}$ in $\Dd$,
        with $A,B \in \Dd'$, then $C$ is isomorphic to some object in $\Dd'$:
        indeed, $A \to B$ is a morphism in $\Dd'$, thus it can be completed
        to a distinguished triangle $A \to B \to C' \to A{[1]}$ in $\Dd'$,
        hence  we have $C \simeq C'$ by the \hyperref[5lemma]{\textbf{5-lemma}}.
    \end{rmnumerate}
\end{rmk}

\begin{df}
    A triangulated subcategory $\Dd' \subset \Dd$ is called \textbf{admissible}
    if the inclusion has a right adjoint $\pi: \Dd \to \Dd'$, that we will call
    \textbf{orthogonal projection}. 
    The (\textbf{right}) \textbf{orthogonal complement} of $\Dd'$
    is the full subcategory $\Dd'^{\perp} \subset \Dd$ of those objects $C \in \Dd$
    such that
    \begin{equation*}
        \Hom_{\Dd}(A,C) = 0\,, \quad \text{for all } A \in \Dd'\,.
    \end{equation*}
    Similarly, one can define ${}^{\perp}\Dd'$ as the full subcategory
    of those objects $C \in \Dd$ such that
    \begin{equation*}
        \Hom_{\Dd}(C,A) = 0\,, \quad \text{for all } A \in \Dd'\,.
    \end{equation*}
\end{df}

Note that the orthogonal projection $\pi$ is an exact functor 
by \hyperref[adj-exact]{Proposition~\ref*{adj-exact}}.


\begin{rmk}
    The orthogonal complement of a triangulated subcategory $\Dd' \subset \Dd$
    naturally inherits a triangulated structure from $\Dd$: 
    indeed, we see that $T\vert_{\Dd'^{\perp}}$ has values in $\Dd'^{\perp}$
    because, for every $A \in \Dd'$ and $n \in \Z$, it holds
    \begin{equation*}
        \Hom_{\Dd}(A,C[n]) \simeq \Hom_{\Dd}(A[-n],C) = 0\,;
    \end{equation*}
    moreover, given an exact triangle
    \begin{equation*}
        C \longrightarrow C' \longrightarrow B \longrightarrow C[1]
    \end{equation*}
    in $\Dd$, where $C,C' \in \Dd'^{\perp}$, then by applying $\Hom_{\Dd}(A,-)$
    with $A \in \Dd'$ we obtain the exact sequence
    \begin{equation*}
        0 \longrightarrow \Hom_{\Dd}(A,B) \longrightarrow 0\,,
    \end{equation*}
    from which we deduce that $B \in \Dd'^{\perp}$.
\end{rmk}

The concept of orthogonality in triangulated category is inspired 
by the familiar orthogonality on vector spaces that we know from linear algebra.
We know that, is $W \subset V$ is a vector subspace, the the total
space admits a decomposition $V = W \oplus W^{\perp}$.
Similarly, whenever a full admissible subcategory $\Dd' \subset \Dd$ exists,
every object in $\Dd$ fits in a sequence between a component in $\Dd'$
and a component in its orthogonal complement
\begin{lemma}[Semi-orthogonal decomposition]\label{admissible-dec}
    A full triangulated subcategory $\Dd' \subset \Dd$
    is admissible if and only if, for all $B \in \Dd$,
    there exist a distinguished triangle
    \begin{equation*}
        A \longrightarrow B \longrightarrow C \longrightarrow A[1]\,,
    \end{equation*}
    with $A \in \Dd'$ and $C \in \Dd'^{\perp}$.
    \begin{proof}
        Assume $\Dd'$ is admissible: call $\pi$ its orthogonal projection
        and set $A:=\pi(B) \in \Dd'$.
        Using the adjunction, the identity $\cat{1}_{A}$ corresponds
        to a map $A \to B$ in $\Dd$, which can be completed
        to a distinguished triangle
        \begin{equation*}
            A \longrightarrow B \longrightarrow C \longrightarrow A[1]\;
        \end{equation*}
        note that for any $A' \in \Dd'$, there is an isomorphism
        \begin{equation*}
            \Hom_{\Dd}(A',A) 
            = \Hom_{\Dd'}(A',A)
            = \Hom_{\Dd'}(A',\pi(B)) 
            \simeq \Hom_{\Dd}(A',B)\,,
        \end{equation*}
        hence, if we apply the functor $\Hom_{\Dd}(A',-)$ on the triangle,
        then we get $\Hom_{\Dd}(A',C)=0$, so $C \in \Dd'^{\perp}$.

        Conversely, for any $B \in \Dd$, 
        define $\pi(B)$ to be an object in $\Dd'$
        that fits in a distinguished triangle
        \begin{equation*}
            \pi(B) \longrightarrow B \longrightarrow C \longrightarrow \pi(B)[1]\,,
        \end{equation*}
        with $C \in \Dd'^{\perp}$. If $A' \to B \to C' \to A'[1]$
        is another triangle of this form, 
        then for every $X \in \Dd'$ we have
        \begin{align*}
            \Hom_{\Dd'}(X,\pi(B)) 
            = \Hom_{\Dd}(X,\pi(B)) 
            \simeq \Hom_{\Dd}(X,B)
            \simeq \Hom_{\Dd}(X,A')
            = \Hom_{\Dd}(X,A')\,,
        \end{align*}
        so by the \hyperref[yoneda]{Yoneda's Lemma} we deduce
        that $A' \simeq \pi(B)$ in $\Dd'$. This shows that $\pi$
        is defined on objects up to isomorphism.
        Finally, we can define $\pi$ on morphisms too:
        given $f:B \to B'$ in $\Dd$,
        we define $\pi(f)$ to be the image of $f$ via the composition
        \begin{equation*}
            \Hom_{\Dd}(B,B') \longrightarrow
            \Hom_{\Dd}(\pi(B),B') 
            \simeq \Hom_{\Dd}(\pi(B),\pi(B'))
            = \Hom_{\Dd'}(\pi(B),\pi(B'))\,,
        \end{equation*}
        where the isomorphism inbetween follows from the fact that
        $\Hom_{\Dd}(\pi(B),C')=0$, for any $C' \in \Dd'^{\perp}$
        in the decomposition of $B'$.
        One can check that this defines a functor $\pi : \Dd \to \Dd'$
        which is right adjoint to the inclusion $\Dd' \subset \Dd$,
        by construction.
    \end{proof}
\end{lemma}

As it turns out, Serre functors and triangulated structures 
are always compatible. 
In the geometric situation considered later, 
this will be obvious, 
for the Serre functors there will by construction be exact.

\begin{thm}[Bondal, Kapranov]\label{BondalKapranov}
    Any Serre functor on a $k$-linear triangulated category is exact.
    \begin{proof}
        See \parencite[Proposition~1.46]{huybrechts}.
    \end{proof}
\end{thm}
%%    
\section{Equivalences of triangulated categories}

In this section we discuss criteria that allow us to decide 
whether a given exact functor is fully faithful or even an equivalence. 

\begin{df}
    Two triangulated categories $\Dd$ and $\Dd'$ are \textbf{equivalent}
    if there exists an exact equivalence $T : \Dd \to \Dd'$,
    i.e. an equivalence which is an exact functor, whose inverse is exact.
    The set $\Aut(\Dd)$ of isomorphism classes of equivalences $F:\Dd \to \Dd$
    is called \textbf{group of autoequivalences} of $\Dd$.
\end{df}

In many geometric situations, we will encounter the following notion:

\begin{df}
    Let $\Dd$ be a triangulated category.
    A \textbf{spanning class} for $\Dd$ is a collection $\Omega$ 
    of objects in $\Dd$ such that, for any $B \in \Dd$, 
    the following two conditions hold:
    \begin{rmnumerate}
        \item if for every $n \in \Z$ and $A \in \Omega$ we have
        $\Hom_{\Dd}(A,B[n]) = 0$, then $B \simeq \cat{0}$;
        \item if for every $n \in \Z$ and $A \in \Omega$ we have
        $\Hom_{\Dd}(B[n],A) = 0$, then $B \simeq \cat{0}$.
    \end{rmnumerate}
\end{df}

\begin{rmk}
    If $\Dd$ is also $k$-linear, endowed with a Serre functor $S$, 
    then the two conditions in the definition are equivalent:
    assume (i) to be true. 
    If for every $n \in \Z$ and $A \in \Omega$ we have
    \begin{equation*}
        0 = \Hom_{\Dd}(B[n],A) \simeq \Hom_{\Dd}(A,S(B[n]))^*\,,
    \end{equation*}
    since by the \hyperref[BondalKapranov]{Bondal-Kapranov Theorem}
    we have $S(B[n]) \simeq S(B)[n]$, from (i) we deduce that
    $S(B) \simeq \cat{0}$, and hence $B \simeq \cat{0}$ because
    $S$ is additive. Similarly one proves (ii) $\implies$ (i).
\end{rmk}

Spanning classes give a sufficient collections of objects
on which we can check whether an exact functor is fully faithful or not.

\begin{prop}\label{span-ff}
    Let $F : \Dd \to \Dd'$ be an exact functor between triangulated categories
    with both left and right adjoints: $G \dashv F \dashv H$.
    Suppose $\Omega$ is a spanning class such that, for every $A,B \in \Omega$, 
    the natural maps
    \begin{equation*}
        F : \Hom_{\Dd}(A,B[n]) \longrightarrow \Hom_{\Dd'}(F(A),F(B)[n])
    \end{equation*}
    are bijections for every $n \in \Z$. Then $F$ is fully faithful.
    \begin{proof}
        For any pair of objects $A, B \in \Dd$,
        we have the commutative diagram
        \begin{equation}\label{adj-square-prop}
            \begin{tikzcd}
                \Hom_{\Dd}(A,B) 
                \ar[r, "\eta_{B} \circ -"] 
                \ar[d, "- \circ \epsilon_{A}"'] 
                \ar[dr, "F"]
                & \Hom_{\Dd}(A,HF(B)) \ar[d, equals] \\
                \Hom_{\Dd}(GF(A),B) \ar[r, equals]
                & \Hom_{\Dd'}(F(A),F(B))\,.
            \end{tikzcd}
        \end{equation}
        We show that 
        the counit $\epsilon_{A}$ is an isomorphism,
        for $A \in \Omega$:
        complete it to a distinguished triangle
        \begin{center}
            \begin{tikzcd}
                GF(A) \ar[r, "\epsilon_{A}"]
                & A \ar[r]
                & A' \ar[r]
                & GF(A){[1]}\,,
            \end{tikzcd}
        \end{center}
        where we have used that $GF(A[1]) \simeq GF(A)[1]$
        by exactness of both $F$ and $G$ 
        (see \hyperref[adj-exact]{Proposition~\ref*{adj-exact}}).
        Apply $\Hom_{\Dd}(-,B[n])$ to it and get %for an arbitrary $B \in \Dd$
        \begin{center}
            \begin{tikzcd}
                \Hom_{\Dd}(A',B{[n]}) \ar[r]
                & \Hom_{\Dd}(A,B{[n]}) \ar[r, "-\circ \epsilon"] \ar[dr, "F"']
                & \Hom_{\Dd}(GF(A),B{[n]}) \ar[d, equals] \\
                & & \Hom_{\Dd'}(F(A),F(B){[n]})\,.
            \end{tikzcd}
        \end{center}
        If we take $B \in \Omega$, then $F$ is bijective, 
        so $-\circ \epsilon$ is an isomorphism, from which
        we deduce that $\Hom_{\Dd}(A',B[n])=0$;
        since this holds for every $B \in \Omega$ and $n \in \Z$,
        then $A' \simeq \cat{0}$ because $\Omega$ spans $\Dd$,
        so we conclude that $\epsilon_{A}:GF(A) \simeq A$.

        It follows that $- \circ \epsilon_{A}$ is an isomorphism,
        for $A \in \Omega$ and any $B \in \Dd$, and hence all the maps 
        in the diagram~\eqref{adj-square-prop} are isomorphisms.
        Now complete $\eta_{B}$ to a distinguished triangle
        \begin{center}
            \begin{tikzcd}
                B \ar[r, "\eta_{B}"]
                & HF(B) \ar[r]
                & B' \ar[r]
                & B{[1]}\,,
            \end{tikzcd}
        \end{center}
        and apply $\Hom_{\Dd}(A,-) \circ [n]$, 
        for all $n \in \Z$ and $A \in \Omega$
        to see that $\Hom_{\Dd}(A,B') = 0$, and hence $B' \simeq \cat{0}$.
        This means that $\eta_{B}:B \simeq HF(B)$, for any $B \in \Dd$. 
        Thus, the arrows in \eqref{adj-square-prop} are isomorphisms
        for all $A,B \in \Dd$, in particular
        \begin{equation*}
            F:\Hom_{\Dd}(A,B) \xrightarrow[]{\sim} \Hom_{\Dd'}(F(A),F(B))\,.
        \end{equation*}
    \end{proof}
\end{prop}

Suppose we already know that the functor is fully faithful. 
What do we need to know in order to be able to decide whether 
it is in fact an equivalence? 
The following lemma provides a first criterion, 
whose assumption however is difficult to check. 

\begin{lemma}\label{lemma-equiv}
    Let $F:\Dd \to \Dd'$ be a fully faithful exact functor
    between triangulated categories and suppose that $F$
    has a right adjoint $F \dashv H$.
    Then $F$ is an equivalence if and only if,
    for every $C \in \Dd'$, $H(C) \simeq \cat{0}$
    implies $C \simeq \cat{0}$.

    The same holds true if $F$ has a left adjoint $G \dashv F$
    with the above property.
    \begin{proof}
        By \hyperref[ff-adj]{Corollary~\ref*{ff-adj}},
        we know that $\eta_{A}:A \xrightarrow[]{\sim} HF(A)$ is an isomorphism,
        for any $A \in \Dd$. We prove that also $\epsilon_{B}:FH(A) \to B$ is
        an isomorphism, and hence $H$ is a quasi-inverse of $F$.

        Given any $B \in \Dd'$, complete $\epsilon_{B}$
        to a distinguished triangle
        \begin{center}
            \begin{tikzcd}
                FH(B) \ar[r, "\epsilon_{B}"]
                & B \ar[r]
                & B' \ar[r]
                & FH(B){[1]}\,;
            \end{tikzcd}
        \end{center}
        since $H$ is exact by \hyperref[adj-exact]{Proposition~\ref*{adj-exact}},
        it gives a distinguished triangle
        \begin{center}
            \begin{tikzcd}
                HFH(B) \ar[r, "H\epsilon_{B}"]
                & H(B) \ar[r]
                & H(B') \ar[r]
                & HFH(B){[1]}\,.
            \end{tikzcd}
        \end{center}
        From \hyperref[unit-counit-identity]{Exercise~\ref*{unit-counit-identity}}
        we know that $H\epsilon_{B}$ is an isomorphism, hence $H(B') \simeq \cat{0}$.
        By assumption $B' \simeq \cat{0}$ and we get the thesis.
    \end{proof}
\end{lemma}

\begin{df}
    A triangulated category $\Dd$ is \textbf{decomposed into
    triangulated subcategories} $\Dd_{1}, \Dd_{2} \subset \Dd$
    if the following three conditions are satisfied:
    \begin{rmnumerate}
        \item both $\Dd_{1}$ and $\Dd_{2}$ contain objects non-isomorphic
        to $\cat{0}$;

        \item for all $A \in \Dd$, there exists a distinguished triangle
        \begin{equation*}
            A_{1} \longrightarrow A \longrightarrow A_{2} \longrightarrow A_{1}[1]\,,
        \end{equation*}
        with $A_{1} \in \Dd_{1}$ and $A_{2} \in \Dd_{2}$;

        \item the two subcategories are ``disjoint'' in the sense that,
        for every $X_{1} \in \Dd_{1}$ and $X_{2} \in \Dd_{2}$, it holds
        \begin{equation*}
            \Hom_{\Dd}(X_{1}, X_{2}) = \Hom_{\Dd}(X_{2}, X_{1}) = 0\,.
        \end{equation*}
    \end{rmnumerate}
    A triangulated category which cannot be decomposed is called
    \textbf{indecomposable}.
\end{df}

\begin{rmk}
    Notice that, in the presence of (iii), the property (ii) states that
    $A$ is a direct sum $A \simeq A_{1} \oplus A_{2}$ 
    (see the \hyperref[split-lemma]{Split Lemma}).
    Thus, condition (ii) is symmetric, despite the chosen order in the statement.
\end{rmk}

\begin{prop}\label{trivial-adj-equiv}
    Let $F:\Dd \to \Dd'$ be a fully faithful exact functor
    between triangulated categories. Suppose that $\Dd$ contains no
    objects isomorphic to $\cat{0}$ and that $\Dd'$ is indecomposable.
    Then $F$ is an equivalence if and only if it admits both right 
    and left adjoints $G \dashv F \dashv H$ such that, for any $B \in \Dd$,
    $H(B) \simeq \cat{0}$ implies $G(B) \simeq \cat{0}$.
    \begin{proof}
        The ``only if part is clear'' because $G \simeq H \simeq F^{-1}$.
        So we need to prove the ``if'' implication: our goal is to show that
        $H$ is a quasi-inverse of $F$.

        First, we introduce two full triangulated subcategories
        $\Dd'_{1}, \Dd'_{2} \subset \Dd'$ defined as follows:
        let $\Dd'_{1}$ the full subcategory of objects $B \in \Dd'$
        such that $B \simeq F(A)$, for some $A \in \Dd$; 
        let $\Dd'_{2}$ be the full subcategory consisting of
        all objects $C \in \Dd'$ such that $H(C) \simeq \cat{0}$.
        One can easily check that both $\Dd'_{1}$ and $\Dd'_{2}$
        are triangulated subcategories of $\Dd'$.

        The proof of \hyperref[lemma-equiv]{Lemma~\ref*{lemma-equiv}}
        shows that, for every $B \in \Dd'_{1}$, the counit $FH(B) \simeq B$
        is an isomorphism and also that every $B \in \Dd'$
        sits in a distinguished triangle of the form
        \begin{equation*}
            B_{1} \longrightarrow
            B \longrightarrow
            B_{2} \longrightarrow
            B_{1}{[1]}\,,
        \end{equation*}
        with $B_{1} \in \Dd'_{1}$ and $B_{2} \in \Dd'_{2}$.
        Since by assumption $H(B_{2}) \simeq \cat{0}$
        implies $G(B_{2}) \simeq \cat{0}$,
        then for all $B_{1} \in \Dd'_{1}$ and $B_{2} \in \Dd'_{2}$
        it holds
        \begin{align*}
            \Hom_{\Dd'}(B_{1}, B_{2})
            \simeq \Hom_{\Dd'}(FH(B_{1}),B_{2}) 
            \simeq \Hom_{\Dd}(H(B_{1}),H(B_{2})) 
            \simeq \Hom_{\Dd}(H(B_{1}),\cat{0}) = 0\,, \\
            \Hom_{\Dd'}(B_{2}, B_{1})
            \simeq \Hom_{\Dd'}(B_{2}, FH(B_{1})) 
            \simeq \Hom_{\Dd}(G(B_{2}),H(B_{1})) 
            \simeq \Hom_{\Dd}(\cat{0}, H(B_{1}))= 0\,,
        \end{align*}
        so $\Dd'_{1}$ and $\Dd'_{2}$ decompose $\Dd'$.
        As $\Dd'$ is indecomposable, either $\Dd'_{1}$
        or $\Dd'_{2}$ is trivial, i.e. contains only objects
        isomorphic to $\cat{0}$.

        Suppose $\Dd'_{1}$ is trivial. Then for every $A \in \Dd$,
        the image $F(A)$ is trivial, and hence
        $A \simeq HF(A) \simeq \cat{0}$ for $F$ is fully faithful:
        this contradicts the non-triviality of $\Dd$.
        Thus $\Dd'_{2}$ must be trivial, which implies
        that $\Dd'_{1} \subset \Dd'$ is an equivalence, i.e.
        for every $B \in \Dd'$, it holds $FH(B) \simeq B$,
        and $H$ is a quasi-inverse of $F$.
    \end{proof}
\end{prop}

The following proposition gives a criterion for equivalences
of triangulated endowed with Serre functors.

\begin{cor}
    Let $\Dd$ and $\Dd'$ be triangulated categories,
    endowed with Serre functors $S_{\Dd}$, resp. $S_{\Dd'}$. 
    Let $F:\Dd \to \Dd'$ be an exact functor
    which admits both right 
    and left adjoints $G \dashv F \dashv H$. 
    Assume there is $\Omega$
    a spanning class of $\Dd$ satisfying the following three properties:
    \begin{rmnumerate}
        \item for all $A,B \in \Omega$, the natural morphisms
        \begin{equation*}
            \Hom_{\Dd}(A,B[n]) \longrightarrow \Hom_{\Dd'}(F(A),F(B)[i])
        \end{equation*}
        are bijective for all $n \in \Z$;
        
        \item Serre functors commute with $F$ on the spanning class,
        that is
        \begin{equation*}
            F \circ S_{\Dd}(A) \simeq S_{\Dd'} \circ F(A)\,, 
            \quad A \in \Omega \,;
        \end{equation*}

        \item the category $\Dd'$ is indecomposable and $\Dd$
        is non-trivial.
    \end{rmnumerate}
    Then $F$ is an equivalence.
    \begin{proof}
        Condition (i) ensures that $F$ is fully faithful by
        \hyperref[span-ff]{Proposition~\ref*{span-ff}}.
        Now we want to prove $F$ is an equivalence by verifying
        the conditions in \hyperref[trivial-adj-equiv]{Proposition~\ref*{trivial-adj-equiv}}.

        Suppose $H(B) \simeq \cat{0}$, for $B \in \Dd'$.
        For every $A \in \Omega$, using adjunction and condition (ii),
        one finds
        \begin{align*}
            0 = \Hom_{\Dd}(A,H(B))
            &\simeq \Hom_{\Dd'}(F(A),B)
            \simeq \Hom_{\Dd'}(B,S_{\Dd'}F(A)))^* \\
            & \simeq \Hom_{\Dd'}(B,FS_{\Dd}(A)))^*
            \simeq \Hom_{\Dd}(G(B),S_{\Dd}(A)))^* \\
            & \simeq \Hom_{\Dd}(A,G(B))\,,
        \end{align*}
        thus $G(B) \simeq \cat{0}$ because $\Omega$ spans $\Dd$;
        more precisely, this argument shows that $G \simeq H$
        by the \hyperref[yoneda]{Yoneda's Lemma}.
    \end{proof}
\end{cor}

Recall that in a $k$-linear category endowed with a
Serre functor, the existence of an adjoint functor 
implies the existence of the other one; hence,
the previous Corollary may be stated assuming the existence
of $H$ only.
%%    
\section{Exceptional sequences and orthogonal decompositions}

In the geometric context, the derived categories 
in question will usually be indecomposable. 
However, there are geometrically relevant situations 
where one can decompose the derived category in a weaker sense. 
This leads to the abstract notion of 
semi-orthogonal decompositions 
of a triangulated category, 
the topic of this section. 
Any full exceptional sequence yields such 
a semi-orthogonal decomposition, 
so we will discuss this notion first.

\begin{df}
    Let $\Dd$ be $k$-linear triangulated category. 
    An object $E \in \Dd$ is \textbf{exceptional} if
    \begin{equation*}
        \Hom_{\Dd}(E,E[n]) =
        \begin{cases}
            k\,, \quad &\text{if } n=0\,;\\
            0\,, \quad &\text{if } n \ne 0\,.
        \end{cases}
    \end{equation*}
    An \textbf{exceptional sequence} is a sequence
    $E_{1}, E_{2}, \dots, E_{m}$ of exceptional objects
    such that for all $i,j$ it holds
    \begin{equation*}
        \Hom_{\Dd}(E_{i},E_{j}[n]) =
        \begin{cases}
            k\,, \quad &\text{if } i=j, n=0\,;\\
            0\,, \quad &\text{if } i>j \text{ or if } i = j, n\ne 0\,.
        \end{cases}
    \end{equation*}
    An exceptional sequence is \textbf{full} if $\Dd$
    is generated by $\Set{E_{i}}$, 
    i.e. any full triangulated subcategory of $\Dd$
    containing all objects $E_{i}$ is equivalent to $\Dd$
    via the inclusion.
\end{df}

\begin{lemma}
    Let $\Dd$ be a $k$-linear triangulated category such that,
    for any $A,B \in \Dd$ the vector space 
    $\bigoplus_{i} \Hom_{\Dd}(A,B[i])$ is finite-dimensional.
    If $E \in \Dd$ is exceptional, 
    then the objects $\bigoplus_{i}E[i]^{\oplus j_{i}}$
    form an admissible subcategory $\langle E \rangle$
    of $\Dd$.
    \begin{proof}
        It is easy to check that $\langle E \rangle$ inherits
        the structure of a triangulated category:
        direct sums behave well in forming distinguished triangles;
        moreover, non-trivial endomorphisms of the exceptional
        object are automorphisms.
        In order to see that it is admissible, 
        we use the tensor product $\otimes_{\Dd}$:
        for every $n \in \Z$, consider the morphism
        induced by shifting $(-n)$ times
        \begin{equation*}
           [-n] : \Hom_{\Dd}(E,A[n]) \xrightarrow[]{\sim} \Hom_{\Dd}(E[-n],A)\,;
        \end{equation*}
        by adjunction, it gives an evaluation morphism
        \begin{equation*}
            \cat{ev}_{n} : \Hom_{\Dd}(E,A[n]) \otimes_{\Dd} E[-n] \longrightarrow A\,,
        \end{equation*}
        for each $n \in \Z$; take the direct sum of these morphisms
        and complete it to a distinguished triangle
        \begin{equation*}
            \bigoplus_{n}\big(\Hom_{\Dd}(E,A[n]) \otimes_{\Dd} E[-n] \big)
            \longrightarrow A \longrightarrow B\,.
        \end{equation*}
        For every $n \in \Z$, let $d_{n}:=\dim \Hom_{\Dd}(E,A[n])$ and notice
        it is finite by assumption; then it holds
        \begin{equation*}
            \Hom_{\Dd}(E,A[n]) \otimes_{\Dd} E[-n] 
            \simeq k^{d_{n}} \otimes_{\Dd} E[-n]
            \simeq \bigoplus_{m=1}^{d_{n}} E[-n]\,,
        \end{equation*}
        hence, if we apply $\Hom_{\Dd}(E[-i],-)$ to the above triangle,
        by exceptionality of $E$ we get
        \begin{align*}
            \Hom_{\Dd}\left(E[-i], \bigoplus_{n}\big(\Hom_{\Dd}(E,A[n]) \otimes_{\Dd} E[-n] \big) \right)
            &\simeq \bigoplus_{n} \left( \Hom_{\Dd}(E[-i], E[-n]) \right)^{d_{n}} \\ %\bigoplus_{m=1}^{d_{n}} \Hom_{\Dd}(E[-i],E[-n]) \\
            &\simeq \left( \Hom_{\Dd}(E[-i],E[-i]) \right)^{d_{i}} \\
            &\simeq k^{d_{i}} \simeq \Hom_{\Dd}(E,A[i]) \\
            &\simeq \Hom_{\Dd}(E[-i],A)\,,
        \end{align*}
        thus, $\Hom_{\Dd}(E,B[i]) \simeq \Hom_{\Dd}(E[-i],B) \simeq 0$.
        Since this holds for every $i \in \Z$, 
        then one has $B \in \langle E \rangle^{\perp}$,
        so we conclude by \hyperref[admissible-dec]{Lemma~\ref*{admissible-dec}}.
    \end{proof}
\end{lemma}

Now we generalize the concept of exceptional sequence:

\begin{df}
    A sequence of full admissible triangulated subcategories
    \begin{equation*}
        \Dd_{1}, \Dd_{2}, \dots, \Dd_{m} \subset \Dd
    \end{equation*}
    is \textbf{semi-orthogonal} if for all $i < j$ 
    it holds $\Dd_{i} \subset \Dd_{j}^{\perp}$.

    We say that $\Dd_{1}, \dots, \Dd_{m}$ is a 
    \textbf{semi-orthogonal decomposition} of $\Dd$
    if $\Dd$ is generated by the $\Dd_{i}$, i.e.
    the smallest full triangulated subcategory containing
    all the $\Dd_{i}$ is equivalent to $\Dd$ via the inclusion.
\end{df}

\begin{ex}
    Let $\Dd' \subset \Dd$ be an admissible full triangulated
    subcategory. Then
    \begin{equation*}
        \Dd_{1} := \Dd'^{\perp}\,, \quad \Dd_{2} := \Dd'\,,
    \end{equation*}
    defines a semi-orthogonal decomposition of $\Dd$.
\end{ex}

\begin{ex}
    Let $E_{1}, \dots, E_{m}$ be an exceptional sequence in $\Dd$.
    Then the admissible trangulated subcategories generated by these objects
    \begin{equation*}
        \Dd_{1} := \langle E_{1} \rangle\,, \quad \dots \quad
        \Dd_{m} := \langle E_{m} \rangle\,,
    \end{equation*}
    form a semi-orthogonal sequence.
    If the sequence is \emph{full}, then $\Dd_{1}, \dots, \Dd_{m}$
    is a semi-orthogonal decomposition of $\Dd$.
\end{ex}

\begin{lemma}
    Any semi-orthogonal sequence of full admissible triangulated subcategories
    $\Dd_{1}, \dots, \Dd_{m} \subset \Dd$ defines a semi-orthogonal
    decomposition for $\Dd$ if and only if any object $A \in \Dd$ such that
    $A \in \Dd_{i}$, for all $i=1,2, \dots, m$, 
    is then trivial, i.e. $A \simeq \cat{0}$.
    \begin{proof}
        Suppose $\Dd_{1}, \dots, \Dd_{m}$ is a semi-orthogonal 
        decomposition of $\Dd$. 
        If $A_{0} \in \bigcap \Dd_{i}^{\perp}$, 
        then each $\Dd_{i} \subset {}^{\perp}A_{0}$;
        by assumption ${}^{\perp}A_{0} = \Dd$ and in particular
        $A_{0} \in {}^{\perp}A_{0}$, thus $\Hom_{\Dd}(A_{0},A_{0}) = 0$.
        This means $A_{0} \simeq \cat{0}$, indeed $\cat{1}_{A_{0}} = 0$.

        Conversely, assume $\bigcap \Dd_{i}^{\perp} = \Set{\cat{0}}$.
        For simplicity, we consider the case $m=2$: given $A_{0} \in \Dd$,
        we want to show $A_{0}$ is in the triangulated subcategory
        generated by $\Dd_{1}$ and $\Dd_{2}$.
        As $\Dd_{2}$ is admissible, 
        by \hyperref[admissible-dec]{Lemma~\ref*{admissible-dec}}
        there is a distinguished triangle
        \begin{center}
            \begin{tikzcd}
                A \ar[r]
                & A_{0} \ar[r]
                & A' \ar[r]
                & A{[1]}\,,
            \end{tikzcd}
        \end{center}
        with $A \in \Dd_{2}$ and $A' \in \Dd_{2}^{\perp}$.
        Now, using that $\Dd_{1}$ is admissible, 
        we can decompose $A'$ as
        \begin{center}
            \begin{tikzcd}
                B \ar[r]
                & A' \ar[r]
                & B' \ar[r]
                & B{[1]}\,,
            \end{tikzcd}
        \end{center}
        with $B \in \Dd_{1}$ and $B' \in \Dd_{1}^{\perp}$.
        As the sequence is semi-orthogonal we have
        $B \in \Dd_{1} \subset \Dd_{2}^{\perp}$,
        and since $A' \in \Dd_{2}^{\perp}$, 
        we deduce that $B' \in \Dd_{2}^{\perp}$
        because it is a full triangulated subcategory.
        So $B' \in \Dd_{1}^{\perp} \cap \Dd_{2}^{\perp}$
        implies $B' \simeq \cat{0}$, from which we deduce
        that $B \simeq A'$. 
        Then $A_{0}$ has a semi-orthogonal decomposition
        with $A \in \Dd_{2}$ and $B \in \Dd_{1}$.
        (The general case follows by applying inductively this argument).
    \end{proof}
\end{lemma}

\missingfigure{Do exercises on Exceptional sequences.}
%
%
%%%%%%%%%%%%%%%%%%%%%%%%%%%%%%%%%%%%%%%%%%%%%%%%%%%%%%%%%%%%
%%%%%%%%%%%%%%%%%%%%%%%%%%%%%%%%%%%%%%%%%%%%%%%%%%%%%%%%%%%%
%%%%%%%%%%%%%%%%%%%%%%%%%%%%%%%%%%%%%%%%%%%%%%%%%%%%%%%%%%%%
%
%\chapter{Derived categories}
%
%%	
\section{The derived category of an abelian category}

We begin by stating the existence of the derived category 
as a theorem, and explain the technical features, 
necessary for any calculation, later on.
In the sequel, we will mostly be interested in 
the derived category of the abelian category of 
(coherent) sheaves or of modules over a ring. 

Let $\Aa$ be an abelian category.
Recall that in the chapter 
of \hyperref[AbelianCategories]{Abelian Categories},
we have defined the category $C^{\bullet}(\Aa)$
of cochain complexes and cohomology functors $H^{n}$
over it.

\begin{df}
    A morphism of complexes
    $f^{\bullet} : A^{\bullet} \to B^{\bullet}$
    is a \textbf{quasi-isomorphism} (shortened \textbf{qis})
    if, for all $n \in \NN$, the induced map
    \begin{equation*}
        H^{n}(f^{\bullet}) : H^{n}(A^{\bullet}) \xrightarrow[]{\sim} H^{n}(B^{\bullet})
    \end{equation*}
    is an isomorphism.
\end{df}

The central idea for the definition of the derived category is this: 
quasi-isomorphic complexes should become isomorphic objects 
in the derived category. 
We shall begin our discussion with the following existence theorem.

\begin{thmdef}
    Given an abelian category $\Aa$,
    there exists a category $\cat{D}(\Aa)$,
    called the \textbf{derived category} of $\Aa$,
    and a functor
    \begin{equation*}
        Q : C^{\bullet}(\Aa) \longrightarrow \cat{D}(\Aa)
    \end{equation*}
    that satisfy the following two conditions:
    \begin{rmnumerate}
        \item if $f^{\bullet}$ is a qis, 
        then $Q(f^{\bullet})$ is an isomorphism;

        \item \textbf{universal property}:
        if a functor $F:C^{\bullet}(\Aa) \to \Dd$
        satisfies \emph{(i)}, then it factorizes uniquely through $Q$,
        i.e. there exists a unique (up to isomorphism) functor 
        $G:\cat{D}(\Aa) \to \Dd$ such that $F \simeq GQ$:
        \begin{equation*}
            \begin{tikzcd}
                C^{\bullet}(\Aa) \ar[rr, "Q"] \ar[dr, "F"']
                && \cat{D}(\Aa) \ar[dl, "G", "\exists !"',dashed]\\
                &\Dd & \,.
            \end{tikzcd}
        \end{equation*}
    \end{rmnumerate}
    \begin{proof}[Construction]       
In order to be able to work with the derived category, 
we have to understand which objects become isomorphic under 
$Q : C^{\bullet}(A) \to \cat{D}(A)$ and, more complicated, 
how to represent morphisms in the derived category. 
Explaining this, will at the same time 
provide a proof the theorem. 
Recall that $\cat{K}(\Aa)$ denotes the homotopy category of complexes.

        First, we set the objects of the derived category to be
        cochain complexes, so
            \begin{equation*}
                \textrm{Obj}(\cat{D}(\Aa)) := \textrm{Obj}(\cat{K}(\Aa)) 
                = \textrm{Obj}(C^{\bullet}(\Aa))\,.
            \end{equation*}

        Now we describe how morphisms should behave.
        Since the derived category is built in such
        a way that quasi-isomorphisms become isomorphisms,
        if $C^{\bullet} \to A^{\bullet}$ is an isomorphism,
        then any morphism of complexes $C^{\bullet} \to B^{\bullet}$
        will have to count as a morphism $A^{\bullet} \to B^{\bullet}$
        in $\cat{D}(\Aa)$. Thus, given two complexes $A^{\bullet}, B^{\bullet}$,
        a representative of a morphism 
        $A^{\bullet} \to B^{\bullet}$ in $\cat{D}(\Aa)$
        is given by a diagram
        \begin{equation*}
            \begin{tikzcd}
                & C^{\bullet} \ar[dl, "\textrm{qis}"] \ar[dr] & \\
                A^{\bullet} & & B^{\bullet}\,,
            \end{tikzcd}
        \end{equation*}
        where $C^{\bullet} \to A^{\bullet}$ is a quasi-isomorphism;
        we will call such a diagram a \textbf{roof} 
        over $A^{\bullet}$ and $B^{\bullet}$.
        Two roofs over $A^{\bullet}$ and $B^{\bullet}$ are
        said to be \textbf{equivalent} if they are 
        dominated by a third roof in $\cat{K}(\Aa)$, 
        i.e. there is a diagram
        \begin{equation*}
            \begin{tikzcd}
                & & C^{\bullet} \arrow[ld, "\textrm{qis}"'] \arrow[rd] 
                \arrow[lldd, "\textrm{qis}"', dashed, bend right=50, shift right] 
                & & \\
                & C_1^{\bullet} \arrow[ld, "\textrm{qis}"'] \arrow[rrrd] 
                & & C_2^{\bullet} \arrow[llld, "\textrm{qis}", crossing over] \arrow[rd] 
                & \\
                A^{\bullet} 
                &  &  &  & B^{\bullet}
            \end{tikzcd}
        \end{equation*}
        which commutes in $\cat{K}(\Aa)$, i.e. compositions are homotopy equivalent.
        This property defines, in fact, an \emph{equivalence relation} on roofs:
        \begin{itemize}
            \item \textbf{reflextivity}: a roof is equivalent to itself because
            we have the diagram
            \begin{equation*}
            \begin{tikzcd}
                & & C^{\bullet} \arrow[ld, equals] \arrow[rd, equals] 
                & & \\
                & C^{\bullet} \arrow[ld, "\textrm{qis}"'] \arrow[rrrd] 
                & & C^{\bullet} \arrow[llld, "\textrm{qis}", crossing over] \arrow[rd] 
                & \\
                A^{\bullet} 
                &  &  &  & B^{\bullet}\,;
            \end{tikzcd}
            \end{equation*}

            \item \textbf{symmetry}: given an equivalence of roofs
            \begin{equation*}
            \begin{tikzcd}
                & & C^{\bullet} \arrow[ld, "\textrm{qis}"'] \arrow[rd] 
                \arrow[lldd, "\textrm{qis}"', dashed, bend right=50, shift right] 
                & & \\
                & C_1^{\bullet} \arrow[ld, "\textrm{qis}"'] \arrow[rrrd] 
                & & C_2^{\bullet} \arrow[llld, "\textrm{qis}", crossing over] \arrow[rd] 
                & \\
                A^{\bullet} 
                &  &  &  & B^{\bullet}\,,
            \end{tikzcd}
            \end{equation*}
            we notice that the map $C^{\bullet} \to C^{\bullet}_{2}$ is, in fact, a qis:
            indeed, by passing to cohomology objects, in $\Aa$ we have
            a commutative diagram
            \begin{equation*}
                \begin{tikzcd}
                    H^*(C^{\bullet}) \ar[rr, "\simeq"] \ar[dr] 
                    & & H^*(A^{\bullet}) \\
                    & H^*(C_{2}^{\bullet}) \ar[ur, "\simeq"] & \,,
                \end{tikzcd}
            \end{equation*}
            implying that $H^*(C^{\bullet}) \simeq H^*(C_{2}^{\bullet})$.
            Thus, the ``equivalence'' diagram is symmetric;

            \item \textbf{transitivity}: given two diagrams
            \begin{equation*}
            \begin{tikzcd}[column sep=small]
                & & C^{\bullet} \arrow[ld, "\textrm{qis}"'] \arrow[rd] 
                & & &
                & & D^{\bullet} \arrow[ld, "\textrm{qis}"'] \arrow[rd] 
                & &\\
                & C_1^{\bullet} \arrow[ld, "\textrm{qis}"'] \arrow[rrrd] 
                & & C_2^{\bullet} \arrow[llld, "\textrm{qis}", crossing over] \arrow[rd] 
                & &
                & C_2^{\bullet} \arrow[ld, "\textrm{qis}"'] \arrow[rrrd] 
                & & C_3^{\bullet} \arrow[llld, "\textrm{qis}", crossing over] \arrow[rd] 
                &\\
                A^{\bullet} 
                &  &  &  & B^{\bullet} \,, &
                A^{\bullet} 
                &  &  &  & B^{\bullet}\,,
            \end{tikzcd}
            \end{equation*}
            by \hyperref[roof-comp]{Proposition~\ref*{roof-comp}}
            we can build a diagram
            \begin{equation*}
            \begin{tikzcd}
                && C_{0}^{\bullet} \ar[dl, "\textrm{qis}"'] \ar[dr] && \\
                & C^{\bullet} \ar[dl, "\textrm{qis}"'] \ar[dr]
                && D^{\bullet} \ar[dl, "\textrm{qis}"'] \ar[dr] & \\
                C_{1}^{\bullet} & & C_{2}^{\bullet} & & C_{3}^{\bullet}
            \end{tikzcd}
            \end{equation*}
            which commutes up to homotopy; thus, we obtain the diagram
            \begin{equation*}
            \begin{tikzcd}
                & & C_{0}^{\bullet} \arrow[ld, "\textrm{qis}"'] \arrow[rd]  
                & & \\
                & C_1^{\bullet} \arrow[ld, "\textrm{qis}"'] \arrow[rrrd] 
                & & C_3^{\bullet} \arrow[llld, "\textrm{qis}", crossing over] \arrow[rd] 
                & \\
                A^{\bullet} 
                &  &  &  & B^{\bullet}\,,
            \end{tikzcd}
            \end{equation*}
            which shows that the roof dominated by $C_{1}^{\bullet}$
            is equivalent to the roof dominated by $C^{\bullet}_{3}$.
        \end{itemize}

        Finally, we can define $\Hom_{\cat{D}(\Aa)}(A^{\bullet}, B^{\bullet})$:
        a morphism $A^{\bullet} \to B^{\bullet}$ in the derived category
        is the equivalence class of a roof 
        over $A^{\bullet}$ and $B^{\bullet}$. Given two morphisms
        $A^{\bullet} \to B^{\bullet}$ 
        and $B^{\bullet} \to C^{\bullet}$ in $\cat{D}(\Aa)$,
        we describe the composition in the derived category:
        taken two representatives
        \begin{equation*}
            \begin{tikzcd}
                & C_{1}^{\bullet} \ar[dl, "\textrm{qis}"'] \ar[dr]
                && C_{2}^{\bullet} \ar[dl, "\textrm{qis}"'] \ar[dr] & \\
                A^{\bullet} & & B^{\bullet} & & C^{\bullet}\,,
            \end{tikzcd}
        \end{equation*}
        the composition $A^{\bullet} \to B^{\bullet} \to C^{\bullet}$
        is the equivalence class of a roof on top of the other two,
        in such a way that the following
        \begin{equation*}
            \begin{tikzcd}
                && C_{0}^{\bullet} \ar[dl, "\textrm{qis}"'] \ar[dr] && \\
                & C_{1}^{\bullet} \ar[dl, "\textrm{qis}"'] \ar[dr]
                && C_{2}^{\bullet} \ar[dl, "\textrm{qis}"'] \ar[dr] & \\
                A^{\bullet} & & B^{\bullet} & & C^{\bullet}
            \end{tikzcd}
        \end{equation*}
        is a commutative diagram in $\cat{K}(\Aa)$: 
        \hyperref[roof-comp]{Proposition~\ref*{roof-comp}}
        ensures that such a diagram always exists; 
        moreover, the equivalence class of the
        appearing roof is unique: indeed, 
        for two choices $C_{0}^{\bullet}$
        and $D_{0}^{\bullet}$ on the top of the above diagram,
        one can use \hyperref[roof-comp]{Proposition~\ref*{roof-comp}}
        again to show that the two choices are equivalent.
        Thus, the composition of two morphisms in $\cat{D}(\Aa)$
        is well defined and one can show it is associative,
        so that $\cat{D}(\Aa)$ defines a category.

        Notice that, for any given qis
        $f^{\bullet}:A^{\bullet} \to B^{\bullet}$,
        its image $Q(f^{\bullet})$ is an isomorphism
        in the category $\cat{D}(\Aa)$ just described:
        indeed, it can be represented by the roof
        $A^{\bullet} = A^{\bullet} \to B^{\bullet}$,
        whose inverse is the equivalence class of
        \begin{equation*}
            \begin{tikzcd}
                & A^{\bullet} \ar[dr, equals] \ar[dl, "f^{\bullet}"'] & \\
                B^{\bullet} & & A^{\bullet}\,,
            \end{tikzcd}
        \end{equation*}
        as we can see from the diagrams:
        \begin{equation*}
            \begin{tikzcd}[column sep=small]
                && A^{\bullet} \ar[dl, equals] \ar[dr, equals] && &
                && A^{\bullet} \ar[dl, equals] \ar[dr, equals] && \\
                & A^{\bullet} \ar[dl, equals] \ar[dr, "f^{\bullet}"]
                && A^{\bullet} \ar[dr, equals] \ar[dl, "f^{\bullet}"'] & &
                & A^{\bullet} \ar[dr, equals] \ar[dl, "f^{\bullet}"']
                && A^{\bullet} \ar[dl, equals] \ar[dr, "f^{\bullet}"] &\\
                A^{\bullet} & & B^{\bullet} & & A^{\bullet} \,, &
                B^{\bullet} & & A^{\bullet} & & B^{\bullet}\,,
            \end{tikzcd}
        \end{equation*}
        where the second composition is indeed in the equivalence class of the
        identity of $B^{\bullet}$ because
        \begin{equation*}
            \begin{tikzcd}
                & & A^{\bullet} \arrow[ld, equals] \arrow[rd, "f^{\bullet}"]  
                & & \\
                & A^{\bullet} \arrow[ld, "f^{\bullet}"'] \arrow[rrrd, "f^{\bullet}"'] 
                & & B^{\bullet} \arrow[llld, equals, crossing over] \arrow[rd, equals] 
                & \\
                B^{\bullet} 
                &  &  &  & B^{\bullet}\,.
            \end{tikzcd}
        \end{equation*}
        Hence, one can easily verify the \emph{\textbf{universal property}}
        holds by setting $G(f:A^{\bullet} \to B^{\bullet}) 
        := \left(F(f^{\bullet}):F(A^{\bullet}) \to F(B^{\bullet}) \right)$,
        for any representative $f^{\bullet}$ of the morphism $f$.
    \end{proof}
\end{thmdef}

\begin{rmk}
    Under the functor $Q$, we identify objects 
    in $\cat{D}(\Aa)$ with objects in $C^{\bullet}(\Aa)$:
    hence, we may speak of complexes 
    $A^{\bullet}, B^{\bullet}, ... \in \cat{D}(\Aa)$.
    As a consequence, the cohomology objects $H^{n}(A^{\bullet})$
    of $A^{\bullet}\in \cat{D}(\Aa)$ are well-defined
    objects of the abelian category $\Aa$ because
    of the universal property of the derived category.
    Moreover, given two homotopic maps
    $f^{\bullet}, g^{\bullet}:A^{\bullet} \to B^{\bullet}$,
    the diagram
    \begin{equation*}
            \begin{tikzcd}
                & & A^{\bullet} \arrow[ld, equals] \arrow[rd, equals]  
                & & \\
                & A^{\bullet} \arrow[ld, equals] \arrow[rrrd, "f^{\bullet}"'] 
                & & A^{\bullet} \arrow[llld, equals, crossing over] \arrow[rd, "g^{\bullet}"] 
                & \\
                A^{\bullet} 
                &  &  &  & B^{\bullet}\,.
            \end{tikzcd}
    \end{equation*}
    commutes in $\cat{K}(\Aa)$, so we deduce that 
    $Q(f^{\bullet}) = Q(g^{\bullet})$.
    By the \textbf{universal property of the homotopy category},
    it follows that there exists a unique factorization
    \begin{equation*}
        \begin{tikzcd}
            C^{\bullet}(\Aa) \ar[rr] \ar[dr, "Q"'] 
            & & \cat{K}(\Aa) \ar[dl] \\
            & \cat{D}(\Aa) & \,.
        \end{tikzcd}
    \end{equation*}
    This means that, for every $n \in \Z$, 
    the $n$-th cohomology $H^{n}$ is a well-defined functor
    on the derived category.
\end{rmk}

\begin{rmk}
    Viewing any object $A \in \Aa$ as a complex 
    concentrated in degree zero yields an equivalence 
    between $\Aa$ and the full subcategory of $\cat{D}(A)$
    that consists of all complexes $B^{\bullet}$ 
    with $H^{n}(B^{\bullet}) = \cat{0}$, for $n \ne 0$.
\end{rmk}

\begin{thm}
    The inclusion $\Aa \hookrightarrow \cat{D}(\Aa)$
    which sends an object $A$ to the complex
    concentrated in degree zero $A{[0]}$ 
    yields an equivalence with the full subcategory 
    $\cat{D}^{0}(\Aa)$ of $\cat{D}(\Aa)$ formed by
    complexes $B^{\bullet}$ such that 
    $H^{n}(B^{\bullet}) \simeq \cat{0}$,
    for $n \ne 0$.
    \begin{proof}
        \textcolor{red}{Guarda Manin - III.5.Prop2}.
    \end{proof}
\end{thm}

\begin{exercise!}\label{derived-trivial}
    Show that $A^{\bullet}$ is isomorphic to $\cat{0}$
    in $\cat{D}(\Aa)$ if and only if 
    $H^{n}(A^{\bullet}) \simeq \cat{0}$ in $\Aa$, 
    for every $n \in \Z$.
    On the other hand, find an example of a complex morphism
    $f^{\bullet} : A^{\bullet} \to B^{\bullet}$ which is
    trivial in cohomology, but $Q(f^{\bullet}) \ne 0$.
    \begin{proof}[Solution]
        We have already noticed that cohomology
        is a well defined functor on the derived category,
        so if $A^{\bullet} \simeq \cat{0}$ 
        in $\cat{D}(\Aa)$, then $H^*(A^{\bullet})$
        is trivial. 
        Conversely, assume $A^{\bullet}$ has trivial cohomology:
        then the zero morphism $A^{\bullet} \to \cat{0}$
        is a qis, hence an isomorphism in $\cat{D}(\Aa)$.

        As an example of a non-trivial morphism,
        with trivial cohomology, consider the following
        commutative diagram in $\Aa = \Mod_{\Z}$:
        \begin{equation*}
            \begin{tikzcd}
                A^{\bullet}\,: \ar[d, "f^{\bullet}"']
                & \cat{0} \ar[r] 
                & \Z/4\Z \ar[r, "\cdot 2"] \ar[d]
                & \Z/4\Z \ar[r, "\cdot 2"] \ar[d, "q"]
                & \Z/4\Z \ar[r] \ar[d]
                & \cat{0} \\
                B^{\bullet} \,:
                & \cat{0} \ar[r]
                & \cat{0} \ar[r]
                & \Z/2\Z \ar[r]
                & \cat{0} \ar[r]
                & \cat{0} 
            \end{tikzcd}\,;
        \end{equation*}
        since $H^0(A^{\bullet}) \simeq \cat{0}$,
        then it is clear that $f^*=0$.
        Nevertheless, $f^{\bullet}$ in not null-homotopic:
        indeed, if there existed a homotopy $h$, 
        then for each $x \in \Z/4\Z$ we would have
        \begin{equation*}
            q(x) = h(2x) = 2h(x) = 0\,,
        \end{equation*}
        which is a contradiction. In particular,
        this means that in $\cat{K}(\Mod_{\Z})$
        there is no commutative diagram of the
        form
        \begin{equation*}
            \begin{tikzcd}
                & & C_{0}^{\bullet} \arrow[ld, "\mathrm{qis}"'] \arrow[rd]  
                & & \\
                & A^{\bullet} \arrow[ld, equals] \arrow[rrrd, "f^{\bullet}"'] 
                & & C^{\bullet} \arrow[llld, "\mathrm{qis}", crossing over] \arrow[rd, "0"] 
                & \\
                A^{\bullet} 
                &  &  &  & B^{\bullet}\,,
            \end{tikzcd}
        \end{equation*}
        so $Q(f^{\bullet}) \ne 0$. \textcolor{red}{Why? Guarda esercizi Danilo.}
    \end{proof}
\end{exercise!}

\begin{exercise}
    \textcolor{red}{Check that the derived category $\cat{D}(\Aa)$
    is additive.}
\end{exercise}

Thus, if $\Aa$ is an abelian category, 
its derived category $\cat{D}(\Aa)$ is additive, 
but in general not abelian;
nevertheless, it is triangulated.

\begin{prop}
    The category $\cat{D}(\Aa)$ is triangulated and the canonical functor
    \begin{equation*}
        \cat{K}(\Aa) \longrightarrow \cat{D}(\Aa)
    \end{equation*}
    is an exact functor of triangulated categories.
    \begin{proof}
        \textcolor{red}{Check Gelfand-Manin IV.2}
    \end{proof}
\end{prop}

\begin{exercise!}\label{SES-TRI}
    Suppose
    \begin{center}
        \begin{tikzcd}
            0 \ar[r]
            & A \ar[r, "f"]
            & B \ar[r, "g"]
            & C \ar[r]
            & 0
        \end{tikzcd}
    \end{center}
        is a short exact sequence in an abelian category $\Aa$.
        Show that under the full embedding into the
        homotopy category $\Aa \hookrightarrow \cat{K}(\Aa)$
        (or the one in the derived category $\Aa \hookrightarrow \cat{D}(\Aa)$),
        this becomes a distinguished triangle 
    \begin{center}
        \begin{tikzcd}
            A \ar[r]
            & B \ar[r]
            & C \ar[r, "\delta"]
            & A{[1]}\,,
        \end{tikzcd}
    \end{center}
        where $\delta$ is the composition of the inverse
        of the qis $\cat{C}(f) \to C$ with the projection $\cat{C}(f) \to A[1]$.
        
        Conversely, if $A,B, C \in \Aa$ form a distinguished triangle 
        \begin{center}
        \begin{tikzcd}
            A \ar[r]
            & B \ar[r]
            & C \ar[r]
            & A{[1]}\,,
        \end{tikzcd}
        \end{center}
        then $0 \to A \to B \to C \to 0$ is a short exact sequence in $\Aa$.
    \begin{proof}[Solution]
        Notice that the cone $\cat{C}(f)$ is given by the complex
        \begin{equation*}
            \begin{tikzcd}
                0 \ar[r] & A \ar[r, "f"] & B \ar[r] & 0\,,
            \end{tikzcd}
        \end{equation*}
        thus $H^{-1}\left(\cat{C}(f)\right) = \ker f$ 
        and $H^{0}\left(\cat{C}(f)\right) = \Coker f$.
        Since the short sequence is exact, it follows that
        \begin{equation*}
            H^{-1}\left(\cat{C}(f)\right) = 0\,, \quad
            H^{0}\left(\cat{C}(f)\right) \simeq C\,,
        \end{equation*}
        hence the diagram
        \begin{equation*}
            \begin{tikzcd}
                0 \ar[r] & A \ar[r, "f"] \ar[d] & B \ar[r] \ar[d, "g"] & 0 \\
                0 \ar[r] & 0 \ar[r] & C \ar[r] & 0
            \end{tikzcd}
        \end{equation*}
        defines a qis $\cat{C}(f) \to C$. 
        Then, by passing in $\cat{K}(\Aa)$ (or equivalently in $\cat{D}(\Aa)$),
        we end up with the commutative diagram
        \begin{equation*}
            \begin{tikzcd}
                A \ar[r, "f"] \ar[d, equals] 
                & B \ar[r] \ar[d, equals] 
                & C \ar[r] \ar[d]
                & A{[1]} \ar[d, equals] \\
                A \ar[r, "f"]
                & B \ar[r]
                & \cat{C}(f) \ar[r] 
                & A{[1]}\,,
            \end{tikzcd}
        \end{equation*}
        which is an isomorphism of triangles by the 
        \hyperref[5lemma]{5-lemma~\ref*{5lemma}}.

        Conversely, if three objects $A,B,C \in \Aa$ form
        a distinguished triangle $A \to B \to C \to A{[1]}$,
        then the induced \hyperref[LECS]{LECS~\ref*{LECS}}
        is a short exact sequence in $\Aa$.
    \end{proof}
\end{exercise!}

\begin{exercise!}\label{derived-LECS}
    Suppose $A^{\bullet} \to B^{\bullet} \to C^{\bullet} \to A^{\bullet}[1]$
    is a distinguished triangle in the derived category $\cat{D}(\Aa)$.
    Show that it naturally induces a long exact sequence
    \begin{center}
        \begin{tikzcd}
            \dots \ar[r]
            & H^{i}(A^{\bullet}) \ar[r]
            & H^{i}(B^{\bullet}) \ar[r]
            & H^{i}(C^{\bullet}) \ar[r]
            & H^{i+1}(A^{\bullet}) \ar[r]
            & \dots
        \end{tikzcd}
    \end{center}
    \begin{proof}[Solution]
        \textcolor{red}{LALALA.}
    \end{proof}
\end{exercise!}


By definition, complexes in the categories 
$\cat{K}(A)$ and $\cat{D}(A)$ are unbounded, 
but often it is more convenient to work with bounded ones,
especially in the algebro-geometric context.

\begin{df}
    Given an abelian category $\Aa$,
    let $C^{*}(\Aa)$, with $* \in \Set{+,-,b}$, 
    be the category of complexes $A^{\bullet}$
    with $A^{n} = 0$ for $n << 0, n >> 0$,
    respectively $|n| >> 0$.
\end{df}

By dividing out first by homotopy equivalence 
and then by qis one obtains
the categories $\cat{K}^{*}(\Aa)$ and $\cat{D}^{*}(\Aa)$,
with $* \in \Set{+,-,b}$. 
Let us consider the natural functors 
$\cat{D}^{*}(A) \to \cat{D}(A)$ 
given by just forgetting the boundedness condition.

\begin{prop}
    The natural functor $\cat{D}^+(\Aa) \hookrightarrow \cat{D}(\Aa)$,
    defines an equivalence of $\cat{D}^{+}(\Aa)$ with
    the full triangulated subcategory of all 
    complexes $A^{\bullet} \in \cat{D}(\Aa)$ with
    $A^{n}=0$ for $n << 0$.
    Analogous statements hold true for $\cat{D}^{-}(\Aa)$
    and $\cat{D}^{b}(\Aa)$.
    \begin{proof}
        \textcolor{red}{KASHIWARA SHAPIRA.}
    \end{proof}
\end{prop}

\begin{exercise!}\label{null-terms}
    Let $A^{\bullet}$ be a complex with $H^{n}(A^{\bullet}) = 0$, for $n > m$.
    Show that $A^{\bullet}$ is quasi-isomorphic 
    (and hence isomorphic as an object in $\cat{D}(\Aa)$)
    to a complex $B^{\bullet}$, with $B^{n}=0$ for $n > m$.
    \begin{proof}[Solution]
        \textcolor{red}{AAAAAAAH}
    \end{proof}
\end{exercise!}

\begin{exercise}
    Let $A^{\bullet}$ be a complex with 
    $m := \Set{n \in \Z| H^{n}(A^{\bullet}) \ne 0} < \infty$.
    Show there exists a morphism
    \begin{equation*}
        \phi : A^{\bullet} \longrightarrow H^{m}(A^{\bullet})[-m]
    \end{equation*}
    in $\cat{D}(\Aa)$ such that 
    $H^{m}(\phi) : H^{m}(A^{\bullet}) \to H^{m}(A^{\bullet})$
    is the identity.
    \begin{proof}[Solution]
        \textcolor{red}{AAAAAAAAAA}
    \end{proof}
\end{exercise}

\begin{exercise}
    Suppose $H^{n}(A^{\bullet}) = 0$ for $n < n_{0}$.
    Show there exists a distinguished triangle
    \begin{equation*}
        \begin{tikzcd}
            H^{n_{0}}(A^{\bullet}){[-n_{0}]} \ar[r]
            & A^{\bullet} \ar[r, "\phi"]
            & B^{\bullet} \ar[r]
            & H^{n_{0}}(A^{\bullet}){[1-n_{0}]}
        \end{tikzcd}
    \end{equation*}
    in $\cat{D}(\Aa)$ with $H^{n}(B^{\bullet})=0$ 
    for $n \le n_{0}$ and $\phi$ inducing isomorphisms
    $H^{n}(A^{\bullet}) \simeq H^{n}(B^{\bullet})$ for $n > n_{0}$.
    \begin{proof}[Solution]
        \textcolor{red}{AAAAAAAAA.}
    \end{proof}
\end{exercise}

\textcolor{red}{Splitting and derived categories in Gelfand Manin.}
%%	
\section{Injective and projective resolutions}\todo{Fix this chapter structure. Follow the notes of Tamas.}

Due to the very construction of the derived category, it is sometimes quite cumbersome to do explicit calculations there. Often, however, it is possible to work with a very special class of complexes for which morphisms in the derived category and in the homotopy category are the same thing. Depending on the kind of functors one is interested in, the notion of injective, respectively, projective objects will be crucial.
As always, let $\Aa$ denote an abelian category.

\begin{lemma}
    A map $f:A \to B$ is a monomorphism if and only if $\ker f \simeq \cat{0}$.
    Dually, $f$ is an epimorphism if and only if $\Coker f \simeq \cat{0}$.
    \begin{proof}
        If $f: A \to B$ is mono, 
        then by the commutativity of the diagram
        \begin{equation*}
            \begin{tikzcd}
            \ker f \ar[r, bend left, shift left=0.3ex] \ar[r, bend right, shift right=0.3ex, "0"'] 
            & A \ar[r, "f"] & B
            \end{tikzcd}
        \end{equation*}
        we deduce that the canonical map $\ker f \to A$
        factors through the zero object, hence $\ker f \simeq \cat{0}$.
        Conversely, if $\ker f$ is the zero object, for any two morphisms
        $g,h:C \to A$ such that $gf=hf$ it holds $(g-h)f=0$,
        which means we have a factorization
        \begin{equation*}
            \begin{tikzcd}
                & \ker f \simeq \cat{0} \ar[dr] & & \\
                C \ar[ur] \ar[rr, "g-h"'] & & A \ar[r, "f"] & B\,,
            \end{tikzcd}
        \end{equation*}
        thus $g=h$ and $f$ is a monomorphism. Dually, one proves the statement
        for epimorphisms.
    \end{proof}
\end{lemma}

The previous lemma justifies the following notation,
already used in the setting of abelian groups:
a map $f:A \to B$ is a monomorphism if and only if the sequence
\begin{equation*}
    \begin{tikzcd}
        \cat{0} \ar[r] & A \ar[r, "f"] & B
    \end{tikzcd}
\end{equation*}
is exact, thus it will be convenient to represent monomorphisms
in terms of this sequence, whose exactness won't be stressed further on.
In a similar fashion, if $f:A \to B$ is an epimorphism,
it will be drawn as
\begin{equation*}
    \begin{tikzcd}
        A \ar[r, "f"] & B \ar[r] & \cat{0}\,.
    \end{tikzcd}
\end{equation*}


\begin{df}
    An object $P \in \Aa$ is called \textbf{projective}
    if, given any epimorphism $A \to B$,
    every map $P \to B$ can be lifted to a map $P \to A$,
    making the following diagram commute
    \begin{equation}\label{proj-diag}
        \begin{tikzcd}
            P \ar[d, "\exists"', dashed] \ar[dr] & & \\ 
            A \ar[r] & B \ar[r] & 0 \,.
        \end{tikzcd}
    \end{equation}
\end{df}

Recall that, given any object $X \in \Aa$, the covariant functor
\begin{equation*}
    \Hom_{\Aa}(X,-) : \Aa \longrightarrow \Ab
\end{equation*}
is \emph{left exact}, i.e. by applying it to any
short exact sequence
\begin{equation*}
    \begin{tikzcd}
        \cat{0} \ar[r]
        & A \ar[r]
        & B \ar[r]
        & C \ar[r]
        & \cat{0}\,,
    \end{tikzcd}
\end{equation*}
we obtain the following exact sequence of abelian groups:
\begin{equation*}
    \begin{tikzcd}
        \cat{0} \ar[r]
        & \Hom_{\Aa}(X,A) \ar[r]
        & \Hom_{\Aa}(X,B) \ar[r]
        & \Hom_{\Aa}(X,C)\,.
    \end{tikzcd}
\end{equation*}

It is easy to see that projectivity is characterized
by the exactness of this functor:
\begin{prop}
    An object $P \in \Aa$ is projective if and only
    if $\Hom_{\Aa}(P,-)$ is an exact functor.
    \begin{proof}
        Since we already know that the $\Hom$
        functor is left exact, it will suffice to check
        exactness on the right. Thus, we notice that
        any diagram of the form
        \eqref{proj-diag}
        %\begin{equation*}
        %\begin{tikzcd}
        %    P \ar[d, "\exists"', dashed] 
        %    \ar[dr] & & \\ 
        %    A \ar[r] & B \ar[r] & 0 \,
        %\end{tikzcd}
        %\end{equation*}
        can be filled with a vertical arrow $P \to A$
        if and only if the induced homomorphism
        $\Hom_{\Aa}(X,A) \to \Hom_{\Aa}(X,B)$ is surjective.
    \end{proof}
\end{prop}

\begin{rmk}
    It will be useful to consider the following slightly more
    general version of the diagram \eqref{proj-diag}:
    if $P$ is projective, from the exactness of $\Hom_{\Aa}(P,-)$
    we deduce that, given any commutative diagram of the form
    \begin{equation*}
        \begin{tikzcd}
            & P \ar[d] \ar[dr, "0"] \ar[dl, "\exists"', dashed] & \\
            A \ar[r] & B \ar[r] & C\,,
        \end{tikzcd}
    \end{equation*}
    whose bottom row is exact, there exists a map $P \to A$ 
    which fits in it.
\end{rmk}

The dual notion of the projective property
is given by \textbf{injectivity}:

\begin{df}
    An object $I \in \Aa$ is called \textbf{injective} if,
    given any monomorphism $A \to B$, 
    every map $A \to I$ can be extended to a map $B \to I$,
    in such a way that the following diagram commutes:
    \begin{equation*}
        \begin{tikzcd}
            \cat{0} \ar[r] & A \ar[r] \ar[dr] & B \ar[d, dashed, "\exists"'] \\
            & & I  \,.
        \end{tikzcd}
    \end{equation*}
\end{df}

\begin{rmk}
    Note that injectivity and projectivity are indeed
    dual notions, in the sense that an object $P \in \Aa$
    is projective if and only if $P$ is injective in the
    opposite category $\Aa^{op}$.
    From this, we deduce a similar characterization
    of injectivity in terms of $\Hom$ functors:
    an object $I \in \Aa$ is injective if and only
    if the contravariant functor 
    \begin{equation*}
        \Hom_{\Aa}(-,I) : \Aa^{op} \longrightarrow \Ab
    \end{equation*}
    is exact.
\end{rmk}

\begin{ex}
    The direct sum of two injective objects is again injective:
    assume $I,J \in \Aa$ are injective and $A \to B$ is a monomorphism;
    then we have the following diagram
    \begin{equation*}
        \begin{tikzcd}[row sep=large]
\cat{0} \arrow[r] & A \arrow[rr] \arrow[d] \arrow[rd] &                                                                                      & B \arrow[lld, "\exists"', dashed, shorten=3mm, crossing over] \arrow[d, "\exists"', dashed] \\
            & I \arrow[rrd, hook, shorten=3mm]               & I \oplus J \arrow[l, two heads] \arrow[r, two heads] \arrow[rd, no head, equals] & J \arrow[d, hook]                                               \\
            &                                   &                                                                                      & I \oplus J          \,,                                           
\end{tikzcd}
    \end{equation*}
    which shows that any map $A \to I \oplus J$ extends to a morphism $B \to I \oplus J$.
    In other words, we use that 
    \begin{equation*}
        \Hom_{\Aa}(I \oplus J, -) \simeq \Hom_{\Aa}(I,-) \oplus \Hom_{\Aa}(J,-)
    \end{equation*}
    is a sum of exact functors, and hence exact.
    
    Dually, one checks that a sum of projective objects is again projective.
\end{ex}

\begin{df}
    We say that an abelian category $\Aa$ 
    \textbf{has enough injective} (resp. \textbf{projective})
    \textbf{objects} if, for any $A \in \Aa$,
    there exists a monomorphism $\cat{0} \to A \to I$,
    with $I \in \Aa$ injective 
    (resp. an epimorphism $P \to A \to \cat{0}$,
    with $P \in \Aa$ projective).

    An \textbf{injective resolution} of an object $A \in \Aa$
    is an exact sequence
    \begin{equation*}
        \begin{tikzcd}
            \cat{0} \ar[r]
            & A \ar[r]
            & I^{0} \ar[r]
            & I^{1} \ar[r]
            & I^{2} \ar[r]
            & \dots \,,
        \end{tikzcd}
    \end{equation*}
    with all $I^{i} \in \Aa$ injective. Similarly,
    a \textbf{projective resolution} of $A \in \Aa$
    is an exact sequence
    \begin{equation*}
        \begin{tikzcd}
            \dots \ar[r]
            & P^{-2} \ar[r]
            & P^{-1} \ar[r]
            & P^{0} \ar[r]
            & A \ar[r]
            & \cat{0}\,,
        \end{tikzcd}
    \end{equation*}
    where all $P^{i} \in \Aa$ are projective.
\end{df}

In other words, an injective resolution of $A \in \Aa$
consists of a quasi isomorphism
$A{[0]} \to I^{\bullet}$,
where $I^{\bullet}$ is a complex of injective objects,
such that $I^{i}=\cat{0}$, for $i < 0$.
In an analogous way, a projective resolution
of $A$ is a particular 
quasi-isomorphism $P^{\bullet} \to A{[0]}$,
where $P^{\bullet}$ is a complex with non-positive terms,
which are projective. It follows that
$A{[0]}, P^{\bullet}$ and $I^{\bullet}$
are isomorphic in $\cat{D}(\Aa)$. In fact,
the idea of the derived category as we understand it today,
is that an object $A \in \Aa$ should be
\emph{identified with all its resolutions}.
We can actually show that resolutions are unique in some sense.

\begin{prop}\label{res-htp}
    Let $f : A \to B$ be a morphism in $\Aa$ and two projective resolutions
    \begin{equation*}
        P^{\bullet} \longrightarrow A{[0]}\,, \quad
        Q^{\bullet} \longrightarrow B{[0]}\,.
    \end{equation*}
    Then there exists a morphism of resolutions 
    $R(f) : P^{\bullet} \to Q^{\bullet}$ which extends $f$,
    i.e. the following diagram in $\Aa$ commutes:
    \begin{equation*}
        \begin{tikzcd}
            P^{0} \ar[r] \ar[d, "R(f)^{0}"'] 
            & A \ar[d, "f"] \\
            Q^{0} \ar[r] & B\,;
        \end{tikzcd}
    \end{equation*}
    moreover, any two such extensions $R(f)$ and $R'(f)$ are homotopic.
    \begin{proof}
        We build the morphism $R(f)$ by induction on its degree:
        to build the first step, consider the diagram
        \begin{equation*}
        \begin{tikzcd}
            P^{0} \ar[r] \ar[d, "R(f)^{0}"', dashed] \ar[dr]
            & A \ar[d, "f"] & \\
            Q^{0} \ar[r] & B \ar[r] & \cat{0} \,;
        \end{tikzcd}
    \end{equation*}
    since $Q^{0} \to B$ is an epimorphism and $P^{0}$
    is projective, then there exists $R(f)^{0}:P^{0} \to Q^{0}$,
    which makes the above diagram commutative.

    Assume we already have $R(f)^{-j} : P^{-j} \to Q^{-j}$, 
    for every $0 \le j < i$, such that
    \begin{equation*}
        d_{Q}^{-j} \circ R(f)^{-j} = R(f)^{-j+1} \circ d_{P}^{-j}\,;
    \end{equation*}
    then, consider the diagram
    \begin{equation*}
        \begin{tikzcd}
            P^{-i} \ar[d, "\exists"', dashed] 
            \ar[dr, "R(f)^{-i+1} \circ d_{P}^{-i}"]
            && \\ %& P^{-i+1} \ar[d, "R(f)^{-i+1}"] & \\
            Q^{-i} \ar[r] & Q^{-i+1} \ar[r] & Q^{-i+2}\,,
        \end{tikzcd}
    \end{equation*}
    whose bottom row is exact; as $P^{-i}$ is projective and
    \begin{equation*}
        d^{-i+1}_{Q} \circ R(f)^{-i+1} \circ d_{P}^{-i}
        = R(f)^{-i+2} \circ d_{P}^{-i-1} \circ d_{P}^{-i-2} = 0\,,
    \end{equation*}
    we deduce there exists $R(f)^{-i}:P^{-i} \to Q^{-i}$
    which makes the diagram commute.

    Now assume $R(f)$ and $R'(f)$ are two morphisms of resolutions
    that extend $f:A \to B$. As in the previous part,
    we build a homotopy $s$ between $R(f)$ and $R'(f)$ inductively:
    start by setting $s^{1} : A \to Q^{0}$ to be the zero morphism,
    and then consider the diagram
        \begin{equation*}
            \begin{tikzcd}
                & P^{0} \ar[d, "R(f)^{0}-R'(f)^{0}"] \ar[dl, dashed, "\exists"']
                & \\
                Q^{-1} \ar[r] & Q^{0} \ar[r] & B \,,
            \end{tikzcd}
        \end{equation*}
    whose bottom row is exact; we see that
    \begin{equation*}
       d_{Q}^{0} \circ (R(f)^{0}-R'(f)^{0}) 
       = (f - f) \circ d^{0}_{P} = 0\,, 
    \end{equation*}
    thus there exists $s^{0}: P^{0} \to Q^{-1}$ that
    makes the diagram commute because $P^{0}$ is projective.
    Notice that at the $0$-th level it holds
    \begin{equation*}
        R(f)^{0}-R'(f)^{0} = s^{0}d^{-1}_{Q} + d^{0}_{P} s^{1}\,.
    \end{equation*}

    Similarly, to build the $(-i)$-th level of the homotopy,
    once $s^{-j}:P^{-j} \to Q^{-i-1}$ are given for all $0 \le j <i$,
    consider the commutative diagram
    \begin{equation*}
            \begin{tikzcd}
                & P^{-i} \ar[d, "R(f)^{-i}-R'(f)^{-i}-s^{-i+1}d_{P}^{-i}"] 
                \ar[dl, dashed, "\exists"']
                & \\
                Q^{-i-1} \ar[r] & Q^{-i} \ar[r] & Q^{-i+1} \,
            \end{tikzcd}
        \end{equation*}
    and notice that
    \begin{align*}
        & d^{-i}_{Q} \circ \big( R(f)^{-i}-R'(f)^{-i}-s^{-i+1} \circ d_{P}^{-i} \big) \\
        =& (R(f)^{-i+1}-R'(f)^{-i+1}) \circ d^{-i}_{P} - (d^{-i}_{Q} \circ s^{-i+1}) \circ d_{P}^{-i}\\
        =& (R(f)^{-i+1}-R'(f)^{-i+1}) \circ d^{-i}_{P}
        - \big( R(f)^{-i+1}-R'(f)^{-i+1} - s^{-i+2} \circ d^{-i+1}_{P} \big) 
        \circ d_{P}^{-i} \\
        =& - s^{-i+2} \circ d^{-i+1}_{P} \circ d_{P}^{-i} = 0\,,
    \end{align*}
    hence by projectivity of $P^{-i}$ we conclude that 
    there exists $s^{-i}:P^{-i} \to Q^{-i-1}$ such that
    \begin{equation*}
        R(f)^{-i}-R'(f)^{-i} = s^{-i+1} \circ d^{-i}_{P} + d^{-i-1} \circ s^{-i}\,.
    \end{equation*}
    \end{proof}
\end{prop}

\begin{cor}\label{unique-res}
    Any two projective resolutions of an object are homotopy equivalent.
    \begin{proof}
        Using the notation of \hyperref[res-htp]{Proposition~\ref*{res-htp}},
        let $B=A$, $f = \cat{1}_{A}$ and then consider 
        $R(\cat{1}_{A}) : P^{\bullet} \to Q^{\bullet}$
        and $R'(\cat{1}_{A}) : Q^{\bullet} \to P^{\bullet}$
        extensions of the identity. 
        Since both $R'(\cat{1}_{A}) \circ R(\cat{1}_{A})$
        and $\cat{1}_{P^{\bullet}}$ are cochain maps
        $P^{\bullet} \to P^{\bullet}$ that extend the identity of $A$,
        by the last part of \hyperref[res-htp]{Proposition~\ref*{res-htp}}
        we conclude that 
        $R'(\cat{1}_{A}) \circ R(\cat{1}_{A}) \sim \cat{1}_{P^{\bullet}}$.
        Similarly, we deduce that $R(\cat{1}_{A}) \circ R'(\cat{1}_{A}) \sim \cat{1}_{Q^{\bullet}}$.
    \end{proof}
\end{cor}

\begin{rmk}
    \begin{enumerate}
        \item In the proof of \hyperref[res-htp]{Proposition~\ref*{res-htp}}
        we never used that $Q^{\bullet}$ was a projective resolution: indeed,
        \hyperref[res-htp]{Proposition~\ref*{res-htp}} remains valid even if
        we assume that $Q^{\bullet} \to B \to 0$ is an exact complex.

        \item By reversing arrows and replacing projective resolutions with
        injective ones, we obtain a dual version of 
        \hyperref[res-htp]{Proposition~\ref*{res-htp}}, 
        as well as a dual version of \hyperref[unique-res]{Corollary~\ref*{unique-res}}.

        \item The meaning of \hyperref[unique-res]{Corollary~\ref*{unique-res}}
        is that a projective resolution of an object is 
        \emph{unique up to homotopy equivalence}, 
        and the same holds true for an injective resolution.
        This means that, for any object $A \in \Aa$, 
        in the homotopy category $\cat{K}(\Aa)$ (and hence in $\cat{D}(\Aa)$)
        there is at most
        one projective resolution $P^{\bullet}$ of $A$
        and at most one injective resolution $I^{\bullet}$ of $A$
        \emph{up to isomorphism}.
    \end{enumerate}
\end{rmk}

The fundamental idea of the derived category,
as we understand it today, is that any object
of the abelian category should be identified
with all of its resolutions. 
In fact, a more general result holds true:
given an abelian category $\Aa$,
let $\Ii$ be the full subcategory of its injective objects 
and consider $\cat{K}^{+}(\Ii)$,
the full subcategory of $\cat{K}^{+}(\Aa)$
spanned by complexes with injective terms.
Consider the natural immersion $\cat{K}^{+}(\Ii) \hookrightarrow \cat{K}^{+}(\Aa)$
with $Q : \cat{K}^{+}(\Aa) \to \cat{D}^{+}(\Aa)$.

\begin{thm}\label{inj-equivalence}
    The functor $\cat{K}^{+}(\Ii) \to \cat{D}^{+}(\Aa)$ is an equivalence
    of $\cat{K}^{+}(\Ii)$ with a full subcategory of $\cat{D}^{+}(\Aa)$.
    Moreover, if $\Aa$ \textbf{has enough injectives}, the above functor
    is an equivalence of $\cat{K}^{+}(\Ii)$ and $\cat{D}^{+}(\Aa)$.
    \begin{proof}
        Check \parencite[III.5.20]{gelfand}.
        The proof is rather long and technical,
        although not very complicated,
        and passes through the definition of the
        derived category as a \emph{localization}
        of the homotopy category.
    \end{proof}
\end{thm}

\begin{cor}
    Suppose $\Aa$ is an abelian category with enough injectives.
    Any complex $A^{\bullet}$ with 
    $H^{n}(A^{\bullet}) = \cat{0}$ for $n<<0$
    is isomorphic in $\cat{D}(\Aa)$ to a complex
    $I^{\bullet}$ of injective objects $I^{i}$,
    such that $I^{i}=\cat{0}$ for $i << 0$.
    \begin{proof}
        Suppose $\Aa$ is an abelian category with enough injectives.
        By \hyperref[null-terms]{Exercise~\ref*{null-terms}}
        we may assume $A^{i} = \cat{0}$, for $i << 0$,
        thus $A^{\bullet} \in \cat{D}^{+}(\Aa)$
        and by the equivalence $\cat{K}^{+}(\Ii) \simeq \cat{D}^{+}(\Aa)$
        we have $A^{\bullet} \simeq I^{\bullet}$,
        for some injective resolution $I^{\bullet}$.
    \end{proof}
\end{cor}

As one might expect, a similar statement holds true for
projective resolutions: if $\Pp$ is the full subcategory 
of projective objects of $\Aa$, then $\cat{K}^{-}(\Pp)$
is the full subcategory of $\cat{K}^{-}(\Aa)$ spanned by
complexes with projective terms, and hence we have the following

\begin{thm}\label{proj-equivalence}
    The functor $\cat{K}^{-}(\Pp) \to \cat{D}^{-}(\Aa)$ is an equivalence
    of $\cat{K}^{-}(\Pp)$ with a full subcategory of $\cat{D}^{-}(\Pp)$.
    Moreover, if $\Aa$ \textbf{has enough projectives}, the above functor
    is an equivalence of $\cat{K}^{-}(\Pp)$ and $\cat{D}^{-}(\Aa)$.

    It follows that every complex $A^{\bullet}$ with 
    $H^{n}(A^{\bullet}) = \cat{0}$ for $n>>0$
    is isomorphic in $\cat{D}(\Aa)$ to a complex
    $P^{\bullet}$ of projective objects $P^{i}$,
    such that $P^{i}=\cat{0}$ for $i >> 0$.
\end{thm}


We now try to relate morphisms between complexes
in the derived category and morphisms between their resolutions,
in order to better understand how to do computations in $\cat{D}(\Aa)$.

\begin{lemma}\label{lemma-qis}
    Let $A^{\bullet} \to B^{\bullet}$ be a qis in $\cat{K}^{+}(\Aa)$.
    Then for every complex of injective objects $I^{\bullet}$ 
    which is bounded below, the induced map
    \begin{equation*}
        \Hom_{\cat{K}(\Aa)}(B^{\bullet}, I^{\bullet})
        \longrightarrow \Hom_{\cat{K}(\Aa)}(A^{\bullet},I^{\bullet})\,,
    \end{equation*}
    is an isomorphism.
    \begin{proof}
        Since $\cat{K}^{+}(\Aa)$ is triangulated, 
        we can complete the qis to a distinguished triangle
        \begin{equation*}
            A^{\bullet} \longrightarrow B^{\bullet}
            \longrightarrow C^{\bullet}
            \longrightarrow A^{\bullet}{[1]}\,,
        \end{equation*}
        and if we apply the cohomological functor
        $\Hom_{\cat{K}(\Aa)}(-,I^{\bullet})$ we obtain a LECS;
        thus, it is enough to show that 
        $\Hom_{\cat{K}(\Aa)}(C^{\bullet},I^{\bullet}) = \cat{0}$
        whenever the complex $C^{\bullet} \in \cat{K}^{+}(\Aa)$
        is acyclic, i.e. any cochain map 
        $f^{\bullet}:C^{\bullet} \to I^{\bullet}$
        is null homotopic.

        Indeed, by following the same procedure as in the proof
        of \hyperref[res-htp]{Proposition~\ref*{res-htp}},
        one can build a homotopy by induction: first, 
        we may assume the first non-zero term of $C^{\bullet}$
        is $C^{0}$ and set $s^{0}:C^{0} \to I^{-1}$ to be the zero map.
        Since $I^{0}$ is injective, the morphism $f^{0}$
        extends to a map $s^{1} : C^{1} \to I^{0}$ such that
        \begin{equation*}
            f^{0} = s^{1} \circ d^{0}_{C}
            = s^{1} \circ d^{0}_{C} + d^{-1}_{I} \circ s^{0}\,.
        \end{equation*}
        If we have already built $s^{j}$ for $0 \le j < i$ such that
        \begin{equation*}
            f^{j-1} = s^{j} \circ d^{j-1}_{C} + d^{j-2}_{I} \circ s^{j-1}\,,
        \end{equation*}
        then consider the commutative diagram
        \begin{equation*}
            \begin{tikzcd}[row sep=large]
                C^{i-2} \ar[r] \ar[d] 
                & C^{i-1} \ar[r] \ar[d, "\alpha"'] 
                & C^{i} \ar[dl, "\exists"', "s^{i}", dashed] \\
                \cat{0} \ar[r] & I^{i-1} & \,,
            \end{tikzcd}
        \end{equation*}
        with $\alpha = f^{i-1}- d_{I}^{i-2} \circ s^{i-1}$;
        since the top row is exact and $I^{i-1}$ is projective,
        there is an extension $s^{i} : C^{i} \to I^{i-1}$
        such that
        \begin{equation*}
            f^{i-1} = s^{i} \circ d^{i-1}_{C} + d^{i-2}_{I} \circ s^{i-1}\,,
        \end{equation*}
        so the $s^{i}$ all together form the desired homotopy.
    \end{proof}
\end{lemma}

\begin{lemma}
    Given $A^{\bullet}, I^{\bullet} \in \cat{K}^{+}(\Aa)$,
    with all $I^{i}$ injective objects, then
    \begin{equation*}
        \Hom_{\cat{K}(\Aa)}(A^{\bullet}, I^{\bullet})
        = \Hom_{\cat{D}(\Aa)}(A^{\bullet}, I^{\bullet})\,.
    \end{equation*}
    \begin{proof}
        Clearly the canonical functor $Q:\cat{K}(\Aa) \to \cat{D}(\Aa)$
        induces a map 
        \begin{equation*}
        \Hom_{\cat{K}(\Aa)}(A^{\bullet}, I^{\bullet})
        \longrightarrow \Hom_{\cat{D}(\Aa)}(A^{\bullet}, I^{\bullet})\,,
        \qquad
        \big(A^{\bullet} \xrightarrow{f^{\bullet}} I^{\bullet}\big)
        \longmapsto \big(A^{\bullet} = A^{\bullet} \xrightarrow{f^{\bullet}} I^{\bullet}\big)\,,
        \end{equation*}
        and hence we have to show that, for every roof
        \begin{equation*}
            \begin{tikzcd}
                & B^{\bullet} \ar[dl, "\textrm{qis}"'] \ar[dr] & \\
                A^{\bullet} \ar[rr, dashed, "\exists !"]
                & & I^{\bullet} \,,
            \end{tikzcd}
        \end{equation*}
        there exists a unique morphism $A^{\bullet} \to I^{\bullet}$
        which makes the diagram commute \emph{up to homotopy}.
        This is equivalent to saying that, 
        once a qis $B^{\bullet} \to A^{\bullet}$ is fixed,
        there is a bijection 
        $\Hom_{\cat{K}(\Aa)}(A^{\bullet}, I^{\bullet})
        = \Hom_{\cat{K}(\Aa)}(B^{\bullet}, I^{\bullet})$,
        but this is the content of \hyperref[lemma-qis]{Lemma~\ref*{lemma-qis}}.
    \end{proof}
\end{lemma}

\textcolor{red}{Qui ho fatto un po' un casino per l'ordine,
perché ho messo il teorema sull'equivalenza prima, aiuto.
Rivedere un po' l'ordine di questa parte, magari seguire Tamas.}
%%	
\section{Derived functors}

Let $F : \Aa \to \Bb$ be an \emph{additive} functor 
between abelian categories; it naturally extends
to a unique functor between the associated categories of
cochain complexes $C^{\bullet}(\Aa) \to C^{\bullet}(\Bb)$
and hence, by dividing out by homotopy equivalence,
it induces a well-defined additive functor 
$\Tilde{F}:\cat{K}(\Aa) \to \cat{K}(\Bb)$:
indeed, if $s$ is a homotopy between two
cochain maps $f^{\bullet},g^{\bullet}:A^{\bullet} \to \Tilde{A}^{\bullet}$, 
then the identity
\begin{align*}
    \Tilde{F}(f^{\bullet}) - \Tilde{F}(g^{\bullet})
    &= \Tilde{F}(f^{\bullet} - g^{\bullet}) \\
    &= \Tilde{F}\left(sd_{A}+d_{\Tilde{A}}s\right) 
    = \Tilde{F}(s)\Tilde{F}\left(d_{A}\right) + \Tilde{F}\left(d_{\Tilde{A}}\right)\Tilde{F}(s)
    = \Tilde{F}(s)d_{F(A)} + d_{F(\Tilde{A})}F(s)
\end{align*}
shows that $\Tilde{F}(s)$ is a homotopy between
$\Tilde{F}(f^{\bullet})$ and $\Tilde{F}(g^{\bullet})$.
Moreover, we can check that 
$\Tilde{F}:\cat{K}(\Aa) \to \cat{K}(\Bb)$ is an
exact functor between triangulated categories:
    \begin{itemize}
        \item[(\textbf{EF1})] since the functor $\Tilde{F}$
        sends a complex $A^{\bullet} \in \cat{K}(\Aa)$ to
        the complex
        \begin{equation*}
            \begin{tikzcd}[column sep=large]
                \dots \ar[r]
                & F(A^{i-1}) \ar[r, "F(d_{A}^{i-1})"]
                & F(A^{i}) \ar[r, "F(d_{A}^{i})"]
                & F(A^{i+1}) \ar[r]
                & \dots
            \end{tikzcd}
        \end{equation*}
        it is clear that $\Tilde{F}$ commutes with the shift functor:
        \begin{equation*}
            \big( \Tilde{F}(A^{\bullet})[1] \big)^{n}
            = \Tilde{F}(A^{\bullet})^{n-1} 
            = F(A^{n-1})
            = F\big( (A^{\bullet}[1])^{n} \big)
            = \Tilde{F}(A^{\bullet}[1])^{n}\,;
        \end{equation*}

        \item[(\textbf{EF2})] recall that the image of a direct sum
        via an additive functor is the direct sum of the images:
        in particular, given a map 
        $f^{\bullet}:A^{\bullet} \to \Tilde{A}^{\bullet}$ in $\cat{K}(\Aa)$,
        then $\Tilde{F}\big(\cat{C}(f^{\bullet})\big) 
        \simeq \cat{C}\big(\Tilde{F}(f^{\bullet})\big)$,
        and hence any distinguished triangle of the form
        \begin{equation*}
            \begin{tikzcd}
                A^{\bullet} \ar[r, "f^{\bullet}"]
                & B^{\bullet} \ar[r]
                & \cat{C}(f^{\bullet}) \ar[r]
                & A^{\bullet}{[1]}
            \end{tikzcd}
        \end{equation*}
        is sent to a distinguished triangle
        \begin{equation*}
            \begin{tikzcd}
                \Tilde{F}(A^{\bullet}) \ar[r, "\Tilde{F}(f^{\bullet})"]
                & \Tilde{F}(B^{\bullet}) \ar[r]
                & \cat{C}\big(\Tilde{F}(f^{\bullet})\big) \ar[r]
                & \Tilde{F}(A^{\bullet}){[1]}\,;
            \end{tikzcd}
        \end{equation*}
        by definition of distinguished triangles in the homotopy
        category, this is enough to conclude that $\Tilde{F}$
        preserves all triangles.
    \end{itemize}

Now consider the following question:
given the diagram
\begin{equation*}
    \begin{tikzcd}
        \cat{K}^{*}(\Aa) \ar[r] \ar[d, "\Tilde{F}"'] & \cat{D}^{*}(\Aa) \\
        \cat{K}^{}(\Bb) \ar[r] & \cat{D}^{}(\Bb)
    \end{tikzcd}
\end{equation*}
there exists a triangulated functor $\cat{D}^{*}(\Aa) \to \cat{D}^{}(\Bb)$
making the diagram commute? When $F:\Aa \to \Bb$ is an exact functor
of abelian categories, then the answer is yes: indeed, $\Tilde{F}$
sends any acyclic complex $A^{\bullet}$ to an acyclic complex because,
for every $n \in \Z$, the short exact sequence
\begin{equation*}
    \begin{tikzcd}
        \cat{0} \ar[r]
        & \ker d_{A}^{n} \ar[r]
        & A^{n} \ar[r, "d_{A}^{n}"]
        & \ker d_{A}^{n+1} \ar[r]
        & \cat{0}
    \end{tikzcd}
\end{equation*}
is sent to the sequence
\begin{equation*}
    \begin{tikzcd}
        \cat{0} \ar[r]
        & \ker d_{F(A)}^{n} \ar[r]
        & \Tilde{F}(A^{\bullet})^{n} \ar[r, "d_{F(A)}^{n}"]
        & \ker d_{F(A)}^{n+1} \ar[r]
        & \cat{0}
    \end{tikzcd}
\end{equation*}
which is exact in $\Bb$, and hence $H^{n}(\Tilde{F}(A^{\bullet}) = \cat{0}$.
Thus, the hypothesis of the following lemma are satisfied:

\begin{lemma}\label{induced-derived-functor}
    Let $G:\cat{K}^{*}(\Aa) \to \cat{K}^{}(\Bb)$
    be an exact functor of triangulated categories.
    Then $G$ naturally induces a commutative diagram
        \begin{equation*}
            \begin{tikzcd}
                \cat{K}^{*}(\Aa) \ar[r, "G"] \ar[d]
                & \cat{K}^{}(\Bb) \ar[d] \\
                \cat{D}^{*}(\Aa) \ar[r] & \cat{D}^{}(\Bb)
            \end{tikzcd}
        \end{equation*}
        if one of the two following conditions is satisfied:
        \begin{rmnumerate}
            \item the image of a qis under $G$ is a qis;
            \item the functor $G$ sends acyclic complexes to acyclic complexes.
        \end{rmnumerate}
        \begin{proof}
            If condition (i) holds, then the composition 
            $\cat{K}^{*}(\Aa) \to \cat{K}^{}(\Bb) \to \cat{D}^{}(\Bb)$
            sends each qis to an isomorphism, hence by the universal
            property of the derived category of $\Aa$,
            it factors through $\cat{D}^{*}(\Aa)$ as in the diagram above.

            Assume now that condition (ii) holds true and consider
            a qis $f^{\bullet} : A^{\bullet} \to B^{\bullet}$;
            the cone $\cat{C}(f^{\bullet})$ is acyclic,
            so by hypothesis $\cat{C}\big(G(f^{\bullet})\big)$ is acyclic too,
            which implies that $G(f^{\bullet})$ is a quasi-isomorphism.
            Thus, $G$ sends quasi-isomorphisms to quasi-isomorphisms,
            and we conclude by part (i).
        \end{proof}
\end{lemma}

\begin{rmk}
    In the Lemma above, the functor $G$ need not 
    come from a functor between the abelian categories!
\end{rmk}

If $F$ is not exact, the image of an acyclic complex in $A$, 
i.e. one that becomes trivial in $\cat{D}(\Aa)$, is not, 
in general, acyclic. Thus, the naive extension of $F$ 
to a functor between the derived categories 
$\cat{D}(\Aa) \to \cat{D}(\Bb)$ does not make sense
and in this case a more complicated construction is needed 
in order to induce a natural functor between the derived categories. 
The new functor, called the \emph{derived functor}, 
will not produce a commutative diagram 
as in \hyperref[induced-derived-functor]{Lemma~\ref*{induced-derived-functor}}, 
but it has the advantage to encode more information 
even when applied to an object in the abelian category. 
Roughly, it explains why the original functor fails to be exact.
In order to ensure existence of the derived functor, 
we will always have to assume some kind of exactness:
for a \textbf{left exact} functor $F : \Aa \to \Bb$,
one constructs the \textbf{right derived functor}
\begin{equation*}
    RF : \cat{D}^{+}(\Aa) \longrightarrow \cat{D}^{}(\Bb)\,,
\end{equation*}
and for a \textbf{right exact} functor $G:\Aa \longrightarrow \Bb$
one constructs the \textbf{left derived} functor
\begin{equation*}
    LG : \cat{D}^{-}(\Aa) \longrightarrow \cat{D}^{}(\Bb)\,.
\end{equation*}
They can be defined by a universal property:

\begin{df}
    Let $F:\cat{K}^{+}(\Aa) \to \cat{K}^{}(\Bb)$ be a triangulated functor.
    A \textbf{right derived functor} for $F$ is a triangulated functor
    \begin{equation*}
        RF : \cat{D}^{+}(\Aa) \longrightarrow \cat{D}^{}(\Bb)\,,
    \end{equation*}
    together with a morphism of functors
        $\epsilon : Q \circ F \to RF \circ Q$ such that,
        for any pair $(\Phi,\eta)$ of a triangulated functor
        $\Phi:\cat{D}^{+}(\Aa) \to \cat{D}^{+}{(\Bb)}$ and natural
        transformation $\eta : Q \circ F \to \Phi \circ Q$,
        there exists a unique morphism of functors
        $\alpha : RF \to \Phi$ such that
        $\eta = \alpha_{Q(-)} \circ \epsilon$;
        more explicitly, for every $A^{\bullet} \in \cat{K}^{+}(\Aa)$,
        in $\cat{D}^{}(\Bb)$ we have the following commutative diagram:
        \begin{equation*}
            \begin{tikzcd}
                Q \circ F(A^{\bullet}) \ar[rr, "\eta_{A^{\bullet}}"]
                \ar[dr, "\epsilon_{A^{\bullet}}"']
                & & \Phi \circ Q(A^{\bullet}) \\
                & RF \circ Q (A^{\bullet}) \ar[ur, "\alpha_{Q(A^{\bullet})}"'] & \,.
            \end{tikzcd}
        \end{equation*}

    Similarly, if $G:\cat{K}^{-}(\Aa) \to \cat{K}^{-}(\Bb)$ is a triangulated functor,
    a \textbf{left derived functor} for $G$ is a triangulated functor
    \begin{equation*}
        LG : \cat{D}^{-}(\Aa) \longrightarrow \cat{D}^{-}(\Bb)\,,
    \end{equation*}
    together with a morphism of functors
        $\epsilon : LG \circ Q \to Q \circ F$ such that,
        for any pair $(\Phi,\eta)$ of a triangulated functor
        $\Phi:\cat{D}^{-}(\Aa) \to \cat{D}^{-}{(\Bb)}$ and natural
        transformation $\eta : \Phi \circ Q \to Q \circ G$,
        there exists a unique morphism of functors
        $\beta : \Phi \to LG$ such that
        $\eta = \epsilon \circ \beta_{Q(-)}$;
        more explicitly, for every $A^{\bullet} \in \cat{K}^{-}(\Aa)$,
        in $\cat{D}^{-}(\Bb)$ we have the following commutative diagram:
        \begin{equation*}
            \begin{tikzcd}
                \Phi \circ Q(A^{\bullet}) \ar[rr, "\eta_{A^{\bullet}}"]
                \ar[dr, "\beta_{Q(A^{\bullet})}"']
                & & Q \circ G(A^{\bullet}) \\
                & LG \circ Q (A^{\bullet}) \ar[ur, "\epsilon_{A^{\bullet}}"'] & \,.
            \end{tikzcd}
        \end{equation*}
\end{df}

Of course, when $LG$ or $RF$ exists, 
it is unique up to unique isomorphism,
for it is defined by a universal property.
A sufficient condition for derived functors
to exist is the existence of enough injectives
and projectives.

\begin{prop}\label{derived-functor}
    If $F:\cat{K}^{+}(\Aa) \to \cat{K}^{}(\Bb)$ is
    a triangulated functor and $\Aa$ has enough
    injective objects, then the right derived functor
    $RF$ exists.

    Similarly, if $\Aa$ has enough projectives
    and $G:\cat{K}^{-}(\Aa) \to \cat{K}^{}(\Bb)$
    is triangulated, then $LG$ exists.
    \begin{proof}
        We do the case of the right derived functor:
        if $\Ii_{\Aa}$ is the full subcategory of injective objects of $\Aa$, 
        then consider $\psi:\cat{D}^{+}(\Aa) \to \cat{K}^{+}(\Ii_{\Aa})$
        a quasi-inverse of the equivalence stated in 
        \hyperref[inj-equivalence]{Theorem~\ref*{inj-equivalence}}
        and define $RF$ to be the composition
        $RF := Q \circ F\vert_{\cat{K}^{+}(\Ii_{\Aa})} \circ \psi$,
        i.e.
        \begin{equation*}
            \begin{tikzcd}
                \cat{K}^{+}(\Ii_{\Aa}) \ar[r, hook] \ar[d, "Q"]
                & \cat{K}^{+}(\Aa) \ar[r, "F"]
                & \cat{K}^{}(\Bb) \ar[d, "Q"] \\
                \cat{D}^{+}(\Aa) \ar[u, "\psi", bend left] 
                \ar[rr, dashed, "RF"] & & \cat{D}^{}(\Bb)\,.
            \end{tikzcd}
        \end{equation*}

        To define $\epsilon: Q \circ F \to RF \circ Q$
        on objects, let $A^{\bullet}$ be a complex bounded below
        and consider a qis $A^{\bullet} \to I^{\bullet}$
        in $\cat{K}^{+}(\Aa)$, 
        where $I^{\bullet} \in \cat{K}^{+}(\Ii_{\Aa})$,
        which exists because of the equivalence 
        $\cat{D}^{+}(\Aa) \simeq \cat{K}^{+}(\Ii_{\Aa})$.
        By applying $Q$ to the induced morphism 
        $F(A^{\bullet}) \to F(I^{\bullet})$, one gets a morphism
        \begin{equation*}
            \epsilon_{A^{\bullet}} :
            Q \circ F(A^{\bullet}) 
            \longrightarrow Q \circ F(I^{\bullet}) 
            \simeq Q \circ F \circ \psi \big( Q(I^{\bullet}) \big)
            \simeq RF \circ Q (A^{\bullet})\,,
        \end{equation*}
        where we used that $Q(I^{\bullet}) \simeq Q(A^{\bullet})$.
        One can check that $\epsilon$ is natural because,
        given any map $f^{\bullet} : A^{\bullet} \to B^{\bullet}$
        and quasi-isomorphisms $A^{\bullet} \to I^{\bullet}$
        and $B^{\bullet} \to J^{\bullet}$, there is a unique
        (up to homotopy) induced map $I^{\bullet} \to J^{\bullet}$
        in $\cat{K}^{+}(\Ii_{\Aa})$, 
        making the following diagram in $\cat{K}^{+}(\Aa)$ commute:
        \begin{equation*}
            \begin{tikzcd}
                A^{\bullet} \ar[r, "\textrm{qis}"] \ar[d, "f^{\bullet}"']
                & I^{\bullet} \ar[d, dashed, "\exists !"] \\
                B^{\bullet} \ar[r, "\textrm{qis}"] & J^{\bullet}\,.
            \end{tikzcd}
        \end{equation*}

        Consider now any $(\Phi,\eta)$ as in the definition of derived functor,
        and we show that $\eta$ factors through $\epsilon$:
        as above, let $A^{\bullet} \simeq I^{\bullet}$, where $I^{\bullet}$.
    \end{proof}
\end{prop}

The above result actually holds in a more general setting:
given a functor $F$ between homotopy categories,
the right and the left derived functors exist whenever
we find a class of objects adapted to the functor.

\begin{df}
    Given a triangulated functor $F:\cat{K}^{*}(\Aa) \to \cat{K}^{}(\Bb)$,
    a triangulated subcategory $\Kk_{F} \subset \cat{K}^{*}(\Aa)$
    is \textbf{$F$-adapted} if it satisfies the following two conditions:
    \begin{rmnumerate}
        \item if $A^{\bullet} \in \Kk_{F}$ is acyclic,
        then $F(A^{\bullet})$ is acyclic;

        \item any $A^{\bullet} \in \cat{K}^{*}(\Aa)$ is quasi-isomophic
        to a complex in $\Kk_{F}$.
    \end{rmnumerate}
\end{df}

We can define the notion of adapted class already
on the level of the abelian category $\Aa$.

\begin{df}
    Let $F:\Aa \to \Bb$ be a left (resp. right) exact functor between abelian categories.
    A class of objects $\Ii_{F} \subset \Aa$ is \textbf{$F$-adapted}
    if the following conditions hold true:
    \begin{rmnumerate}
        \item the class $\Ii_{F}$ is closed under finite sums, 
        i.e. given $A, B \in \Ii_{F}$, then $A \oplus B \in \Ii_{F}$;
        \item if $A^{\bullet} \in \cat{K}^{+}(\Ii_{F})$ 
        (resp. $\cat{K}^{-}(\Ii_{F})$) is acyclic,
        then $F(A^{\bullet})$ is acyclic too;
        \item any object of $\Aa$ can be embedded into an
        object of $\Ii_{F}$.
    \end{rmnumerate}
\end{df}

One can prove that, whenever $\Ii_{F}$ is an adapted class for a
left exact functor $F$, then the localization 
of $\cat{K}^{+}(\Ii_{F})$ by the class of quasi-isomorphisms
is equivalent to the derived category $\cat{D}^{+}(\Aa)$;
for a complete description of this fact, 
check \parencite[Proposition~III.6.4]{gelfand}:
the idea is that $F$ sends qis of complexes in $\Ii_{F}$
to qis by condition (ii), while condition (iii) ensures
that we can identify any object $A \in \Aa$ with an
object of $\cat{K}^{+}(\Ii_{F})$, as it happens
with injetive resolutions.
In fact, this is not by chance: whenever $\Aa$ has enough
injectives, the class $\Ii_{\Aa}$ spanned by injective objects
is $F$-adapted for \emph{any} left exact functors, indeed
we know that $I \oplus J$ is injective whenever $I, J \in \Ii$;
left exactness guarantees (ii) and (iii) is the definition of
injective resolution.

Whenever $\Ii_{F} \subset \Aa$ is an $F$-adapted class 
with respect to a left exact functor $F:\Aa \to \Bb$,
one can prove that the full subcategory
$\cat{K}^{+}(\Ii_{F})$ is an adapted
with respect to the induced homotopy functor
$\Tilde{F}:\cat{K}^{+}(\Aa) \to \cat{K}(\Bb)$,
thus there exists the right derived functor,
that we will write as
\begin{equation*}
    RF : \cat{D}^{+}(\Aa) \longrightarrow \cat{D}(\Bb)\,.
\end{equation*}
An analogous statement is true for 
right exact functors and the class of projective objects;
for a complete treatment, see \parencite[Theorem~III.6.8]{gelfand}.
By assuming these general constructions,
we can deduce the following

\begin{cor}
    Let $F:\Aa \to \Bb$ be an additive functor between abelian categories.
    \begin{itemize}
        \item If $\Aa$ has enough injective objects and $F$ is left exact,
        then the right derived functor $RF: \cat{D}^{+}(\Aa) \to \cat{D}(\Bb)$ exists.
        \item If $\Aa$ has enough projective objects and $F$ is right exact,
        then the left derived functor $LF: \cat{D}^{-}(\Aa) \to \cat{D}(\Bb)$ exists.
    \end{itemize}
\end{cor}

\begin{df}
    Whenever the right derived functor $RF$ of a left exact functor $F:\Aa \to \Bb$ exists,
    for every $i \in \Z$ we define
    \begin{equation*}
        R^{i}F : \cat{D}^{+}(\Aa) \longrightarrow \Bb\,,
        \qquad A^{\bullet} \longmapsto H^{i}\big(RF(A^{\bullet})\big)\,.
    \end{equation*}
    By precomposing with the canonical embedding $\Aa \subset \cat{D}^{+}(\Aa)$,
    we can define, for every $i \in \Z$, the \textbf{$i$-th higher derived functor}
    $R^{i}F : \Aa \longrightarrow \Bb$.
\end{df}

The proof of \hyperref[derived-functor]{Proposition~\ref*{derived-functor}}
shows a way to compute the higher derived functors of $F$: indeed,
whenever $\Aa$ has enough injectives, for any $A \in \Aa$
we can pick an injective resolution $\cat{0} \to A \to I^{\bullet}$ and,
by the the natural isomorphism $\epsilon : Q \circ F \simeq RF \circ Q$,
we have 
\begin{equation*}
    RF^{i}(A) = H^{i}\big( RF(A[0]) \big) 
    %\simeq H^{i}\big(Q(I^{\bullet})\big) 
    \simeq H^{i}\big(F(I^{\bullet})\big)\,.
\end{equation*}
It follows that, if $i<0$, the higher derived functors are $RF^{i}=0$
and $R^{0}F(A) \simeq F(A)$ by left exactness:
\begin{equation*}
    R^{0}F(A) = H^{0}\big(F(I^{\bullet})\big) 
    = \ker\big(F(I^{0}) \to F(I^{1})\big) \simeq F(A)
\end{equation*}

\begin{cor}
    Let $F : \Aa \to \Bb$ be a left exact functor,
    where $\Aa$ has enough injectives. Then every short exact sequence in $\Aa$
    \begin{equation*}
        \begin{tikzcd}
            \cat{0} \ar[r]
            & A \ar[r] 
            & B \ar[r]
            & C \ar[r]
            & \cat{0}
        \end{tikzcd}
    \end{equation*}
    gives rise in $\Bb$ to the long exact sequence
    \begin{equation*}
        \begin{tikzcd}[column sep=small]
            \cat{0} \ar[r]
            & F(A) \ar[r] 
            & F(B) \ar[r]
            & F(C) \ar[r]
            & R^{1}F(A) \ar[r]
            & R^{1}F(B) \ar[r]
            & \dots \\
            \dots \ar[r]
            & R^{i}F(A) \ar[r] 
            & R^{i}F(B) \ar[r]
            & R^{i}F(C) \ar[r]
            & R^{i+1}F(A) \ar[r]
            & R^{i+1}F(B) \ar[r]
            & \dots
        \end{tikzcd}
    \end{equation*}
    \begin{proof}
        According to \hyperref[SES-TRI]{Exercise~\ref*{SES-TRI}},
        any short exact sequence in $\Aa$ gives a distinguished triangle
        $A \to B \to C \to A[1]$ in $\cat{D}^{+}(\Aa)$.
        Since $RF$ is triangulated, the triangle 
        $RF(A) \to RF(B) \to RF(C) \to RF(A)[1]$ 
        is distinguished in $\cat{D}^{+}(\Bb)$,
        thus it induces a LECS by \hyperref[derived-LECS]{Exercise~\ref*{derived-LECS}},
        which is the desired long exact sequence in $\Bb$.
    \end{proof}
\end{cor}

\begin{exercise}
    Let $F$ be a left exact functor and $\Ii_{F}$ an $F$-adapted class in $\Aa$.
    We say that $A \in \Aa$ is \textbf{$F$-acyclic} if $R^{i}F(A) \simeq \cat{0}$
    for all $i \ne 0$. Show that we obtain an $F$-adapted class by
    enlarging $\Ii_{F}$ by all $F$-acyclic objects.
    \begin{proof}
        It is enough to check the three conditions in the definition of 
        $F$-adapted class:
        \begin{rmnumerate}
            \item by additivity of the functors $R^{i}F$, it holds
            \begin{equation*}
                R^{i}F(A \oplus B) \simeq R^{i}F(A) \oplus R^{i}F(B)\,,
            \end{equation*}
            thus the sum of finitely many $F$-acyclic objects is still $F$-acyclic.
            Nevertheless, the sum $I \oplus A$ of some $I \in \Ii_{F}$ 
            and $A$ an $F$-acyclic object needs not be neither in $\Ii_{F}$,
            nor $F$-acyclic. Thus, when we say ``enlarge'',
            we mean we need to consider also direct sums of these objects;

            \item consider 

            \item any objects of $\Aa$ can be embedded into some object of $\Ii_{F}$,
            so it still remains true if we enlarge the class.
        \end{rmnumerate}
    \end{proof}
\end{exercise}

\begin{ex}[\textbf{Kernel}]
    Let $\Aa$ be an abelian category.
    Let $\Aa^{\{\ast \to \ast\}}$ be the category whose
    objects are maps $f:A \to A'$ in $\Aa$;
    a morphism $\Phi$ from $f:A \to A'$ to $g:B \to B'$
    is a pair $(\phi,\phi')$ of arrows $\phi:A \to B$
    and $\phi':A' \to B'$ such that the following square commutes:
    \begin{equation*}
        \begin{tikzcd}
            A \ar[r, "\phi"] \ar[d,"f"'] & B \ar[d, "g"]\\
            A' \ar[r, "\phi'"] & B'\,.
        \end{tikzcd}
    \end{equation*}
    Given any two morphisms $\Phi = (\phi,\phi') : f \to g$
    and $\Psi = (\psi,\psi'):g \to h$, set $\Psi \circ \Phi$
    to be the componentwise composition, i.e.
    $$\Psi \circ \Phi := (\psi \circ \phi,\psi' \circ \phi')\,.$$
    One can check this composition is associative,
    making $\Aa^{\{\ast \to \ast\}}$ into a category.
    Moreover, $\Aa^{\{\ast \to \ast\}}$ is abelian:
    \begin{itemize}
        \item[(\textbf{A1})] if we write $\cat{0}$
        for the zero object of $\Aa$, then the zero morphism
        $0: \cat{0} \to \cat{0}$ is the zero object of 
        $\Aa^{\{\ast \to \ast\}}$;

        \item[(\textbf{A2})] the existence of finite products
        and coproducts in $\Aa^{\{\ast \to \ast\}}$ 
        follows by their existence in $\Aa$: 
        indeed, given $f:A \to A'$ and $g:B \to B'$,
        their product is given by $f \oplus g : A \oplus B \to A' \oplus B'$
        because, for any object $t:C \to C'$ and 
        any pair of maps $t \to f, t \to g$, we have the factorization
        \begin{equation*}
            \begin{tikzcd}
                   & C \arrow[ld] \arrow[rrd] \arrow[dd, "t"', near end] \arrow[rd, "\exists !"', dashed] &                                                                  &                   \\
A \arrow[dd, "f"'] &                                                                            & A \oplus B \arrow[ll, crossing over] \arrow[r]  & B \arrow[dd, "g"] \\
                   & C' \arrow[ld] \arrow[rrd] \arrow[rd, "\exists !"', dashed]                 &                                                                  &                   \\
A'                 &                                                                            & A' \oplus B' \arrow[ll] \arrow[r] \arrow[from=uu, "\exists !"', dashed, crossing over]                               & B'         \,,          
\end{tikzcd}
        \end{equation*}
        which can be rewritten as the following commutative diagram
        in $\Aa^{\{\ast \to \ast\}}$
        \begin{equation*}
            \begin{tikzcd}
                & t \ar[d, dashed, "\exists !"'] \ar[dr] \ar[dl] & \\
                f & f \oplus g \ar[l] \ar[r] & g\,.
            \end{tikzcd}
        \end{equation*}
        The coproduct of $f$ and $g$ is given again by $f \otimes g$
        because $A \times B \simeq A \amalg B$ in $\Aa$, which is indeed the direct sum;

        \item[(\textbf{A3})] we already know that $\Aa$ has kernels, 
        and hence for any $\Phi:f \to g$ in 
        $\Aa^{\{\ast \to \ast\}}$ we have
        \begin{equation*}
            \begin{tikzcd}
                \ker \phi \ar[r, "i", hook] \ar[d, "\exists !"', "k", dashed]
                & A \ar[r, "\phi"] \ar[d, "f"] 
                & B \ar[d, "g"] \\
                \ker \phi' \ar[r, hook]
                & A' \ar[r, "\phi'"]
                & B'\,,
            \end{tikzcd}
        \end{equation*}
        where the vertical arrow $k$ is the unique map
        described by the universal property of $\ker \phi'$,
        which exists by $\phi' \circ f \circ i 
        = g \circ \phi \circ i = 0$.
        We claim that $k = \ker \Phi$: given any $t:C \to C'$
        and $\Psi:t \to f$ such that $\Phi \circ \Psi = 0$,
        then in $\Aa$ we have the commutative diagram
        \begin{equation*}
            \begin{tikzcd}
C \arrow[rrd, shorten=1mm] \arrow[rrrd, "0", shorten=3mm] \arrow[rd, "\exists !"', dashed] \arrow[dd, "t"'] &                                            &                                       &                   \\
                                                                                  & \ker \phi \arrow[r, hook]  & A \arrow[r, "\phi"']  & B \arrow[dd, "g"] \\
C' \arrow[rrd, shorten=1mm] \arrow[rrrd, "0", shorten=3mm] \arrow[rd, "\exists !"', dashed]                 &                                            &                                       &                   \\
                                                                                  & \ker \phi' \arrow[from=uu, "k"', crossing over] \arrow[r, hook]                 & A' \arrow[from=uu, "f"', crossing over]\arrow[r, "\phi'"']                & B'     \,,          
\end{tikzcd}
        \end{equation*}
        which can be written in $\Aa^{\{\ast \to \ast\}}$ as the commutative diagram
        \begin{equation*}
            \begin{tikzcd}
                t \ar[rd, "\Psi"'] \ar[rrd, "0"] \ar[d, dashed] & & \\
                k \ar[r] & f \ar[r, "\Phi"'] & g\,,
            \end{tikzcd}
        \end{equation*}
        which proves $k = \ker \Phi$. 
        Similarly, one shows that $\Coker \Phi$ is given by the induced map
        on cokernels;
        
        \item[(\textbf{A4})] consider $\Phi = (\phi,\phi') : f \to g$ 
        in $\Aa^{\{\ast \to \ast\}}$. Since axiom \hyperref[A4]{(\textbf{A4})} 
        holds in $\Aa$, then $\operatorname{coim} \phi \simeq \operatorname{im} \phi$
        and $\operatorname{coim} \phi' \simeq \operatorname{im} \phi'$, which means
        that $\operatorname{coim} \Phi \simeq \operatorname{im} \Phi$ beacuse
        isomorphisms in $\Aa^{\{\ast \to \ast\}}$ are given by pairs of isomorphisms.
    \end{itemize}

    We now define the functor
    \begin{equation*}
        \ker: \Aa^{\{\ast \to \ast\}} \longrightarrow \Aa\,,
        \qquad \big( f \to g \big) 
        \longmapsto \big( \ker f \to \ker g \big)\,.
    \end{equation*}
    This is an additive functor between abelian categories, 
    and we claim it is left exact: given a short exact sequence 
    $0 \to f \to g \to h \to 0$ in $\Aa^{\{\ast \to \ast\}}$,
    in $\Aa$ we get a commutative diagram of the form
    \begin{equation*}
        \begin{tikzcd}
            %\cat{0} \ar[r]
            & \ker f \ar[d, hook] \ar[r, dashed, "i"]
            & \ker g \ar[d, hook] \ar[r, dashed, "j"]
            & \ker h \ar[d, hook] 
            & \\
            \cat{0} \ar[r]
            & A \ar[d, "f"] \ar[r]
            & B \ar[d, "g"] \ar[r]
            & C \ar[d, "h"] \ar[r]
            & \cat{0} \\
            \cat{0} \ar[r]
            & A' \ar[r]
            & B' \ar[r]
            & C' \ar[r]
            & \cat{0} \,.
        \end{tikzcd}
    \end{equation*}
    Notice that $i:\ker f \to \ker g$ is a monomorphism, 
    for given $t:W \to \ker f$ such that $i \circ t = 0$,
    then the composition $W \to A \to B$ is zero again,
    so that $t$ factors through $\ker\big(A \to B) \simeq \cat{0}$.
    Thus $\ker i \simeq \cat{0}$ and this shows $\ker$ is exact on the left.
    To show exactness in $\ker g$, noticce that 
    the exactness of the middle row implies $j \circ i = 0$,
    so that we have an induced map $\operatorname{im} i \to \ker j$.
    Conversely, we can find a map $\ker j \to \operatorname{im} i$
    which will be the inverse of the above 
    $\ker j \to \operatorname{im} i$
    because of the uniqueness guaranteed by the universal property.
    To get the desired map, one sees that the composition
    \begin{equation*}
        \ker j \to \ker g \to B \xhookrightarrow{\psi} C
    \end{equation*}
    is the zero map, and hence we get a factorization through
    $\ker j \to \ker \phi$; by exactness, 
    $\operatorname{im} \phi \simeq \ker \psi$, and by composing
    $\ker j \to \operatorname{im} \phi \to \operatorname{im}\phi'$
    we get the zero morphism (because it commutes with $\ker j \to \ker g \to B \to B'$), and hence we can lift $\ker j \to \operatorname{im} i$.
    \begin{equation*}
\begin{tikzcd}
                                                         &   \operatorname{im} i                                                               & \ker j \arrow[rd, hook] \arrow[rrrdd, "0", bend left] \arrow[ddd, dashed, crossing over] \ar[l, "\exists !"', dashed] &                                      &  &                   \\
\ker f \arrow[d, hook] \arrow[rrr, "i"] \ar[ru, dashed]                 &                                                                 &                                                                           & \ker g \arrow[d, hook]               &  &                   \\
A \arrow[rrr, "\phi"] \arrow[rd, dashed] \arrow[dd, "f"'] &                                                                 &                                                                           & B \arrow[rr, "\psi"] \arrow[dd, "g"] &  & C \arrow[dd, "h"] \\
                                                         & \operatorname{im}\phi \ar[from=uuu, crossing over] \arrow[r, no head, equals] & \ker \psi \arrow[ru, hook] \arrow[from=uuu, dashed, crossing over]                                              &                                      &  &                   \\
A' \arrow[rrr, "\phi'"]  \arrow[rd, dashed]               &                                                                 &                                                                           & B' \arrow[rr]                        &  & C'                \\
                                                         & \operatorname{im}\phi' \arrow[rru, hook]   \arrow[from=uu, crossing over]                     &                                                                           &                                      &  &        \,.          
\end{tikzcd}
    \end{equation*}

    Now that we know $\ker$ is a left exact functor,
    assume $\Aa$ has enough injective objects; it follows that
    $\Aa^{\{\ast \to \ast\}}$ has enough injectives too:
    indeed, a map $I \to J$ between injective objects in $\Aa$
    is injective as an object of $\Aa^{\{\ast \to \ast\}}$,
    and by the dual version of \hyperref[res-htp]{Proposition~\ref*{res-htp}}
    we can always find an injective resolution
        \begin{equation}\label{ladder-ker}
            \begin{tikzcd}
                \cat{0} \ar[r]
                & A \ar[r] \ar[d, "f"]
                & I^{0} \ar[r] \ar[d, "j^{0}"] 
                & I^{1} \ar[r] \ar[d, "j^{1}"]
                & I^{2} \ar[r] \ar[d, "j^{2}"]
                & \dots \\
                \cat{0} \ar[r]
                & A' \ar[r] 
                & J^{0} \ar[r]  
                & J^{1} \ar[r] 
                & J^{2} \ar[r] 
                & \dots
            \end{tikzcd}
        \end{equation}
    for any object $f \in \Aa^{\{\ast \to \ast\}}$.
    This means that we can compute $\ker$'s higher right derived functors:
    given any $f : A \to A'$, 
    we already know that $R^{0}\ker f \simeq \ker f$,
    so now consider an injective resolution $f \to j^{\bullet}$
    as in \eqref{ladder-ker}.\todo{Finish this exercise.}
\end{ex}

%%	

\section{Spectral sequences}

\todo{Guarda Fomenko per intro e esempi, Manin per approfondimento, tutto lungo Huybrechts.}
%
%%%%%%%%%%%%%%%%%%%%%%%%%%%%%%%%%%%%%%%%%%%%%%%%%%%%%%%%%%%%
%%%%%%%%%%%%%%%%%%%%%%%%%%%%%%%%%%%%%%%%%%%%%%%%%%%%%%%%%%%%
%%%%%%%%%%%%%%%%%%%%%%%%%%%%%%%%%%%%%%%%%%%%%%%%%%%%%%%%%%%%
% to here!

\chapter{Differential graded algebras}
	
	
For the whole chapter, fix a commutative ring $\Lambda$ 
with unit $1 \ne 0$. We are going to study new
additional structures on $\Lambda$-algebras, 
namely \emph{differential} and \emph{graded} structures;
the aim of this is to define and understand their main properties,
and after that we will be able to
approach the related concept of graded modules over these algebras. 
In the geometric setting, our main interest will be the special case
of the semisimple algebra $\Lambda = k$,
for some field and $m \le 1$.

The differential structure on these new objects
allows us to talk about \emph{cohomology}.
It is well known that isomorphic complexes
yield the same information in cohomology,
while the converse is false in general;
thus, we are going to focus on a class of very well-behaved 
graded algebras, \emph{intrinsically formal} algebras:
strictly speaking, these objects are determined
by their cohomology. 

Intrinsic formality of DGA-algebras
can be characterized in terms of vanising
of some \emph{Hochschild cohomology groups},
thus we safe a section to define this notion.

\section{DG-algebras}

	Every $\Lambda$-algebra in this chapter is associative,
	not necessarily commutative,
	with unit $1 \ne 0$.
	
	\begin{df}
		A $\Lambda$-algebra $A$ is \textbf{graded algebra}
		if there exist submodules $\Set{A^{k} | k \in \Z}$,
		such that $1 \in A^{0}$ and,
		for every $i,j \in \Z$, we have multiplications
		\begin{equation*}
			A^{i} \otimes_{\Lambda} A^{j} \longrightarrow A^{i+j}\,,
			\quad x \otimes y \longmapsto xy\,,
		\end{equation*}
		such that $A = \bigoplus_{k \in \Z} A^{k}$.
		An element of $A^{k}$ is said \textbf{homogeneous of degree $k$};
		sometimes we will simply write $|x|$ instead of $k$,
		for a homogeneous element $x \in A^{k}$.
		
		A graded $\Lambda$-algebra is \textbf{graded-commutative} 
		(or \textbf{anti-}, sometimes called \textbf{super-}commutative),
		if it holds
		\begin{equation*}
			yx = (-1)^{kh} xy\,,
			\quad \text{for every } x \in A^{k}, y \in A^{h}\,.
		\end{equation*}
	\end{df}
	
	The concept of differential graded algebra $A$ is not new:
	it boils down to a cochain complex whose differential
	must satisfy some compatibility axiom with respect to the multiplication.
	
	\begin{df}
		A \textbf{differential} on a graded algebra $A$ is a
		$\Lambda$-linear endomorphism $d:A \to A$
		such that $d^{2} = 0$ and, for every $k \in \Z$,
		it holds $d(A^{k}) \subset A^{k+1}$;
		moreover, $d$ must satisfy the following \textbf{graded Liebniz rule}:
		\begin{equation}\label{graded-liebniz}
		 	d(xy) = (dx)y + (-1)^{k}x (dy)\,,
		 	\quad \text{for } x \in A^{k}\,.
		 \end{equation} 
		A \textbf{DG-algebra} $\Aa = (A,d)$
		is a graded $\Lambda$-algebra $A$ endowed with a differential $d$.
	\end{df}
	
	\begin{rmk}
		Notice that \eqref{graded-liebniz} implies that $d(1)=0$.
	\end{rmk}	
	
	The condition $d(A^{k}) \subset A^{k+1}$ in the above definition
	allows us to interpret a DG-algebra $\Aa = (A,d)$ as a
	sequence
	\begin{equation*}
		\begin{tikzcd}
			\dots \ar[r] & A^{k-1} \ar[r, "d"] \ar[rr, bend right=20, "0"']
			& A^{k} \ar[r, "d"] 
			& A^{k+1} \ar[r] & \dots
		\end{tikzcd}
	\end{equation*}
	thus, any DG-algebra is a cochain complex $\Aa = A^{\bullet}$
	of $\Lambda$-modules,
	whose coboundary map is the same $d$ at each level;
	this means we can compute its cohomology
	$H^{*}(\Aa)$. The interesting fact is that 
	this is not just a module:
	indeed, the relation \eqref{graded-liebniz} implies that
	the multiplication induced on $H^{*}(\Aa)$ 
	is well defined and moreover
	\begin{equation*}
		H^{k}(\Aa) \otimes H^{h}(\Aa) \longrightarrow H^{k+h}(\Aa)\,,
		\quad [x] \cdot [y] = [xy]
	\end{equation*}
	shows that the cohomology inherits a graded $\Lambda$-algebra 
	structure $H^{*}(\Aa) = \oplus_{k} H^{k}(\Aa)$.
	By endowing it with a trivial differential $d=0$,
	we conclude that the cohomology $H^{*}(\Aa)$ of a DG-algebra $\Aa$
	is again a DG-algebra.
	
	
	We define $\cat{DG}_{\Lambda}$ to be the category whose objects
	are DG-algebras over $\Lambda$, 
	and morphisms $f:\Aa \to \Bb$ between them
	given by homomorphisms of unital $\Lambda$-algebras
	\begin{equation*}
		f : A \longrightarrow B\,, \quad 
		\text{such that } f\left(1_{A}\right) = 1_{B}\,,
	\end{equation*}
	which are also maps of complexes, that is 
	$f(A^{k}) \subset B^{k}$ and $f \circ d = \delta \circ f$.
	One can check that $\cat{DG}_{\Lambda}$ is an abelian category
	(we might expect this because it is a category of modules).
	
	\begin{df}
		Given a DG-algebra $\Aa = (A,d)$, we define its \textbf{opposite algebra}
		$\Aa^{op} = (A^{op},d)$ as the DG-algebra whose elements and differential
		are the same of $\Aa$, but we consider a new multiplication
		\begin{equation*}
			a \cdot^{op} b := (-1)^{|a|\,|b|} ba\,,
		\end{equation*}
		where $ba$ denotes the usual multiplication in $\Aa$.
		Notice that $\Aa$ is graded-commutative if and only if $\Aa = \Aa^{op}$.
	\end{df}
	
	\begin{ex}
		Let $X$ be a smooth real manifold.
		The algebra $\Omega^{\bullet}(X)$
		of smooth differential forms endowed with 
		the \emph{exterior derivative} is a DG-algebra
		and $\Omega^{\bullet}$ determines a functor
		\begin{equation*}
			\Omega^{\bullet} : \cat{Man} \longrightarrow \cat{DG}_{\R}\,,
		\end{equation*}
		where $\cat{Man}$ is the category of smooth manifolds and smooth maps.
	\end{ex}
	
	\begin{ex}
		Taking cohomology defines a covariant functor 
		$H^{*}:\cat{DG}_{\Lambda} \to \cat{DG}_{\Lambda}$.
	\end{ex}
	
	\begin{ex}
		Given a finite dimensional $k$-vector space $V$, 
		define $T^{0} := k$ and for each $k \ge 1$ set
		\begin{equation*}
			T^{k}(V) := T^{\otimes k} 
			= \underbrace{V \otimes_{k} \dots \otimes_{k} V}_{k \text{ times}}\,.
		\end{equation*}
		Then $T(V) := \bigoplus_{k \ge 0} T^{k}(V)$ naturally 
		inherits an associative multiplication
		\begin{equation*}
			T^{k}(V) \otimes_{k} T^{h}(V) \longrightarrow T^{k+h}(V)\,;
		\end{equation*}
		from the canonical isomorphism $k \otimes_{k} V \simeq V$ 
		we deduce that $1 \in T^{0}(V)$ is the unit of the multiplication,
		hence $T(V)$ is a graded algebra over $k$,
		called \textbf{tensor algebra} of $V$.
		
		Let $\{e_{1}, \dots, e_{n}\}$ be a basis of $V$,
		and set $A^{-k}:=T^{k}(V)$.
		We can define a differential $d:T(V) \to T(V)$
		by setting on basis elements 
		\begin{equation*}
			d(e_{i}) = (-1)^{i}\,, \quad i \in \Set{1, 2, \dots, n}\,,
		\end{equation*}
		and then extending it componentwise to a map
		\begin{equation*}
			d : A^{-k} \longrightarrow A^{-k+1}\,, \quad
			d(e_{i_{1}} \otimes \dots \otimes e_{i_{k}})
			= \sum_{j=1}^{k} e_{i_1} \otimes \dots 
			\otimes d(e_{i_{j}}) \otimes \dots \otimes e_{i_{k}}\,.
		\end{equation*}
	\end{ex}
	
	We write $\iota_{\Aa} : \Lambda \to A$ to be the structure map
	$\iota(\lambda) := \lambda \cdot 1_{A}$.
	
	\begin{df}
		An \textbf{augmentation} on a DG-algebra $\Aa$
		is a morphism $\epsilon : \Aa \to \Lambda$
		of unital $\Lambda$-algebras such that $\epsilon \circ d = 0$ and
		$\epsilon \circ \iota_{\Aa} = \cat{1}_{\Lambda}$.
		Its kernel is a two-sided ideal of $\Aa$,
		called the \textbf{augmentation ideal} $\Aa^{+} := \ker \epsilon$.
	\end{df}
	
	Notice that the condition $\epsilon d = 0$ implies
	that there is a well defined aumentation on $H^*(\Aa)$.
	Since $\epsilon$ is a retraction on $\Lambda$,
	we may identify the ground ring $\Lambda$ with a subring of $A$;
	in particular, notice that $\Lambda$ is a direct summand of $A$
	by the \hyperref[split-lemma]{Split Lemma}.
	Thus, an augmentation is determined by its restriction to $A^{0}$.
	
	From now on, we will consider DG-algebras with terms of non-negative degree,
	that is $A^{k} = \cat{0}$ for $k<0$. In this case,
	the augmetation ideal coincides with the
	submodule of positive degree elements
	$\Aa^{+} = \bigoplus_{k \ge 0} A^{k}$.
	If moreover $\Aa$ is \textbf{connected},
	that is $A^{k} = \cat{0}$ for $k<0$ and $\iota_{\Aa}:\Lambda \to A^{0}$
	is an isomorphism, then $\Aa$ has a unique augmentation map
	given by $\iota_{\Lambda}^{-1}$.
	

	
	
	\begin{df}
		A DG-algebra $\Aa$ endowed with an augmentation $\epsilon$
		will be called \textbf{DGA-algebra} 
		(which stands for \textbf{differential augmented graded algebra}).
		Given two DGA-algebras 
		$\Aa = (A,d,\epsilon)$ and $\Aa' = (A',d',\epsilon')$,
		a \textbf{DGA-homomorphism} is a morphism
		of DG-algebras compatible with augmentations,
		that is $\epsilon' \circ f = \epsilon$.
		\begin{equation*}
			\begin{tikzcd}
				A \ar[rr, "f"] \ar[dr, "\epsilon"'] & & A' \ar[dl, "\epsilon'"] \\
				& \Lambda & \,.
			\end{tikzcd}
		\end{equation*}
	\end{df}
	
	We will write $\cat{DGA}_{\Lambda}$ for the category of
	DGA-algebras over $\Lambda$.
	Notice that a DGA-homomorphism induces a morphism $f*:H^*(\Aa) \to H^*(\Aa')$
	of DGA-algebras, hence cohomology defines a covariant functor
	$H^*:\cat{DGA}_{\Lambda} \to \cat{DGA}_{\Lambda}$.
	
	\begin{ex}
		Given a topological space $X$, let $S_{\bullet}(X)$ denote its
		singular simplicial complex. For each pair of spaces $X,Y$,
		one can define a chain map 
		$\sigma : S_{\bullet}(X) \otimes_{\Z} S_{\bullet}(Y) \to S_{\bullet}(X \times Y)$.
		Now, assume $X$ is a topological group, 
		with a continuous multiplication
		\begin{equation*}
			\mu: X \times X \longrightarrow X\,,
		\end{equation*}
		with neutral element $e \in X$;
		then the composition
		\begin{equation*}
			\begin{tikzcd}
				S_{\bullet}(X) \otimes_{\Z} S_{\bullet}(X) \ar[r, "\sigma"]
				& S_{\bullet}(X \times X) \ar[r, "\mu_{\bullet}"]
				& S_{\bullet}(X)
			\end{tikzcd}
		\end{equation*}
		defines an associative multiplication on $S_{\bullet}(X)$,
		whose neutral element is the $0$-simplex $e$,
		called \textbf{Pontrjagin product}.
		Moreover, there exists an augmentation $\epsilon : S(X) \to \Z$
		which sends each $0$-simplex (i.e. a point of $X$) to $1 \in \Z$,
		and vanishes on higher dimensional simplices;
		this makes $S_{\bullet}(X)$ into a DGA-algebra\footnote{In this case, $S_{\bullet}(X)$ is a
		DGA-algebra whose differential follows the \emph{homological} convention,
		while our definition is based on the cochain complex convention.}.
	\end{ex}
	
	\begin{prop}
		The category $\cat{DGA}_{\Lambda}$ is a %symmetric 
		monoidal category.
		\begin{proof}
			Consider any two DGA-algebras $\Aa = (A^{\bullet},d,\epsilon)$
			and $\Bb = (B^{\bullet}, \delta, \eta)$.
			We define their tensor product $\Aa \otimes \Bb$
			to be the cochain complex $A^{\bullet} \otimes B^{\bullet}$
			as defined in \hyperref[tensor-complex]{Example~\ref{tensor-complex}},
			i.e. the complex whose degree $n$ elements are
			\begin{equation*}
				\left( A^{\bullet} \otimes B^{\bullet} \right)^{n}
				= \bigoplus_{k \in \Z} A^{k} \otimes_{\Lambda} B^{n-k}\,,
			\end{equation*}
			equipped with the differential $D$ defined by
			\begin{equation*}
				D(a \otimes b) = da \otimes b + (-1)^{|a|}a \otimes \delta b\,.
				% \quad \text{for } a \in A^{k}\,.
			\end{equation*}
			We can define the multiplication
			\begin{equation*}
				(a \otimes b) \cdot (a' \otimes b')
				:= (-1)^{|a|\,|a'|}(aa') \otimes (bb')\,,
			\end{equation*}
			in such a way that $(A^{\bullet} \otimes B^{\bullet},D)$
			becomes a DG-algebra: indeed, it holds
			\begin{align*}
				D\big((a \otimes b) \cdot (a' \otimes b')\big)
				&= (-1)^{\deg(a)\deg(a')} D\big( (aa') \otimes (bb') ) \\
				&= (-1)^{\deg(a)\deg(a')} \big( d(aa') \otimes bb'
				+ (-1)^{\deg(aa')} aa' \otimes \delta(bb') \big) \\
				&= (-1)^{\deg(a)\deg(a')} 
				\Big( (da)a' \otimes bb' + (-1)^{\deg(a)}a(da') \otimes bb' \\
				&+ (-1)^{\deg(a) + \deg(a')} 
				\big(aa' \otimes (\delta b)b' 
				+ (-1)^{\deg(b)} aa' \otimes b(\delta b')\big)\Big) \\
				&= 
				\big( (-1)^{\deg(a')} ( da \otimes b %\cdot (a' \otimes b')
				+ (-1)^{\deg(a)} a \otimes \delta b) \cdot (a' \otimes b') \big) \\
				&+ \big((a \otimes b) \cdot (da' \otimes b' + a' \otimes \delta b' )\big) \\
				&= (-1)^{\deg(a')} \big( D(a \otimes b) \cdot (a' \otimes b')
				+ \big)\\
				&= D(a \otimes b) \cdot (a' \otimes b')
				+(-1)^{\deg(a \otimes b)} (a \otimes b) \cdot D(a' \otimes b')
			\end{align*}\todo{Please fix this.}
			so the differential satisfies the Liebniz rule \eqref{graded-liebniz}.
			Finally, the map $\alpha$ defined on every simple tensor 
			$a \otimes b$ by
			\begin{equation*}
				\alpha( a \otimes b ) := \epsilon(a)\eta(b) = 0\,,
			\end{equation*}
			is an augmentation on $\Aa \otimes \Bb$ because
			it clearly vanishes on elements of $(\Aa \otimes \Bb)^{0}
			= A^{0} \otimes B^{0}$, and
			by composing it with the differential one gets
			\begin{equation*}
				\alpha \circ D
				= (\epsilon \circ d) \eta
				\pm \epsilon (\eta \circ \delta)
				= 0\,.
			\end{equation*}
			This shows that $\Aa \otimes \Bb 
			= (A^{\bullet} \otimes B^{\bullet}, D, \alpha)$
			is a DGA-algebra.
			One can check that $\cat{DGA}_{\Lambda}$ equipped with $\otimes$
			becomes a monoidal category by
			an analgous argument as for $C^{\bullet}(\cat{Mod}_{\Lambda})$.
		\end{proof}
	\end{prop}
	

	
	
	
	
	
	
	
	\section{DG-modules}

	\begin{df}
		Let $\Aa = (A,d)$ be a DG-algebra.
		A \textbf{left DG-module} $\Mm$ over $\Aa$, or simply \textbf{$\Aa$-DG-module},
		is a cochain\footnote{Some authors, for instance Henri Cartan, use the homological indexing, which means they consider a differential of degree $-1$. These two concepts are basically the same: from our definition } 
		complex $\Mm = (M^{\bullet},d_{M})$ in $C^{\bullet}(\cat{Mod}_{A})$
		such that:
		\begin{itemize}
			\item the multiplication by an homogeneous element $a \in A^{k}$
			is a group homomorphism of degree $k$, that is
			\begin{equation*}
				A^{k} \otimes_{\Aa} M^{h} \longrightarrow M^{k+h}\,, \quad
				a \otimes m \longmapsto am\,;
			\end{equation*}
			
			\item the differential $d_{M}$ satisfies the following Liebniz rule:
			\begin{equation}\label{graded-mod-liebniz}
				d_{M}(am) = (da)m + (-1)^{|a|}a(d_{M}m)\,,
				\quad \text{for } a \in \Aa, m \in \Mm\,.
			\end{equation}
		\end{itemize}
		A \textbf{morphism of $\Aa$-DG-modules} $f:\Mm \to \Nn$
		is simply a cochain map $f:M^{\bullet} \to N^{\bullet}$
		of complexes of $A$-modules.
		
		One defines a \textbf{right DG-module} $\Mm$
		over $\Aa$ (in short a \textbf{DG-$\Aa$-module})
		in an analogous way, but the Liebniz rule for the
		right multiplication becomes
		\begin{equation*}
			d_{M}(ma) = (d_{M}m)a + (-1)^{|m|}m(da)\,,
			\quad \text{for } a \in \Aa, m \in \Mm\,.
		\end{equation*}
	\end{df}
	
	\begin{ex}
		Left DG-modules over a fixed DG-algebra $\Aa$, together with maps of complexes,
		form a category called ${}_{\Aa}\cat{DGMod}$. 
		If $\Aa = A^{0}$, then there is no differential structure
		on the ground ring, thus an $\Aa$-DG-modules is just a cochain complex of $A^{0}$-modules.
		In particular, ${}_{\Z}\cat{DGMod} = C^{\bullet}(\Ab)$.
	\end{ex}
	
	\begin{ex}
		Given any $\Aa$-DG-module $\Mm$, its cohomology $H^{*}(\Mm)$ is naturally
		a cochain complex (with trivial differential), which has a natural module structure
		over %the DG-algebra 
		$H^*(\Aa)$.
	\end{ex}
	
	\begin{df}
		Given a DG-module $\Mm$ over a DGA-algebra $\Aa=(A,d,\epsilon)$,
		an \textbf{augmentation} on $\Mm$ is a $\Lambda$-linear homomorphism
		$\epsilon_{\Mm}:\Mm \to \Lambda$ such that:
		\begin{rmnumerate}
			\item $\epsilon_{M} \circ d_{M} = 0$;
			\item vanishes in positive degree, i.e. if $|m|>0$, then $\epsilon_{M}(m) = 0$;
			\item it is compatible with augmentation on $\Aa$, that is
			for every $a \in \Aa$ and every $m \in \Mm$, 
			it holds $\epsilon_{\Mm}(am) = \epsilon(a)\epsilon_{\Mm}(m)$.
		\end{rmnumerate}
		An $\Aa$-DG-module $\Mm$ endowed with an augmentation is called \textbf{$\Aa$-DGA-module}.
	\end{df}	
	
	\begin{ex}
		Let $G$ be a topological group acting (on the left) on a topological space $X$.
		The action $G \times X \to X$ induces a morphism on singular complexes
		\begin{equation*}
			S_{\bullet}(G) \otimes_{\Z} S_{\bullet}(X) \longrightarrow S_{\bullet}(X)\,,
		\end{equation*}
		which makes $S_{\bullet}(X)$ into a left DGA-module over $S_{\bullet}(G)$.
	\end{ex}
	
	\begin{df}
		Consider a homomorphism $f:(\Aa,\epsilon) \to (\Bb, \eta)$ of DGA-algebras over $\Lambda$,
		an $\Aa$-DGA-module $\Mm$ and a $\Bb$-DGA-module $\Nn$.
		A \textbf{DGA-homomorphism compatible} with $f$ is a $\Lambda$-linear
		map of complexes $g:\Mm \to \Nn$ of degree $0$ such that: 
		\begin{rmnumerate}
			\item it is comptible with restriction of scalars, i.e. for every $a \in \Aa$
			and every $m \in \Mm$ it holds $g(am) = f(a)g(m)$;
			\item it preserves augmentations, that is $\eta_{\Nn} \circ g = \epsilon_{\Mm}$.
		\end{rmnumerate}
	\end{df}
	
	From now on we borrow all the terminology used for complexes of modules. 
	For example, we say that a $\Aa$-DG-module $\Mm$ is \textbf{acyclic} 
	if $H^{*}(\Mm) = \cat{0}$;
	we say that a morphism $f:\Mm \to \Nn$ of $\Aa$-DG-modules 
	is a \textbf{quasi-isomorphism} if it induces isomorphisms 
	$f^*:H^{*}(\Mm) \simeq H^{*}(\Nn)$, and so on and so forth.
	
	When a DG-module $\Mm$ is endowed with an augmentation,
	its structure becomes ``enough rigid'': in fact,
	whenever we have a base of homogeneous elements of $\Mm$
	(e.g. if the ground ring $\Lambda = k$ is a field),
	we can always define a DGA-homomorphism from $\Mm$
	to an acyclic module $\Nn \to \Lambda \to 0$; moreover, the quasi-isomorphism
	class of this map is \emph{unique}.
	The following result by Henri Cartan explains more precisely
	what it means.
	
	\begin{thm}
		Let $\Aa$ be DGA-algebra and $\Mm$ be a DGA-module over it, 
		with a free basis of homogeneous elements; similarly,
		consider $\Mm'$ to be a DGA-algebra over the DGA-algebra $\Aa'$,
		and assume $\Mm'$ has a homogoeneous free basis.
		Given a DGA-homomorphism $f:\Aa \to \Aa'$,
		if both $\Mm$ and $\Mm'$ are acyclic, then there exists
		a map $g:\Mm \to \Mm'$ compatible with $f$; moreover,
		if $f*:H_{*}(\Aa) \to H_{*}(\Aa')$ is an isomorphism,
		then also $g_{*}:H_{*}(\Mm) \to H_{*}(\Mm')$ is an isomorphism.
		\begin{proof}
			For details, look at \parencite{cartanDGA}.
			Notice that Cartan uses the homological convention.
			The proof shows that a map $g$ as above is defined
			on a basis $\{m_{i}\}$ of $\Mm$ by the formula
			\begin{equation*}
				g\left( \sum_{i} a_{i} m_{i} \right) 
				:= \sum_{i} (fa_{i}) m'_{i}\,,
			\end{equation*}
			where each homogeneous element $m_{i}$ of degree $k$
			is sent to an element $m'_{i} \in \Mm'$ of the same degree,
			which is built by induction on $k$ by following the rules:
			\begin{itemize}
				\item for each $m_{i} \in M_{0}$, consider $m'_{i} \in M'_{0}$
				such that $\epsilon_{\Mm}(m_{i}) = \epsilon_{\Mm'}(m'_{i})$;
				\item if $|m_{i}| \ge 1$, then pick $m'_{i}$ such that
				$d'm'_{i} = g(dm_{i})$.
			\end{itemize}
			This construction is obviously non-unique, but it can be
			shown that any two
			homomorphisms built this way are \emph{homotopic}.
		\end{proof}
	\end{thm}
	
	By considering the case $f=\cat{1}_{\Aa}$, one obtains the following
	\begin{cor}
		Any two acyclic free DGA-modules $\Mm$ and $\Mm'$ over a DGA-algebra $\Aa$
		are quasi-isomorphic.
	\end{cor}
	
	\begin{df}
		We define the \textbf{translation functor} 
		$[1] : {}_{\Aa}\cat{DGMod} \to {}_{\Aa}\cat{DGMod}$
		by sending any $\Aa$-DG-module $\Mm$ to the graded module
		\begin{equation*}
			\left( \Mm[1] \right)^{n} := M^{n+1}\,,
			\quad d_{M[1]} := -d_{M}\,,
		\end{equation*}
		in which the $A$-module structure on $M[1]$ is \textbf{twisted},
		i.e. we define the scalar multiplication as
		\begin{equation*}
			a \ast m := (-1)^{|a|} am\,,
		\end{equation*}
		where $am$ is the multiplication in $M$.
	\end{df}
	
	
	At this point we have enough structure to talk about \emph{triangles}:
	in fact, our next goal is to develop enough theory to be able
	to state and prove that the homotopy category $\Kk(\Aa)$
	is a triangulated category.
	
	
	\begin{df}
		Two morphisms $f,g: \Mm \to \Nn$ in ${}_{\Aa}\cat{DGMod}$
		are \textbf{homotopic} if there exists a homotopy
		$s:\Mm \to \Nn[-1]$ of $A$-modules (possibly \textbf{not} of $\Aa$-DG-modules)
		such that $$f-g=sd_{M} + d_{N}s\,,$$
		in which case we write $f \sim g$.
	\end{df}
	
	Since null-homotopic morphisms form a $2$-sided ideal in 
	$\Hom_{{}_{\Aa}\cat{DGMod}}(\Mm,\Nn)$, we may quotient
	by the equivalence relation $f \sim g$ given by homotopy
	to obtain the \textbf{homotopy category} $\Kk(\Aa)$,
	whose objects are $\Aa$-DG-modules and morphisms between them are
	\begin{equation*}
		\Hom_{\Kk(\Aa)}(\Mm,\Nn) := \Hom_{{}_{\Aa}\cat{DGMod}}(\Mm,\Nn)/\sim\,.
	\end{equation*}
	
	Given a morphism $f:\Mm \to \Nn$ in ${}_{\Aa}\cat{DGMod}$,
	we may build the \textbf{cone} of $f$ in the usual way, namely
	$\cat{C}(f) := N \oplus M[1]$ with differential $(d_{N} + f, -d_{M})$;
	this naturally inherits a structure of $\Aa$-DG-module,
	thus we may define a \textbf{strict triangle}
	\begin{equation*}
		\begin{tikzcd}
			\Mm \ar[r, "f"] & \Nn \ar[r] & \cat{C}(f) \ar[r] & \Mm{[1]}\,.
		\end{tikzcd}
	\end{equation*}
	An \textbf{exact triangle} in $\Kk(\Aa)$
	is a diagram $\Mm' \to \Nn' \to \Cc' \to \Mm'[1]$
	which is isomorphic (in he homotopy category) to a strict
	triangle as above: explicitly, there is a diagram
	\begin{equation*}
		\begin{tikzcd}
			\Mm' \ar[r] \ar[d, "\simeq"] & \Nn' \ar[r] \ar[d, "\simeq"] & \Cc' \ar[r] \ar[d, "\simeq"] & \Mm'{[1]} \ar[d, "\simeq"] \\
			\Mm \ar[r, "f"] & \Nn \ar[r] & \cat{C}(f) \ar[r] & \Mm{[1]}\,,
		\end{tikzcd}
	\end{equation*}
	which commutes up to homotopy, in which vertical maps are homotopy equivalences.
	
	As it happens for the homotopy category $\cat{K}(\Bb)$ of an abelian category $\Bb$,
	it turns out that $\Kk(\Aa)$ becomes a triangulated category 
	if we equip it with
	the translation functor $[1]$ and take the exact triangles as the family
	of distinguished triangles. In this category, quasi-isomorphisms of $\Aa$-DG-modules
	form a \emph{multiplicative system}, and hence we can 
	build the \textbf{derived category} $\Dd(\Aa)$ by formally inverting
	quasi-isomorphisms in $\Kk(\Aa)$: as it happens for $\cat{D}(\Bb)$,
	morphisms are pictured as \emph{roofs}, and $\Dd(\Aa)$ also inherits
	a triangulated structure.
	
	\begin{rmk}
		Even though $\Kk(\Aa)$, resp. $\Dd(\Aa)$, is called 
		the homotopy category of $\Aa$-DG-modules, resp. the derived category,
		the reader must be careful that these constructions are \emph{not}
		the usual ones described in 
		\hyperref[Derived-categories]{Section~\ref{Derived-categories}} 
		for abelian categories, 
		for $\Kk(\Aa)$ is not isomorphic to $\cat{K}({}_{\Aa}\cat{DGMod})$ in general
		(thus, we use a different symbol). Indeed, objects in $\Kk(\Aa)$ are
		DG-modules seen already as complexes, while objects in $\cat{K}({}_{\Aa}\cat{DGMod})$
		are complexes of DG-modules, which can be interpreted as bicomplexes.
		This explains why we should morally prove again 
		that $\Kk(\Aa)$ and $\Dd(\Aa)$ are triangulated,
		as it happens in \parencite[Part II]{bernstein-lunts}, 
		but in fact constructions are analogous for the classic homotopy and derived categories.
	\end{rmk}
	
	Let $\Aa=(A,d)$ be a DG-algebra.
	 Given two left DG-modules $\Mm$ and $\Nn$ over $\Aa$,
	we define the \textbf{internal hom} to be 
	the cochain complex $Hom^{\bullet}(\Mm,\Nn)$	of abelian groups
	\begin{equation*}
		Hom^{n}(\Mm,\Nn) := \Hom_{A}(M, N[n]) 
		= \Set{f:M \to N[n] | f \text{ morphism of %graded 
		} A\text{-modules}}\,,
	\end{equation*}
	where the differential $d$ is defined on $f \in Hom^{n}(\Mm,\Nn)$ by
	\begin{equation*}
		df := d_{\Nn} \circ f - (-1)^{n} f \circ d_{\Mm}\,.
	\end{equation*}
	We define the tensor product of $\Aa$-DG-modules in a similar fashion
	as for DG-algebras: given $\Mm$ and $\Nn$ left DG-modules over $\Aa$, 
	we define their \textbf{tensor product} $\Mm \otimes_{\Aa} \Nn$
	to be the total complex associated to $M^{\bullet} \otimes_{A} \Nn^{\bullet}$,
	that is the $A$-module $M \otimes_{A} N$ with the differential
	\begin{equation*}
		d(m \otimes n) := (d_{\Mm}m) \otimes n + (-1)^{|m|}m (d_{\Nn}n)\,.
	\end{equation*}
	
	\begin{ex}
		A $0$-cocycle of $Hom^{\bullet}(\Mm,\Nn)$ is a map of complexes:
		indeed, by definition $f:\Mm \to \Nn$ is such that $d_{\Nn}f - fd_{\Mm}$.
		Since $0$-coboundaries are null homotopic maps, one deduces that
		\begin{equation*}
			H^{0}\left(Hom^{\bullet}(\Mm,\Nn) \right)
			= \Hom_{\Kk(\Aa)}(\Mm,\Nn)\,.
		\end{equation*}
	\end{ex}
	
	If $\Aa$ is a graded-commutative DG-algebra, 
	then both $Hom^{\bullet}(\Mm,\Nn)$
	and $\Mm \otimes_{\Aa} \Nn$ have a natural $\Aa$-DG-module structure,
	in which scalar multiplications are defined for every $a \in \Aa$ by
	\begin{equation*}
		(a \cdot f) : m \mapsto af(m)\,, \quad
		a \cdot (m \otimes n) := (-1)^{|a|\,|m|} am \otimes n\,.
	\end{equation*}
	One can check that both $Hom^{\bullet}(\Nn,-)$ and $- \otimes_{\Aa} \Nn$ are
	endofunctors of ${}_{\Aa}\cat{DGMod}$ which send null-homotopic maps
	to null-homotopic maps, thus they descend to additive functors
	\begin{equation*}
		- \otimes_{\Aa} \Nn\,, Hom^{\bullet}(\Nn,-) : \Kk(\Aa) \longrightarrow \Kk(\Aa)\,,
	\end{equation*}
	which are in fact \emph{triangulated} functors.
	Moreover, given any three $\Aa$-DG-modules $\Mm,\Nn$ and $\Pp$,
	there exist functorial isomorphisms
	\begin{align*}
		(\Mm \otimes_{\Aa} \Nn) \otimes_{\Aa} \Pp &\simeq \Mm \otimes_{\Aa} (\Nn \otimes_{\Aa} \Pp)\,,\\
		Hom^{\bullet}(\Mm,Hom^{\bullet}(\Nn,\Pp)) &\simeq Hom^{\bullet}(M \otimes_{\Aa} \Nn,\Pp)\,\\
		\Hom_{{}_{\Aa}\cat{DGMod}}(\Mm,Hom^{\bullet}(\Nn,\Pp)) &\simeq \Hom_{{}_{\Aa}\cat{DGMod}}(M \otimes_{\Aa} \Nn,\Pp)\,,\\
		\Hom_{\Kk(\Aa)}(\Mm,Hom^{\bullet}(\Nn,\Pp)) &\simeq \Hom_{\Kk(\Aa)}(M \otimes_{\Aa} \Nn,\Pp)\,.
	\end{align*}
	One can sum up all these properties by saying that ${}_{\Aa}\cat{DGMod}$ is a
	\emph{symmetrical monoidal closed category}, where the monoidal structure
	is given by the tensor product $\otimes_{\Aa}$, with $\Aa$ as a neutral element,
	and the internal hom as its right adjoint. For details, see for instance
	\cite[\href{https://stacks.math.columbia.edu/tag/0FQ2}{Tag 0FQ2}]{stacksDGA}.
	
	As it happens for the classical notion of modules,
	whenever we have a homomorphsim of DG-algebras $f:\Aa \to \Bb$,
	we can give $\Bb$ a structure of $\Aa$-DG-module by setting the multiplication
	$a \cdot b := f(a)b$. Hence, we can relate the category of DG-modules over
	$\Aa$ and the ones over $\Bb$ via two functors:
	\begin{itemize}
		\item by \textbf{restriction of scalars}, which consists in considering
		any $\Bb$-DG-module as a left module over $\Aa$, via the induced
		multiplication defined above:
		\begin{equation*}
			f_{*}:{}_{\Bb}\cat{DGMod} \longrightarrow {}_{\Aa}\cat{DGMod}\,,
			\quad \Mm \longmapsto f_{*}\Mm\,;
		\end{equation*}
		
		\item by \textbf{extension of scalars}, given by the assignment
		\begin{equation*}
			f^{*}:{}_{\Aa}\cat{DGMod} \longrightarrow {}_{\Bb}\cat{DGMod}\,,
			\quad \Mm \longmapsto \Bb \otimes_{\Aa} \Mm\,.
		\end{equation*}
	\end{itemize}
	The two functors are adjoint to each other, 
	namely for every $\Aa$-DG-module $\Mm$
	and every $\Bb$-DG-module $\Nn$, 
	there are isomorphisms
	\begin{equation*}
		\Hom_{{}_{\Bb}\cat{DGMod}}(f^{*}\Mm,\Nn) \simeq \Hom_{{}_{\Aa}\cat{DGMod}}(\Mm,f_{*}\Nn)\,,
	\end{equation*}
	and one can check that they preserve homotopy equivalences, 
	thus induce adjoint functors between the homotopy categories.
	If both $\Aa$ and $\Bb$ are graded-commutative, then
	for every $\Mm, \Mm'$ DG-modules over $\Aa$ it holds
	\begin{equation*}
		\Bb \otimes_{\Aa} (\Mm \otimes_{\Aa} \Mm')
		\simeq (\Bb \otimes_{\Aa} \Mm) \otimes_{\Bb} (\Bb \otimes_{\Aa} \Mm')\,.
	\end{equation*}
	
	\begin{df}
		A left DG-module $\Pp$ over $\Aa$ is called \textbf{$\Kk$-projective}
		if one of the two following conditions holds:
		\begin{rmnumerate}
			\item $\Hom_{\Kk(\Aa)}(\Pp,-) = \Hom_{\Dd(\Aa)}(\Pp,-)$;
			\item for every acyclic $\Aa$-DG-module $\Cc$, 
			then the complex $Hom^{\bullet}(\Pp,\Cc)$ is acyclic.
		\end{rmnumerate}
	\end{df}
	
	The conditions i) and ii) are equivalent (see \parencite[Lemma~10.12.2.2]{bernstein-lunts}).
	Given any DG-module $\Mm$ over $\Aa$, there exists a canonical construction,
	called the \textbf{bar construction}, of a $\Kk$-projective module $B(\Mm)$
	together with a quasi-isomorphism $B(\Mm) \to \Mm$. 
	Thanks to this, whenever $\Aa$ is graded-commutative
	one may define the right derived functor
	\begin{equation*}
		RHom^{\bullet}(\Mm,-) : \Dd(\Aa) \longrightarrow \Dd(\Aa)\,,
		\quad \Nn \longmapsto RHom^{\bullet}(\Mm,\Nn) := Hom^{\bullet}(B(\Mm),\Nn)
	\end{equation*}
	and the left derived functor of the tensor product, i.e.
	\begin{equation*}
		- \overset{\cat{L}}{\otimes}_{\Aa} \Mm : \Dd(\Aa) \longrightarrow \Dd(\Aa)\,,
		\quad \Nn \longmapsto \Nn \overset{\cat{L}}{\otimes}_{\Aa} \Mm := \Nn \otimes_{\Aa} B(\Mm)\,,
	\end{equation*}
	thus, for any homomorphism $f:\Aa \to \Bb$ of DG-algebras,
	one can define extension of scalars between derived categories;
	this induces an adjunction
	\begin{equation*}
		\Hom_{\Dd(\Bb)}(f^{*}\Mm,\Nn) \simeq \Hom_{\Dd(\Aa)}(\Mm,f_{*}\Nn)\,.
	\end{equation*}
	
	\begin{thm}
		If $f:\Aa \to \Bb$ is a quasi-isomorphism of DG-algebras,
		then the extension of scalars induces an exact equivalence
		$f^*: \Dd(\Aa) \to \Dd(\Bb)$ 
		of triangulated categories.
		\begin{proof}
			The complete proof can be found in \parencite[Theorem~10.12.5.1]{bernstein-lunts}.
		\end{proof}
	\end{thm}
	
	
	
	
	
	
	
	
	
	
	
	
	
	
	
	
	
	
\section{Ainfty-algebras}

	Read the brief description in the article of braid groups and compare with
	something by Borislav Mladenov, which I found by chance.
	
\section{Hochschild (co)homology}

	First read the article by Sarah Witherspoon.
	Then take from Tamas, the Bonn course and maybe Keller.
	% https://www.ams.org/notices/202006/rnoti-p780.pdf
	% https://pbelmans.ncag.info/assets/hh-2018-notes.pdf
	% https://webusers.imj-prg.fr/~bernhard.keller/publ/HochschildCohomologyAndDerivedCategories.pdf
	
\section{Intrinsic formality}

	Read from the article of Seidel and Thomas,
	state and prove\textbf{Theorem 4.7}.
	

%%%%%%%%%%%%%%%%%%%%%%%%%%%%%%%%%%%%%%%%%%%%%%%%%%%%%%%%%%%
%%%%%%%%%%%%%%%%%%%%%%%%%%%%%%%%%%%%%%%%%%%%%%%%%%%%%%%%%%%
%%%%%%%%%%%%%%%%%%%%%%%%%%%%%%%%%%%%%%%%%%%%%%%%%%%%%%%%%%%

\chapter{Spherical objects}

	

\section{Monoid action on a category}
	
	The aim of this paragraph is to describe 
	how to define by generators and relations
	an action of the braid group on a category;
	then we will apply this result to 
	the derived category of a scheme $\cat{D}^{b}(X)$.


	
	
%%%%%%%%%%%%%%%%%%%%%%%%%%%%%%%%%%%%%%%%%%%%%%%%%%%%%%%%%%%
%%%%%%%%%%%%%%%%%%%%%%%%%%%%%%%%%%%%%%%%%%%%%%%%%%%%%%%%%%%
%%%%%%%%%%%%%%%%%%%%%%%%%%%%%%%%%%%%%%%%%%%%%%%%%%%%%%%%%%%


\missingfigure{To print bibliography, you must change 
`Quick build' settings to pdf + bib + pdf(x2) + pdfviewer.}


\nocite{*}


\backmatter\KOMAoption{chapterprefix}{false}
\printbibliography[heading=bibintoc, title={References}]
\end{document}