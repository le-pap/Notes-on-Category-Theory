

Applications of DG methods to homological algebra 
often hinge on constructing a chain of quasi-isomorphisms 
connecting two given DG-algebras. 
For instance, in the situation explained in the previous  section, 
one can try to use the DG-algebra $end(E)$ to study the twists
$T_{E_{i}}$ via the functor $\Psi_{E}$. 
What really matters for this purpose is only the quasi-isomorphism type of $end(E)$. 
In general, quasi-isomorphism type is a rather subtle invariant.
 However, there are some cases where the cohomology already determines the quasi-isomorphism type.
 
\begin{df}
	A DG-algebra $\Aa$ is called \textbf{formal} if
	it is quasi-isomorphic to its own cohomology algebra $H^{*}(\Aa)$
	thought as a DG-algebra with zero differentials.
\end{df}

\begin{ex}
	Any algebra $A$ seen as a DG-algebra concentrated in degree $0$ is formal.
\end{ex}

\begin{df}
	A graded algebra $A$ is called \textbf{intrinsically formal}
	if any two DG-algebras $\Bb, \Cc$ such that $H^{*}(\Bb) \simeq A \simeq H^{*}(\Cc)$,
	then $\Bb$ and $\Cc$ are quasi-isomorphic.
\end{df}

Equivalently, one can say that $A$ is intrinsically formal if any DG-algebra $\Bb$
whose cohomology is $A$ is formal.

\begin{ex}
	Any graded algebra $A$ concentrated in degree zero is intrinsically formal.\todo{Explain why.}
\end{ex}

The aim of this section is to find a characterization of intrinsic formality,
which can be computed via Hochschild cohomology of the right $A$-modules obtianed
by shifting $A$ suitably.