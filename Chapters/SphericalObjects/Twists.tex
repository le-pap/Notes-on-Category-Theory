
\section{Twist functors and spherical objects}

Recall that, given any two complexes $C^{\bullet}, D^{\bullet}$ with terms in $\Aa$,
we denote by 
	\begin{equation*}
		\Hom_{\cat{C}^{\bullet}(\Aa)}^{*}(C^{\bullet},D^{\bullet})
		= \bigoplus_{n \in \Z} \Hom_{\cat{C}^{\bullet}(\Aa)}(C^{\bullet},D^{\bullet}[n])\,,
	\end{equation*}
and we do so for all other categories whose objects are complexes.
Since $\Aa$ is a $k$-linear category, the set defined above is a $k$-vector space.

\begin{df}
	We call an object $E^{\bullet} \in \xK$ \textbf{twistable} if it satisfies
	the following finiteness conditions:
	\begin{itemize}
		\item[(\textbf{S1})]\label{S1} %the complex $E^{\bullet}$
		it is bounded;
		
		\item[(\textbf{S2})]\label{S2} for any $F^{\bullet}$ in $\xK$,
		both $\Hom_{\xK}^{*}(E^{\bullet},F^{\bullet})$ and $\Hom_{\xK}^{*}(F^{\bullet},E^{\bullet})$
		have finite dimension over $k$.
	\end{itemize}
\end{df}

\begin{df}
	Given a twistable complex $E^{\bullet}$,
	we define the \textbf{twist functor} around $E^{\bullet}$ 
	to be the cone of the following evaluation map:
	\begin{equation*}
		T_{E} : \xK \longrightarrow \xK\,,
		\quad T_{E}(F^{\bullet}) := 
		\cat{C}\left( \Hom^{*}(E^{\bullet},F^{\bullet}) \otimes E^{\bullet} 
		\xrightarrow{ev} F^{\bullet} \right)\,.
	\end{equation*}
\end{df}

A remark on the well-definition of this functor may be needed:
since $E^{\bullet}$ is bounded and $F^{\bullet}$ is bounded from below,
then the complex $\Hom^{*}(E^{\bullet},F^{\bullet})$ is also 
bounded from below, thus $\Hom^{*}(E^{\bullet},F^{\bullet})$
is an object in $C^{b}(\cat{Vect}_{k})$.
As $E^{\bullet}$ has injective terms, by property (\hyperref[C3]{\textbf{C3}}) 
we deduce that $\Hom^{*}(E^{\bullet},F^{\bullet}) \otimes E^{\bullet}$
is again a bounded below complex made of injective objects.
Notice that $\Hom^{*}(E^{\bullet},F^{\bullet})$ is quasi-isomorphic
to the complex $\Hom^{*}_{\xK}(E^{\bullet},F^{\bullet})$,
which is finite-dimensional by (\hyperref[S2]{\textbf{S2}}),
and since $- \otimes E^{\bullet}$ preserves quasi-isomorphisms
it holds
\begin{equation*}
	H^{*}\left( \Hom^{*}(E^{\bullet},F^{\bullet}) \otimes E^{\bullet} \right)
	\simeq H^{*}\left( \Hom^{*}_{\xK}(E^{\bullet},F^{\bullet}) \otimes E^{\bullet} \right)
	\simeq \Hom^{*}_{\xK}(E^{\bullet},F^{\bullet}) \otimes H^*(E^{\bullet})\,,
\end{equation*}
where the last isomorphism is due to the additivity of $H^{*}$.
It follows that the above cohomology is bounded, with objects in $\Bb$
by (\hyperref[C2]{\textbf{C2}}), 
therefore $\Hom^{*}(E^{\bullet},F^{\bullet}) \otimes E^{\bullet} \in \xK$.
By inspecting the associated LECS, one sees that $T_{E}(F^{\bullet}$ lies in $\xK$,
so $T_{E}$ is a well-defined functor. Moreover, it is triangulated\todo{Check this!}.


\begin{prop}
	A qis between twistable objects $E_{1}^{\bullet} \to E_{2}^{\bullet}$
	gives rise to an isomorphism of twist functors $T_{E_{1}} \simeq T_{E_{2}}$.
	\begin{proof}
		Given any $F^{\bullet} \in \xK$, 
		since the complexes $\Hom^{*}(E_{i}^{\bullet}),F^{\bullet}) \otimes E_{i}^{\bullet}$
		with $i,j \in \{1,2\}$ are quasi-isomorphic, then by the diagram
		\begin{equation*}
			\begin{tikzcd}
				\Hom^{*}(E_{1}^{\bullet},F^{\bullet}) \otimes E_{1}^{\bullet} \ar[r, "ev"]
				& F^{\bullet} \ar[r] \ar[d, equals]
				& T_{E_{1}}(F^{\bullet}) \\
				\Hom^{*}(E_{2}^{\bullet},F^{\bullet}) \otimes E_{1}^{\bullet} \ar[r] \ar[u] \ar[d]
				& F^{\bullet} \ar[r] \ar[d, equals]
				& C^{\bullet} \ar[u, dashed] \ar[d, dashed] \\
				\Hom^{*}(E_{2}^{\bullet},F^{\bullet}) \otimes E_{2}^{\bullet} \ar[r, "ev"]
				& F^{\bullet} \ar[r]
				& T_{E_{2}}(F^{\bullet}) 
			\end{tikzcd}
		\end{equation*}
		one deduces that $T_{E_{1}}(F^{\bullet})$ and $T_{E_{2}}(F^{\bullet})$
		are quasi-isomorphic. Since $\xK$ is a category of complexes of injectives,
		quasi-isomorphisms are in fact isomorphisms.
	\end{proof}
\end{prop}

\begin{cor}
	For any $j \in \Z$, the functor $T_{E[j]}$ is isomorphic to $T_{E}$.
	\begin{proof}
		Given any $F^{\bullet} \in \xK$, the $n$-th term of the
		complex $\Hom^{*}(E^{\bullet}[j],F^{\bullet}) \otimes E^{\bullet}[j]$
		is
		\begin{align*}
			\Big( \Hom^{*}(E^{\bullet}[j],F^{\bullet}) \otimes E^{\bullet}[j] \Big)^{n}
			&= \bigoplus_{k \in \Z} \Hom^{k}(E^{\bullet}[j],F^{\bullet}) \otimes (E^{\bullet}[j])^{-k+n} \\
			&\simeq \bigoplus_{k \in \Z} \Hom^{k}(E^{\bullet},F^{\bullet}[-j]) \otimes E^{-k+j+n} \\
			&= \bigoplus_{k \in \Z} \Hom^{k-j}(E^{\bullet},F^{\bullet}) \otimes E^{-(k-j)+n} \\
			&= \bigoplus_{p \in \Z} \Hom^{p}(E^{\bullet},F^{\bullet}) \otimes E^{-p+n} \\
			&= \Big( \Hom^{*}(E^{\bullet},F^{\bullet}) \otimes E^{\bullet} \Big)^{n}\,.
		\end{align*}
		The sign convention is such that all these isomorphisms together define
		an isomorphism of complexes, thus the associated twist functors are isomorphic.
	\end{proof}
\end{cor}

\begin{df}
	Given a twistable object $E^{\bullet} \in \xK$, 
	we define the \textbf{dual twist functor} to be %the cone
	\begin{equation*}
		T'_{E}: \xK \longrightarrow \xK\,, \quad
		T'_{E}(F^{\bullet}) := \cat{C}\left( F^{\bullet} \xrightarrow{ev'} 
		[\Hom^{*}(F^{\bullet},E^{\bullet}),E^{\bullet}]  \right)
	\end{equation*}
\end{df}

By a similar argument as for $T_{E}$, the dual twist functor actually
takes values in $\xK$, and the name ``\emph{dual}'' is justified by the following

\begin{lemma}
	For every $E^{\bullet}$ twistable, the functor $T'_{E}$ is left adjoint to $T_{E}$.
	\begin{proof}
		Given $F^{\bullet},G^{\bullet} \in \xK$, 
		we need to show the	exists a natural isomorphism
		\begin{equation*}
			\Hom^{*}_{\xK}(T'_{E}(F^{\bullet}),G^{\bullet}) 
			\simeq \Hom^{*}_{\xK}(F^{\bullet},T_{E}(G^{\bullet}))\,.
		\end{equation*}
		If we find a qis $\Hom^{*}(T'_{E}(F^{\bullet}),G^{\bullet}) 
		\to \Hom^{*}(F^{\bullet},T_{E}(G^{\bullet}))$,
		then the claim follows by taking $H^{0}$ on both sides.
			
		By applying $\Hom^{*}(F^{\bullet},-)$ to the distinguished triangle
		\begin{equation*}
			\begin{tikzcd}
			\Hom^{*}(E^{\bullet},G^{\bullet}) \otimes E^{\bullet} \ar[r, "ev"]
			& G^{\bullet} \ar[r]
			& T_{E}(G^{\bullet}) \ar[r]
			& \Big( \Hom^{*}(E^{\bullet},G^{\bullet}) \otimes E^{\bullet} \Big){[1]}\,,
			\end{tikzcd}
		\end{equation*}
		we notice that $\Hom^{*}(F^{\bullet},T_{E}(G^{\bullet})) \simeq \cat{C}(ev_{*})$;
		dually, it holds
		\begin{equation*}
			\Hom^{*}(T'_{E}(F^{\bullet}),G^{\bullet}) \simeq 
			\cat{C}\left( \Hom^{*} \big([\Hom^{*}(F^{\bullet},E^{\bullet}),E^{\bullet}],G^{\bullet} \big)
			\xrightarrow{(ev')^{*}} \Hom^{*}(F^{\bullet},G^{\bullet}) \right)\,.
		\end{equation*}
		Thus, if we consider the composition morphism
		\begin{equation*}
			\circ : \Hom^{*}(E^{\bullet},G^{\bullet}) \otimes_{k} \Hom^{*}(F^{\bullet},E^{\bullet})
			\longrightarrow \Hom^{*}(F^{\bullet},G^{\bullet})\,, \quad
			\phi^{\bullet} \otimes \psi^{\bullet} \longmapsto \phi^{\bullet} \circ \psi^{\bullet}\,,
		\end{equation*}
		thanks to the natural quasi-isomorphisms \eqref{tensor-adjs},
		we can build a zigzag of qis, which are natural both in $F^{\bullet}$ and in $G^{\bullet}$:
		\begin{equation*}
			\Hom^{*}(E^{\bullet},G^{\bullet}) \otimes E^{\bullet}
			\longleftarrow \cat{C}(ev_{*})
			\longleftarrow \cat{C}(\circ)
			\longrightarrow \cat{C}\big( (ev')^{*} \big)
			\longrightarrow \Hom^{*}(T'_{E}(F^{\bullet}),G^{\bullet})\,,
		\end{equation*}
		hence the thesis.
	\end{proof}
\end{lemma}



\begin{df}
	An object $E^{\bullet} \in \xK$ is \textbf{$n$-spherical} for some $n > 0$
	if it satisfies (\hyperref[S1]{\textbf{S1}}), (\hyperref[S2]{\textbf{S2}})
	and the following two properties:
	\begin{itemize}
		\item[(\textbf{S3})]\label{S3} its endomorphism algebra is the cohomology of the $n$-sphere
		$\Hom^{*}_{\xK}(E^{\bullet},E^{\bullet}) \simeq H^{*}_{\textrm{sing}}(S^{n};k)$, that is
		\begin{equation*}
			\Hom^{j}_{\xK}(E^{\bullet},E^{\bullet}) \simeq
			\begin{cases}
				k \,, \quad &\text{if } j=0,n\,; \\
				0 \,, \quad &\text{otherwise}.
			\end{cases}
		\end{equation*}
		
		\item[(\textbf{S4})]\label{S4} \textbf{Poincaré duality}: for all $F^{\bullet}$
		and all $j \in \Z$, the composition morphism
		\begin{equation*}
			\Hom^{j}(F^{\bullet},E^{\bullet}) \otimes_{k} \Hom^{n-j}(E^{\bullet},F^{\bullet})
			\longrightarrow \Hom^{n}(E^{\bullet},E^{\bullet}) \simeq k
		\end{equation*}
		is a non degenerate pairing.
	\end{itemize}
\end{df}


\begin{thm}
	The spherical twist around an $n$-spherical object is an exact self-equivalence of $\xK$.
	\begin{proof}
		Given $F^{\bullet} \in \xK$, then $T_{E}T'_{E}(F^{\bullet})$
	\end{proof}
\end{thm}


