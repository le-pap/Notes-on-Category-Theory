
\section{Equivalences}

\begin{df}
    Given two categories $\Aa$ and $\Bb$ and
    a functor $F : \Aa \to \Bb$.
    For any two objects $A, A' \in \Aa$ 
    consider the map
    \begin{equation*}
        F_{A,A'} : \Hom_{\Aa}(A,A') \longrightarrow \Hom_{\Bb}(F(A),F(A'))\,.
    \end{equation*}
    The functor $F$ is called:
    \begin{itemize}
        \item \textbf{full} if $F_{A,A'}$ is surjective
        for any $A, A' \in \Aa$;

        \item \textbf{faithful} if $F_{A,A'}$ is injective
        for any $A, A' \in \Aa$;

        \item \textbf{fully faithful} if $F$ is both full and faithful.
    \end{itemize}
\end{df}

Recall that a \emph{morphism of functors},
or \emph{natural transformation},
$\phi : F \to F'$ between two functors
$F,F' : \Aa \to \Bb$ is a collection of arrows
$\phi_{A} \in \Hom_{\Bb}(F(A), F'(A))$, for every $A \in \Aa$,
which is functorial in $A$, i.e. given $f : A \to B$ in $\Aa$,
the diagram
\begin{center}
    \begin{tikzcd}
        F(A) \ar[r, "\phi_{A}"] \ar[d, "F(f)"'] 
        & F'(A) \ar[d, "F'(f)"] \\
        F(B) \ar[r, "\phi_{B}"] & F'(B)
    \end{tikzcd}
\end{center}
commutes. Thus, once two categories $\Aa$ and $\Bb$
are fixed, one may verify that functors from $\Aa$ to $\Bb$
and natural transformations form a category 
$\cat{Fun}(\Aa, \Bb)$. In here, two functors $F,F'$
are \textbf{isomorphic} if and only if
there exists a natural transformation $\phi:F \to F'$
such that $\phi_{A}$ is an isomorphism in $\Bb$, 
for every $A \in \Aa$.

\begin{df!}\label{cat-equivalence}
    A functor $F : \Aa \to \Bb$ is called 
    an \textbf{equivalence} if there exists a
    functor $F^{-1} : \Bb \to \Aa$ such that
    $F^{-1} \circ F \simeq \cat{1}_{\Aa}$ in $\cat{Fun}(\Aa, \Aa)$
    and
    $F \circ F^{-1} \simeq \cat{1}_{\Bb}$ in $\cat{Fun}(\Bb, \Bb)$,
    where $\cat{1}_{\Aa}$ and $\cat{1}_{\Bb}$
    denote the identity functors. The functor $F^{-1}$ is called a \textbf{quasi-inverse} of $F$.

    Two categories $\Aa$ and $\Bb$ are called \textbf{equivalent} if there exists an
    equivalence $F:\Aa \to \Bb$.
\end{df!}

In particular, notice that 
for any $A,A' \in \Aa$,
%for any arrow $f: A \to A'$ in $\Aa$,
%there exists a commutative diagram
%\begin{center}
%    \begin{tikzcd}
%        F^{-1} \circ F(A) \ar[rr, "F^{-1} \circ F(f)"] \ar[d, "\simeq"']
%        & & F^{-1} \circ F(A') \ar[d, "\simeq"] \\
%        A \ar[rr, "f"] & & A'\,,
%    \end{tikzcd}
%\end{center}
the map $F_{A,A'} : \Hom_{\Aa}(A, A') \to \Hom_{\Bb}(F(A),F(A'))$
is bijective, whose inverse is the composition
\begin{equation*}
    \Hom_{\Bb}(F(A),F(A')) \longrightarrow
    \Hom_{\Aa}(F^{-1} \circ F(A), F^{-1} \circ F(A')) \simeq
    \Hom_{\Aa}(A, A')\,.
\end{equation*}
Thus, an equivalence is fully faithful. It holds a partial converse:

\begin{prop}
    A fully faithful functor $F : \Aa \to \Bb$ is an equivalence
    if and only if every $B \in \Bb$ is isomorphic
    to an object of the form $F(A)$, for some $A \in \Aa$.
    \begin{proof}
        If $F$ is an equivalence, then by definition there exists
        a quasi-inverse $F^{-1}$ and for every $B \in \Bb$ we have
        $B \simeq F(F^{-1}(B))$.

        Conversely, we build a quasi-inverse $F^{-1}$ 
        in the following way: for every $B \in \Bb$,
        let $\phi_{B} : B \simeq F(A_{B})$ a fixed isomorphism,
        with $A_{B} \in \Aa$. Since $F$ is fully faithful,
        for any two $B, B' \in \Bb$ the map $F_{A_{B}, A_{B'}}$ 
        is bijective, hence we can define $F^{-1}$ to be
        \begin{equation*}
            \Big( f : B \to B' \Big)
            \longmapsto
            \left( \left(F_{A_{B},A_{B'}}\right)^{-1}(\phi_{B'} \circ f \circ \phi_{B}^{-1}) : A_B \to A_B' \right)\,,
        \end{equation*}
        and then one verifies that $F^{-1}$ works.
    \end{proof}
\end{prop}

\begin{cor}
    Any fully faithful functor $F:\Aa \to \Bb$
    defines and equivalence between $\Aa$ and
    the full subcategory $\Bb_{\Aa} \subset \Bb$
    of all objects of $\Bb$ isomorphic to $F(A)$,
    for some $A \in \Aa$.
\end{cor}

The following result is fundamental to understand
when a functor $F$ can be represented by an object,
i.e. the functor is nothing but the morphisms hitting
some \emph{universal object}.
Let $\cat{Fun}(\Aa^{op})$ be the category of all
contravariant functors from $\Aa$ to $\CSet$,
whose objects are $F : \Aa^{op} \to \CSet$.
There exists a natural functor
\begin{equation*}
    h : \Aa \longrightarrow \cat{Fun}(\Aa^{op})\,,
    \quad A \longmapsto \Hom_{\Aa}(-,A)\,,
\end{equation*}
called the \textbf{Yoneda embedding}.

\begin{thm}[Yoneda's Lemma]\label{yoneda}
    For any fixed $A \in \Aa$, the map 
    \begin{equation*}
        \Hom_{\cat{Fun}(\Aa^{op})}\left( h(A), F \right)
        \longrightarrow F(A)\,, \quad
        \phi \longmapsto \phi_{A}(\cat{1}_{A})
    \end{equation*}
    is bijective and it is functorial
    both in $A$ and in $F$.
    It follows that the Yoneda embedding $h$ defines 
    an equivalence between $\Aa$ and the full subcategory 
    of \textbf{representable functors}.

    \begin{proof}
        See \parencite{riehl}.
    \end{proof}
\end{thm}

As a consequence, it follows that an object $A \in \Aa$ is
determined by all the morphisms that hit it: more
precisely, whenever there exist $A,A' \in \Aa$ such that
for every $T \in \Aa$ there are natural isomorphisms
\begin{equation*}
    \Hom_{\Aa}(T,A) \simeq \Hom_{\Aa}(T,A')\,,
\end{equation*}
then we conclude that $A \simeq A'$ in $\Aa$.