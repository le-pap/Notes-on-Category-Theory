
\section{\texorpdfstring{$k$}{k}-linear categories}
% FOR MATH SYMBOLS IN TITLES 
% https://latex.org/forum/viewtopic.php?t=21256


As the categories we will eventally be interested in
have a geometric origin, i.e. are defined in terms
of certain variety over some base field $k$,
we usually deal with the following special kind of categories.

\begin{df}
    Let $k$ be an arbitrary field. 
    A \textbf{$k$-linear category} is an additive category
    $\Aa$ such that the groups $\Hom_{\Aa}(A,A')$
    are $k$-linear vector spaces, for any $A,A' \in \Aa$.
    Moreover, we require the compositions to be $k$-bilinear.
\end{df}

In short, a $k$-linear category $\Aa$ is a $\cat{Vect}_{k}$-enriched category,
where we require $\Aa(A,A') = \Hom_{\Aa}(A,A')$.

\begin{ex}
    Since $\Vv = \Vect{k}$ endowed with $\otimes_{k}$
    is a symmetric monoidal closed category,
    then it is enriched over itself,
    i.e. it is $k$-linear.
\end{ex}

\begin{df}
    Given $\Aa,\Bb$ two $k$-linear categories,
    an additive functor $F : \Aa \to \Bb$ 
    is called \textbf{$k$-linear}
    if the map $F_{A,A'} : \Hom_{\Aa}(A,A') \to \Hom_{\Bb}(F(A),F(A'))$ 
    is a $k$-linear map for any two
    $A,A' \in \Aa$. As in the additive case, 
    $k$-linear functors from $\Aa$ to $\Bb$ 
    and natural transformations
    form a category $\cat{Fun}_{k}(\Aa, \Bb)$.
\end{df}


As special case of the 
\hyperref[enriched-yoneda]{\textbf{Theorem~\ref*{enriched-yoneda}}},
we will adapt the \hyperref[yoneda]{\textbf{Yoneda's Lemma}~\ref*{yoneda}} 
to the $k$-linear setting:
first, we recall that a \textbf{$k$-linear} equivalence
$F : \Aa \to \Bb$ between two $k$-linear categories
is an equivalence which is a $k$-linear functor,
whose quasi-inverse $F^{-1}$ is again $k$-linear.

%\begin{thm}[Additive Yoneda]
%    For an additive category $\Aa$, the Yoneda embedding
%    \begin{equation*}
%        \Aa \longrightarrow \cat{Fun}_{+}(\Aa^{op},\Ab)\,, \quad
%        A \longmapsto \Hom_{\Aa}(-,A)
%    \end{equation*}
%    defines an equivalence between $\Aa$ and
%    the category of contravariant \emph{additive} functors
%    between $\Aa$ and the category of abelian groups.
%\end{thm}

%Similarly, it holds:

\begin{thm}[\textbf{Linear Yoneda's Lemma}]\label{linear-yoneda}
    Let $k$ be a field. For a $k$-linear category $\Aa$, 
    the Yoneda embedding
    \begin{equation*}
        \Aa \longrightarrow \cat{Fun}_{k}(\Aa^{*},\cat{Vect}_{k})\,, \quad
        A \longmapsto \Hom_{\Aa}(-,A)
    \end{equation*}
    defines an equivalence between $\Aa$ and
    the category of contravariant \emph{$k$-linear} functors
    between $\Aa$ and the category of vector spaces over $k$.
\end{thm}

\begin{lemma}\label{tensor-w-zero}
    Any $k$-linear category $\Dd$ is tensored over
    finite-dimensional vector spaces, i.e.
    given $n \in \NN$ and $A \in \Dd$,
    then
    \begin{equation*}
        k^{n} \otimes_{\Dd} A = A^{\oplus n}\,.
    \end{equation*}
    In particular, for every $A \in \Dd$, one has
    $0 \otimes_{\Dd} A \simeq \cat{0}$.
    \begin{proof}
        Given any object $B \in \Dd$, one notices that
        \begin{align*}
            \Hom_{\Dd}(k^{n} \otimes_{\Dd} A, B)
            \simeq \Hom_{k}(k^{n}, \Hom_{\Dd}(A,B)) 
            \simeq \Hom_{\Dd}(A,B)^{\oplus n}
            \simeq \Hom_{\Dd}\left( A^{\oplus n}, B \right)\,,
        \end{align*}
        thus, $k^{n} \otimes_{\Dd} A = A^{\oplus n}$ 
        by the covariant version of the \hyperref[linear-yoneda]{Linear Yoneda's Lemma}.
    \end{proof}
\end{lemma}

\begin{df}
    Let $\Aa$ be a $k$-linear category.
    A \textbf{Serre functor} is a $k$-linear equivalence
    $S : \Aa \to \Aa$ such that, for any two objects
    $A,A' \in \Aa$, there exists an isomorphism
    of $k$-vector spaces
    \begin{equation*}
        \sigma_{A,A'} : \Hom_{\Aa}(A,A') 
        \xrightarrow[]{\sim} \Hom_{\Aa}(A',S(A))^*\,,
    \end{equation*}
    which is functorial both in $A$ and $A'$.
\end{df}

A Serre functor induces a pairing
\begin{equation*}
    \Hom_{\Aa}(A',S(A)) \times \Hom_{\Aa}(A,A') \longrightarrow k\,,
    \quad (f,g) \longmapsto \langle f | g \rangle 
    := \sigma_{A,A'}(g)(f)\,.
\end{equation*}

In order to avoid any trouble with the dual, 
one usually assumes that all Hom’s in A are \emph{finite-dimensional}. 
Under this hypothesis it is easy to see that a Serre functor, 
if it exists, is unique up to isomorphism. 
More generally one has the following

\begin{lemma}\label{serre-eq-comm}
    Let $\Aa$ and $\Bb$ be $k$-linear categories with
    finite dimensional $\Hom$'s. Assume $\Aa$, resp. $\Bb$,
    is endowed with a Serre functor $S_{\Aa}$, resp. $S_{\Bb}$.
    Then any $k$-linear equivalence $F:\Aa \to \Bb$ commutes
    with Serre duality, i.e. there exists an isomorphism
    \begin{equation*}
        F \circ S_{\Aa} \simeq S_{\Bb} \circ F\,.
    \end{equation*}
    \begin{proof}
        We show the two functors are isomorphic
        by applying the \hyperref[yoneda]{Yoneda Lemma~\ref*{yoneda}}:
        given any two $A, A' \in \Aa$, 
        we compute the Serre duality in $\Aa$
        and we use that $F$
        is fully faithful to get
        \begin{align*}
            \Hom_{\Aa}(A,A') \simeq \Hom_{\Aa}(A',S_{\Aa}(A))^* 
            \simeq \Hom_{\Bb}\big(F(A'),F(S_{\Aa}(A))\big)^*\,.
        \end{align*}
        On the other hand, if we first apply $F$ and then Serre duality
        we obtain
        \begin{align*}
            \Hom_{\Aa}(A,A') 
            \simeq \Hom_{\Bb}(F(A),F(A'))
            \simeq \Hom_{\Bb}\big(F(A'),S_{\Bb}(F(A))\big)^*\,.
        \end{align*}
        Since $F$ is essentially surjective, 
        for every $B \in \Bb$ we have isomorphisms
        \begin{align*}
            \Hom_{\Bb}\big(F(A'),S_{\Bb}(F(A))\big)^*
            \simeq \Hom_{\Bb}\big(F(A'),S_{\Bb}(F(A))\big)^*\,,
        \end{align*}
        thus, by passing to the double dual we conclude.
    \end{proof}
\end{lemma}

\begin{rmk}
    Let $\Aa$ and $\Bb$ be $k$-linear categories 
    as in the hypothesis of \hyperref[serre-eq-comm]{Lemma~\ref*{serre-eq-comm}}.
    If $F: \Aa \to \Bb$ is a functor such that $G \dashv F$,
    then Serre duality gives us a way to build a right adjoint for $F$:
    indeed, we have $F \dashv S_{\Aa} \circ  G \circ S_{\Bb}^{-1}$.
    To see this,  we have natural isomorphisms
    \begin{align*}
        \Hom_{\Aa}(A, S_{\Aa} \circ  G \circ S_{\Bb}^{-1}(B)) 
        &\simeq \Hom_{\Bb}(G \circ S_{\Bb}^{-1}(B),A)^{*} \\
        &\simeq \Hom_{\Bb}(S_{\Bb}^{-1}(B),F(A))^{*} \\
        &\simeq \Hom_{\Bb} \big(F(A), S_{\Bb} \circ S_{\Bb}^{-1}(B)\big)^{**} \\
        &\simeq \Hom_{\Bb}(F(A),B)\,.
    \end{align*}
\end{rmk}

 A similar argument allows the construction of 
 a left adjoint if a right adjoint 
 $F \dashv H$ is given. 
 In particular, for functors between categories with 
 Serre functors the existence of the left or 
 the right adjoint implies the existence of the other one.
