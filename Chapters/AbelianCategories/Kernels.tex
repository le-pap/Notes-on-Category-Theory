\section{Zero objects and kernels}

\begin{df}
    By a \textbf{zero object} in a category $\Aa$
    we mean an object $\cat{0}$ which is both 
    an initial and a terminal object.
\end{df}

\begin{ex}
    In the category of abelian groups, or any category of modules over a ring, both notions coincide and correspond to the group (respectively the module) reduced to {0}.
\end{ex}

\begin{df}
    Consider a category $\Aa$ with a zero object $\cat{0}$.
    A morphism $f : A \to B$ is a \textbf{zero morphism}
    when it factors through the zero object, 
    which means that the following triangle commutes
    \begin{equation*}
        \begin{tikzcd}
            A \ar[rr, "f"] \ar[dr] & & B \\
            & \cat{0} \ar[ur] & \,.
        \end{tikzcd}
    \end{equation*}
\end{df}

Notice that the factorization is unique by definition
zero object; in particular, for each pair of objects
$A$ and $B$ in $\Aa$, there always exists a unique
zero morphism $0_{AB} \in \Hom_{\Aa}(A,B)$.
We will soon drop the indices, 
whenever it is clear from the context.

Notice that the notion of having ``zero morphisms'' 
allows us to talk about \emph{complexes}, which are
sequences of objects and morphisms whose composite 
is zero. If we want to talk about \emph{exactness},
we need \emph{kernels} and \emph{images}.

If we consider a morphism $f : G \to H$
in the category of abelian groups, 
we know that the kernel of $f$ is a subgroup of $G$, 
which is defined by
\begin{equation*}
    \ker f = \Set{g \in G | f(g) = 0 = 0_{GH}(g)}\,.
\end{equation*}
In particular, we see that the diagram
\begin{equation*}
    \begin{tikzcd}
        \ker f \ar[r, "\subset"]
        & G \ar[r, shift left=0.7ex, "f"] \ar[r, shift right=0.7ex, "0_{GH}"']
        & H\,
    \end{tikzcd}
\end{equation*}
is an equalizer.

\begin{df}
    Let $\Aa$ be a category with a zero object $\cat{0}$.
    The \textbf{kernel} of a morphism $f : A \to B$
    is the equalizer of $f$ and the zero morphism $0_{AB}$,
    whenever it exists. As equalizers are unique up to isomorphisms,
    we will denote the kernel of $f$ by $\ker f$:
    \begin{equation*}
    \begin{tikzcd}
        \ker f \ar[r]
        & A \ar[r, shift left=0.7ex, "f"] \ar[r, shift right=0.7ex, "0_{AB}"']
        & B\,.
    \end{tikzcd}
\end{equation*}
\end{df}

\begin{rmk}
    Since it is an equalizer, the map $\ker f \to A$ is a 
    monomorphism; in the example of $\Aa = \cat{Ab}$ it
    is indeed an inclusion of a subgroup. Hence, this agrees
    with the intuition that the kernel is something
    ``contained'' in the domain $A$. 
    The converse does not hold true: there exist
    monomorphisms that are not kernels. For instance,
    if $\Aa = \cat{Grp}$ is the category of groups,
    then any inclusion of a subgroup $H \subset G$ is
    a monomorphism in $\Aa$, but if $H$ is not normal,
    then it cannot be a kernel.
\end{rmk}

We define the dual notion of a kernel.

\begin{df}
    Let $\Aa$ be a category with a zero object $\cat{0}$.
    The \textbf{cokernel} of a morphism $f : A \to B$
    is the coequalizer of $f$ and the zero morphism $0_{AB}$,
    whenever it exists. As coequalizers are unique up to isomorphisms,
    we will denote the cokernel of $f$ by $\Coker f$:
    \begin{equation*}
    \begin{tikzcd}
        A \ar[r, shift left=0.7ex, "f"] \ar[r, shift right=0.7ex, "0_{AB}"']
        & B \ar[r] & \Coker f\,.
    \end{tikzcd}
\end{equation*}
\end{df}

We have almost everything we need to develop
the homological algebra language. Indeed, in order
to talk about ``\emph{exactness}'' of
composable arrows, 
we define \textbf{images} and \textbf{coimages}.

\begin{df}
    Let $\Aa$ be a category with a zero object $\cat{0}$,
    and assume that all equalizers and coequalizers exist.
    Given a morphism $f:A \to B$, we call \textbf{image} of $f$
    the kernel of $B \to \Coker f$ and we denote it by $\im f$.
    Thus, we have a factorization
    \begin{equation*}
        \begin{tikzcd}
            A \ar[r, "f"] \ar[rd, dashed] & B \ar[r, "c"] & \Coker f\\
            & \im f := \ker c \ar[u, hook] & \,.
        \end{tikzcd}
    \end{equation*}
    Dually, we define the \textbf{coimage} of $f$ as the cokernel of the kernel of $f$,
    thus we have the factorization
    \begin{equation*}
        \begin{tikzcd}
            \ker f \ar[r, "k"] & A \ar[r, "f"] \ar[d,two heads] & B \\
            & \operatorname{coim} f := \Coker k \ar[ur, dashed] & \,.
        \end{tikzcd}
    \end{equation*}
\end{df}

\begin{ex}
    Given a morphism $f : A \to B$ in $\Aa = \cat{Ab}$, then $\im f$ coincides
    with the set theoretic image of the map, that is
    \begin{equation*}
        \im f = \Set{f(a) | a \in A}\,;
    \end{equation*}
    this forms a subgroup of $B$ and we may identify the cokernel of $f$
    with the quotient
    \begin{equation*}
        \Coker f = B/\im f\,.
    \end{equation*}
    In particular, if we consider the inclusion $\ker f \subset A$,
    its cokernel is
    \begin{equation*}
        \operatorname{coim} f = A/\ker f\,.
    \end{equation*}
\end{ex}

\begin{prop}\label{coim-im}
    For any $f : A \to B$, there
    exists a unique map $\operatorname{coim}f \to \im f$ that
    makes the following square commute:
        \begin{equation*}
            \begin{tikzcd}
                A \ar[r, "f"] \ar[d] & B \\
                \operatorname{coim} f \ar[r] & \im f \ar[u] \,.
            \end{tikzcd}
        \end{equation*}
    \begin{proof}
        By definition of kernel, the composition $\ker f \to A \overset{f}{\to} B$
        is the zero map, hence we get the factorization
        \begin{equation*}
        \begin{tikzcd}
            \ker f \ar[r] & A \ar[r, "f"] \ar[d,two heads, "\pi"'] & B \\
            & \operatorname{coim} f \ar[ur, dashed, "f'"'] & \,.
        \end{tikzcd}
        \end{equation*}
        Now we notice that the diagram
        \begin{equation*}
        \begin{tikzcd}
            A \ar[r,two heads, "\pi"] 
            & \operatorname{coim} f  \ar[r, "0"] \ar[dr, "f'"'] 
            & \Coker f \\
            & & B \ar[u, "c"'] 
        \end{tikzcd}
        \end{equation*}
        commutes, and since $\pi$ is an epimorphism, then $cf' = 0$.
        Since the image is the kernel of $c$, there exists a unique map $\ol{f}$
        which makes the following diagram commute
        \begin{equation*}
        \begin{tikzcd}
            \ker f \ar[r] & A \ar[r, "f"] \ar[d,two heads, "\pi"'] 
            & B \ar[r, "c"] & \Coker f \\
            & \operatorname{coim} f \ar[ur, "f'"'] \ar[r, dashed, "\ol{f}"'] 
            & \im f \ar[u, hook] & \,.
        \end{tikzcd}
        \end{equation*}
    \end{proof}
\end{prop}

\begin{rmk}
    In abstract categorical nonsense, this statement
    may look at first a bit counter intuitive,
    but it is in fact a very familiar statement in algebra:
    if we consider the case $\Aa = \cat{Ab}$, then
    the proposition tells that every homomorphism $f:A \to B$
    factors as
    \begin{equation*}
        \begin{tikzcd}
             A \ar[r, "f"] \ar[d,two heads, "\pi"'] 
            & B \\
            A/\ker f \ar[r, "\ol{f}"', "\simeq"] 
            & \im f \ar[u, hook] \,,
        \end{tikzcd}
        \end{equation*}
    where the map $\ol{f}$ is an isomorphism.
    This is the well-known 
    \textbf{First Isomorphism Theorem} (\textbf{for abelian groups})!
    Be careful that, in general, the map $\ol{f}$ needs \textbf{not}
    be an iso, and this property is what characterizes 
    \textbf{abelian categories}.
\end{rmk}