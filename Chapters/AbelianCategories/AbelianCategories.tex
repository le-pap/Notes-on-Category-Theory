
\section{Abelian categories}

\begin{df!}\label{df-abel}
    A category $\Aa$ is \textbf{abelian}
    when it satisfies the following properties:
    \begin{itemize}
        \item[(\textbf{A1})] $\Aa$ has a \textbf{zero object} $\cat{0}$;
        \item[(\textbf{A2})] every pair of objects of $\Aa$ has a product and a coproduct;
        \item[(\textbf{A3})] every morphism in $\Aa$ has both \textbf{kernel} and a \textbf{cokernel};
        \item[(\textbf{A4})] for every $f \in \Hom_{\Aa}(A,B)$,
        the canonical map $\operatorname{coim}f \to \im f$
        defined in \hyperref[coim-im]{\textbf{Proposition~\ref*{coim-im}}}
        is an isomorphism.
    \end{itemize}
\end{df!}

\begin{rmk}
    Axiom (\textbf{A4}) does not follow by the previous ones,
    as shown in the following examples;
    a category that satisfies (\textbf{A1}), (\textbf{A2}) and (\textbf{A3}) is usually called \textbf{pre-abelian}.
    As already noticed, (\textbf{A4}) tells us
    that in $\Aa$ it holds a sort of \textbf{First Isomorphism Theorem},
    but many authors (for instance \parencite[]{borceux})
    use an equivalent definition, where the last axiom 
    is substituted by:
    \begin{itemize}
         \item[(\textbf{A4}*)] every monomorphism of $\Aa$ is a kernel
        and any epimorphism of $\Aa$ is a cokernel.
    \end{itemize}
    An explanation for this equivalence may be found in \parencite[]{ab-eq-def}.
\end{rmk}

\begin{ex}
    Let $k$ be a field and $\cat{Vect}_{k}$ be the
    category of $k$-vector spaces.
    We define the category $\cat{Fil}(\cat{Vect}_{k})$ 
    of \textbf{filtered vector spaces}
    to be the category whose objects are decreasing filtrations
    \begin{equation*}
        V_{\bullet} := V  \supset \dots \supset V_{n} 
        \supset V_{n+1} \supset \dots \supset 0\,,
    \end{equation*}
    indexed on $\Z$, and morphisms 
    $\phi_{\bullet} : V_{\bullet} \to W_{\bullet}$ 
    are given by $k$-linear maps $\phi : V \to W$
    such that $\phi(V_{n}) \subset W_{n}$ for every $n \in \Z$.
    Then $\cat{Fil}(\cat{Vect}_{k})$ has the zero object
    $0_{\bullet}$ given by the trivial
    filtration $0 = 0 = \dots = 0$,
    and every pair $V_{\bullet}, W_{\bullet}$ has biproduct
    \begin{equation*}
        V_{\bullet} \oplus W_{\bullet} 
        = (V \oplus W)_{\bullet} 
        = V_{} \oplus W_{} \supset \dots \supset V_{n} \oplus W_{n}
        \supset V_{n+1} \oplus W_{n+1} \supset \dots \supset 0\,.
    \end{equation*}
    Moreover, this category satisfies the axiom (\textbf{A3}):
    for any morphism $\phi_{\bullet} : V_{\bullet} \to W_{\bullet}$,
    we have
    \begin{align*}
        \ker \phi_{\bullet} &= \ker \phi \supset \dots \supset (\ker \phi \cap V_{n}) 
        \supset (\ker \phi \cap V_{n+1}) \supset \dots 0\,,\\
        \Coker \phi_{\bullet} &= W/\im f \supset \dots \supset W_{n}/\im\left(f\vert_{V_{n}}\right)
        \supset W_{n+1}/\im \left(f\vert_{V_{n+1}} \right) \supset \dots \supset 0\,.
    \end{align*}
    Nevertheless, $\cat{Fil}(\cat{Vect}_{k})$ is \textbf{not}
    an abelian category, for axiom (\textbf{A4}) does not hold:
    consider $V_{\bullet}, W_{\bullet}$ to be the two filtrations
    \begin{equation*}
        V_{n} := \begin{cases}
            k\,, \quad &\text{if } n>0\,;\\
            0\,, \quad &\text{if } n \le 0\,;
        \end{cases}
        \qquad
        W_{n} := \begin{cases}
            k\,, \quad &\text{if } n\ge0\,;\\
            0\,, \quad &\text{if } n < 0\,,
        \end{cases}
    \end{equation*}
    and $\phi_{\bullet} : V_{\bullet} \to W_{\bullet}$ to be the morphism
    induced by the identity $1_{k} : k \to k$. 
    Then both $\ker \phi_{\bullet}$ and $\Coker \phi_{\bullet}$
    are the zero filtration, hence
    \begin{equation*}
        \operatorname{coim}\phi_{\bullet} = V_{\bullet}\,, \quad
        \im \phi_{\bullet} = W_{\bullet}\,,
    \end{equation*}
    but the two objects are not isomorphic, for any arrow
    $\psi_{\bullet}:W_{\bullet} \to V_{\bullet}$ has to vanish on $W_{0}=k$,
    so it is necessarily induced by the zero map $k \to 0$.
\end{ex}

\begin{ex}
    Let $\Aa = \cat{TopAb}$ be the category of topological abelian groups,
    that is abelian groups endowed with a topology in such a way that
    their operation $+$ and the inverse $x \mapsto -x$ are continuous maps;
    morphisms in $\cat{TopAb}$ are continuous homomorphisms.
    As for the category $\cat{Ab}$, one sees that $\cat{TopAb}$
    is an additive category; moreover, kernels and cokernels always exist,
    and for every arrow $f:G \to H$, one might check that
    $\ker f$ is the group theoretic kernel of $f$,
    endowed with the subspace topology $\ker f \subset G$,
    while
    \begin{equation*}
        \Coker f = \left. H \middle/ \ol{f(G)} \right. {\,,}
    \end{equation*}
    where $\ol{f(G)}$ denotes the closure in $H$.
    Thus, $\cat{TopAb}$ is a preabelian category, which is \textbf{not} abelian:
    indeed, consider the additive groups $\Q$ and $\R$ with the euclidean topology.
    The inclusion $\iota : \Q \subset \R$ is injective, hence $\ker \iota = 0$,
    from which we deduce that $\operatorname{coim} \iota = \Q/0 = \Q$.
    As $\Q$ is dense in $\R$, we see that 
    $\Coker \iota = \left. \R \middle/ \ol{\Q}\right. = 0$,
    so $\im \iota = \ker(\R \to 0) = \R$, thus $\operatorname{coim} \iota$
    and $\im \iota$ are not isomorphic.
\end{ex}

\begin{ex!}\label{equiv-modr}
    Let $R$ be a commutative ring with unit and $\cat{P}(R)$
    be the preadditive category defined in 
    \hyperref[ring-cat]{\emph{Example~\ref*{ring-cat}}}.
    The category $\cat{Add}(\cat{P}(R),\cat{Ab})$ of additive functors
    and natural tranformations is abelian, indeed it is equivalent
    to $\Mod_{R}$. 
    
    Given an additive functor $F:\cat{P}(R) \to \cat{Ab}$,
    the image $F(\ast)$ is an abelian group, endowed with a scalar multiplication
    \begin{equation*}
        R \times F(\ast) \longrightarrow F(\ast)\,, \quad
        (r,g) \longmapsto F(r)(g)\,.
    \end{equation*}
    Since $F$ is additive, one checks that 
    $$(r+r')g = F(r+r')(g) = F(r)(g)+F(r')(g) = rg +r'g$$
    and since it is a functor, is follows $1_{R}g = g$ and $r(g+g') = rg+rg'$,
    so $F(\ast)$ is an $R$-module. A natural transformation
    $\phi: F \to G$ consists of a group homomorphism
    $\phi' : F(\ast) \to G(\ast)$ such that for every $r \in R$
    the following square commutes:
    \begin{equation}\label{r-linear}
        \begin{tikzcd}
            F(\ast) \ar[r, "\phi'"] \ar[d, "r \cdot "] & G(\ast) \ar[d, "r\cdot"] \\
            F(\ast) \ar[r, "\phi'"] & G(\ast)\,,
        \end{tikzcd}
    \end{equation}
    that is $\phi'(rg) = r\phi'(g)$, hence $\phi'$ is $R$-linear.
    
    Conversely, given any $R$-module $M$, one defines an
    additive functor $F:\cat{P}(R) \to \cat{Ab}$ by setting $F(\ast) = M$
    and, for every $r \in R$, the morphisms $F(r) : M \to M$
    are given by the multiplication $F(r)(m):=r\cdot m$.
    Any $R$-linear map $\phi':M \to N$ in $\Mod_{R}$
    describes the same diagram as in \eqref{r-linear},
    so it induces a natural transformation between
    additive functors $F \to G$, whose images are respectively 
    $M$ and $N$. These two constructions are one the inverse of the other,
    thus the equivalence is proved.
\end{ex!}

%%% COMPARE WITH DEFINITION IN https://stacks.math.columbia.edu/tag/00ZX

The properties listed in 
\hyperref[df-abel]{\textbf{Definition~\ref*{df-abel}}}
give us the appropriate language to talk about exact sequences,
and hence abelian categories seem the appropriate setting
we were hoping to build to develop a notion of homology;
but after a closer inspection, one might notice
that something is missing... the algebra!
It is not clear from \hyperref[df-abel]{\textbf{Definition~\ref*{df-abel}}} 
if there exists of any algebraic operation in $\Aa$ or
on its morphism, so that we can actually talk about ``groups''.
Quite surprisingly, it turns out that any abelian
category is $\cat{Ab}$-enriched!

\begin{thm}
    Every abelian category is an additive category.
    \begin{proof}[Idea of proof]
    Let $\Aa$ be an abelian category.
    The main steps of the proof are the following:
        \begin{enumerate}
            \item for each object $A$ in $\Aa$, 
            let $\Delta_{A}:A \to A \times A$ be the diagonal morphism
            and $q_{A} : A \times A \to \Coker \Delta_{A}$ its cokernel.
            One checks that $\Coker \Delta_{A}$ is isomorphic to $A$
            via the composition $\operatorname{pr}_{1} \circ \langle 1_{A},0 \rangle$,
            where $\operatorname{pr}_{1}$ is the projection on
            the first component of the product;
            
            \item we define a map from the product to $A$, by composing
            \begin{equation*}
                \sigma_{A} : A \times A \overset{q}{\longrightarrow} \Coker \Delta_{A} 
                \longrightarrow A\,,
            \end{equation*}
            which maybe though of as the ``\emph{subtraction}'' in $A$:
            for any two arrows $f,g : B \to A$, we define $f-g$ to be the composite
            \begin{equation*}
            \begin{tikzcd}
                f-g : B \ar[r,"{\langle f, g \rangle}"] 
                & A \times A \ar[r, "\sigma_{A}"] & A\,.
            \end{tikzcd}
            \end{equation*}
            \item For every pair of objects $A,B$ in $\Aa$, define 
            \begin{equation*}
                + : \Hom_{\Aa}(B,A) \times \Hom_{\Aa}(B,A) \longrightarrow \Hom_{\Aa}(B,A)\,,
                \quad (f,g) \longmapsto f + g := f - (0_{BA} - g)\,.
            \end{equation*}
            One has to verify that this operation is associative, 
            with neutral element $0_{BA}$ and commutative, so that
            every set of morphisms $\Hom_{\Aa}(B,A)$ is an abelian group.
            It follows that $\Aa$ is a preadditive category,
            hence additive because it has a zero and both products
            and coproducts (which are isomorphic by \hyperref[biproduct]{\textbf{Proposition~\ref*{biproduct}}}).
        \end{enumerate}
        For a full and detailed proof, check \parencite[1.6~Additivity of abelian categories]{borceux}.
    \end{proof}
\end{thm}

As a consequence of \hyperref[coim-im]{\textbf{Proposition~\ref*{coim-im}}},
one has the following result.

\begin{thm}[\textbf{Image factorization}]
    Let $\Aa$ be an abelian category. Any morphism $f$ in $\Aa$
    can be factored uniquely as the composition $f = i \circ p$,
    where $i$ is a monomorphism and $p$ is a epimorphism.
    \begin{proof}
        We can apply \hyperref[coim-im]{\textbf{Proposition~\ref*{coim-im}}} 
        and take $i : \im f \hookrightarrow B$ and $p = \ol{f} \pi$,
        where $\pi : A \to \operatorname{coim}f$.
        All these maps are unique up to isomorphisms.
    \end{proof}
\end{thm}

