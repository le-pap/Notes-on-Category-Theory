
\section{Triangulated categories}

The notion of triangulated category generalizes the structure 
that exact triangles give to $\cat{K}(\Aa)$. 
One should think of exact triangles as 
substitutes for short exact sequences.

Consider a category $\Dd$ equipped with 
an equivalence $T:\Dd \to \Dd$.
\begin{df}
    A \textbf{triangle} on an ordered triple
    $(A,B,C)$ of objects of $\Dd$ is a 
    triple $(\alpha,\beta,\delta)$ of morphisms
    \begin{center}
        \begin{tikzcd}
            A \ar[r,"\alpha"]
            & B \ar[r, "\beta"]
            & C \ar[r, "\delta"]
            & T(A)\,;
        \end{tikzcd}
    \end{center}
    the triangle $(\alpha,\beta,\delta)$ 
    is usually displayed as follows:
    \begin{center}
        \begin{tikzcd}
            A \ar[rr,"\alpha"]
            & & B \ar[dl, "\beta"] \\
            & C \ar[ul, "\delta"] &\,.
        \end{tikzcd}
    \end{center}
    A \textbf{morphism of triangles} is a triple $(f,g,h)$
    forming the following commutative diagram %in $\Dd$:
    \begin{center}
        \begin{tikzcd}
            A \ar[r] \ar[d, "f"]
            & B \ar[r] \ar[d, "g"]
            & C \ar[r] \ar[d, "h"]
            & T(A) \ar[d, "T(f)"] \\
            A' \ar[r]
            & B' \ar[r]
            & C' \ar[r]
            & T(A')\,.
        \end{tikzcd}
    \end{center}
\end{df}

\begin{notation}
    When dealing with the automorphism $T$,
    it will be convenient to write
    $A[1] := T(A)$ for any object $A \in \Dd$;
    with this idea, if we apply $T$ (or its quasi-inverse)
    many times, we will write
    \begin{equation*}
        A[n] := T^{n}(A)\,, \quad n \in \Z\,.
    \end{equation*}
    Similarly, for any morphism $f \in \Hom_{\Dd}(A,B)$,
    we put $f[n] := T^{n}(f)$.
\end{notation}

\begin{df}
    Let $\Dd$ be an additive category.
    The structure of a \textbf{triangulated category} on $\Dd$
    is given by an additive equivalence 
    $T : \Dd \longrightarrow \Dd$,
    called \textbf{shift functor}, 
    and a set of \textbf{distinguished triangles}
    $(\alpha,\beta,\delta)$
    which are subject to the following four axioms:
    \begin{itemize}
        \item[(\textbf{TR1})]\label{TR1}
        it consists of three points.
        \begin{rmnumerate}
            \item For any $A \in \Dd$, the triangle $(\cat{1}_{A},0,0)$
                %\begin{center}
                %    \begin{tikzcd}
                %        A \ar[rr,"\cat{1}_{A}"]
                %        & & A \ar[dl, "0"] \\
                %        & \cat{0} \ar[ul, "0"] &
                %    \end{tikzcd}
                %\end{center}
                is distinguished.
                
            \item If $(\alpha,\beta,\delta)$ is a distinguished triangle
            and $(\alpha',\beta',\delta')$ is isomorphic to it,
            then $(\alpha',\beta',\delta')$ is distinguished too.
            
            \item Any morphism $\alpha : A \to B$ can be completed
            to a distinguished triangle.
        \end{rmnumerate}
        
        
        \item[(\textbf{TR2})]\label{TR2}
        \emph{\textbf{Rotation}} $-$ 
        If $(\alpha,\beta,\delta)$ is a distinguished triangle on $(A,B,C)$,
        then both its ``\emph{rotates}''
        $(\beta,\delta,-\alpha[1])$ and $(-\delta[-1],\alpha,\beta)$
        are distinguished triangles on $(B,C,A[1])$ and $(C[-1],A,B)$, respectively.
        
        \item[(\textbf{TR3})]\label{TR3}
        \emph{\textbf{Morphisms}} $-$ 
        Suppose there exists a commutative diagram of 
        distinguished triangles with vertical arrows $f$ and $g$:
        \begin{center}
            \begin{tikzcd}
                A \ar[r] \ar[d, "f"]
                & B \ar[r] \ar[d, "g"]
                & C \ar[r] \ar[d, "\exists \, h", dashed]
                & A{[1]} \ar[d, "f{[1]}"] \\
                A' \ar[r]
                & B' \ar[r]
                & C' \ar[r]
                & A'{[1]}\,;
            \end{tikzcd}
        \end{center}
        then there exists a (\textbf{not} necessarily unique) 
        morphism $h:C \to C'$ which 
        completes it to a morphism of triangles.
        
        \item[(\textbf{TR4})]\label{TR4}
        \emph{\textbf{The octahedral axiom}} $-$ 
        Given six objects $A,B,C,A',B',C' \in \Dd$
        and three triangles
        \begin{center}
            \begin{tikzcd}[column sep=small]
                A \ar[rr, "\alpha"] & & B \ar[dl, "i"]
                & & 
                B \ar[rr, "\beta"] & & C \ar[dl, "j"]
                & & 
                A \ar[rr, "\beta \alpha"]& & C \ar[dl, "l"] \\
                & C' \ar[ul,"\de"] & \,, &
                & & A' \ar[ul,"d"] & \,, & 
                & & B' \ar[ul,"\delta"] & \,,
            \end{tikzcd}
        \end{center}
        then there exists a fourth distinguished triangle
        \begin{center}
            \begin{tikzcd}
                C' \ar[rr,"f"]
                & & B' \ar[dl, "g"] \\
                & A' \ar[ul, "i{[1]} \circ d"] &\,,
            \end{tikzcd}
        \end{center}
        such that in the following octahedron 
        \begin{equation}\label{octahedron}
            \begin{tikzcd}[row sep=huge, column sep=huge]
            & B \arrow[dr, squiggly, "\beta"] \arrow[ddl, squiggly, dashed, "i", pos=0.25] \arrow[from=ddr, crossing over, dashed, "\alpha", pos=0.75]  & \\
            A' \arrow[ur, "d"] \arrow[d, "i{[1]} \circ d"'] 
            & & C \arrow[ll, crossing over, "j"'] \\
            C' \arrow[rr, dashed, "\de"] \arrow[dr, squiggly, "f"'] 
            & & A \arrow[u, "\beta \alpha"']  \\
             & B' \arrow[uul, crossing over, "g"', pos=0.6] 
             \arrow[ur, "\delta"'] \arrow[from=uur, squiggly, crossing over, "l"', pos=0.4]  &
            \end{tikzcd}
        \end{equation}
        we have:
        \begin{rmnumerate}
            \item the four distinguished triangles above forming four of the faces;
            
            \item the remaining four faces commute, that boils down to
            \begin{equation*}
                j = g \circ l\,, \quad \de = \delta \circ f\,;
            \end{equation*}
            
            \item the most exterior paths connecting $B$ and $B'$ commute,
            that is
            \begin{equation*}
                f \circ i = l \circ \beta \,, \quad d \circ g = \alpha \circ \delta\,.
            \end{equation*}
        \end{rmnumerate}
    \end{itemize}
\end{df}

\begin{rmk}
    Some comments on axiom 
    \hyperref[TR4]{(\textbf{TR4})} are needed,
    because it is with no doubt 
    the most mysterious
    and confusing one.
    First, we may ``unpack'' the octahedron~\eqref{octahedron}
    to give a nicer visualization:
    notice that the triangle $(f,g,i{[1]}\circ d)$
    completes the commutative plane diagram
    \begin{center}
        \begin{tikzcd}
            A \ar[r, "\alpha"] \ar[d, equals]
            & B \ar[r, "i"] \ar[d, "\beta"]
            & C' \ar[r, "\de"] \ar[d, "\exists"',"f", dashed]
            & A{[1]} \ar[d, equals] \\
            A \ar[r]
            & C \ar[r,"l"] \ar[d, "j"]
            & B' \ar[r, "\delta"] \ar[d, "\exists"',"g", dashed] \ar[d, dashed]
            & A{[1]} \ar[d, "\alpha{[1]}"]\\
            & A' \ar[r, equals] \ar[d, "d"] 
            & A' \ar[r, "d"] \ar[d, "\exists"', dashed]
            & B{[1]} \\
            & B{[1]} \ar[r]
            & C'{[1]}
            & \,.
        \end{tikzcd}
    \end{center}
    
    We may interpret the \hyperref[TR4]{\emph{\textbf{Octahedral Axiom}}}
    as a generalization of the \textbf{Third Isomorphism Theorem} we
    find in abelian categories: indeed, if we think of distinguished
    triangles as an analogous of exact sequences in abelian categories, 
    then the three triangles in \hyperref[TR4]{(\textbf{TR4})} 
    tell us that
    \begin{equation*}
        C' \simeq \left. B \middle/ A \right. \,, \quad
        B' \simeq \left. C \middle/ A \right. \,, \quad
        A' \simeq \left. C \middle/ B \right. \,, 
    \end{equation*}
    thus the \hyperref[TR4]{\emph{\textbf{Octahedral Axiom}}}
    states that $(C/A)/(B/A) \simeq C/B$.
    We may visualize it in the following nice diagram:
    the three lines passing through $A,B$ and $C$ completely
    determine the sequence $C' \to B' \to A'$ on the right:
    \begin{center}
% https://tikzcd.yichuanshen.de/#N4Igdg9gJgpgziAXAbVABwnAlgFyxMJZAJgBoBmAXVJADcBDAGwFcYkQAhEAX1PU1z5CKACykR1Ok1bsAwjz4gM2PASIA2CpIYs2iTgHIF-FUKIAGUuu3S9IAILGlA1cOQAOKzd3t7R3iaCaqKk5t4y+rL+kjBQAObwRKAAZgBOEAC2SOQ0OBBIliCM9ABGMIwACi5m+owwyTggNDoRIAA6bUxoABb0TmmZBbn5iACszbbsWP3pWYg5IHlIAIwBIANzy8NIZFI+tTODiIVLiFt7rR1lOH1rGyvbiGIXdgBWTUWl5VWmwSCpWDi3Uad1mSHGixGuxadmShzmu1Ozxh7DiPEo3CAA
\begin{tikzcd}[column sep=small, row sep=small]
                                         &  &                                           &  & C' \arrow[rrdddd, "f", dashed]                   &  &                       &  &    \\
                                         &&&&&&&&\\
                                         &  &                                           &  &                                         &  &                       &  &    \\
                                         &  &                                           &  &                                         &  &                       &  &    \\
                                         &  & B \arrow[rruuuu, "i"] \arrow[rrd, "\beta"] &  &                                         &  & B' \arrow[rrddd, "g", dashed] &  &    \\
                                         &  &                                           &  & C \arrow[rru, "l"] \arrow[rrrrdd, "j"'] &  &                       &  &    \\
                                         &  &                                           &  &                                         &  &                       &  &    \\
A \arrow[rruuu, "\alpha"] \arrow[rrrruu] &  &                                           &  &                                         &  &                       &  & A'\,.
\end{tikzcd}
    \end{center}
\end{rmk}

\begin{rmk}
    The definition of triangulated category
    may be modified by changing the forth axiom:
    for instance, consider the statement
    \begin{itemize}
        \item[(\textbf{TR4'})]  given a commutative diagram
        \begin{center}
            \begin{tikzcd}
                A \ar[r, "\alpha"] \ar[d, "f"]
                & B \ar[r, "\beta"] \ar[d, "g"]
                & C \ar[r, "\delta"] \ar[d, "\exists"',"h", dashed]
                & A{[1]} \ar[d] \\
                A' \ar[r, "\alpha'"]
                & B' \ar[r, "\beta'"]
                & C' \ar[r, "\delta'"]
                & A'{[1]}
            \end{tikzcd}
        \end{center}
        whose rows are distinguished triangles,
        there exists a morphism $h:C \to C'$
        that makes the diagram commutative and
        makes the mappying cone a distinguished triangle:
        \begin{center}
            \begin{tikzcd}[ampersand replacement=\&, column sep=large]
                B \oplus A' \ar[r, "{\begin{pmatrix}
                    -\beta & 0 \\ g & \alpha'
                \end{pmatrix}}"]
                \& C \oplus B' \ar[r, "{\begin{pmatrix}
                    -\delta & 0 \\ h & \beta'
                \end{pmatrix}}"]
                \& A{[1]} \oplus C' \ar[r, "{\begin{pmatrix}
                    -\alpha{[1]} & 0 \\ f{[1]} & \delta'
                \end{pmatrix}}"]
                \& B{[1]} \oplus A'{[1]}\,.
            \end{tikzcd}
        \end{center}
    \end{itemize}
    Assuming (\textbf{TR1}), (\textbf{TR2}) and (\textbf{TR3}),
    it turns out that (\textbf{TR4'}) is equivalent to (\textbf{TR4}).
    Moreover, it is clear that (\textbf{TR4'}) implies (\textbf{TR3})
    and its strength relies on the fact that it describes the way we can
    complete a morphism of distinguished triangles.
    There are many other different (but equivalent) axioms we can choose
    to modify the \emph{\textbf{Octahedral Axiom}}: for details,
    check \parencite[]{neeman}.
\end{rmk}

\begin{ex}
    Recall that, given a category $\Cc$, its opposite category $\Cc^{op}$
    has the same objects as $\Cc$, but arrows are ``reversed'', i.e.
    $f^{op}: B \to A$ is an morphism in $\Cc^{op}$ if and only if
    there exists $f:A \to B$ in $\Cc$. In fact, this shows that
    we have a bijection
    \begin{equation*}
        (-)^{op} : \Hom_{\Cc}(A,B) \simeq \Hom_{\Cc^{op}}(B,A)\,.
    \end{equation*}
    If $\Dd$ is a triangulated category, the opposite category $\Dd^{op}$
    is triangulated as well: if $T$ is the shift functor in $\Dd$,
    set 
    \begin{equation*}
        t : \Dd^{op} \longrightarrow \Dd^{op}\,,
        \quad
        \left( B \xrightarrow[]{f^{op}} A \right)
        \longmapsto
        \left( T^{-1}B \xrightarrow[]{(T^{-1}f)^{op}} T^{-1}A \right)
    \end{equation*}
    to be the shift functor in $\Dd^{op}$ and say that a sequence
    \begin{center}
        \begin{tikzcd}
            C \ar[r, "g^{op}"]
            & B \ar[r, "f^{op}"]
            & A \ar[r, "h^{op}"]
            & tC
        \end{tikzcd}
    \end{center}
    is a distinguished triangle in $\Dd^{op}$
    if and only if 
    \begin{center}
        \begin{tikzcd}
            C{[-1]} \ar[r, "h"]
            & A \ar[r, "f"]
            & B \ar[r, "g"]
            & C
        \end{tikzcd}
    \end{center}
    is a distinguished triangle in $\Dd$.
    It is easy to check that these two notions define 
    a triangulated structure on $\Dd^{op}$:
    indeed, all the properties \hyperref[TR1]{(\textbf{TR1})-(\textbf{TR4})}
    hold because they are true in $\Dd$, so
    we can build ``opposite'' triangles easily;
    strictly speaking, distinguished triangles
    in $\Dd^{op}$ are the same of $\Dd$,
    but with reversed indexing.
\end{ex}

\begin{ex}
    For any category $\Cc$, let $\Cc^{\Z}$ be the
    category of graded objects in $\Cc$, 
    i.e. families $A_{*} = \Set{A_{n}}_{n \in \Z}$ indexed
    on the integers, a morphism $f: A_* \to B_*$
    being a family of morphisms $f_{n} : A_{n} \to B_{n}$,
    for each $n \in \Z$. This category is naturally
    endowed with a shift functor $T$ given by the
    translation $TA_{*} := A[-1]_*$, that is
    \begin{equation*}
        TA_{n} := A[-1]_n = A_{n-1}\,,
        \quad n \in \Z\,.
    \end{equation*}

    We now define distinguished triangles in $\Cc^{\Z}$
    to be triplets $(\alpha,\beta,\delta)$ such
    that, for every $n \in \Z$, the following
    sequence is exact:
    \begin{center}
        \begin{tikzcd}
            A_{n} \ar[r, "\alpha"]
            & B_{n} \ar[r, "\beta"]
            & C_{n} \ar[r, "\delta"]
            & A_{n-1} \ar[r, "\alpha"]
            & B_{n-1}\,.
        \end{tikzcd}
    \end{center}
    
    Consider now $\Cc = \Ab$. Then $\Ab^{\Z}$ is an abelian category
    (it is essentially $C_{\bullet}(\Ab)$ with no boundary maps),
    and it clearly satisfies both \hyperref[TR1]{(\textbf{TR1})(i)}
    and \hyperref[TR1]{(\textbf{TR1})(ii)}. Note that
    any morphism $\alpha:A_{*} \to B_{*}$ is embedded in a triangle
    of the form
    \begin{center}
        \begin{tikzcd}
            A_* \ar[rr, "\alpha"] 
            & & B_* \ar[ld] \\
            & \Coker(\alpha) \oplus \left(\ker(\alpha){[-1]} \right) \ar[ul] & \,,
        \end{tikzcd}
    \end{center}
    which is given on the $n$-th level by
    \begin{center}
        \begin{tikzcd}
            A_{n} \ar[r, "\alpha_{n}"]
            & B_{n} \ar[r, "q_{n} \oplus 0"]
            & \Coker(\alpha_{n}) \oplus \left(\ker(\alpha_{n-1}) \right) 
            \ar[r, "0 + \iota_{n-1}"]
            & A_{n-1} \ar[r, "\alpha_{n-1}"]
            & B_{n-1}\,,
        \end{tikzcd}
    \end{center}
    where $q$ is the quotient, while $\iota$ the inclusion. 
    Thus, \hyperref[TR1]{(\textbf{TR1})(iii)} holds true.

    If $(\alpha,\beta, \delta)$ is a distinguished triangle,
    then the $n$-th level of
    its rotation $(\beta, \delta, \alpha[-1])$ is given by
    \begin{center}
        \begin{tikzcd}
            B_{n} \ar[r, "\beta"]
            & C_{n} \ar[r, "\delta"]
            & A_{n-1} \ar[r, "\alpha"]
            & B_{n-1} \ar[r, "\beta"]
            & C_{n-1} \,,
        \end{tikzcd}
    \end{center}
    which is an exact sequence because both
    \begin{center}
        \begin{tikzcd}
            A_{n} \ar[r, "\alpha"]
            & B_{n} \ar[r, "\beta"]
            & C_{n} \ar[r, "\delta"]
            & A_{n-1} \ar[r, "\alpha"]
            & B_{n-1}
        \end{tikzcd}
    \end{center}
    and
    \begin{center}
        \begin{tikzcd}
            A_{n-1} \ar[r, "\alpha"]
            & B_{n-1} \ar[r, "\beta"]
            & C_{n-1} \ar[r, "\delta"]
            & A_{n-2} \ar[r, "\alpha"]
            & B_{n-2}
        \end{tikzcd}
    \end{center}
    are exact. The same holds true for the other rotation,
    hence we have \hyperref[TR2]{(\textbf{TR2})}. 
    
    Nevertheless, the axiom \hyperref[TR3]{(\textbf{TR3})}
    is not valid in $\Ab^{\Z}$: consider any abelian group $G$
    to be a graded object concentrated in degree 0; then,
    the diagram
    \begin{center}
        \begin{tikzcd}
            \Z \ar[r, "\cdot 2"] \ar[d, "\cdot 2"]
            & \Z \ar[r] \ar[d, "\Delta"]
            & \Z/2\Z \ar[r]
            & \cat{0} \ar[d, equals] \\
            \Z \ar[r, "\Delta"]
            & \Z^{2} \ar[r, "+"] 
            & \Z \ar[r]
            & \cat{0}\,,
        \end{tikzcd}
    \end{center}
    commutes, but it cannot be completed to a morphism of triangles.
    Thus, the category $\Ab^{\Z}$ is not triangulated.
\end{ex}

\begin{ex!}\label{Z-Vect}\todo{Check Octahedral Axiom.}
    Let $k$ be an arbitrary field, and $\Cc = \Vect{k}$ be the category
    of vector spaces over $k$. Then $\Vect{k}^{\Z}$ is triangulated:
    indeed, axioms \hyperref[TR1]{(\textbf{TR1})} and
    \hyperref[TR2]{(\textbf{TR2})} hold true for the same
    reasons as in the previous example.

    We now verify \hyperref[TR3]{(\textbf{TR3})}:
    given a commutative diagram
    \begin{center}
        \begin{tikzcd}
            A_{n} \ar[r, "\alpha"] \ar[d, "f_n"]
            & B_{n} \ar[r, "\beta"] \ar[d, "g_n"]
            & C_{n} \ar[r, "\delta"] 
            & A_{n-1} \ar[d, "f_{n-1}"] \\
            U_{n} \ar[r, "u"]
            & V_{n} \ar[r, "v"]
            & W_{n} \ar[r, "w"]
            & U_{n-1}
        \end{tikzcd}
    \end{center}
    we can complete it to a morphism of distinguished
    triangles by taking a basis $\Lambda$ of $C_{n}$ 
    and defining $h_{n}:C_{n} \to W_{n}$ via diagram chasing
    as follows:
    \begin{center}
        \begin{tikzcd}
            C_{n} \ar[r, "\delta"] \ar[d, "h_{n}", dashed]
            & A_{n-1} \ar[d, "f_{n-1}"] \ar[r, "\alpha"]
            & B_{n-1} \ar[d, "g_{n-1}"] 
            & & c \in \Lambda \ar[r, mapsto] \ar[d, "h_{n}:="', dashed]
            & \delta c \ar[r, mapsto] \ar[d, mapsto] 
            & 0 \ar[d, mapsto] \\
            W_{n} \ar[r, "w"]
            & U_{n-1} \ar[r, "u"]
            & V_{n-1} \,,
            & & u & f_{n-1}\delta c \ar[r, mapsto] \ar[l, dashed]
            & u\,f_{n-1}\delta c  = 0\,,
        \end{tikzcd}
    \end{center}
    for any basis element $c \in \Lambda$, we note
    that $f\delta c \in \ker u$ by the commutativity
    of the right box; since the bottom row is exact,
    we may lift $f\delta c$ to some $u \in U_{n-1}$,
    so we finally define $h_{n}c := u$. 
    The diagram commutes by construction.

    Finally, we check the \emph{\textbf{Octahedral Axiom}}:
    consider three distinguished triangles
        \begin{center}
            \begin{tikzcd}[column sep=small]
                A_{*} \ar[rr, "\alpha"] & & B_{*} \ar[dl, "i"]
                & & 
                B_{*} \ar[rr, "\beta"] & & C_{*} \ar[dl, "j"]
                & & 
                A_{*} \ar[rr, "\beta \alpha"]& & C_{*} \ar[dl, "l"] \\
                & W_{*} \ar[ul,"\de"] & \,, &
                & & U_{*} \ar[ul,"d"] & \,, & 
                & & V_{*} \ar[ul,"\delta"] & \,,
            \end{tikzcd}
        \end{center}
    and build $f:W_{*} \to V_{*}$ as follows: at the $n$-th level,
    choose a basis $\Lambda$ for $W_{n}$ and,
    for every $c \in \Lambda$, set
    \begin{equation*}
        f_{n} : W_{n} \longrightarrow V_{n}\,, \quad
        f_{n}(c) := 
        \begin{cases}
            v\,, \quad
            &\text{if there exists } v \in \delta_{n}^{-1}(\de_{n}(c))\,; \\
            0\,, \quad &\text{otherwise},
        \end{cases}
    \end{equation*}
    and extend it by linearity; by construction,
    this gives a morphism $f:W_{*} \to V_{*}$ such that
    $\delta f = \de$. Similarly, one builds $g:V_{*} \to U_{*}$
    such that $j=gl$.
\end{ex!}

\begin{exercise}
    If $(\alpha,\beta,\delta)$ is an distinguished triangle,
    show that the compositions $\beta\alpha, \delta\beta$ 
    and $\alpha[1]\,\delta$ are zero in $\Dd$.
    %This is the reason why we call the triangles exact.
    \begin{proof}[Solution]
        By \hyperref[TR1]{(\textbf{TR1})(i)}, 
        the triangle $(\cat{1}_{A},0,0)$ is distinguished,
        thus we may compare
        \begin{center}
            \begin{tikzcd}
                A \ar[r, equals] \ar[d, equals]
                & A \ar[r] \ar[d, "\alpha"]
                & \cat{0} \ar[r] \ar[d]
                & A{[1]} \ar[d, equals] \\
                A \ar[r, "\alpha"]
                & B \ar[r, "\beta"]
                & C \ar[r, "\delta"]
                & A{[1]}\,.
            \end{tikzcd}
        \end{center}
        By the commutativity of the middle box, we deduce $\beta\alpha=0$.
        Similarly, if we use the \emph{Rotation axiom}~\hyperref[TR2]{(\textbf{TR3})}
        we find commutative diagrams
        \begin{center}
            \begin{tikzcd}
                %B \ar[r, equals] \ar[d, equals] &
                B \ar[r] \ar[d, "\beta"]
                & \cat{0} \ar[r] \ar[d]
                & B{[1]} \ar[d, equals] 
                & & C \ar[r] \ar[d, "\delta"]
                & \cat{0} \ar[r] \ar[d]
                & C{[1]} \ar[d, equals]\\
                %B \ar[r, "\beta"] &
                C \ar[r, "\delta"]
                & A{[1]} \ar[r, "\alpha{[1]}"]
                & B{[1]}
                & & A{[1]} \ar[r, "\alpha{[1]}"]
                & B{[1]} \ar[r, "\beta{[1]}"] 
                & C{[1]} \,,
            \end{tikzcd}
        \end{center}
        from which we see that $\delta\beta = 0 = \alpha{[1]}\,\delta$.
    \end{proof}
\end{exercise}

\begin{exercise!}[\textbf{The $5$-lemma}]\label{5lemma}
    Let $(f,g,h)$ be a morphism of distinguished triangles. 
    If two maps are isomorphisms, then the third is an isomorphism as well.
    \begin{proof}[Solution]
        Up to rotating triangles, we may assume
        without loss of generality that both $f$ and $g$ are isomorphism.
        We show that $h$ is an isomorphism too:
        for any object $X \in \Dd$, we apply the functor 
        $\Hom_{\Dd}(-,X)$ to the diagram
        \begin{center}
            \begin{tikzcd}
                A \ar[r] \ar[d, "f", equals]
                & B \ar[r] \ar[d, "g", equals]
                & C \ar[r] \ar[d, "h"]
                & A{[1]} \ar[d, "f{[1]}", equals] \\
                A' \ar[r]
                & B' \ar[r]
                & C' \ar[r]
                & A'{[1]}\,,
            \end{tikzcd}
        \end{center}
        so that we obtain the following commutative diagram in $\Ab$:
        \begin{center}
            \begin{tikzcd}[column sep=small]
                \Hom_{\Dd}(B'{[1]},X) \ar[r] \ar[d, equals]
                & \Hom_{\Dd}(A'{[1]},X) \ar[r] \ar[d, equals]
                & \Hom_{\Dd}(C',X) \ar[r] \ar[d, "- \circ h"]
                & \Hom_{\Dd}(B',X) \ar[r] \ar[d, equals]
                & \Hom_{\Dd}(A',X) \ar[d, equals] \\
                \Hom_{\Dd}(B{[1]},X) \ar[r]
                & \Hom_{\Dd}(A{[1]},X) \ar[r] 
                & \Hom_{\Dd}(C,X) \ar[r] 
                & \Hom_{\Dd}(B,X) \ar[r] 
                & \Hom_{\Dd}(A,X) \,.
            \end{tikzcd}
        \end{center}
        By the classic \textbf{$5$-lemma} for abelian groups,
        we know the central arrow is an isomorphism;
        thus, when $X = C$, we deduce there exists $k:C' \to C$
        such that $kh = \cat{1}_{C}$. Moreover, if we plug $X=C'$,
        we see that $(\cat{1}_{C'} - hk) \circ h = h - h(kh) = 0$,
        so we conclude $\cat{1}_{C'} = hk$ because precomposition
        is an isomorphism.
    \end{proof}
\end{exercise!}

\begin{rmk}
    As a consequence, we see that the completion 
    of any $\alpha: A \to B$ to a triangle,
    as declared in the axiom \hyperref[TR1]{(\textbf{TR1})(iii)},
    is unique up to isomorphism: taken triangles
    $(\alpha,\beta,\delta)$ on $(A,B,C)$ and
    $(\alpha,\beta',\delta')$ on $(A,B,C')$,
    then by \hyperref[TR3]{(\textbf{TR3})} one has
    \begin{center}
            \begin{tikzcd}
                A \ar[r, "\alpha"] \ar[d, equals]
                & B \ar[r, "\beta"] \ar[d, equals]
                & C \ar[r, "\delta"] \ar[d, "\exists"', "h", dashed]
                & A{[1]} \ar[d, equals] \\
                A \ar[r, "\alpha"]
                & B \ar[r, "\beta'"]
                & C' \ar[r, "\delta'"]
                & A{[1]}\,,
            \end{tikzcd}
        \end{center}
    and the \hyperref[5lemma]{$5$-lemma~\ref*{5lemma}} tells us $h$ is an isomorphism.
    This means that every distinguished triangle is determined,
    up to isomorphism, by just one of its maps! In particular,
    the data of the \hyperref[TR4]{\emph{\textbf{Octahedral Axiom}}}
    are completely determined by $A \xrightarrow[]{\alpha} B \xrightarrow[]{\beta} C$.
\end{rmk}

\begin{exercise!}[The Split Lemma]\label{split-lemma}
    Let $A \to B \to C \to A[1]$ be a distinguished triangle 
    in a triangulated category $\Dd$. 
    Suppose that $C \to A[1]$ is trivial. 
    Show that then the triangle is split, 
    i.e. is given by a direct sum 
    decomposition $B \simeq A \oplus C$.
    \begin{proof}[Solution]
        Consider the commutative diagram
        \begin{center}
            \begin{tikzcd}
            C\left[-1\right] \ar[r, "0"]  \ar[d, equals]
            & A \ar[r] \ar[d, equals]
            & B \ar[r] \ar[d, dashed, "\exists"', "h"]
            & C \ar[d, equals] \\
            %& A\left[ 1 \right] \ar[d, equals] \\
            C\left[-1\right] \ar[r, "0"] 
            & A \ar[r, "j_{A}"]
            & A \oplus C \ar[r, "\pi_{C}"]
            & C \,.
            %& A\left[ 1 \right]\,.
            \end{tikzcd}
        \end{center}
        By axiom \hyperref[TR3]{(\textbf{TR3})},
        there exists an arrow $h$ which completes
        the diagram to a morphism of triangles.
        Then by the \hyperref[5lemma]{5-lemma}
        we conclude $h$ must be an isomorphism.
    \end{proof}
\end{exercise!}

\begin{prop}
    Given an abelian category $\Aa$, the category $\cat{K}(\Aa)$ is triangulated.
    \begin{proof}
        We show that a triangulated structure on $\cat{K}(\Aa)$ is given 
        by letting the shift functor $TA^{\bullet} := A^{\bullet}[1]$
        to be the translation and the family of distinguished triangles
        to be given by exact triangles.

        It is easy to check that both axioms \hyperref[TR1]{(\textbf{TR1})}
        and \hyperref[TR2]{(\textbf{TR2})} hold. Without loss of generality,
        it is enough to check \hyperref[TR3]{(\textbf{TR3})} on
        strict triangles: the diagram
        \begin{center}
            \begin{tikzcd}
                A^{\bullet} \ar[r, "u"] \ar[d, "f"]
                & B^{\bullet} \ar[r] \ar[d, "g"]
                & \cat{C}(u) \ar[r] \ar[d, "\exists"', "h", dashed]
                & A^{\bullet}{[1]} \ar[d, "f{[1]}"] \\
                \tilde{A}^{\bullet} \ar[r, "w"]
                & \tilde{B}^{\bullet} \ar[r]
                & \cat{C}(w) \ar[r]
                & \tilde{A}^{\bullet}{[1]}\,
            \end{tikzcd}
        \end{center}
        can be completed by defining the cochain map
        \begin{equation*}
            h := (f[1],g) : \cat{C}(u) \longrightarrow \cat{C}(w)\,.
        \end{equation*}
        It remains to prove the \hyperref[TR4]{\textbf{\emph{Octahedral Axiom}~(\textbf{TR4})}}.
        As before, we may assume the given triangles are strict:
        \begin{center}
            \begin{tikzcd}[column sep=small]
                A^{\bullet} \ar[rr, "\alpha"] & & B^{\bullet} \ar[dl, "j_{B}"]
                & & 
                B^{\bullet} \ar[rr, "\beta"] & & C^{\bullet} \ar[dl, "j_{C}"]
                & & 
                A^{\bullet} \ar[rr, "\beta \alpha"]& & C^{\bullet} \ar[dl, "j_{C}"] \\
                & \cat{C}(\alpha) \ar[ul,"\pi_{A{[1]}}"] & \,, &
                & & \cat{C}(\beta) \ar[ul,"\pi_{B{[1]}}"] & \,, & 
                & & \cat{C}(\beta\alpha) \ar[ul,"\pi_{A{[1]}}"] & \,,
            \end{tikzcd}
        \end{center}
        thus we may define
        \begin{center}
            \begin{tikzcd}
                \cat{C}(\alpha) \ar[rr, "{(\cat{1}_{A[1]},\beta)}"] 
                & & \cat{C}(\beta\alpha) \ar[dl, "{(\alpha[1], \cat{1}_{C})}"]\\
                & \cat{C}(\beta) \ar[ul, "j_{B[-1]}\pi_{B[1]}"] & \,.
            \end{tikzcd}
        \end{center}
        By construction, the octahedron described in \hyperref[TR4]{(\textbf{TR4})}
        commutes, so we conclude if we show that the above triangle is a distinguished triangle.
        Set $f := (\cat{1}_{A[1]},\beta)$ and consider $\cat{C}(f)$;
        on degree $n$ it is
        \begin{equation*}
            \cat{C}(f)^{n} = \cat{C}(\alpha)^{n+1} \oplus \cat{C}(\beta)^{n}
            = (A^{n+2} \oplus B^{n+1}) \oplus (A^{n+1} \oplus C^{n})\,,
        \end{equation*}
        thus we can embed $\cat{C}(\beta)^{n}=B^{n+1} \oplus C^{n}$ in it.
        This gives us a natural inclusion $\iota$ which fits in the commutative diagram
        \begin{center}
            \begin{tikzcd}
                \cat{C}(\alpha) \ar[r, "f"] \ar[d, equals]
                & \cat{C}(\beta\alpha) \ar[r] \ar[d, equals]
                & \cat{C}(\beta) \ar[r] \ar[d, "\iota"]
                & \cat{C}(\alpha){[1]} \ar[d, equals] \\
                \cat{C}(\alpha) \ar[r, "f"]
                & \cat{C}(\beta\alpha) \ar[r]
                & \cat{C}(f) \ar[r]
                & \cat{C}(\alpha){[1]}\,,
            \end{tikzcd}
        \end{center}
        in which $\iota$ is a homotopy equivalence:
        define 
        \begin{equation*}
            \phi^{\bullet}:\cat{C}(f) \longrightarrow \cat{C}(\beta)\,, \quad
            \phi^{n}\big(a',b,a,c\big) = (\alpha^{n+1}(a) + b, c)\,,
        \end{equation*}
        and note that $\phi^{\bullet}\iota(b,c)=\phi(0,b,0,c) = (b,c)$, 
        i.e. it holds $\phi^{\bullet}\iota = \cat{1}_{\cat{C}(\beta)}$.
        On the other hand, the map
        \begin{equation*}
            s : \cat{C}(f)^{n} \longrightarrow \cat{C}(f)^{n-1}\,,
            \quad s(a',b,a,c) = (a,0,0,0)
        \end{equation*}
        defines a homotopy between $\iota\phi^{\bullet}$ and $\cat{1}_{\cat{C}(f)}$:
        the coboundary map of the mappying cone $\cat{C}(f)$ is
        represented by the matrix
        \begin{equation*}
            d_{\cat{C}(f)} 
            = \begin{pmatrix}
                -d_{\cat{C}(\alpha)} & 0 \\ f^{\bullet} & d_{\cat{C}(\beta\alpha)}
            \end{pmatrix}
            = \begin{pmatrix}
                d_{A} & 0 & 0 & 0 \\
                \alpha & -d_{B} & 0 & 0 \\
                \cat{1} & 0 & -d_{A} & 0 \\
                0 & -\beta & \beta\alpha & d_{C}
            \end{pmatrix}\,,
        \end{equation*}
        hence we compute
        \begin{align*}
            ds + sd 
            = \begin{pmatrix}
                d_{A} \\ \alpha \\ -\cat{1}_{A} \\ 0
            \end{pmatrix} 
            + \begin{pmatrix}
                -d_{A} - \cat{1}_{A} \\ 0 \\ 0 \\ 0
            \end{pmatrix}
            = \begin{pmatrix}
                -\cat{1}_{A} \\ \alpha \\ -\cat{1}_{A} \\ 0
            \end{pmatrix} 
            = \begin{pmatrix}
                0 \\ \alpha + \cat{1}_{B} \\ 0 \\ \cat{1}_{C}
            \end{pmatrix} 
            - \begin{pmatrix}
               \cat{1}_{A} \\ \cat{1}_{B} \\ \cat{1}_{A} \\ \cat{1}_{C}
            \end{pmatrix}
            = \iota\phi^{\bullet} - \cat{1}_{\cat{C}(f)}\,.
        \end{align*}
        We conclude that $\iota$ is an isomorphism in $\cat{K}(\Aa)$.
    \end{proof}
\end{prop}

\begin{cor}
    Let $\Cc \subset C^{\bullet}(\Aa)$ be a full subcategory
    and $\cat{K}(\Cc)$ be its corresponding quotient category.
    Suppose that $\Cc$ is an additive category and is closed
    under translations and under contruction of mapping cones.
    Then $\cat{K}(\Cc)$ is a triangulated category.
    In particular, $\cat{K}^{\flat}(\Aa), \cat{K}^{-}(\Aa)$ and
    $\cat{K}^{+}(\Aa)$ are triangulated.
\end{cor}

\begin{df}
    An additive functor $F: \Cc \to \Dd$ 
    between triangulated categories
    with shift functors $T_{\Cc}$, and $T_{\Dd}$
    respectively, is called \textbf{exact} 
    (or \textbf{triangulated}) if it satisfies
    the following:
    \begin{itemize}
        \item[(\textbf{EF1})]\label{EF1} 
        there exists an isomorphism of functors
        \begin{equation*}
            F \circ T_{\Cc} \xrightarrow[]{\sim} T_{\Dd} \circ F\,;
        \end{equation*}

        \item[(\textbf{EF2})]\label{EF2} 
        any distinguished triangle $A \to B \to C \to A[1]$
        in $\Cc$ is mapped to a distinguished triangle
        \begin{center}
            \begin{tikzcd}
                F(A) \ar[r]
                & F(B) \ar[r]
                & F(C) \ar[r]
                & F(A){[1]}\,
            \end{tikzcd}
        \end{center}
        in $\Dd$, where we use the isomorphism $F(A[1]) \simeq F(A)[1]$
        given by \hyperref[EF1]{(\textbf{EF1})}.
    \end{itemize}
\end{df}

\begin{rmk}
    Once again, the notions of a triangulated category 
    and of an exact functor have to be adjusted 
    when one is interested in additive categories over a field $k$.
    In this case, the shift functor should be $k$-linear 
    and one usually considers only $k$-linear exact functors.
\end{rmk}

\begin{ex}
    Recall that in \hyperref[Z-Vect]{Exercise~\ref*{Z-Vect}}
    we proved that $\Vect{k}^{\Z}$ is a triangulated category.
    The total cohomology $H^* : \cat{K}(\Vect{k}) \to \Vect{k}^{\Z}$
    is a morphism of triangulated categories: indeed, 
    $H^{*}$ is additive and for every $n \in \Z$
    and every object $V^{\bullet}$ it holds
    \begin{equation*}
        (H^{*}TV^{\bullet})^{n}
        = H^{n}(V^{\bullet}[1]) 
        = H^{n-1}(V^{\bullet}) 
        = \left(H^{*}(V^{\bullet})[1]\right)^{n}
        = (TH^{*}V^{\bullet})^{n}\,,
    \end{equation*}
    so $H^{*}$ commutes with shifts; 
    axiom \hyperref[EF2]{(\textbf{EF2})}
    is the \hyperref[LHS]{Long Exact Sequence~\ref*{LHS}} induced in cohomology. 
\end{ex}

\begin{df}
    Let $\Dd$ be a triangulated category and
    $\Aa$ an abelian category.
    An additive functor $F : \Dd \to \Aa$ is called
    a \textbf{covariant cohomological functor}
    if, whenever $(\alpha, \beta, \delta)$ is a
    distinguished triangle on $(A,B,C)$,
    the long sequence
    \begin{center}
        \begin{tikzcd}
            \dots \ar[r]
            & H(A{[n]}) \ar[r, "H\alpha"]
            & H(B{[n]}) \ar[r, "H\beta"]
            & H(C{[n]}) \ar[r, "H\delta"]
            & H(A{[n+1]}) \ar[r]
            & \dots
        \end{tikzcd}
    \end{center}
    is exact in $\Aa$.
    We often write $H^{n}(A) := H(A[n])$.

    A \textbf{contravariant cohomological functor}
    on $\Dd$ is a covariant cohomological functor
    $F : \Dd^{op} \to \Aa$ (remember $\Dd^{op}$ is triangulated).
\end{df}

\begin{ex}
    The \textbf{zero-th cohomology} $H^{0} : \cat{K}(\Aa) \to \Aa$
    is a cohomological functor.
\end{ex}

\begin{ex}
    Let $\Dd$ be a triangulated category. For any $X \in \Dd$,
    the functor 
    $$\Hom_{\Dd}(X,-) : \Dd \longrightarrow \Ab$$ 
    is cohomological: indeed,
    given a distinguished triangle $(\alpha, \beta, \delta)$ 
    on $(A,B,C)$, the sequence
    \begin{center}
        \begin{tikzcd}
            \Hom_{\Dd}(X,A) \ar[r, "\alpha_{*}"]
            & \Hom_{\Dd}(X,B) \ar[r, "\beta_{*}"]
            & \Hom_{\Dd}(X,C) 
        \end{tikzcd}
    \end{center}
    is exact because $\beta\alpha=0$ implies $\im \alpha^{*} \subset \ker \beta^{*}$,
    and conversely, whenever $g \in \Hom_{\Dd}(X,B)$ is such that
    $\alpha_{*}g = \alpha \circ g = 0$, by \hyperref[TR3]{(\textbf{TR3})} applied
    to the rotated triangle we have
    \begin{center}
        \begin{tikzcd}
            X{[-1]}\ar[r] \ar[d]
            & \cat{0} \ar[d] \ar[r] 
            & X \ar[r, equals] \ar[d, "\exists"', "f", dashed]
            & X \ar[r] \ar[d, "g"]
            & \cat{0} \ar[d] \\
            B{[-1]} \ar[r]
            & C{[-1]} \ar[r]
            & A \ar[r, "\alpha"]
            & B \ar[r, "\beta"]
            & C\,,
        \end{tikzcd}
    \end{center}
    thus $g = \alpha_{*}f$ and we conclude that $\im \alpha^{*} = \ker \beta^{*}$.
    Finally, by shifting the triangle thanks to \hyperref[TR2]{(\textbf{TR2})},
    it follows that the sequence is exact everywhere else. 
    In a similar fashion, one proves that $\Hom_{\Dd}(-,X)$
    is a contravariant cohomological functor.

    Moreover, if we additionally assume that $\Dd$ is $k$-linear, for some field $k$,
    then for any $X \in \Dd$ yields a functor
    $$\Hom_{\Dd}(X,-) : \Dd \longrightarrow \Vect{k}$$
    which is cohomological and gives rise to a long exact sequence of
    vector spaces over $k$. The same holds true for the
    contravariant functor $\Hom_{\Dd}(-,X)$.
\end{ex}

\begin{rmk}
    Once again, the notions of a triangulated category 
    and of an exact functor have to be adjusted when 
    one is interested in \emph{additive categories 
    over a field} $k$. 
    In this case, the shift functor should be $k$-linear 
    and one usually considers only $k$-linear exact functors.
%Also note that in this case the two 
%long exact cohomology sequences  associated to a distinguished %triangle are long exact sequences of k-vector spaces.
\end{rmk}

% 
\begin{ex}[Verdier]
    Let $\Dd$ be a triangulated category.
    Show that every commutative square
    \begin{center}
        \begin{tikzcd}
            A \ar[r,"i"] \ar[d, "u"]
            & B \ar[d] \\
            A' \ar[r] & B'
        \end{tikzcd}
    \end{center}
    in $\Dd$ can be completed to a diagram
    \begin{center}
        \begin{tikzcd}
            A \ar[r,"i"] \ar[d, "u"]
            & B \ar[r, "j"] \ar[d] 
            & C \ar[r, "k"] \ar[d] 
            & A{[1]} \ar[d, "u{[1]}"]\\
            A' \ar[r] \ar[d, "v"]
            & B' \ar[r] \ar[d] 
            & C' \ar[r] \ar[d] 
            & A'{[1]} \ar[d, "v{[1]}"] \\
            A'' \ar[r] \ar[d, "w"]
            & B'' \ar[r] \ar[d] 
            & C'' \ar[r, squiggly] \ar[d, squiggly] 
            & A''{[1]} \ar[d, "w{[1]}", squiggly] \\
            A{[1]} \ar[r, "i{[1]}"]
            & B{[1]}\ar[r, "j{[1]}"]
            & C{[1]} \ar[r, "k{[1]}", squiggly]
            & A{[2]}\,, \\
        \end{tikzcd}
    \end{center}
    in which all the rows and columns are exact
    and all the squares commute,
    except the squigglyd box on the bottom right,
    which is anticommutative.
    \begin{proof}
        Thanks to axiom~\hyperref[TR1]{(\textbf{TR1})(iii)},
        we may embed every map of the box in some distinguished
        triangle, thus we get a commutative diagram
        \begin{center}
        \begin{tikzcd}
            A \ar[r,"i"] \ar[d, "u"]
            & B \ar[r, "j"] \ar[d] 
            & C \ar[r, "k"]
            & A{[1]} \ar[d, "u{[1]}"]\\
            A' \ar[r] \ar[d, "v"]
            & B' \ar[r] \ar[d] 
            & C' \ar[r] 
            & A'{[1]} \ar[d, "v{[1]}"] \\
            A'' \ar[d, "w"]
            & B'' \ar[d] 
            & 
            & A''{[1]} \ar[d, "w{[1]}", squiggly] \\
            A{[1]} \ar[r, "i{[1]}"]
            & B{[1]}\ar[r, "j{[1]}"]
            & C{[1]} \ar[r, "k{[1]}", squiggly]
            & A{[2]}\,; \\
        \end{tikzcd}
        \end{center}
        since there exists $A'' \to B''$ which completes 
        the left rectangle to
        a morphism of triangles, by appying 
        axiom~\hyperref[TR1]{(\textbf{TR1})(iii)}
        once again we obtain
        \begin{equation}\label{verdier-diag}
        \begin{tikzcd}
            A \ar[r,"i"] \ar[d, "u"]
            & B \ar[r, "j"] \ar[d] 
            & C \ar[r, "k"]
            & A{[1]} \ar[d, "u{[1]}"]\\
            A' \ar[r] \ar[d, "v"]
            & B' \ar[r] \ar[d] 
            & C' \ar[r] 
            & A'{[1]} \ar[d, "v{[1]}"] \\
            A'' \ar[r] \ar[d, "w"]
            & B'' \ar[d] \ar[r]
            & C'' \ar[r]
            & A''{[1]} \ar[d, "w{[1]}", squiggly] \\
            A{[1]} \ar[r, "i{[1]}"]
            & B{[1]}\ar[r, "j{[1]}"]
            & C{[1]} \ar[r, "k{[1]}", squiggly]
            & A{[2]}\,. \\
        \end{tikzcd}
        \end{equation}
        As a first intuition, one would be tempted
        to complete the third column by exploiting
        \hyperref[TR3]{(\textbf{TR3})}: even though
        we would end up with a commutative diagram,
        the problem is that the column obtained this way
        needs not be a distinguished triangle.
        Thus, the key is to build a triangle
        on $(C'',C,C')$ thanks to the first part
        of the \hyperref[TR4]{octahedral axiom~(\textbf{TR4})}.

        First, embed the composition $A \to B'$ into 
        a triangle on $(A,B',D)$, for some $D \in \Dd$,
        to get
        \begin{center}
        \begin{tikzcd}
            A \ar[r,"i"]  \ar[dr] %\ar[d, "u"]
            & B \ar[r, "j"] \ar[d] 
            & C \ar[r, "k"] \ar[dd, dashed, "\exists"']
            & A{[1]} \\
            %A' \ar[r] \ar[d, "v"]
            & B' \ar[dr] \ar[d] 
            & 
            & \\
            %A'' \ar[d, "w"]
            & B'' \ar[d] \ar[dr, equals]
            & D \ar[dr] \ar[d, dashed, "\exists"']
            &\\
            %A{[1]} \ar[r, "i{[1]}"]
            & B{[1]}
            & B'' \ar[d, dashed, "\exists"']
            & A{[1]} \\
            & & C{[1]} & \,,
        \end{tikzcd}
        \end{center}
    thus \hyperref[TR4]{(\textbf{TR4})} ensures
    there is a distinguished triangle on $(C,D,B'')$.
    From the diagram
    \begin{center}
        \begin{tikzcd}
            A \ar[d, "u"] \ar[dr]
            & 
            & 
            & 
            & \\
            A' \ar[r] \ar[d, "v"]
            & B' \ar[r] \ar[dr] 
            & C' \ar[r] \ar[dr, equals]
            & A'{[1]} 
            & \\
            A'' \ar[d, "w"]  \ar[rr, dashed, "\exists"]
            & 
            & D \ar[dr]  \ar[r, dashed, "\exists"]
            & C' \ar[r, dashed, "\exists"]
            & A''{[1]}\\
            A{[1]} 
            & 
            & 
            & A{[1]}
            & \, \\
        \end{tikzcd}
        \end{center}
        we obtain a distinguished triangle on $(A'',D,C')$.
        Finally, we see from
        \begin{center}
        \begin{tikzcd}
            A'' \ar[r]  \ar[dr] %\ar[d, "u"]
            & D \ar[r] \ar[d] 
            & C' \ar[r] \ar[dd, "\exists"', dashed]
            & A{[1]} \\
            %A' \ar[r] \ar[d, "v"]
            & B'' \ar[dr] \ar[d] 
            & 
            & \\
            %A'' \ar[d, "w"]
            & C{[1]} \ar[d] \ar[dr, equals]
            & C'' \ar[dr] \ar[d, dashed, "\exists"']
            &\\
            %A{[1]} \ar[r, "i{[1]}"]
            & D{[1]}
            & C{[1]} \ar[d, dashed, "\exists"']
            & A{[1]} \\
            & & C'{[1]} & \,,
        \end{tikzcd}
        \end{center}
        we find a distinguished triangle on $(C',C'',C)$,
        and by rotating it, we get the desired triangle
        on $(C,C',C'')$ which fits in the diagram~\eqref{verdier-diag}
        by construction.
    \end{proof}
\end{ex}

\begin{prop}\label{adj-exact}
    Let $F:\Dd \to \Dd'$ be an exact functor 
    between triangulated categories.
    If $F \dashv H$, then $H: \Dd' \to \Dd$ is exact.
    Similarly, if $G \dashv F$, then $G: \Dd' \to \Dd$.
    \begin{proof}
        Let $T$, respectively $T'$, denote the shift functor
        of $\Dd$, resp. $\Dd'$. We first show that $H$
        satisfies \hyperref[EF1]{(\textbf{EF1})}: 
        by exactness of $F$, there is a natural isomorphism
        $FT \simeq T'F$, which yields $(T')^{-1}F \simeq FT^{-1}$.
        Hence, for every $A, B \in \Dd$
        it holds
        \begin{align*}
            \Hom_{\Dd}(A,HT'(B))
            &\simeq \Hom_{\Dd'}(F(A),T'(B)) \\
            &\simeq \Hom_{\Dd'}((T')^{-1}F(A),B) \\
            &\simeq \Hom_{\Dd'}(FT^{-1}(A),B) \\
            &\simeq \Hom_{\Dd}(T^{-1}(A),H(B)) \\
            &\simeq \Hom_{\Dd}(A,TH(B))\,,
        \end{align*}
        thus, the \hyperref[yoneda]{Yoneda's Lemma}
        yields an isomorphism $HT' \simeq TH$.

        Consider now a distinguished triangle
        \begin{center}
            \begin{tikzcd}
                A \ar[r]
                & B \ar[r]
                & C \ar[r]
                & A{[1]}
            \end{tikzcd}
        \end{center}
        in $\Dd'$; we need to show that $H$ maps it
        in a distinguished triangle in $\Dd$.
        By (\textbf{EF1}) we know that $H(A[1]) \simeq H(A)[1]$,
        hence by the axiom \hyperref[TR1]{(\textbf{TR1})(iii)} we can
        complete $H(A) \to H(B)$ to a distinguished triangle
        \begin{center}
            \begin{tikzcd}
                H(A) \ar[r]
                & H(B) \ar[r]
                & C_{0} \ar[r]
                & H(A){[1]}\,.
            \end{tikzcd}
        \end{center}
        Usign the counit $\epsilon : FH \to \cat{1}_{\Dd'}$
        and the assumption that $F$ is exact, 
        we get the following commutative diagram,
        whose rows are distinguished triangles:
        \begin{center}
            \begin{tikzcd}
                FH(A) \ar[r] \ar[d, "\epsilon_{A}"]
                & FH(B) \ar[r] \ar[d, "\epsilon_{B}"]
                & F(C_{0}) \ar[d, "h", dashed] \ar[r]
                & FH(A){[1]} \ar[d, "\epsilon_{A}{[1]}"] \\
                A \ar[r]
                & B \ar[r]
                & C \ar[r]
                & A{[1]}\,,
            \end{tikzcd}
        \end{center}
        thus, by \hyperref[TR3]{(\textbf{TR3})} it can be completed
        to a morphism $(\epsilon_{A}, \epsilon_{B}, h)$ of distinguished triangles.
        If we apply $H$ to this diagram, we can ``attach''
        the first triangle to it by using the unit
        $\eta : \cat{1}_{\Dd} \to HF$ in the following way:
        \begin{center}
            \begin{tikzcd}[column sep=large]
                H(A) \ar[r] \ar[d, "\eta_{H(A)}"] \ar[dd, equals, bend right=75]
                & H(B) \ar[r] \ar[d, "\eta_{H(B)}"]
                & C_{0} \ar[r] \ar[d, "\eta_{C_{0}}"] 
                & H(A){[1]} \ar[d] \ar[dd, equals, bend left=75]\\ %, "\eta_{H(A)}{[1]}"] \\
                HFH(A) \ar[r] \ar[d, "\epsilon_{A}"]
                & HFH(B) \ar[r] \ar[d, "\epsilon_{B}"]
                & HF(C_{0}) \ar[d, "H(h)"] \ar[r]
                & HFH(A){[1]} \ar[d] \\ %, "\epsilon_{A}{[1]}"] \\
                H(A) \ar[r]
                & H(B) \ar[r]  \ar[from=uu, equals, bend right=75, crossing over]
                & H(C) \ar[r]
                & H(A){[1]}\,,
            \end{tikzcd}
        \end{center}
        where the compositions $\epsilon \circ \eta_{H(-)}$ are the identities
        by \hyperref[unit-counit-identity]{Exercise~\ref*{unit-counit-identity}}.
        Unfortunately, we cannot conclude by applying 
        the \hyperref[5lemma]{\textbf{5-lemma}}
        because the bottom triangle is not distinguished. Nevertheless,
        remember that the functor $\Hom_{\Dd}(X,-)$ is cohomological,
        for any $X \in \Dd$; thus, by using the adjunction
        we obtain the diagram of abelian groups
        \begin{center}
            \begin{tikzcd}[column sep=small]
                \Hom_{\Dd'}(F(X),A) \ar[r] \ar[d, equals]
                & \Hom_{\Dd'}(F(X),B) \ar[r] \ar[d, equals]
                & \Hom_{\Dd}(X,C_{0}) \ar[d] \ar[r]
                & \Hom_{\Dd'}(F(X),A{[1]}) \ar[d,equals] \\
                \Hom_{\Dd'}(F(X),A) \ar[r] 
                & \Hom_{\Dd'}(F(X),B) \ar[r]
                & \Hom_{\Dd}(X,H(C))  \ar[r]
                & \Hom_{\Dd'}(F(X),A{[1]}) \,,
            \end{tikzcd}
        \end{center}
        so by applying the classical 5 lemma for abelian groups
        we deduce that $\Hom_{\Dd'}(X,C_{0}) \simeq \Hom_{\Dd'}(X,H(C))$,
        thus from the \hyperref[yoneda]{Yoneda's Lemma} it follows
        $C_{0}\simeq H(C)$. Finally, by \hyperref[TR1]{(\textbf{TR1})(ii)}
        we see that $H(A) \to H(B) \to H(C) \to H(A){[1]}$ is distinguished,
        so we conclude that $H$ satisfies \hyperref[EF2]{(\textbf{EF2})}.
    \end{proof}
\end{prop}

\begin{df}
    Given a triangulated category $\Dd$, a subcategory $\Dd' \subset \Dd$
    is a \textbf{triangulated subcategory} if $\Dd'$ admits
    a structure of triangulated category such that the inclusion is exact.
\end{df}

\begin{rmk}
    \begin{rmnumerate}
        \item Notice in particular that the family of distinguished triangles
        of $\Dd'$ must be included in the distinguished triangles of $\Dd$.

        \item If $\Dd' \subset \Dd$ is full, then it is a triangulated subcategory
        if and only if $\Dd'$ is invariant under the shift functor and,
        for every distinguished triangle $A \to B \to C \to A{[1]}$ in $\Dd$,
        with $A,B \in \Dd'$, then $C$ is isomorphic to some object in $\Dd'$:
        indeed, $A \to B$ is a morphism in $\Dd'$, thus it can be completed
        to a distinguished triangle $A \to B \to C' \to A{[1]}$ in $\Dd'$,
        hence  we have $C \simeq C'$ by the \hyperref[5lemma]{\textbf{5-lemma}}.
    \end{rmnumerate}
\end{rmk}

\begin{df}
    A triangulated subcategory $\Dd' \subset \Dd$ is called \textbf{admissible}
    if the inclusion has a right adjoint $\pi: \Dd \to \Dd'$, that we will call
    \textbf{orthogonal projection}. 
    The (\textbf{right}) \textbf{orthogonal complement} of $\Dd'$
    is the full subcategory $\Dd'^{\perp} \subset \Dd$ of those objects $C \in \Dd$
    such that
    \begin{equation*}
        \Hom_{\Dd}(A,C) = 0\,, \quad \text{for all } A \in \Dd'\,.
    \end{equation*}
    Similarly, one can define ${}^{\perp}\Dd'$ as the full subcategory
    of those objects $C \in \Dd$ such that
    \begin{equation*}
        \Hom_{\Dd}(C,A) = 0\,, \quad \text{for all } A \in \Dd'\,.
    \end{equation*}
\end{df}

Note that the orthogonal projection $\pi$ is an exact functor 
by \hyperref[adj-exact]{Proposition~\ref*{adj-exact}}.


\begin{rmk}
    The orthogonal complement of a triangulated subcategory $\Dd' \subset \Dd$
    naturally inherits a triangulated structure from $\Dd$: 
    indeed, we see that $T\vert_{\Dd'^{\perp}}$ has values in $\Dd'^{\perp}$
    because, for every $A \in \Dd'$ and $n \in \Z$, it holds
    \begin{equation*}
        \Hom_{\Dd}(A,C[n]) \simeq \Hom_{\Dd}(A[-n],C) = 0\,;
    \end{equation*}
    moreover, given an exact triangle
    \begin{equation*}
        C \longrightarrow C' \longrightarrow B \longrightarrow C[1]
    \end{equation*}
    in $\Dd$, where $C,C' \in \Dd'^{\perp}$, then by applying $\Hom_{\Dd}(A,-)$
    with $A \in \Dd'$ we obtain the exact sequence
    \begin{equation*}
        0 \longrightarrow \Hom_{\Dd}(A,B) \longrightarrow 0\,,
    \end{equation*}
    from which we deduce that $B \in \Dd'^{\perp}$.
\end{rmk}

The concept of orthogonality in triangulated category is inspired 
by the familiar orthogonality on vector spaces that we know from linear algebra.
We know that, is $W \subset V$ is a vector subspace, the the total
space admits a decomposition $V = W \oplus W^{\perp}$.
Similarly, whenever a full admissible subcategory $\Dd' \subset \Dd$ exists,
every object in $\Dd$ fits in a sequence between a component in $\Dd'$
and a component in its orthogonal complement
\begin{lemma}[Semi-orthogonal decomposition]\label{admissible-dec}
    A full triangulated subcategory $\Dd' \subset \Dd$
    is admissible if and only if, for all $B \in \Dd$,
    there exist a distinguished triangle
    \begin{equation*}
        A \longrightarrow B \longrightarrow C \longrightarrow A[1]\,,
    \end{equation*}
    with $A \in \Dd'$ and $C \in \Dd'^{\perp}$.
    \begin{proof}
        Assume $\Dd'$ is admissible: call $\pi$ its orthogonal projection
        and set $A:=\pi(B) \in \Dd'$.
        Using the adjunction, the identity $\cat{1}_{A}$ corresponds
        to a map $A \to B$ in $\Dd$, which can be completed
        to a distinguished triangle
        \begin{equation*}
            A \longrightarrow B \longrightarrow C \longrightarrow A[1]\;
        \end{equation*}
        note that for any $A' \in \Dd'$, there is an isomorphism
        \begin{equation*}
            \Hom_{\Dd}(A',A) 
            = \Hom_{\Dd'}(A',A)
            = \Hom_{\Dd'}(A',\pi(B)) 
            \simeq \Hom_{\Dd}(A',B)\,,
        \end{equation*}
        hence, if we apply the functor $\Hom_{\Dd}(A',-)$ on the triangle,
        then we get $\Hom_{\Dd}(A',C)=0$, so $C \in \Dd'^{\perp}$.

        Conversely, for any $B \in \Dd$, 
        define $\pi(B)$ to be an object in $\Dd'$
        that fits in a distinguished triangle
        \begin{equation*}
            \pi(B) \longrightarrow B \longrightarrow C \longrightarrow \pi(B)[1]\,,
        \end{equation*}
        with $C \in \Dd'^{\perp}$. If $A' \to B \to C' \to A'[1]$
        is another triangle of this form, 
        then for every $X \in \Dd'$ we have
        \begin{align*}
            \Hom_{\Dd'}(X,\pi(B)) 
            = \Hom_{\Dd}(X,\pi(B)) 
            \simeq \Hom_{\Dd}(X,B)
            \simeq \Hom_{\Dd}(X,A')
            = \Hom_{\Dd}(X,A')\,,
        \end{align*}
        so by the \hyperref[yoneda]{Yoneda's Lemma} we deduce
        that $A' \simeq \pi(B)$ in $\Dd'$. This shows that $\pi$
        is defined on objects up to isomorphism.
        Finally, we can define $\pi$ on morphisms too:
        given $f:B \to B'$ in $\Dd$,
        we define $\pi(f)$ to be the image of $f$ via the composition
        \begin{equation*}
            \Hom_{\Dd}(B,B') \longrightarrow
            \Hom_{\Dd}(\pi(B),B') 
            \simeq \Hom_{\Dd}(\pi(B),\pi(B'))
            = \Hom_{\Dd'}(\pi(B),\pi(B'))\,,
        \end{equation*}
        where the isomorphism inbetween follows from the fact that
        $\Hom_{\Dd}(\pi(B),C')=0$, for any $C' \in \Dd'^{\perp}$
        in the decomposition of $B'$.
        One can check that this defines a functor $\pi : \Dd \to \Dd'$
        which is right adjoint to the inclusion $\Dd' \subset \Dd$,
        by construction.
    \end{proof}
\end{lemma}

As it turns out, Serre functors and triangulated structures 
are always compatible. 
In the geometric situation considered later, 
this will be obvious, 
for the Serre functors there will by construction be exact.

\begin{thm}[Bondal, Kapranov]\label{BondalKapranov}
    Any Serre functor on a $k$-linear triangulated category is exact.
    \begin{proof}
        See \parencite[Proposition~1.46]{huybrechts}.
    \end{proof}
\end{thm}