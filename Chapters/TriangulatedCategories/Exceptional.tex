
\section{Exceptional sequences and orthogonal decompositions}

In the geometric context, the derived categories 
in question will usually be indecomposable. 
However, there are geometrically relevant situations 
where one can decompose the derived category in a weaker sense. 
This leads to the abstract notion of 
semi-orthogonal decompositions 
of a triangulated category, 
the topic of this section. 
Any full exceptional sequence yields such 
a semi-orthogonal decomposition, 
so we will discuss this notion first.

\begin{df}
    Let $\Dd$ be $k$-linear triangulated category. 
    An object $E \in \Dd$ is \textbf{exceptional} if
    \begin{equation*}
        \Hom_{\Dd}(E,E[n]) =
        \begin{cases}
            k\,, \quad &\text{if } n=0\,;\\
            0\,, \quad &\text{if } n \ne 0\,.
        \end{cases}
    \end{equation*}
    An \textbf{exceptional sequence} is a sequence
    $E_{1}, E_{2}, \dots, E_{m}$ of exceptional objects
    such that for all $i,j$ it holds
    \begin{equation*}
        \Hom_{\Dd}(E_{i},E_{j}[n]) =
        \begin{cases}
            k\,, \quad &\text{if } i=j, n=0\,;\\
            0\,, \quad &\text{if } i>j \text{ or if } i = j, n\ne 0\,.
        \end{cases}
    \end{equation*}
    An exceptional sequence is \textbf{full} if $\Dd$
    is generated by $\Set{E_{i}}$, 
    i.e. any full triangulated subcategory of $\Dd$
    containing all objects $E_{i}$ is equivalent to $\Dd$
    via the inclusion.
\end{df}

\begin{lemma}
    Let $\Dd$ be a $k$-linear triangulated category such that,
    for any $A,B \in \Dd$ the vector space 
    $\bigoplus_{i} \Hom_{\Dd}(A,B[i])$ is finite-dimensional.
    If $E \in \Dd$ is exceptional, 
    then the objects $\bigoplus_{i}E[i]^{\oplus j_{i}}$
    form an admissible subcategory $\langle E \rangle$
    of $\Dd$.
    \begin{proof}
        It is easy to check that $\langle E \rangle$ inherits
        the structure of a triangulated category:
        direct sums behave well in forming distinguished triangles;
        moreover, non-trivial endomorphisms of the exceptional
        object are automorphisms.
        In order to see that it is admissible, 
        we use the tensor product $\otimes_{\Dd}$:
        for every $n \in \Z$, consider the morphism
        induced by shifting $(-n)$ times
        \begin{equation*}
           [-n] : \Hom_{\Dd}(E,A[n]) \xrightarrow[]{\sim} \Hom_{\Dd}(E[-n],A)\,;
        \end{equation*}
        by adjunction, it gives an evaluation morphism
        \begin{equation*}
            \cat{ev}_{n} : \Hom_{\Dd}(E,A[n]) \otimes_{\Dd} E[-n] \longrightarrow A\,,
        \end{equation*}
        for each $n \in \Z$; take the direct sum of these morphisms
        and complete it to a distinguished triangle
        \begin{equation*}
            \bigoplus_{n}\big(\Hom_{\Dd}(E,A[n]) \otimes_{\Dd} E[-n] \big)
            \longrightarrow A \longrightarrow B\,.
        \end{equation*}
        For every $n \in \Z$, let $d_{n}:=\dim \Hom_{\Dd}(E,A[n])$ and notice
        it is finite by assumption; then it holds
        \begin{equation*}
            \Hom_{\Dd}(E,A[n]) \otimes_{\Dd} E[-n] 
            \simeq k^{d_{n}} \otimes_{\Dd} E[-n]
            \simeq \bigoplus_{m=1}^{d_{n}} E[-n]\,,
        \end{equation*}
        hence, if we apply $\Hom_{\Dd}(E[-i],-)$ to the above triangle,
        by exceptionality of $E$ we get
        \begin{align*}
            \Hom_{\Dd}\left(E[-i], \bigoplus_{n}\big(\Hom_{\Dd}(E,A[n]) \otimes_{\Dd} E[-n] \big) \right)
            &\simeq \bigoplus_{n} \left( \Hom_{\Dd}(E[-i], E[-n]) \right)^{d_{n}} \\ %\bigoplus_{m=1}^{d_{n}} \Hom_{\Dd}(E[-i],E[-n]) \\
            &\simeq \left( \Hom_{\Dd}(E[-i],E[-i]) \right)^{d_{i}} \\
            &\simeq k^{d_{i}} \simeq \Hom_{\Dd}(E,A[i]) \\
            &\simeq \Hom_{\Dd}(E[-i],A)\,,
        \end{align*}
        thus, $\Hom_{\Dd}(E,B[i]) \simeq \Hom_{\Dd}(E[-i],B) \simeq 0$.
        Since this holds for every $i \in \Z$, 
        then one has $B \in \langle E \rangle^{\perp}$,
        so we conclude by \hyperref[admissible-dec]{Lemma~\ref*{admissible-dec}}.
    \end{proof}
\end{lemma}

Now we generalize the concept of exceptional sequence:

\begin{df}
    A sequence of full admissible triangulated subcategories
    \begin{equation*}
        \Dd_{1}, \Dd_{2}, \dots, \Dd_{m} \subset \Dd
    \end{equation*}
    is \textbf{semi-orthogonal} if for all $i < j$ 
    it holds $\Dd_{i} \subset \Dd_{j}^{\perp}$.

    We say that $\Dd_{1}, \dots, \Dd_{m}$ is a 
    \textbf{semi-orthogonal decomposition} of $\Dd$
    if $\Dd$ is generated by the $\Dd_{i}$, i.e.
    the smallest full triangulated subcategory containing
    all the $\Dd_{i}$ is equivalent to $\Dd$ via the inclusion.
\end{df}

\begin{ex}
    Let $\Dd' \subset \Dd$ be an admissible full triangulated
    subcategory. Then
    \begin{equation*}
        \Dd_{1} := \Dd'^{\perp}\,, \quad \Dd_{2} := \Dd'\,,
    \end{equation*}
    defines a semi-orthogonal decomposition of $\Dd$.
\end{ex}

\begin{ex}
    Let $E_{1}, \dots, E_{m}$ be an exceptional sequence in $\Dd$.
    Then the admissible trangulated subcategories generated by these objects
    \begin{equation*}
        \Dd_{1} := \langle E_{1} \rangle\,, \quad \dots \quad
        \Dd_{m} := \langle E_{m} \rangle\,,
    \end{equation*}
    form a semi-orthogonal sequence.
    If the sequence is \emph{full}, then $\Dd_{1}, \dots, \Dd_{m}$
    is a semi-orthogonal decomposition of $\Dd$.
\end{ex}

\begin{lemma}
    Any semi-orthogonal sequence of full admissible triangulated subcategories
    $\Dd_{1}, \dots, \Dd_{m} \subset \Dd$ defines a semi-orthogonal
    decomposition for $\Dd$ if and only if any object $A \in \Dd$ such that
    $A \in \Dd_{i}$, for all $i=1,2, \dots, m$, 
    is then trivial, i.e. $A \simeq \cat{0}$.
    \begin{proof}
        Suppose $\Dd_{1}, \dots, \Dd_{m}$ is a semi-orthogonal 
        decomposition of $\Dd$. 
        If $A_{0} \in \bigcap \Dd_{i}^{\perp}$, 
        then each $\Dd_{i} \subset {}^{\perp}A_{0}$;
        by assumption ${}^{\perp}A_{0} = \Dd$ and in particular
        $A_{0} \in {}^{\perp}A_{0}$, thus $\Hom_{\Dd}(A_{0},A_{0}) = 0$.
        This means $A_{0} \simeq \cat{0}$, indeed $\cat{1}_{A_{0}} = 0$.

        Conversely, assume $\bigcap \Dd_{i}^{\perp} = \Set{\cat{0}}$.
        For simplicity, we consider the case $m=2$: given $A_{0} \in \Dd$,
        we want to show $A_{0}$ is in the triangulated subcategory
        generated by $\Dd_{1}$ and $\Dd_{2}$.
        As $\Dd_{2}$ is admissible, 
        by \hyperref[admissible-dec]{Lemma~\ref*{admissible-dec}}
        there is a distinguished triangle
        \begin{center}
            \begin{tikzcd}
                A \ar[r]
                & A_{0} \ar[r]
                & A' \ar[r]
                & A{[1]}\,,
            \end{tikzcd}
        \end{center}
        with $A \in \Dd_{2}$ and $A' \in \Dd_{2}^{\perp}$.
        Now, using that $\Dd_{1}$ is admissible, 
        we can decompose $A'$ as
        \begin{center}
            \begin{tikzcd}
                B \ar[r]
                & A' \ar[r]
                & B' \ar[r]
                & B{[1]}\,,
            \end{tikzcd}
        \end{center}
        with $B \in \Dd_{1}$ and $B' \in \Dd_{1}^{\perp}$.
        As the sequence is semi-orthogonal we have
        $B \in \Dd_{1} \subset \Dd_{2}^{\perp}$,
        and since $A' \in \Dd_{2}^{\perp}$, 
        we deduce that $B' \in \Dd_{2}^{\perp}$
        because it is a full triangulated subcategory.
        So $B' \in \Dd_{1}^{\perp} \cap \Dd_{2}^{\perp}$
        implies $B' \simeq \cat{0}$, from which we deduce
        that $B \simeq A'$. 
        Then $A_{0}$ has a semi-orthogonal decomposition
        with $A \in \Dd_{2}$ and $B \in \Dd_{1}$.
        (The general case follows by applying inductively this argument).
    \end{proof}
\end{lemma}

\missingfigure{Do exercises on Exceptional sequences.}