
\section{The homotopy category of complexes} %\texorpdfstring{$\cat{K}(\Aa)$}{K(A)}}

There are many formal similarities between
homological algebra and algebraic topology. 
The Dold-Kan correspondence, for example, 
provides a dictionary between 
positive complexes and simplicial theory. 
The algebraic notions of chain homotopy, mapping cones, 
and mapping cylinders have their historical 
origins in simplicial topology.

Let $\Aa$ be an abelian category and consider the
category $C^{\bullet}(\Aa)$ of cochain complexes in $\Aa$.
We now define the \textbf{homotopy category} $\cat{K}(\Aa)$ of $C^{\bullet}(\Aa)$
as follows: we take as objects the same of $C^{\bullet}(\Aa)$,
i.e. cochain complexes, 
and the morphisms of $\cat{K}(\Aa)$ to be
\emph{chain homotopy equivalence classes of maps} in $C^{\bullet}(\Aa)$:
for any two cochains $A^{\bullet}, B^{\bullet}$, it holds
\begin{equation*}
    \Hom_{\cat{K}(\Aa)}(A^{\bullet}, B^{\bullet}) :=
    \left. \Hom_{C^{\bullet}(\Aa)}(A^{\bullet}, B^{\bullet}) \middle/ \sim \right. \,,
\end{equation*}
where the relation $\sim$ is given by homotopy equivalence.
Notice that the quotient naturally inherits
the sum: given $f^{\bullet} \sim \tilde{f}^{\bullet}$ 
and $g^{\bullet} \sim \tilde{g}^{\bullet}$,
if $s$ is a homotopy between $f^{\bullet}$ and $\tilde{f}^{\bullet}$
and $t$ is a homotopy between $g^{\bullet}$ and $\tilde{g}^{\bullet}$,
then $s+t$ is a homotopy between $f^{\bullet} + g^{\bullet}$ 
and $\tilde{f}^{\bullet} + \tilde{g}^{\bullet}$, indeed
\begin{align*}
    (s+t)d_{A}^{\bullet} + d_{B}^{\bullet}(s+t)
    &= (sd_{A}^{\bullet} + d_{B}^{\bullet}s)
    + (td_{A}^{\bullet} + d_{B}^{\bullet}t) \\
    &= (f^{\bullet} - \tilde{f}^{\bullet}) + (g^{\bullet} - \tilde{g}^{\bullet}) \\
    &= (f^{\bullet} + g^{\bullet}) - (\tilde{f}^{\bullet} + \tilde{g}^{\bullet})\,, 
\end{align*}
thus $f^{\bullet} + g^{\bullet} \sim \tilde{f}^{\bullet} + \tilde{g}^{\bullet}$.
It follows that $\cat{K}(\Aa)$ is an additive category and the quotient functor
\begin{equation*}
    C^{\bullet}(\Aa) \longrightarrow \cat{K}(\Aa)
\end{equation*}
is an additive functor. 

Sometimes it is useful to consider 
categories of complexes having special properties: 
if $\Cc$ is any \emph{full} subcategory of $C^{\bullet}(\Aa)$, 
let $\cat{K}(\Cc)$ denote the \emph{full} subcategory of $\cat{K}(\Aa)$
whose objects are the cochain complexes in $\Cc$. 
Then $\cat{K}(\Aa)$ is a quotient category of $\Cc$;
moreover, if $\Cc$ contains the zero object and it is closed
under direct sum $\oplus$, then both 
$\Cc$ and $\cat{K}(\Cc)$ are additive categories
and the quotient $\Cc \to \cat{K}(\Cc)$ is an additive functors.

\begin{df}
    We write $\cat{K}^{\flat}(\Aa), \cat{K}^{-}(\Aa)$ and $\cat{K}^{+}(\Aa)$ 
    for the full subcategories of $\cat{K}(\Aa)$ 
    corresponding to the full subcategories 
    $C^{\flat}(\Aa), C^{-}(\Aa)$ and $C^{+}(\Aa)$ 
    of bounded, bounded above, 
    and bounded below cochain complexes respectively.
\end{df}

Having introduced the cast of categories, we turn to their properties.

\begin{lemma}
    For every $n \in \Z$, the $n$-th cohomology $H^{n}$
    is a well defined functor on the quotient category:
    \begin{equation*}
        H^{n} : \cat{K}(\Aa) \longrightarrow \Aa\,.
    \end{equation*}
    \begin{proof}
        We know that, if $f^{\bullet} \sim g^{\bullet}$,
        then $H^{n}(f^{\bullet}) = H^{n}(f^{\bullet})$,
        thus cohomology descends to the quotient.
    \end{proof}
\end{lemma}

The homotopy category $\cat{K}(\Aa)$ is characterized by the following
\begin{prop}[\textbf{Universal property}]
    Let $\Dd$ be a category and $F : C^{\bullet}(\Aa) \to \Dd$
    be any functor that sends chain homotopy equivalences
    to isomorphisms. Then $F$ factors uniquely through $\cat{K}(\Aa)$,
    that is
    \begin{center}
        \begin{tikzcd}
            C^{\bullet}(\Aa) \ar[r] \ar[d, "F"'] 
            & \cat{K}(\Aa) \ar[dl, dashed, "\exists !"]\\
            \Dd\,.
        \end{tikzcd}
    \end{center}
    \begin{proof}
        Let $A^{\bullet}$ be any cochain complex in $\Aa$.
        As we have seen in 
        \hyperref[cyl-equivalence]{Exercise~\ref*{cyl-equivalence}},
        the inclusion $\iota : A^{\bullet} \to \cat{cyl}(\cat{1}_{A^{\bullet}})$
        is a homotopy equivalence, hence by assumption
        $F\iota$ is an isomorphism with inverse $F\beta^{\bullet}$,
        where $\beta(a,a',a'') =  a'$. Moreover, notice that also the map
        \begin{equation*}
            j: A^{\bullet} \longrightarrow \cat{cyl}(\cat{1}_{A^{\bullet}})\,,
            \quad a \longmapsto (a,0,0)
        \end{equation*}
        is such that $\beta^{\bullet}j=\cat{1}_{B^{\bullet}}$,
        so in particular it holds
        \begin{equation*}
            Fj = F(\iota\,\beta^{\bullet})Fj
            = F\iota\,F(\beta^{\bullet}j) = F\iota\,.
        \end{equation*}

        Suppose now there is a cochain homotopy
        $s$ between $f^{\bullet}, g^{\bullet} : A^{\bullet} \to B^{\bullet}$;
        by \hyperref[cyl-criterion]{Exercise~\ref*{cyl-criterion}},
        is extends to a map $\gamma^{\bullet}:\cat{cyl}(\cat{1}_{A^{\bullet}}) 
        \to B^{\bullet}$ such that
        \begin{equation*}
            \gamma^{\bullet} j = f\,, \quad \gamma^{\bullet} \iota = g\,,
        \end{equation*}
        thus in $\Dd$ we have
        \begin{equation*}
            Ff = F\gamma^{\bullet}\,,Fj = F\gamma^{\bullet}\,,F\iota = Fg\,.
        \end{equation*}
        We conclude that $F$ factors through the quotient $\cat{K}(\Aa)$.
    \end{proof}
\end{prop}

We introduce now a useful operation we can perform on (co)chain complexes:
translating indices. This concept of ``shift'' functor will be 
taken to define triangulated categories later.

\begin{df}
    Let $A^{\bullet} \in C^{\bullet}(\Aa)$ be a cochain complex. 
    For any $p \in \Z$, we define the \textbf{$p^{\text{th}}$-translate}
    $A^{\bullet}[p]$ of $A^{\bullet}$ to be the complex whose $n$-th level
    objects and differentials are
    \begin{equation*}
        (A^{\bullet}[p])^{n} := A^{n+p}\,, \quad d_{[p]}^{n} = (-1)^{p}d^{n+p}\,.
    \end{equation*}
    Dually, if $A_{\bullet}$ is a chain complex,
    we define its \textbf{$p^{\text{th}}$-translate} as
    \begin{equation*}
        (A_{\bullet}[p])_{n} := A_{n+p}\,, \quad d^{[p]}_{n} = (-1)^{p}d_{n+p}\,.
    \end{equation*}
\end{df}

By shifting indices on (co)chain maps accordingly,
the $p^{\text{th}}$-translation defines a functor
\begin{equation*}
    [p] : C^{\bullet}(\Aa) \longrightarrow C^{\bullet}(\Aa)
\end{equation*}
which is in fact an equivalence of categories
(one checks that $[-p]$ is a quasi-inverse).
Note that translations shift (co)homology: indeed
\begin{equation}\label{shift-coh}
    H^{n}(A^{\bullet}[p]) = H^{n+p}(A^{\bullet})\,,
    \quad H_{n}(A_{\bullet}[p]) = H_{n-p}(A_{\bullet})\,.
\end{equation}

\begin{df}
    Let $f^{\bullet} : A^{\bullet} \to B^{\bullet}$ be a cochain map.
    The mapping cone of $f^{\bullet}$ fits into a
    short exact sequence
    \begin{center}
        \begin{tikzcd}
            \cat{0} \ar[r]
            & B^{\bullet} \ar[r, "g^{\bullet}"]
            & \cat{C}(f^{\bullet}) \ar[r, "\delta"]
            & A[-1]^{\bullet} \ar[r]
            & \cat{0}\,.
        \end{tikzcd}
    \end{center}
    The \textbf{strict triangle} on $f^{\bullet}$
    is the triple $(f^{\bullet}, g^{\bullet}, \delta)$
    of maps in $\cat{K}(\Aa)$, displayed as
    \begin{center}
        \begin{tikzcd}
            A^{\bullet} \ar[rr, "f^{\bullet}"]
            & & B^{\bullet} \ar[dl, "g^{\bullet}"] \\
            & \cat{C}(f^{\bullet}) \ar[ul, "\delta"] & \,.
        \end{tikzcd}
    \end{center}
    
    Given three cochain complexes $X^{\bullet}, Y^{\bullet}, Z^{\bullet}
    \in \Aa$ and maps $$u^{\bullet} : X^{\bullet} \to Y^{\bullet}, \quad
    v^{\bullet} : Y^{\bullet} \to Z^{\bullet}, \quad
    w^{\bullet} : Z^{\bullet} \to X[-1]^{\bullet}$$ 
    in $\cat{K}(\Aa)$,
    the triple $(u,v,w)$ is an \textbf{exact triangle} 
    on $(X^{\bullet},Y^{\bullet},Z^{\bullet})$
    if it is ``isomorphic'' to a strict triangle of $f^{\bullet}$, 
    for some $f^{\bullet} : A^{\bullet} \to B^{\bullet}$, 
    in the sense that there exists a diagram
    \begin{center}
        \begin{tikzcd}
            X^{\bullet} \ar[r, "u^{\bullet}"] \ar[d, "\alpha^{\bullet}"]
            & Y^{\bullet} \ar[d, "\beta^{\bullet}"] \ar[r, "v^{\bullet}"]
            & Z^{\bullet} \ar[d, "\gamma^{\bullet}"] \ar[r, "w^{\bullet}"]
            & X{[-1]}^{\bullet} \ar[d, "\alpha{[-1]}^{\bullet}"] \\
            A^{\bullet} \ar[r, "f^{\bullet}"]
            & B^{\bullet} \ar[r]
            & \cat{C}(f^{\bullet}) \ar[r]
            & A{[-1]}^{\bullet}
        \end{tikzcd}
    \end{center}
    which commutes in $\cat{K}(\Aa)$, where the vertical arrows
    are isomorphisms (in $\cat{K}(\Aa)$).
\end{df}

\begin{lemma}[\textbf{LECS}]
    Given an exact triangle $(u,v,w)$ on 
    $(A^{\bullet}, B^{\bullet}, C^{\bullet})$,
    the long cohomology sequence
    \begin{center}
        \begin{tikzcd}
            \dots \ar[r]
            & H^{i}(A^{\bullet}) \ar[r, "u^*"]
            & H^{i}(B^{\bullet}) \ar[r, "v^*"]
            & H^{i}(C^{\bullet}) \ar[r, "w^*"]
            & H^{i+1}(A^{\bullet}) \ar[r]
            & \dots
        \end{tikzcd}
    \end{center}
    is exact, where we have identified
    $H^{i}(A^{\bullet}[-1]) = H^{i+1}(A^{\bullet})$
    as in \eqref{shift-coh}.
    \begin{proof}
        For a strict triangle, the result is
        clear because $H^{i}$ is additive 
        and the sequence is split.
        For any other triangle, exactness
        follows because $H^i$ is a functor on $\cat{K}(\Aa)$,
        hence it preserves isomorphisms up to homotopy.
    \end{proof}
\end{lemma}

The following two technical results will be crucial
for the definition of morphisms in the derived category
of $\Aa$.

%\begin{prop}\label{double-cone}
%    Let $f^{\bullet} : A^{\bullet} \to B^{\bullet}$
%    be a cochain map and write 
%    $\iota : B^{\bullet} \to \cat{C}(f)$ for the canonical inclusion
%    and $\pi : \cat{C}(f) \to A^{\bullet}[1]$ for the projection.
%    There exists a homotopy equivalence
%    $g^{\bullet} :A^{\bullet}[1] \to \cat{C}(\iota)$ such that the following
%    diagram
%    \begin{equation*}
%        \begin{tikzcd}
%            B^{\bullet} \ar[r, "\iota"] \ar[d, equals]
%            & \cat{C}(f) \ar[r, "\pi"] \ar[d, equals] 
%            & A^{\bullet}{[1]} \ar[d, "g^{\bullet}", dashed] \ar[r, "-f^{\bullet}"]
%            & B^{\bullet}{[1]} \ar[d, equals] \\
%            B^{\bullet} \ar[r, "\iota"]
%            & \cat{C}(f) \ar[r, "\iota_{\iota}"] 
%            & \cat{C}(\iota) \ar[r, "\pi_{\iota}"]
%            & B^{\bullet}{[1]} 
%        \end{tikzcd}
%    \end{equation*}
%    commutes up to homotopy.
%    \begin{proof}
%        Set $g^{\bullet} = -f^{\bullet}[1] \oplus \cat{1}_{A^{\bullet}[1]} \oplus 0$
%        and notice that it defines a map of complexes: for each $n \in \Z$,
%        it holds
%        \begin{align*}
%            g^{n+1}d_{A}[1]^{n}
%            &= \begin{pmatrix}
%                (-f^{n+2}) (-d_{A}^{n+1}) \\ -d_{A}^{n+1} \\ 0
%            \end{pmatrix}
%            = \begin{pmatrix}
%                f^{n+2} d_{A}^{n+1} \\ -d_{A}^{n+1} \\ 0
%            \end{pmatrix}
%            = \begin{pmatrix}
%                d_{B}^{n+2} f^{n+1}\\ -d_{A}^{n+1} \\ 0
%            \end{pmatrix} \\
%            &= \begin{pmatrix}
%                -d_{B}^{n+2} & & \\ & -d_{A}^{n+1} & \\ \cat{1}_{B^{n+2}} & f^{n+1} & d_{B}^{n+1}
%            \end{pmatrix}
%            \begin{pmatrix}
%                -f^{n+1} \\ \cat{1}_{A^{n+1}} \\ 0
%            \end{pmatrix}
%            = d_{\iota}^{n+1} \, g^{n}\,.
%        \end{align*}
%        One can check that the canonical projection 
%        morphism $q:\cat{C}(\iota) \to A^{\bullet}{[1]}$ defines
%        a homotopy inverse to $g^{\bullet}$.
%        
%        Notice that the right-most square commutes in $C^{\bullet}(\Aa)$,
%        i.e. $\pi_{\iota}g^{\bullet} = -f^{\bullet}$, but in general
%        $g^{\bullet}\pi \ne \iota_{\iota}$; nevertheless,
%        that box commutes up to homotopy: indeed, 
%        from the identity $\pi=q\iota_{\iota}$,
%        we deduce that
%        \begin{equation*}
%            g^{\bullet}\pi = g^{\bullet}q\iota_{\iota} \simeq \iota_{\iota}\,,
%        \end{equation*}
%        which translates to an equality in $\cat{K}(\Aa)$.
%    \end{proof}
%\end{prop}

\begin{prop}[Rooves composition]\label{roof-comp}
    Given two morphisms $f^{\bullet}:A^{\bullet} \to B^{\bullet}$
    and $g^{\bullet}:C^{\bullet} \to B^{\bullet}$,
    there exists a commutative diagram in $\cat{K}(\Aa)$
    \begin{equation*}
        \begin{tikzcd}
            C_{0}^{\bullet} \ar[r, "\Tilde{f}^{\bullet}"] \ar[d, "\Tilde{g}^{\bullet}"']
            & C^{\bullet} \ar[d, "g^{\bullet}"] \\
            A^{\bullet} \ar[r, "f^{\bullet}"]
            & B^{\bullet}\,,
        \end{tikzcd}
    \end{equation*}
    such that:
    \begin{itemize}
        \item if $f^{\bullet}$ is a qis, then $\Tilde{f}^{\bullet}$ is a qis too;
        \item if $g^{\bullet}$ is a qis, then $\Tilde{g}^{\bullet}$ is a qis too.
    \end{itemize}
    \begin{proof}
        Consider the two following strict triangles:
        \begin{equation*}
            \begin{tikzcd}
                A^{\bullet} \ar[rr, "f^{\bullet}"]
                & & B^{\bullet} \ar[dl, "\iota"] 
                &
                & \cat{C}(\iota \circ g^{\bullet}) \ar[rr, "\varpi"]
                & & C^{\bullet} \ar[dl, "\iota \circ g^{\bullet}"] \\
                & \cat{C}(f^{\bullet}) \ar[ul, "\pi"] &
                & \,, & &  \cat{C}(f^{\bullet}) \ar[ul, "j"] & \,,
            \end{tikzcd}
        \end{equation*}
        and define $\gamma^{\bullet} : \cat{C}(\iota \circ g^{\bullet}) \to A^{\bullet}[1]$
        on the $n$-th level as
        \begin{equation*}
            \gamma^{n} := \begin{pmatrix}
                0 & \cat{1}_{A^{n+1}} & 0
            \end{pmatrix}\,;
        \end{equation*}
        one can check that $\gamma^{\bullet}$ actually defines a cochain map
        and, in fact, it makes the following diagram
        \begin{equation*}
            \begin{tikzcd}
                \cat{C}(\iota \circ g^{\bullet}){[-1]} 
                \ar[d, "\gamma^{\bullet}{[-1]}"', dashed] \ar[r, "\varpi"]
                & C^{\bullet} \ar[r] \ar[d, "g^{\bullet}"]
                & \cat{C}(f^{\bullet}) \ar[d, equals] \ar[r, "j"]
                & \cat{C}(\iota \circ g^{\bullet}) \ar[d, "\gamma^{\bullet}"', dashed] \\
                A^{\bullet} \ar[r, "f^{\bullet}"]
                & B^{\bullet} \ar[r, "\iota"]
                & \cat{C}(f^{\bullet}) \ar[r, "\pi"]
                & A^{\bullet}{[1]}\,
            \end{tikzcd}
        \end{equation*}
        commute in $\cat{K}(\Aa)$: in fact, 
        the rightmost box commutes in $C^{\bullet}(\Aa)$ because
        \begin{equation*}
            \gamma^{\bullet} \circ j
            = \begin{pmatrix}
                0 & \cat{1}_{A^{\bullet}{[1]}} & 0
            \end{pmatrix}
            \begin{pmatrix}
                0 & 0 \\ \cat{1}_{A^{\bullet}{[1]}} & 0 \\ 0 & \cat{1}_{B^{\bullet}}
            \end{pmatrix}
            = \begin{pmatrix}
                \cat{1}_{A^{\bullet}{[1]}} & 0
            \end{pmatrix}
            = \pi\,,
        \end{equation*}
        while the map $h:\cat{C}(\iota \circ g^{\bullet}){[-1]} \to B^{\bullet}[-1]$
        given by 
        $h = \begin{pmatrix}
            0 & 0 & \cat{1}_{B^{\bullet}{[-1]}}
        \end{pmatrix}$ 
        is a homotopy between $g^{\bullet} \circ \varpi$
        and $f^{\bullet} \circ \gamma^{\bullet}[-1]$ because
        \begin{align*}
            hd_{-(\iota \circ g)} + d_{B^{\bullet}[-1]}h 
            &= h
            \begin{pmatrix}
                -d_{C} & 0 \\ -(\iota \circ g^{\bullet}) & d_{f}
            \end{pmatrix}
            - d_{B} \begin{pmatrix}
                0 & 0 & \cat{1}_{B^{\bullet}[-1]}
            \end{pmatrix}   \\
            &= \begin{pmatrix}
                0 & 0 & \cat{1}_{B^{\bullet}[-1]}
            \end{pmatrix}
            \begin{pmatrix}
                -d_{C} & 0 & 0 \\ 0 & -d_{A} & 0 \\ -g^{\bullet} & f^{\bullet} & d_{B}
            \end{pmatrix}
            +  \begin{pmatrix}
                0 & 0 & -d_{B}
            \end{pmatrix}   \\
            &= \begin{pmatrix}
                -g^{\bullet} & f^{\bullet} & 0
            \end{pmatrix} 
            = \begin{pmatrix}
                0 & f^{\bullet} & 0
            \end{pmatrix} 
            - \begin{pmatrix}
                g^{\bullet} & 0 & 0
            \end{pmatrix}
            = f^{\bullet} \circ \gamma^{\bullet}[-1] - g^{\bullet} \circ \varpi\,.
        \end{align*}
        This shows that the triple $(\gamma^{\bullet}[-1],g^{\bullet}, \cat{1})$
        defines a morphism of triangles in $\cat{K}(\Aa)$.
        Thus, by setting 
        $$C_{0}^{\bullet} := \cat{C}(\iota \circ g^{\bullet})[-1]\,, \quad
        \Tilde{f}^{\bullet} := \varpi \,, \quad  
        \Tilde{g}^{\bullet} = \gamma^{\bullet}[-1]\,,$$
        we get the desired commutative square in $\cat{K}(\Aa)$.
        
        Moreover, by applying the cohomology functor
        we obtain the following commutative diagram in $\Aa$:
            \begin{equation*}
                \begin{tikzcd}
                    \dots \ar[r]
                    & H^{n-1}(\cat{C}(f^{\bullet})) \ar[r, "\Tilde{f}^*"] \ar[d, equals]
                    & H^{n}(C^{\bullet}_{0}) \ar[r] \ar[d, "\Tilde{g}^*"]
                    & H^{n}(C^{\bullet}) \ar[r] \ar[d, "g^*"]
                    & H^{n}(\cat{C}(f^{\bullet})) \ar[r] \ar[d, equals]
                    & \dots \\
                    \dots \ar[r]
                    & H^{n-1}(\cat{C}(f^{\bullet})) \ar[r]
                    & H^{n}(A^{\bullet}) \ar[r, "f^*"]
                    & H^{n}(B^{\bullet}) \ar[r]
                    & H^{n}(\cat{C}(f^{\bullet})) \ar[r]
                    & \dots
                \end{tikzcd}
            \end{equation*}
        where the rows are long exact sequences.
        Thus, we notice that if $f^{\bullet}$ is a qis,
        then $\cat{C}(f^{\bullet})$
        has trivial cohomology, which implies that 
        $\Tilde{f}^* : H^{*}(C^{\bullet}_{0}) \simeq H^{*}(C^{\bullet})$,
        so $\Tilde{f}^{\bullet}$ is a qis. On the other hand, 
        if $g^{\bullet}$ is a qis, then by the
        \hyperref[5-lemma]{Five Lemma} we deduce that
        also $\Tilde{g}^{\bullet}$ is a qis.
    \end{proof}
\end{prop}

\begin{exercise}
    One might be tempted to define $C_{0}^{\bullet}$
    in the above \hyperref[roof-comp]{Proposition~\ref*{roof-comp}}
    as the fibered product
    \begin{equation*}
        C_{0}^{\bullet} := A^{\bullet} \times_{B^{\bullet}} C^{\bullet}\,.
    \end{equation*}
    Show that, in general, this choice does not work
    and it does not guarantee the nice properties of the
    $C_{0}^{\bullet}$ built above.
    \begin{proof}[An example]
        Let $B^{\bullet}$ be the complex
        \begin{equation*}
            \begin{tikzcd}
                \cat{0} \ar[r]
                & B^{0} \ar[r,"d", two heads] 
                & B^{1} \ar[r]
                & \cat{0}\,,
            \end{tikzcd}
        \end{equation*}
        where $d$ is an epimorphism between non-trivial objets,
        with non trivial kernel $A:=\ker d \ne \cat{0}$.
        Denote by $A^{\bullet}$ the complex with $A$ 
        concentrated in degree $0$: then, there is
        a natural inclusion 
        $\iota:A^{\bullet} \hookrightarrow B^{\bullet}$.
        Set $C^{\bullet} := B^{\bullet}[1]$ and consider the
        cochain map $\Delta:C^{\bullet} \to B^{\bullet}$
        given by the diagram
        \begin{equation*}
            \begin{tikzcd}
                \cat{0} \ar[r]
                & \cat{0} \ar[r] \ar[d]
                & B^{0} \ar[r, "d", two heads] \ar[d, "d", two heads]
                & B^{1} \ar[r] \ar[d]
                & \cat{0} \\
                \cat{0} \ar[r] 
                & B^{0} \ar[r, "d", two heads]
                & B^{1} \ar[r]
                & \cat{0} \ar[r]
                & \cat{0}\,.
            \end{tikzcd}
        \end{equation*}
        One can check that the fibered product $A^{\bullet} \times_{B^{\bullet}} C^{\bullet}$
        is given by the complex
        \begin{equation*}
            \begin{tikzcd}
                \cat{0} \ar[r]
                & A \ar[r, hook]
                & B^{0} \ar[r, "d", two heads]
                & B^{1} \ar[r]
                & \cat{0}\,,
            \end{tikzcd}
        \end{equation*}
        with projection morphisms on $A^{\bullet}$ and $C^{\bullet}$
        being the obvious ones.

        Now consider the $0$-th cohomology of these complexes:
        \begin{equation*}
            H^{0}(B^{\bullet}) 
            = \mathrm{coker}\left( \cat{0} \to A \right)
            \simeq A = H^{0}(A^{\bullet})\,,
        \end{equation*}
        and $H^{0}(C^{\bullet}) \simeq H^{-1}(B^{\bullet}) = \cat{0}$.
        Since the $0$-th cohomology object
        of the fibered product is again $A$,
        by applying the functor $H^{0}$ to the usual
        cartesian square one gets
        \begin{equation*}
            \begin{tikzcd}
                A \ar[r, equals] \ar[d] & A \ar[d, "\iota^*", equals] \\
                \cat{0} \ar[r] & A\,,
            \end{tikzcd}
        \end{equation*}
        which does not commute in $\Aa$. This means that
        \begin{equation*}
            \begin{tikzcd}
                A^{\bullet} \times_{B^{\bullet}} C^{\bullet}
                \ar[r] \ar[d] & A^{\bullet} \ar[d,"\iota"]\\
                C^{\bullet} \ar[r] & B^{\bullet}
            \end{tikzcd}
        \end{equation*}
        does not commute in $\cat{K}(\Aa)$.
    \end{proof}
\end{exercise}