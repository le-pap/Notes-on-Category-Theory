
\section{Equivalences of triangulated categories}

In this section we discuss criteria that allow us to decide 
whether a given exact functor is fully faithful or even an equivalence. 

\begin{df}
    Two triangulated categories $\Dd$ and $\Dd'$ are \textbf{equivalent}
    if there exists an exact equivalence $T : \Dd \to \Dd'$,
    i.e. an equivalence which is an exact functor, whose inverse is exact.
    The set $\Aut(\Dd)$ of isomorphism classes of equivalences $F:\Dd \to \Dd$
    is called \textbf{group of autoequivalences} of $\Dd$.
\end{df}

In many geometric situations, we will encounter the following notion:

\begin{df}
    Let $\Dd$ be a triangulated category.
    A \textbf{spanning class} for $\Dd$ is a collection $\Omega$ 
    of objects in $\Dd$ such that, for any $B \in \Dd$, 
    the following two conditions hold:
    \begin{rmnumerate}
        \item if for every $n \in \Z$ and $A \in \Omega$ we have
        $\Hom_{\Dd}(A,B[n]) = 0$, then $B \simeq \cat{0}$;
        \item if for every $n \in \Z$ and $A \in \Omega$ we have
        $\Hom_{\Dd}(B[n],A) = 0$, then $B \simeq \cat{0}$.
    \end{rmnumerate}
\end{df}

\begin{rmk}
    If $\Dd$ is also $k$-linear, endowed with a Serre functor $S$, 
    then the two conditions in the definition are equivalent:
    assume (i) to be true. 
    If for every $n \in \Z$ and $A \in \Omega$ we have
    \begin{equation*}
        0 = \Hom_{\Dd}(B[n],A) \simeq \Hom_{\Dd}(A,S(B[n]))^*\,,
    \end{equation*}
    since by the \hyperref[BondalKapranov]{Bondal-Kapranov Theorem}
    we have $S(B[n]) \simeq S(B)[n]$, from (i) we deduce that
    $S(B) \simeq \cat{0}$, and hence $B \simeq \cat{0}$ because
    $S$ is additive. Similarly one proves (ii) $\implies$ (i).
\end{rmk}

Spanning classes give a sufficient collections of objects
on which we can check whether an exact functor is fully faithful or not.

\begin{prop}\label{span-ff}
    Let $F : \Dd \to \Dd'$ be an exact functor between triangulated categories
    with both left and right adjoints: $G \dashv F \dashv H$.
    Suppose $\Omega$ is a spanning class such that, for every $A,B \in \Omega$, 
    the natural maps
    \begin{equation*}
        F : \Hom_{\Dd}(A,B[n]) \longrightarrow \Hom_{\Dd'}(F(A),F(B)[n])
    \end{equation*}
    are bijections for every $n \in \Z$. Then $F$ is fully faithful.
    \begin{proof}
        For any pair of objects $A, B \in \Dd$,
        we have the commutative diagram
        \begin{equation}\label{adj-square-prop}
            \begin{tikzcd}
                \Hom_{\Dd}(A,B) 
                \ar[r, "\eta_{B} \circ -"] 
                \ar[d, "- \circ \epsilon_{A}"'] 
                \ar[dr, "F"]
                & \Hom_{\Dd}(A,HF(B)) \ar[d, equals] \\
                \Hom_{\Dd}(GF(A),B) \ar[r, equals]
                & \Hom_{\Dd'}(F(A),F(B))\,.
            \end{tikzcd}
        \end{equation}
        We show that 
        the counit $\epsilon_{A}$ is an isomorphism,
        for $A \in \Omega$:
        complete it to a distinguished triangle
        \begin{center}
            \begin{tikzcd}
                GF(A) \ar[r, "\epsilon_{A}"]
                & A \ar[r]
                & A' \ar[r]
                & GF(A){[1]}\,,
            \end{tikzcd}
        \end{center}
        where we have used that $GF(A[1]) \simeq GF(A)[1]$
        by exactness of both $F$ and $G$ 
        (see \hyperref[adj-exact]{Proposition~\ref*{adj-exact}}).
        Apply $\Hom_{\Dd}(-,B[n])$ to it and get %for an arbitrary $B \in \Dd$
        \begin{center}
            \begin{tikzcd}
                \Hom_{\Dd}(A',B{[n]}) \ar[r]
                & \Hom_{\Dd}(A,B{[n]}) \ar[r, "-\circ \epsilon"] \ar[dr, "F"']
                & \Hom_{\Dd}(GF(A),B{[n]}) \ar[d, equals] \\
                & & \Hom_{\Dd'}(F(A),F(B){[n]})\,.
            \end{tikzcd}
        \end{center}
        If we take $B \in \Omega$, then $F$ is bijective, 
        so $-\circ \epsilon$ is an isomorphism, from which
        we deduce that $\Hom_{\Dd}(A',B[n])=0$;
        since this holds for every $B \in \Omega$ and $n \in \Z$,
        then $A' \simeq \cat{0}$ because $\Omega$ spans $\Dd$,
        so we conclude that $\epsilon_{A}:GF(A) \simeq A$.

        It follows that $- \circ \epsilon_{A}$ is an isomorphism,
        for $A \in \Omega$ and any $B \in \Dd$, and hence all the maps 
        in the diagram~\eqref{adj-square-prop} are isomorphisms.
        Now complete $\eta_{B}$ to a distinguished triangle
        \begin{center}
            \begin{tikzcd}
                B \ar[r, "\eta_{B}"]
                & HF(B) \ar[r]
                & B' \ar[r]
                & B{[1]}\,,
            \end{tikzcd}
        \end{center}
        and apply $\Hom_{\Dd}(A,-) \circ [n]$, 
        for all $n \in \Z$ and $A \in \Omega$
        to see that $\Hom_{\Dd}(A,B') = 0$, and hence $B' \simeq \cat{0}$.
        This means that $\eta_{B}:B \simeq HF(B)$, for any $B \in \Dd$. 
        Thus, the arrows in \eqref{adj-square-prop} are isomorphisms
        for all $A,B \in \Dd$, in particular
        \begin{equation*}
            F:\Hom_{\Dd}(A,B) \xrightarrow[]{\sim} \Hom_{\Dd'}(F(A),F(B))\,.
        \end{equation*}
    \end{proof}
\end{prop}

Suppose we already know that the functor is fully faithful. 
What do we need to know in order to be able to decide whether 
it is in fact an equivalence? 
The following lemma provides a first criterion, 
whose assumption however is difficult to check. 

\begin{lemma}\label{lemma-equiv}
    Let $F:\Dd \to \Dd'$ be a fully faithful exact functor
    between triangulated categories and suppose that $F$
    has a right adjoint $F \dashv H$.
    Then $F$ is an equivalence if and only if,
    for every $C \in \Dd'$, $H(C) \simeq \cat{0}$
    implies $C \simeq \cat{0}$.

    The same holds true if $F$ has a left adjoint $G \dashv F$
    with the above property.
    \begin{proof}
        By \hyperref[ff-adj]{Corollary~\ref*{ff-adj}},
        we know that $\eta_{A}:A \xrightarrow[]{\sim} HF(A)$ is an isomorphism,
        for any $A \in \Dd$. We prove that also $\epsilon_{B}:FH(A) \to B$ is
        an isomorphism, and hence $H$ is a quasi-inverse of $F$.

        Given any $B \in \Dd'$, complete $\epsilon_{B}$
        to a distinguished triangle
        \begin{center}
            \begin{tikzcd}
                FH(B) \ar[r, "\epsilon_{B}"]
                & B \ar[r]
                & B' \ar[r]
                & FH(B){[1]}\,;
            \end{tikzcd}
        \end{center}
        since $H$ is exact by \hyperref[adj-exact]{Proposition~\ref*{adj-exact}},
        it gives a distinguished triangle
        \begin{center}
            \begin{tikzcd}
                HFH(B) \ar[r, "H\epsilon_{B}"]
                & H(B) \ar[r]
                & H(B') \ar[r]
                & HFH(B){[1]}\,.
            \end{tikzcd}
        \end{center}
        From \hyperref[unit-counit-identity]{Exercise~\ref*{unit-counit-identity}}
        we know that $H\epsilon_{B}$ is an isomorphism, hence $H(B') \simeq \cat{0}$.
        By assumption $B' \simeq \cat{0}$ and we get the thesis.
    \end{proof}
\end{lemma}

\begin{df}
    A triangulated category $\Dd$ is \textbf{decomposed into
    triangulated subcategories} $\Dd_{1}, \Dd_{2} \subset \Dd$
    if the following three conditions are satisfied:
    \begin{rmnumerate}
        \item both $\Dd_{1}$ and $\Dd_{2}$ contain objects non-isomorphic
        to $\cat{0}$;

        \item for all $A \in \Dd$, there exists a distinguished triangle
        \begin{equation*}
            A_{1} \longrightarrow A \longrightarrow A_{2} \longrightarrow A_{1}[1]\,,
        \end{equation*}
        with $A_{1} \in \Dd_{1}$ and $A_{2} \in \Dd_{2}$;

        \item the two subcategories are ``disjoint'' in the sense that,
        for every $X_{1} \in \Dd_{1}$ and $X_{2} \in \Dd_{2}$, it holds
        \begin{equation*}
            \Hom_{\Dd}(X_{1}, X_{2}) = \Hom_{\Dd}(X_{2}, X_{1}) = 0\,.
        \end{equation*}
    \end{rmnumerate}
    A triangulated category which cannot be decomposed is called
    \textbf{indecomposable}.
\end{df}

\begin{rmk}
    Notice that, in the presence of (iii), the property (ii) states that
    $A$ is a direct sum $A \simeq A_{1} \oplus A_{2}$ 
    (see the \hyperref[split-lemma]{Split Lemma}).
    Thus, condition (ii) is symmetric, despite the chosen order in the statement.
\end{rmk}

\begin{prop}\label{trivial-adj-equiv}
    Let $F:\Dd \to \Dd'$ be a fully faithful exact functor
    between triangulated categories. Suppose that $\Dd$ contains no
    objects isomorphic to $\cat{0}$ and that $\Dd'$ is indecomposable.
    Then $F$ is an equivalence if and only if it admits both right 
    and left adjoints $G \dashv F \dashv H$ such that, for any $B \in \Dd$,
    $H(B) \simeq \cat{0}$ implies $G(B) \simeq \cat{0}$.
    \begin{proof}
        The ``only if part is clear'' because $G \simeq H \simeq F^{-1}$.
        So we need to prove the ``if'' implication: our goal is to show that
        $H$ is a quasi-inverse of $F$.

        First, we introduce two full triangulated subcategories
        $\Dd'_{1}, \Dd'_{2} \subset \Dd'$ defined as follows:
        let $\Dd'_{1}$ the full subcategory of objects $B \in \Dd'$
        such that $B \simeq F(A)$, for some $A \in \Dd$; 
        let $\Dd'_{2}$ be the full subcategory consisting of
        all objects $C \in \Dd'$ such that $H(C) \simeq \cat{0}$.
        One can easily check that both $\Dd'_{1}$ and $\Dd'_{2}$
        are triangulated subcategories of $\Dd'$.

        The proof of \hyperref[lemma-equiv]{Lemma~\ref*{lemma-equiv}}
        shows that, for every $B \in \Dd'_{1}$, the counit $FH(B) \simeq B$
        is an isomorphism and also that every $B \in \Dd'$
        sits in a distinguished triangle of the form
        \begin{equation*}
            B_{1} \longrightarrow
            B \longrightarrow
            B_{2} \longrightarrow
            B_{1}{[1]}\,,
        \end{equation*}
        with $B_{1} \in \Dd'_{1}$ and $B_{2} \in \Dd'_{2}$.
        Since by assumption $H(B_{2}) \simeq \cat{0}$
        implies $G(B_{2}) \simeq \cat{0}$,
        then for all $B_{1} \in \Dd'_{1}$ and $B_{2} \in \Dd'_{2}$
        it holds
        \begin{align*}
            \Hom_{\Dd'}(B_{1}, B_{2})
            \simeq \Hom_{\Dd'}(FH(B_{1}),B_{2}) 
            \simeq \Hom_{\Dd}(H(B_{1}),H(B_{2})) 
            \simeq \Hom_{\Dd}(H(B_{1}),\cat{0}) = 0\,, \\
            \Hom_{\Dd'}(B_{2}, B_{1})
            \simeq \Hom_{\Dd'}(B_{2}, FH(B_{1})) 
            \simeq \Hom_{\Dd}(G(B_{2}),H(B_{1})) 
            \simeq \Hom_{\Dd}(\cat{0}, H(B_{1}))= 0\,,
        \end{align*}
        so $\Dd'_{1}$ and $\Dd'_{2}$ decompose $\Dd'$.
        As $\Dd'$ is indecomposable, either $\Dd'_{1}$
        or $\Dd'_{2}$ is trivial, i.e. contains only objects
        isomorphic to $\cat{0}$.

        Suppose $\Dd'_{1}$ is trivial. Then for every $A \in \Dd$,
        the image $F(A)$ is trivial, and hence
        $A \simeq HF(A) \simeq \cat{0}$ for $F$ is fully faithful:
        this contradicts the non-triviality of $\Dd$.
        Thus $\Dd'_{2}$ must be trivial, which implies
        that $\Dd'_{1} \subset \Dd'$ is an equivalence, i.e.
        for every $B \in \Dd'$, it holds $FH(B) \simeq B$,
        and $H$ is a quasi-inverse of $F$.
    \end{proof}
\end{prop}

The following proposition gives a criterion for equivalences
of triangulated endowed with Serre functors.

\begin{cor}
    Let $\Dd$ and $\Dd'$ be triangulated categories,
    endowed with Serre functors $S_{\Dd}$, resp. $S_{\Dd'}$. 
    Let $F:\Dd \to \Dd'$ be an exact functor
    which admits both right 
    and left adjoints $G \dashv F \dashv H$. 
    Assume there is $\Omega$
    a spanning class of $\Dd$ satisfying the following three properties:
    \begin{rmnumerate}
        \item for all $A,B \in \Omega$, the natural morphisms
        \begin{equation*}
            \Hom_{\Dd}(A,B[n]) \longrightarrow \Hom_{\Dd'}(F(A),F(B)[i])
        \end{equation*}
        are bijective for all $n \in \Z$;
        
        \item Serre functors commute with $F$ on the spanning class,
        that is
        \begin{equation*}
            F \circ S_{\Dd}(A) \simeq S_{\Dd'} \circ F(A)\,, 
            \quad A \in \Omega \,;
        \end{equation*}

        \item the category $\Dd'$ is indecomposable and $\Dd$
        is non-trivial.
    \end{rmnumerate}
    Then $F$ is an equivalence.
    \begin{proof}
        Condition (i) ensures that $F$ is fully faithful by
        \hyperref[span-ff]{Proposition~\ref*{span-ff}}.
        Now we want to prove $F$ is an equivalence by verifying
        the conditions in \hyperref[trivial-adj-equiv]{Proposition~\ref*{trivial-adj-equiv}}.

        Suppose $H(B) \simeq \cat{0}$, for $B \in \Dd'$.
        For every $A \in \Omega$, using adjunction and condition (ii),
        one finds
        \begin{align*}
            0 = \Hom_{\Dd}(A,H(B))
            &\simeq \Hom_{\Dd'}(F(A),B)
            \simeq \Hom_{\Dd'}(B,S_{\Dd'}F(A)))^* \\
            & \simeq \Hom_{\Dd'}(B,FS_{\Dd}(A)))^*
            \simeq \Hom_{\Dd}(G(B),S_{\Dd}(A)))^* \\
            & \simeq \Hom_{\Dd}(A,G(B))\,,
        \end{align*}
        thus $G(B) \simeq \cat{0}$ because $\Omega$ spans $\Dd$;
        more precisely, this argument shows that $G \simeq H$
        by the \hyperref[yoneda]{Yoneda's Lemma}.
    \end{proof}
\end{cor}

Recall that in a $k$-linear category endowed with a
Serre functor, the existence of an adjoint functor 
implies the existence of the other one; hence,
the previous Corollary may be stated assuming the existence
of $H$ only.