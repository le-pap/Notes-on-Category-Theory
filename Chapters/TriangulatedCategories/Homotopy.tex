
\section{Chain homotopy}

The ideas in this section and the next are motivated 
by homotopy theory in topology. 

Let $\Aa$ be an abelian category and consider 
two cochain complexes $C^{\bullet}, D^{\bullet}$ in $\Aa$. 
Consider a degree $-1$ map $s : C^{\bullet} \to D^{\bullet}$,
that is a collection of morphisms
\begin{equation*}
    s^{n} : C^{n} \longrightarrow D^{n-1}\,,
    \quad n \in \Z\,.
\end{equation*}
For every $n \in \Z$, set $f^{n} := d_{D}^{n-1}s^{n} + s^{n+1}d_{C}^{n}$
to get a morphism $f^{n}:C^{n} \to D^{n}$:
\begin{center}
    \begin{tikzcd}
        C^{n-1} \ar[r]
        & C^{n} \ar[r, "d"] \ar[d, "f^{n}"] \ar[dl, "s"']
        & C^{n+1} \ar[dl,"s"] \\
        D^{n-1} \ar[r, "d"]
        & D^{n} \ar[r]
        & D^{n+1} \,.
    \end{tikzcd}
\end{center}

Then the collection of $f^{n}$ defines
a cochain map $f^{\bullet} : C^{\bullet} \to D^{\bullet}$;
indeed, dropping the superscripts for clarity, we compute
\begin{align*}
    d f = d(ds + sd) = dsd = (ds + sd) d = fd\,.
\end{align*}

\begin{df}
    A chain map $f^{\bullet}:C^{\bullet} \to D^{\bullet}$
    is \textbf{null homotopic} if it is of the form
    $f^{\bullet} = ds + sd$, for some map $s$ of degree $-1$.
    We will call $s$ a \textbf{cochain contraction} of $f^{\bullet}$.
\end{df}

\begin{lemma}
    If $f^{\bullet}:C^{\bullet} \to D^{\bullet}$
    is \textbf{null homotopic}, then every induced map
    in cohomology 
    $f^{*}:H^{n}(C^{\bullet}) \to H^{n}(D^{\bullet})$
    is zero.
    \begin{proof}
        If we focus on the $n$-th level, 
        we have a commutative diagram
        \begin{equation*}
            \begin{tikzcd}
                & \ker d^{n}_{C} \ar[d, hook] & \\
                C^{n-1} \ar[r, "d^{n-1}_{C}"] \ar[d]
                & C^{n} \ar[r, "d^{n}_{C}"] \ar[d, "f^{n}"'] \ar[dl, "s"']
                & C^{n+1} \ar[d] \ar[dl, "s"'] \\
                D^{n-1} \ar[r, "d^{n-1}_{D}"'] \ar[d, two heads]
                & D^{n} \ar[r, "d^{n}_{D}"']
                & D^{n+1} \\
                \im d^{n-1}_{D} \ar[ru, hook] & &
            \end{tikzcd}
        \end{equation*}
        which shows that $\ker d^{n}_{C} \hookrightarrow \im d^{n-1}_{D}$;
        thus, by passing to the cokernel, 
        the map $\ker d^{n}_{C} \to H^{n}(D^{\bullet})$ is zero.
        Since this holds for every $n \in \Z$, we conclude that $f^{*} = 0$.
    \end{proof}
\end{lemma}

Notice that this contraction construction gives
us a way to ``proliferate'' cochain maps:
indeed, given any morphism $g^{\bullet}: C^{\bullet} \to D^{\bullet}$,
then $g^{\bullet} + (ds+sd)$ is again a cochain map.
By the previous \textbf{Lemma}, 
it turns out that $g^{\bullet} + (ds+sd)$ is in fact
not very different from $g^{\bullet}$,
because they induce the same maps in cohomology!

\begin{df}
    Two maps $f^{\bullet}, g^{\bullet}: C^{\bullet} \to D^{\bullet}$
    are \textbf{cochain homotopic}, 
    and write $f^{\bullet} \simeq g^{\bullet}$,
    if their difference is null homotopic, that is
    \begin{equation*}
        f^{\bullet} - g^{\bullet} = ds + sd\,,
    \end{equation*}
    for some degree $-1$ map $s$. In this case,
    we call $s$ a (\textbf{cochain}) \textbf{homotopy}
    from $f^{\bullet}$ to $g^{\bullet}$.

    Finally, $f^{\bullet}: C^{\bullet} \to D^{\bullet}$
    is a (\textbf{cochain}) \textbf{homotopy equivalence}
    if there exists $g^{\bullet} : D^{\bullet} \to C^{\bullet}$
    such that
    \begin{equation*}
        g^{\bullet}f^{\bullet} \simeq \cat{1}_{C^{\bullet}}\,,
        \quad f^{\bullet}g^{\bullet} \simeq \cat{1}_{D^{\bullet}}\,.
    \end{equation*}
\end{df}

\begin{cor}
    If $f^{\bullet}, g^{\bullet}: C^{\bullet} \to D^{\bullet}$
    are \textbf{cochain homotopic}, then $f^* = g^*$.
    In particular, if $f^{\bullet}$ is a homotopy equivalence,
    then the two complexes have the same cohomology: 
    $H^{n}(C^{\bullet}) = H^{n}(D^{\bullet})$.
\end{cor}

\begin{df}
    A morphism of complexes
    $f^{\bullet} : A^{\bullet} \to B^{\bullet}$
    is a \textbf{quasi-isomorphism} (shortened \textbf{qis})
    if, for all $n \in \NN$, the induced map
    \begin{equation*}
        H^{n}(f^{\bullet}) : H^{n}(A^{\bullet}) \xrightarrow[]{\sim} H^{n}(B^{\bullet})
    \end{equation*}
    is an isomorphism.
\end{df}

Thus, the previous corollary may be stated as 
``\emph{homotopy equivalences are quasi-isomorphisms}''.
