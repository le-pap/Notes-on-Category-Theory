
\section{Mapping cones and cylinders}\todo{Check sign conventions and computations of the differentials.}

Let $\Aa$ be an abelian category, 
and consider the category $C^{\bullet}(\Aa)$
of cochain complexes in $\Aa$.

\begin{df}
    Let $f^{\bullet} : A^{\bullet} \to B^{\bullet}$
    be a map of cochain complexes.
    The \textbf{mapping cone} of $f^{\bullet}$
    is the cochain complex $\cat{C}(f^{\bullet})$
    whose degree $n$ part is
    \begin{equation*}
        \cat{C}(f^{\bullet})^{n} := A^{n+1} \oplus B^{n}\,,
    \end{equation*}
    and the coboundary maps 
    $d_{f}^{n} : \cat{C}(f^{\bullet})^{n} \longrightarrow \cat{C}(f^{\bullet})^{n+1}$
    are given by the matrices
    \begin{equation}\label{cone-coboundary}
        % d^{n} : \cat{C}(f^{\bullet})^{n} \longrightarrow \cat{C}(f^{\bullet})^{n+1}\,,
        %\quad 
%        \begin{pmatrix} a \\ b\end{pmatrix}
%        \longmapsto
        d^{n}_{f} := \begin{pmatrix} -d^{n+1}_{A} & 0 \\ f^{n+1} & d^{n}_{B} \end{pmatrix}
%        \begin{pmatrix} a \\ b\end{pmatrix}
%        = \begin{pmatrix} -d^{n+1}_{A}(a) \\ d^{n}_{B}(b) + f^{n+1}(a) \end{pmatrix}
    \end{equation}
\end{df}

\begin{df}
    Let $f^{\bullet} : A^{\bullet} \to B^{\bullet}$
    be a map of cochain complexes.
    The \textbf{mapping cylinder} of $f^{\bullet}$
    is the cochain complex $\cat{cyl}(f^{\bullet})$
    whose degree $n$ part is
    \begin{equation*}
        \cat{cyl}(f^{\bullet})^{n}
        := A^{n} \oplus A^{n+1} \oplus B^{n}.
    \end{equation*}
    The coboundary map $d_{\cat{cyl}} : \cat{cyl}(f^{\bullet}) 
    \to \cat{cyl}(f^{\bullet})$ is given 
    on the $n$-th level by the matrix
    \begin{equation*}
        d_{\cat{cyl}}^n :=
        \begin{pmatrix}
            d_{A}^{n} & \cat{1}_{A^{n+1}} & 0 \\
            0 & -d_{A}^{n+1} & 0 \\
            0 & -f^{n+1} & d^{n}_{B}
        \end{pmatrix}\,.
    \end{equation*}
\end{df}

\begin{exercise!}\label{cyl-criterion}
    Show that two cochain maps 
    $f^{\bullet}, g^{\bullet} : A^{\bullet} \to B^{\bullet}$
    are cochain homotopic if and only if
    they extend to a map $(f,s,g) : \cat{cyl}(\cat{1}_{A^{\bullet}}) \to B^{\bullet}$.
    \begin{proof}[Solution]
        Given a homotopy $s$ from $f^{\bullet}$ to $g^{\bullet}$,
        we may define the map
        \begin{equation*}
            \Psi^{\bullet} : \cat{cyl}(\cat{1}_{A^{\bullet}}) \longrightarrow
            B^{\bullet}\,, \quad
            \Psi^{n}(a, a', a'') = f^{n}(a) + s^{n+1}(a') + g^{n}(a'')\,.
        \end{equation*}
        Then $\Psi^{\bullet}$ is a cochain map:
        indeed, if we drop the indices for simplicity,
        we compute
        \begin{align*}
            \Psi^{\bullet} d
            = \Psi^{\bullet}
            \begin{pmatrix}
            d & \cat{1} & 0 \\
            0 & -d & 0 \\
            0 & -\cat{1} & d
            \end{pmatrix}
            &= \Psi^{\bullet}(d + \cat{1}, -d, -\cat{1}+d) \\
            &= f^{\bullet}d + f^{\bullet} - sd - g^{\bullet} + g^{\bullet} d \\
            &= f^{\bullet}d + ds + g^{\bullet} d \\
            &= df^{\bullet} + ds + dg^{\bullet} 
            =d \Psi^{\bullet}\,.
        \end{align*}
        The equations $\Psi^{n}(a, 0, 0) = f^{n}(a)$ and
        $\Psi^{n}(0, 0, a'') = g^{n}(a'')$ show that $\Psi^{\bullet}$
        extends both $f^{\bullet}$ and $g^{\bullet}$.

        Conversely, suppose an extension $\Psi^{\bullet} : \cat{cyl}(\cat{1}_{A^{\bullet}}) \to B^{\bullet}$ is given and write
        $$j_{2} = (0, \cat{1},0) : A^{\bullet+1} \to \cat{cyl}(\cat{1}_{A^{\bullet}})$$
        for the inclusion. Then $s := \Psi j_{2}$ defines a homotopy
        between $f^{\bullet}$ and $g^{\bullet}$, indeed
        \begin{align*}
            ds = d\Psi^{\bullet}j_{2} 
            = \Psi^{\bullet} d j_{2}
            = \Psi^{\bullet} 
            \begin{pmatrix}
            d & \cat{1} & 0 \\
            0 & -d & 0 \\
            0 & -\cat{1} & d
            \end{pmatrix}
            \begin{pmatrix}
            0 \\ \cat{1} \\ 0 
            \end{pmatrix}
            = \Psi^{\bullet}(\cat{1}, -d, -\cat{1})
            = f^{\bullet} - sd - g^{\bullet}\,.
        \end{align*}
    \end{proof}
\end{exercise!}

\begin{lemma}
    The inclusion map 
    $$\iota := 0 \oplus 0 \oplus \cat{1}_{B^{\bullet}} : B^{\bullet}
    \to \cat{cyl}(f^{\bullet})$$
    is a quasi-isomorphism.
    \begin{proof}
        The above inclusion fits in the following
        short exact sequence:
        \begin{center}
            \begin{tikzcd}
                \cat{0} \ar[r]
                & B^{\bullet} \ar[r, "\iota"]
                & \cat{cyl}(f^{\bullet}) \ar[r, "\pi"]
                & \cat{C}(\cat{1}_{A^{\bullet}}) \ar[r]
                & \cat{0}\,,
            \end{tikzcd}
        \end{center}
        where $\pi$ is the projection on the first two components, switched:
        indeed, the inclusion $\iota$ is clearly a cochain map,
        and so is the projection because for every $n \in \Z$
        it holds
        \begin{align*}
            \pi \circ d_{\cat{cyl}}^n (x)
            &= 
            \begin{pmatrix}
                0 & 1 & 0 \\
                1 & 0 & 0
            \end{pmatrix}
            \begin{pmatrix}
            d_{A}^{n} & \cat{1}_{A^{n+1}} & 0 \\
            0 & -d^{n+1}_{A} & 0 \\
            0 & -f^{n+1} & d
            \end{pmatrix}
            x
            \begin{pmatrix}
                0 & 1 \\ 1 & 0 \\ 0 & 0
            \end{pmatrix} \\
            &= 
            \begin{pmatrix}
                -d_{A}^{n+1} & 0 \\ \cat{1}_{A^{n+1}} & d_{A}^{n}
            \end{pmatrix}
            \begin{pmatrix}
                0 & 1 & 0 \\
                1 & 0 & 0
            \end{pmatrix}
            x
            \begin{pmatrix}
                0 & 1 \\ 1 & 0 \\ 0 & 0
            \end{pmatrix}
            = d^{n}_{\cat{1}} \circ \pi(x)
        \end{align*}
        Thus, by the cohomological version of 
        \hyperref[LHS]{Theorem~\ref*{LHS}} we get 
        the long exact sequence
        \begin{center}
            \begin{tikzcd}[column sep=small]
                \dots \ar[r] 
                & H^{n-1}\big(\cat{C}(\cat{1}_{A^{\bullet}})\big) \ar[r]
                & H^{n}(A^{\bullet}) \ar[r]
                & H^{n}\big(\cat{cyl}(f^{\bullet})\big) \ar[r]
                & H^{n}\big(\cat{C}(\cat{1}_{A^{\bullet}})\big) \ar[r]
                & \dots
            \end{tikzcd}
        \end{center}
        If we show that the cone has trivial cohomology,
        then it follows that $H^{*}(B^{\bullet}) \simeq  
        H^{*}\big(\cat{cyl}(f^{\bullet})\big)$.
        By the definition of the coboundary maps~\eqref{cone-coboundary},
        % the cocycles $Z^{\bullet}$ and the coboundaries $B^{\bullet}$ 
        % for the cone complex are
        % \begin{equation*}
        %    Z^{n}\big(\cat{C}(-\cat{1}_{A^{\bullet}})\big)
        %    = B^{n+1}(A^{\bullet}) \oplus A^{n}\,,
        % \end{equation*}
        any $n$-cocycle is of the form $(-d_{A}a,a)$, for some $a \in A^{n}$,
        and in fact it is also a coboundary:
        \begin{equation*}
            \begin{pmatrix}
                -d^{n}_{A}(a) \\ a
            \end{pmatrix}
            =
            \begin{pmatrix}
                -d^{n}_{A} & 0 \\ \cat{1}_{A^n} & d^{n-1}_{A}
            \end{pmatrix}
            \begin{pmatrix}
                a \\ 0
            \end{pmatrix}
            = d^{n} \begin{pmatrix}
                a \\ 0
            \end{pmatrix}\,,
        \end{equation*}
        hence we conclude $H^{*}\big(\cat{C}(\cat{1}_{A^{\bullet}})\big) \simeq 0$.
    \end{proof}
\end{lemma}

\begin{exercise!}\label{cyl-equivalence}
    Show that the map $\beta^{\bullet} : \cat{cyl}(f^{\bullet}) \to B^{\bullet}$,
    defined at the $n$-th level by
    \begin{equation*}
        \beta^{n}(a,a',b) = f^{n}(a) + b\,,
    \end{equation*}
    defines a cochain map such that $\beta^{\bullet} \iota = \cat{1}_{B^{\bullet}}$.
    Then show that the formula
    \begin{equation*}
        s(a,a',b) = (0,a,0)
    \end{equation*}
    defines a cochain homotopy from the identity of $\cat{cyl}(f^{\bullet})$
    to $\iota \beta^{\bullet}$.
    Conclude that $\iota$ is in fact a cochain homotopy equivalence.
    \begin{proof}
        It is clear that $\beta^{\bullet}\iota = \cat{1}_{B^{\bullet}}$;
        we show it is a map of complexes:
        \begin{equation*}
            \beta^{\bullet} d_{\cat{cyl}} 
            = \beta^{\bullet} 
            \begin{pmatrix}
                d_{A} & \cat{1}_{A^{\bullet + 1}} & 0 \\
                0 & -d_{A} & 0 \\
                0 & -f^{\bullet} & d_{B}
            \end{pmatrix}
            =f^{\bullet}d_{A} + f^{\bullet} - f^{\bullet} + d_{B}
            = d_{B}(f^{\bullet} + \cat{1})
            = d_{B} \beta^{\bullet}\,.
        \end{equation*}
        To conclude $\iota$ is a homotopy equivalence,
        we show $\cat{1}_{\cat{cyl}} \simeq \iota\beta^{\bullet}$:
        \begin{align*}
            sd + ds
            &= (0, d_{A} + \cat{1}_{A^{\bullet+1}}, 0)
            + (\cat{1}_{A}, -d_{A}, -f^{\bullet})\\
            &= \left(\cat{1}_{A}, \cat{1}_{A^{\bullet+1}}, \cat{1}_{B} -(f^{\bullet}+\cat{1}_{B})\right)
            = \cat{1}_{\cat{cyl}} - \iota\beta^{\bullet}\,.
        \end{align*}
    \end{proof}
\end{exercise!}