
\section{Enriched categories}

When a symmetric monoidal category $(\Vv, \otimes)$ is fixed, 
we show how to enrich over it the notions of category, 
functor and natural transformation.

\begin{df}
    A \textbf{$\Vv$-enriched category} $\Cc$
    (or simply \textbf{$\Vv$-category})
    is given by the following data:
    \begin{itemize}
        \item a class $\Cc$ of \textbf{objects};

        \item for any pair $A,B \in \Cc$ of objects,
        there exists an \textbf{arrows} object $\Cc(A,B) \in \Vv$;

        \item for any triple $A,B,C \in \Cc$,
        there exists a \textbf{composition morphism} in $\Vv$:
            \begin{equation*}
                \cat{c}_{ABC} : \Cc(A,B) \otimes \Cc(B,C) \longrightarrow \Cc(A,C)\,;
            \end{equation*}

        \item any object $A \in \Cc$ has a \textbf{unit morphism}
        in $\Vv$, that is
            \begin{equation*}
                \cat{u}_{A} : e \longrightarrow \Cc(A,A)\,,
            \end{equation*}
        where $e$ is the unit object of $\Vv$.
    \end{itemize}
    Moreover, the previous data satisfy the following
    ``coherence'' axioms:
    \begin{itemize}
        \item[(\textbf{E1})]\label{E1} \textbf{associativity}:
        the composition is associative, i.e. 
        for any five objects $A,B,C,D,E$ in $\Cc$,
        the following diagram commutes:
        \begin{center}
            \begin{tikzcd}[column sep=tiny, row sep=large]
             & \left( \Cc(A,B) \otimes \Cc(B,C) \right) \otimes \Cc(C,D) 
             \ar[rd, "\cat{c}_{ABC} \otimes \cat{u}_{\Cc(C,D)}"] 
             \ar[dl, "\alpha"'] &  \\
            \Cc(A,B) \otimes \left( \Cc(B,C) \otimes \Cc(C,D) \right)\ar[d, "\cat{u}_{\Cc(A,B)} \otimes \cat{c}_{BCD}"']
            && \Cc(A,C) \otimes \Cc(C,D) \ar[d, "\cat{c}_{ACD}"] \\
              \Cc(A,B) \otimes \Cc(B,D) \ar[rr, "\cat{c}_{ABD}"]
            & & \Cc(A,D)  
            \end{tikzcd}
        \end{center}

        \item[(\textbf{E2})]\label{E2} \textbf{unital}:
        the unit morphism is the neutral element for the composition,
        i.e. for any pair $A,B \in \Cc$ we have a commutative diagram
            \begin{equation*}
                \begin{tikzcd}
                    e \otimes \Cc(A,B) 
                    \ar[r, "l_{\Cc(A,B)}"] \ar[d, "\cat{u}_{A} \otimes \cat{1}_{\Cc(A,B)}"]
                    & \Cc(A,B) \ar[d, equals] 
                    & \Cc(A,B) \otimes e 
                    \ar[l, "r_{\Cc(A,B)}"'] \ar[d, "\cat{1}_{\Cc(A,B)} \otimes \cat{u}_{B}"] \\
                    \Cc(A,A) \otimes \Cc(A,B) \ar[r, "\cat{c}_{AAB}"']
                    & \Cc(A,B)
                    & \Cc(A,B) \otimes \Cc(B,B) \,. \ar[l, "\cat{c}_{ABB}"]
                \end{tikzcd}
            \end{equation*}
    \end{itemize}
\end{df}

\begin{ex}
    A preadditive category is $\Ab$-enriched.
\end{ex}

\begin{ex!}\label{one-obj-ab}
    A one object $\Ab$-category $R$ is a unitary ring: indeed, 
    it consists of an object $\ast$ and an abelian group
    \begin{equation*}
        R := R(\ast,\ast) \in \Ab\,,
    \end{equation*}
    with a multiplication
    \begin{equation*}
        \cdot := \cat{c}_{\ast \ast \ast} : R \otimes_{\Z} R \longrightarrow R
    \end{equation*}
    given by the composition in $R$,
    which is associative by \hyperref[E1]{(\textbf{E1})}.
    Moreover, the unit morphism
    \begin{equation*}
        \cat{u} : \Z \longrightarrow R\,,
        \quad 1 \longmapsto u
    \end{equation*}
    detects the neutral element for the multiplication,
    for
    \begin{equation*}
        u \cdot x = \cat{c}(\cat{u}(1) \otimes x) 
        = x 
        = \cat{c}(x \otimes \cat{u}(1)) = x \cdot u\,,
    \end{equation*}
    by \hyperref[E2]{(\textbf{E2})}.
\end{ex!}

\begin{ex}
    Any locally small category $\Aa$ is enriched over $\Vv=\cat{Set}$:
    indeed, morphisms between two objects form sets and, for each $A \in \Aa$,
    the unit morphism is simply
    \begin{equation*}
        \cat{u}_{A}: \Set{\ast} \longrightarrow \Hom_{\Aa}(A,A)\,,
        \quad \ast \longmapsto \cat{1}_{A}\,.
    \end{equation*}
\end{ex}

\begin{ex!}\label{smcc-enriched}
    A symmetric monoidal closed category $\Vv$ is enriched over itself:
    for every pair $A,B \in \Vv$, one sets
    \begin{equation*}
        \Vv(A,B) := [A,B] \in \Vv\,,
    \end{equation*}
    and takes units and compositions as defined in 
    \hyperref[v-morphisms]{Remark~\ref*{v-morphisms}}.
    One can check that both (\textbf{E1}) and (\textbf{E2}) hold.
\end{ex!}

\begin{ex}[Dual category]
    If the monoidal category $\Vv$ is symmetric,
    a $\Vv$-category $\Cc$ gives rise to a 
    \textbf{dual} $\Vv$-category $\Cc^{*}$,
    whose objects are the same as $\Cc$,
    and for each pair $A,B \in \Cc^{*}$
    one sets
    \begin{equation*}
        \Cc^*(A,B) := \Cc(B,A)\,.
    \end{equation*}
    Compositions are then defined by
    \begin{equation*}
        \cat{c}^*_{ABC} : \Cc^*(A,B) \otimes \Cc^*(B,C)
        \xrightarrow[]{\tau} \Cc(C,B) \otimes \Cc(B,A)
        \xrightarrow[]{\cat{c}_{CBA}} \Cc(C,A) = \Cc^*(A,C)\,,
    \end{equation*}
    thus the unit is $\cat{u}_{A}^* = \cat{u}_{A}$, for every $A \in \Cc^*$;
    one checks that $\Cc^*$ is indeed a $\Vv$-category.
    
\end{ex}

Now we adapt the notion of a functor between enriched categories.

\begin{df}
    Let $(\Vv, \otimes)$ be a monoidal category 
    and $\Aa,\Bb$ two $\Vv$-categories.
    A \textbf{$\Vv$-functor} $F:\Aa \to \Bb$ 
    is defined by the following data:
    \begin{itemize}
        \item an object $F(A) \in \Bb$, for each $A \in \Aa$;
        \item for every pair of objects $A,A' \in \Aa$,
        there exists a morphism
        \begin{equation*}
            F_{AA'} : \Aa(A,A') \longrightarrow \Bb(F(A),F(A'))\,
        \end{equation*}
        in $\Vv$ such that the following axioms are satisfied:
        \begin{rmnumerate}
            \item \textbf{composition:} 
            for every three objects $A,A',A'' \in \Aa$,
            the following diagram commutes
            \begin{equation*}
                \begin{tikzcd}[column sep=huge]
                    \Aa(A,A') \otimes \Aa(A',A'') 
                    \ar[r, "\cat{c}_{AA'A''}"] \ar[d, "F_{AA'} \otimes F_{A'A''}"']
                    & \Aa(A,A'') \ar[d, "F_{AA''}"] \\
                    \Bb(F(A),F(A')) \otimes \Bb(F(A'),F(A''))
                    \ar[r, "\cat{c}_{F(A)F(A')F(A'')}"]
                    & \Bb(F(A),F(A''))\,;
                \end{tikzcd}
            \end{equation*}

            \item \textbf{unit to unit:}
            for each object $A \in \Aa$, 
            there is a commutative diagram
            \begin{equation*}
                \begin{tikzcd}
                     e \ar[r, "\cat{u}_{A}"] \ar[dr, "\cat{u}_{F(A)}"']
                     & \Aa(A,A) \ar[d, "F_{AA}"] \\
                     & \Bb(F(A),F(A))\,.
                \end{tikzcd}
            \end{equation*}
        \end{rmnumerate}
    \end{itemize}
\end{df}

\begin{ex!}\label{left-R-mod}
    Consider a one object $\Ab$-category $R$. 
    A covariant $\Ab$-functor
    \begin{equation*}
        M : R \longrightarrow \Ab\,,
        \quad \left( \ast \xrightarrow[]{r} \ast \right)
        \longmapsto \left( M(\ast) \xrightarrow[]{r \times -} M(\ast) \right)
    \end{equation*}
    describes a left $R$-module: indeed,
    we have seen in \hyperref[one-obj-ab]{Example~\ref*{one-obj-ab}}
    that $R$ is a unitary ring, 
    so $M(\ast)$ is an abelian group
    endowed with the external multiplication
    \begin{equation*}
        r \times m := M(r)(m)\,, \quad \text{for every } m \in M(\ast)\,.
    \end{equation*}
\end{ex!}

Finally, we get to the notion of natural transformation,
which is a bit more involved to state.

\begin{df}
    Let $F,G : \Aa \to \Bb$ be two $\Vv$-functors between two $\Vv$-categories.
    A \textbf{$\Vv$-natural transformation} $\phi:F \implies G$
    consists in giving, for every object $A \in \Aa$,
    a morphism 
    \begin{equation*}
        \phi_{A} : e \longrightarrow \Bb(F(A),G(A))
    \end{equation*}
    in $\Vv$ such that the following diagram commutes,
    for each pair $A,A''$:
    \begin{equation*}
        \begin{tikzcd}[column sep=small]
            & \Aa(A,A') 
            \ar[dr, "r_{A}^{-1}"] \ar[dl, "l_{A}^{-1}"'] & \\
            e \otimes \Aa(A,A') \ar[d, "\phi_{A} \otimes G_{AA'}"]
            && \Aa(A,A') \otimes e \ar[d, "F_{AA} \otimes \phi_{A'}"] \\
            \Bb(F(A),G(A)) \otimes \Bb(G(A),G(A'))
            \ar[dr, "\cat{c}_{F(A)G(A)G(A')}"']
            && \Bb(F(A),F(A')) \otimes \Bb(F(A'),G(A'))
            \ar[dl, "\cat{c}_{F(A)F(A')G(A')}"] \\
            & \Bb(F(A),G(A')) & \,.
        \end{tikzcd}
    \end{equation*}
\end{df}

\begin{ex}
    Consider a one object $\Ab$-category $R$, 
    as in \hyperref[one-obj-ab]{Example~\ref*{one-obj-ab}},
    and two $\Ab$-functors $M,N : R \to \Ab$,
    that is two left $R$-modules by 
    \hyperref[left-R-mod]{Example~\ref*{left-R-mod}}.
    Then an $\Ab$-natural transformation $f : M \implies N$
    is a group homomorphism $f : M(\ast) \to M(\ast)$
    such that, for every $r \in R$, the following square
    \begin{equation*}
        \begin{tikzcd}
            M(\ast) \ar[r, "f"] \ar[d, "r \times -"']
            & N(\ast) \ar[d, "r\times -"] \\
            M(\ast) \ar[r, "f"] & N(\ast)
        \end{tikzcd}
    \end{equation*}
    commutes, that is
    \begin{equation*}
        r \times f(m) = f(r \times m)\,, \quad m \in M(\ast).
    \end{equation*}
    Thus, $f$ is nothing but a left $R$-module homomorphism!
    In fact, one can notice that the category 
    of $\Ab$-functors from $R$ to $\Ab$,
    with $\Ab$-natural transformations, is
    \begin{equation*}
        \cat{Fun}_{\Ab}(R,\Ab) \simeq \Mod_{R}\,.
    \end{equation*}
\end{ex}

In the section on \hyperref[AdditiveCategories]{Additive categories}
in the previous chapter,
we came across to the 
\hyperref[additive-yoneda]{Additive Yoneda's Lemma},
which is an adaptation of the classical
\hyperref[yoneda]{Yoneda's Lemma} to the setting
of (pre)additive categories.
In fact, it is nothing but a special case 
of an \textbf{enriched} version of this result;
by skipping some more technical details,
we are now ready to state it 
in its full generality:

\begin{thm}[\textbf{Enriched Yoneda's Lemma}]\label{enriched-yoneda}
    Let $\Vv$ be a symmetric monoidal closed category
    and $\Aa$ a small\footnote{A $\Vv$-category $\Aa$ is \textbf{small} if objects form a set $|\Aa|$.}
    $\Vv$-category.
    For every object $A \in \Aa$ and 
    every $\Vv$-functor $F:\Aa^* \to \Vv$,
    the $\Vv$-natural transformations 
    from $\Aa(-,A)$ to $F$ form an object
    \begin{equation*}
        \Vv-\cat{Nat}(\Aa(-,A),F) \in \Vv \,;
        \footnote{In fact, one can prove that $\Vv$-functors between $\Vv$-categories
        $\Cc$ and $\Dd$, endowed with $\Vv$-natural transformations,
        give rise to a $\Vv$-category $\cat{Fun}_{\Vv}(\Cc,\Dd)$.}
    \end{equation*}
    moreover, there exists an isomorphism
        \begin{equation*}
            \Vv-\cat{Nat}(\Aa(-,A),F) \xrightarrow[]{\sim} F(A)
        \end{equation*}
    in $\Vv$, which is $\Vv$-natural in both $A$ and $F$.
\end{thm}