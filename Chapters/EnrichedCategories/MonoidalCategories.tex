
\section{Monoidal categories}

%\emph{The following is taken from \parencite[Chapter~VII]{mclane}.}

%\vspace{6pt}

We want to give the definition of a category $\Vv$ 
endowed with a product $\otimes$ that behaves similarly
to a product in a \textbf{monoid}, that is a bifunctor 
$\otimes : \Vv \times \Vv \to \Vv$ which is \emph{associative}
\begin{equation*}
    X \otimes ( Y \otimes Z) = (X \otimes Y) \otimes Z
\end{equation*}
and there is an object $e$ which is the ``\emph{neutral element}''
with respect to this product, i.e.
\begin{equation*}
    e \otimes X = X = X \otimes e\,,
\end{equation*}
for all objects $X,Y,Z$ in $\Vv$.
As we know, requiring equalities in the above equations
may be too strict in the categorical setting,
hence we will relax this condition accordingly
by requiring that these properties hold \emph{up to natural isomorphism};
by experience, we know for instance that the category of vector spaces
over a field $k$ has this exact behaviour with the tensor product:
indeed, we have a canonical isomorphism
\begin{equation*}
    U \otimes_{k} ( V \otimes_{k} W) = (U \otimes_{k} V) \otimes_{k} W
\end{equation*}
for all $k$-vector spaces $U,V,W$, and the $1$-dimensional line
$k$ is the neutral element. 
Closer analysis shows that more care is requisite in this identification 
- one must use the right isomorphism, 
and one must verify that the resulting identification of multiple 
products can be made in a "coherent" way.

\begin{df}
    A \textbf{monoidal category} $\Vv = (\Vv, \otimes, e, \alpha, l, r)$
    is a category $\Vv$, a bifunctor $\otimes:\Vv \times \Vv \to \Vv$,
    an object $e$ and three natural isomorphisms:
    \begin{itemize}
        \item \textbf{associator:} 
        $\alpha : - \otimes ( - \otimes - ) \simeq (- \otimes -) \otimes -$;
        \item  \textbf{left unit law:} $l: e \otimes - \simeq \cat{1}_{\Vv} $;
        \item  \textbf{right unit law:} $r: - \otimes e \simeq \cat{1}_{\Vv} $;
    \end{itemize}
    such that they satisfy the following properties:
    \begin{itemize}
        \item[(\textbf{M5})]\label{M5} \textbf{pentagon equation:} for every four objects $X,Y,Z$ and $W$ 
        in $\Vv$, the following penthagon containing all possible dispositions
        of brackets among $4$ letters commutes:
        \begin{center}
        \begin{tikzcd}[column sep=0.1in]
             & (X \otimes Y) \otimes (Z \otimes W) \ar[rd, "\alpha"] &  \\
            X \otimes \big(  Y \otimes ( Z \otimes W ) \big) 
            \ar[ru, "\alpha"] \ar[d, "\cat{1}_{X} \otimes \alpha"']
            && \big( ( X \otimes Y )  \otimes Z \big) \otimes W\\
              X \otimes \big( (Y \otimes Z) \otimes W \big) \ar[rr, "\alpha"]
            & & \big( X \otimes ( Y \otimes Z) \big) \otimes W \ar[u, "\alpha \otimes \cat{1}_{W}"] 
        \end{tikzcd}
        \end{center}
        \item[(\textbf{M3})]\label{M3} \textbf{triangle equations:} 
        for every two objects $X, Y$ in $\Vv$, we want the neutral element $e$
        to be compatible with the associativity in the sense that the following triangle
        \begin{center}
        \begin{tikzcd}
            X \otimes (e \otimes Y) \ar[rr, "\alpha"] \ar[dr, "\cat{1}_{X} \otimes l"'] 
            & & (X \otimes e) \otimes Y \ar[dl, "r \otimes \cat{1}_{Y}"] \\
            & X \otimes Y & 
        \end{tikzcd}
        \end{center}
        is commutative.
    \end{itemize}
\end{df}

\begin{ex!}\label{finite-prod}
    Every category $\Vv$ with finite products is \textbf{monoidal}:
    for every $X, Y$ in $\Vv$, we define $X \otimes Y := X \times Y$ any chosen product,
    and the neutral element $e$ to be a terminal object. Then $l$ is given by composing 
    with the projection on the second factor, while $r$ by projecting on the first;
    $\alpha$ is the unique isomorphism given by the universal property of products.
    Then one checks that both property (\textbf{M5}) and (\textbf{M3}) 
    hold by noticing that all maps
    commute with the projections of the four (resp. three) fold products.
\end{ex!}

\begin{df}
    Let $\Vv$ be a monoidal category with product $\otimes$ and denote by
    $\tau : \Vv \times \Vv \to \Vv \times \Vv$ the ``switch'' functor
    that permutes the two components, i.e. for any two objects
    $X,Y \in \Vv$ it is defined as
    \begin{equation*}
        \tau(X,Y) := (Y,X)\,.
    \end{equation*}
    Then $\Vv$ is
    said to be \textbf{symmetric} if there is a natural isomorphism
    $\gamma : (-\otimes-) \simeq (-\otimes-) \circ \tau$, i.e.
    for every pair of object $X,Y$, there is an isomorphism
    $$\gamma_{X,Y} : X \otimes Y \simeq Y \otimes X\,,$$
    natural in both $X$ and $Y$; 
    moreover, it satisfies the following coherence
    conditions:
    \begin{itemize}
        \item[(\textbf{S1})]\label{S1}%
        the transformation $\gamma$ is involutive in the sense that
        \begin{equation*}
            \gamma_{X,Y} \circ \gamma_{Y,X} = \cat{1}_{X \otimes Y}\,;
        \end{equation*}
        \item[(\textbf{S2})]\label{S2}%
        it is compatible with the unital laws:
        \begin{equation*}
            r_{X} = l_{X} \circ \gamma_{X,e}\,;
        \end{equation*} 
        \item[(\textbf{S3})]\label{S3}%
        \textbf{exagon equation:} 
        for all $X,Y,Z$ in $\Vv$, the following diagram commutes:
        \begin{center}
        \begin{tikzcd}
            X \otimes (Y \otimes Z) \ar[r, "\alpha"] \ar[d, "\cat{1}_{X} \otimes \gamma"']%_{Y,Z}"']
            & (X \otimes Y) \otimes Z \ar[r, "\gamma"]
            & Z \otimes (X \otimes Y) \ar[d, "\alpha"] \\
            X \otimes (Z \otimes Y) \ar[r, "\alpha"]%_{X,Z \otimes Y}"] 
            & (X \otimes Z) \otimes Y \ar[r, "\gamma \otimes \cat{1}_{Y}"] 
            & (Z \otimes X) \otimes Y \,.
        \end{tikzcd}
        \end{center}
    \end{itemize}
\end{df}

\begin{ex}
    In fact, categories with finite products in \hyperref[finite-prod]{Example~\ref*{finite-prod}}
    are symmetric monoidal categories.
\end{ex}

\begin{ex}
    For each commutative unital ring $R$, the category $\Vv=\Mod_{R}$
    of modules over $R$ is monoidal with respect to the tensor product $\otimes_{R}$
    is a symmetric monoidal category. The neutral element is $e = R$.
    Our main interest is the case of $R=k$ a field 
    and vector spaces $\cat{Vect}_{k}$ over it.
\end{ex}

\begin{ex}
    The category $\Vv=\cat{CGHaus}_{\ast}$ of pointed compactly generated
    Hausdorff spaces is a symmetric monoidal category, 
    whose product $\otimes = \wedge$ is the \textbf{smash product}
    and neutral element given by $e = (S^{0}, +1)$ the discrete set with two points.
\end{ex}

\begin{ex}
    Remember that a poset $(P, \le)$ gives rise to a category $\cat{P}_{\le}$
    whose objects are elements of $P$, and arrows given by the ordering, i.e.
    for any $x,y \in P$, then there exists $x \to y$ if and only if $x \le y$.
    Assume $P$ has a top element $1$. If $P$ is a meet-semilattice,
    i.e. for any two elements $x,y \in P$ there exists their greatest lower bound
    $x \wedge y$, then the category $\Vv=\cat{P}_{\le}$ endowed with the operation
    $\wedge$ is a symmetric monoidal category, whose neutral element is $e=1$.
\end{ex}

\begin{ex!}\label{tensor-complex}
    Let $\Aa=\Mod_{R}$ and 
    consider any two cochain complexes $X^{\bullet}, Y^{\bullet} 
    \in \cat{C}^{\bullet}(\Aa)$.
    We define the \textbf{tensor product} $X^{\bullet} \otimes Y^{\bullet}$
    to be the complex whose $n$-th level is given by
        \begin{equation*}
            \left( X^{\bullet} \otimes Y^{\bullet} \right)^{n}
            := \bigoplus_{p+q=n} X^{p} \otimes_{R} Y^{q}\,,
        \end{equation*}
    and the coboundary map $d^{n}:\left( X^{\bullet} \otimes Y^{\bullet} \right)^{n}
    \to \left( X^{\bullet} \otimes Y^{\bullet} \right)^{n+1}$
    is defined on simple tensors as
    \begin{equation*}
        d^{n}(x \otimes y) := d_{X}^{p}(x) \otimes y + (-1)^{p}x \otimes d_{Y}^{q}(y)\,,
        \quad \text{with } x \in X^{p}, y \in Y^{q}\,.
    \end{equation*}
    One verifies that $(\Cc^{\bullet}(\Aa), \otimes)$ is a monoidal category,
    with unital element $e=R$, seen as a complex concentrated in degree zero.
    Moreover, $\Cc^{\bullet}(\Aa)$ is also symmetric:
    indeed, the isomorphisms $\gamma_{X^{\bullet},Y^{\bullet}}$
    are given by graded commutativity.
    
    One can generalize this example to any abelian category $\Aa$
    which is, in addition, a symmetrical monoidal category.
\end{ex!}

\begin{df}
    A symmetric monoidal category $(\Vv, \otimes)$ is \textbf{closed} if,
    for each object $B \in \Vv$, the functor
    \begin{equation*}
        - \otimes B: \Vv \longrightarrow \Vv
    \end{equation*}
    has a right adjoint, written $[B,-] : \Vv \to \Vv$.
\end{df}

\begin{rmk}
    The notion of monoidal closed category can be seen as
    a generalized setting in which it holds the ``$\otimes - \Hom$''
    adjunction that is familiar in the category of $R$-modules,
    as recalled in the next example.
\end{rmk}

\begin{ex}
    Given a commutative unital ring $R$, 
    the symmetric monoidal category $\Vv = \Mod_{R}$
    is closed: indeed, given any three $R$-modules $M,N$ and $P$, 
    it is well known that there exists a natural isomorphism
    \begin{equation*}
        \Hom_{R}(M \otimes_{R} N, P) \simeq \Hom_{R}\left(M, \Hom_{R}(N,P)\right)\,,
    \end{equation*}
    thus we deduce that $\Hom_{R}(N,-)$ is a right adjoint to $-\otimes_{R} N$.
\end{ex}


\begin{ex}
    By \hyperref[finite-prod]{Example~\ref*{finite-prod}},
    the category $\Vv = \cat{Set}$ has finite products, 
    hence it is a monoidal category, endowed with
    $\otimes = \times$ the cartesian product.
    Clearly, it is symmetric and, moreover,
    the \textbf{Exponential law} shows that $\cat{Set}$
    is monoidal closed: precisely, for any $Y$ a fixed set,
    the exponentiation functor
    \begin{equation*}
        (-)^{Y}:\cat{Set} \longrightarrow \cat{Set}\,,
        \quad Z \longmapsto Z^{Y} := \Hom_{\cat{Set}}(Y,Z)
    \end{equation*}
    is a left adjoint to $- \times Y$,
    for we have natural bijections
    \begin{equation*}
        Z^{X \times Y} \simeq \left(Z^{Y}\right)^{X}
    \end{equation*}
\end{ex}

\begin{rmk!}\label{v-morphisms}
    Taking the previous example into account, for every $X,Y \in \Vv$,
    we might think of $[X,Y] \in \Vv$ as the object in $\Vv$
    that classifies morphisms between $X$ and $Y$.
    In particular, notice that by definition one has
    \begin{equation*}
        \Hom_{\Vv}(X,X) \simeq \Hom_{\Vv}(e \otimes X,X) \simeq \Hom_{\Vv}(e,[X,X])\,,
    \end{equation*}
    which yield a \textbf{unit morphism} $\cat{u}_{X}:e \to [X,X]$, 
    corresponding with the identity $\cat{1}_{B}$.
    On the other hand, from the identifications
    \begin{equation*}
        \Hom_{\Vv}(Y,Y) \simeq \Hom_{\Vv}(Y \otimes e, Y) \simeq \Hom_{\Vv}(Y,[e,Y]),
    \end{equation*}
    one gets the isomorphism $i_{Y}:Y \to [e,Y]$, 
    which allows us to think of $[e,Y]$ as ``\emph{points of $Y$}'':
    note the similarity with $Y \simeq \Hom_{\cat{Set}}(\ast,Y)$ 
    in the category of sets, or even $G \simeq \Hom_{\Z}(\Z,G)$
    in $\Ab$. Finally, there is a stronger hint which validates
    this interpretation: for every $X,Y \in \Vv$, the adjunction
    \begin{equation*}
        \Hom_{\Vv}([X,Y], [X,Y]) \simeq \Hom_{\Vv}([X,Y] \otimes X, Y)
    \end{equation*}
    sends the identity $\cat{1}_{[X,Y]}$ 
    to the \textbf{evaluation morphism}
    \begin{equation*}
        \cat{ev}_{XY} : [X,Y] \otimes X \longrightarrow Y\,.
    \end{equation*}
    Thanks to this, for every triple $X,Y,Z \in \Vv$
    we can define \textbf{composition morphisms}
    \begin{equation*}
        \cat{c}_{XYZ} : [X,Y] \otimes [Y,Z] \longrightarrow [X,Z]
    \end{equation*}
    as the morphism corresponding via adjunction
    to the composite
    \begin{equation*}
            {[X,Y] \otimes [Y,Z] \otimes X} 
            \simeq {[X,Y] \otimes X \otimes [Y,Z]} 
            \to {Y \otimes [Y,Z]}
            \simeq {[Y,Z] \otimes Y}
            \to Z\,.
    \end{equation*}
\end{rmk!}

\begin{prop}
    In a symmetric monoidal closed category $\Vv$:
    \begin{rmnumerate}
        \item the unit morphisms $\cat{u}_{X}:e \to [X,X]$ are natural in $X$;
        \item the morphisms $i_{Y} : Y \to [e,Y]$ are isomorphisms, 
        and they are natural in $Y$.
    \end{rmnumerate}
    \begin{proof}
        The inverse of $i_{Y}$ is given by the composite
        \begin{equation*}
            \begin{tikzcd}
                {[e,Y]} \ar[r, "r_{[e,Y]}^{-1}"]
                & {[e,Y] \otimes e} \ar[r, "\cat{ev}_{eY}"]
                & Y\,.
            \end{tikzcd}
        \end{equation*}
        We omit the routine computations.
    \end{proof}
\end{prop}