
\section{Path algebras}

Let $k$ be a fixed field for the rest of this section.

\begin{df}
	A \textbf{quiver} $Q = (Q_{0},Q_{1},s,t)$ is a \emph{directed graph},
	that is $Q$ consists of a set $Q_{0}$ of \textbf{vertices},
	a set $Q_{1}$ of \textbf{arrows} together with two maps $s,t:Q_{1} \to Q_{0}$
	which associate to each arrow $\alpha \in Q_{1}$ its \textbf{source} $s(\alpha)$
	and its \textbf{target} $t(\alpha)$.
\end{df}

\begin{df}
	A \textbf{non-trivial path} $\rho$ in a quiver $Q$ is a finite sequence 
	$\alpha_{1}\alpha_{2} \dots \alpha_{m}$ of arrows such that
	$t(\alpha_{i}) = s(\alpha_{i+1})$ for every $1 \le i < m$.
	It can be represented as
	\begin{equation*}
		\begin{tikzcd}
			\bullet \ar[r, bend left, "\alpha_{1}"]
			& \bullet \ar[r, bend left, "\alpha_{2}"]
			& \dots \ar[r, bend left, "\alpha_{m}", near end]
			& \bullet\,,
		\end{tikzcd}
	\end{equation*}
	which helps visualizing that the path source is
	$s(\rho) = s(\alpha_{1})$ and its target is $t(\rho)=t(\alpha_{m})$. 
	We call $m \ge 1$ its length. For every vertex $v \in Q_{0}$,
	there exists a \textbf{trivial path} $e_{v}$ of length $0$
	which starts and ends at $v$.
\end{df}

As the above picture suggests,
whenever the target of a path $\rho$
coincides with the source of a second path $\sigma$,
then we can join the two paths to form
a new longer path $\rho\sigma$.
This yields a well defined operation
on paths, so that we can build an algebra.

\begin{df}
	Given a quiver $Q$, the \textbf{path algebra} of $Q$ 
	is the associative $k$-algebra $kQ$ whose basis (as a $k$-vector space)
	given by paths in $Q$, and product given by
	\begin{equation*}
		(\alpha_{1}\alpha_{2} \dots \alpha_{m}) \cdot (\beta_{1}\beta_{2} \dots \beta_{l}) = 
		\begin{cases}
			\alpha_{1}\alpha_{2} \dots \alpha_{m}\beta_{1}\beta_{2} \dots \beta_{l}\,,
			\quad &\text{if } t(\alpha_{m}) = s(\beta_{1})\,;\\
			0\,, \quad &\text{otherwise}\,,
		\end{cases}
	\end{equation*}
	and for every $\alpha \in Q_{1}$ one sets
	$e_{s(\alpha)}\alpha=\alpha$ and $\alpha e_{t(\alpha)} = \alpha$.
\end{df}

We will be interested in \emph{finite} quivers, that is 
quivers whose vertices and edges are finite sets.

\begin{ex}
	Let $Q$ be a quiver of $m$ disjoint vertices
	\begin{equation*}
		\begin{tikzcd}
			\bullet_{1} & \bullet_{2} & \dots & \bullet_{m}\,.
		\end{tikzcd}
	\end{equation*}
	The path algebra $kQ$ is the $m$-dimensional $k$-vector space
	with basis $e_{1}, \dots, e_{m}$ and since the following relations hold
	$$e_{i} \cdot e_{j} = \delta_{ij}\,, \quad e_{i}^{2} = e_{i}\,,$$
	one deduces that this path algebra is the semisimple algebra $kQ \simeq k^{m}$.
\end{ex}

\begin{ex}
	Let $Q$ be the quiver with one vertex and one loop
	\begin{equation*}
		\begin{tikzcd}
			{}_{1} \bullet \ar[loop, "T"']\,.
		\end{tikzcd}
	\end{equation*}
	Then the loop $T$ generates many paths $T^{2}, T^{3}, \dots$
	unrelated to one another; they form a countable basis for the free algebra $kQ$,
	which can be identified with the polynomials $k[T]$.
\end{ex}

\begin{ex}
	If $Q$ is a finite quiver such that there exists at most
	one path $v_{i} \to v_{j}$ for every $1 \le i, j \le m$,
	then by plugging a scalar into the $(i,j)$-th entry of an
	$m \times m$ matrix we can identify 
	\begin{equation*}
		kQ \simeq
		\Set{C \in M_{m \times m}(k) | C_{ij}=0 \text{ if there is no path } v_{i} \to v_{j}}\,.
	\end{equation*}
	For instance, the path algebra of the quiver 
	\begin{equation*}
		\begin{tikzcd}
			\bullet_{1} 
			& \bullet_{2} 
			& \dots & \bullet_{m}
		\end{tikzcd}
	\end{equation*}
	is the subalgebra of upper triangular $m \times m$ matrices.
\end{ex}

We now list some of the properties of these path algebras.

\begin{prop}
	Let $Q$ be a finite quiver with vertices $\Set{1,2, \dots,m}$.
	\begin{rmnumerate}
		\item The trivial paths $e_{i}$ are \emph{orthogonal idempotents},
		that is $e_{i}e_{j} = \delta_{ij} e_{i}$.
		
		\item The path algebra $kQ$ is unital,
		with unit given by $1 = \sum_{i=1}^{m} e_{i}$.
		
	\item The subalgebra $e_{i} \cdot kQ$ is generated by paths \emph{starting} at $i$,
	the subalgebra $kQ \cdot e_{j}$ is the vector space of paths \emph{ending} at $j$,
	while $e_{i} \cdot kQ \cdot e_{j}$ are paths starting at $i$ and ending at $j$.
	One can also consider $kQ \cdot e_{i} \cdot kQ$ as the paths passing through the vertex $i$.
	
		\item since $kQ = \oplus_{i=1}^{m} e_{i} \cdot kQ$, we deduce that each
		$e_{i} \cdot kQ$ is a projective right $kQ$-module.
		
		\item If $M$ is a right $kQ$-module, 
		then $\Hom_{kQ}\left(e_{i} \cdot kQ, M \right) \simeq M \cdot e_{i}$.
		
		\item The $e_{i}$ are \emph{inequivalent}, i.e. if $e_{i} \cdot kQ \simeq e_{j} \cdot kQ$,
		then $i = j$.
		\begin{proof}
			Only statement (vi) needs a proof. If there exists an isomorphism
			$f : e_{i} \cdot kQ \simeq e_{j} \cdot kQ$,
			then by (v) is can be seen as a path $f \in e_{j} \cdot kQ \cdot e_{i}$,
			whose inverse $g$ is a path starting at $e_{i}$ and ending at $e_{j}$.
			Thus their composition $fg=e_{i}$ should pass through $j$, which is possible only if
			$i=j$ because $e_{i}$ is a trivial path.
		\end{proof}
	\end{rmnumerate}
\end{prop}

