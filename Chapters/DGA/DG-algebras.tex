
For the whole chapter, fix a commutative ring $\Lambda$ 
with unit $1 \ne 0$. We are going to study new
additional structures on $\Lambda$-algebras, 
namely \emph{differential} and \emph{graded} structures;
the aim of this is to define and understand their main properties,
and after that we will be able to
approach the related concept of graded modules over these algebras. 
In the geometric setting, our main interest will be the special case
of the semisimple algebra $\Lambda = k$,
for some field and $m \le 1$.

The differential structure on these new objects
allows us to talk about \emph{cohomology}.
It is well known that isomorphic complexes
yield the same information in cohomology,
while the converse is false in general;
thus, we are going to focus on a class of very well-behaved 
graded algebras, \emph{intrinsically formal} algebras:
strictly speaking, these objects are determined
by their cohomology. 

Intrinsic formality of DGA-algebras
can be characterized in terms of vanising
of some \emph{Hochschild cohomology groups},
thus we safe a section to define this notion.

\section{DG-algebras}

	Every $\Lambda$-algebra in this chapter is associative,
	not necessarily commutative,
	with unit $1 \ne 0$.
	
	\begin{df}
		A $\Lambda$-algebra $A$ is \textbf{graded algebra}
		if there exist submodules $\Set{A^{k} | k \in \Z}$,
		such that $1 \in A^{0}$ and,
		for every $i,j \in \Z$, we have multiplications
		\begin{equation*}
			A^{i} \otimes_{\Lambda} A^{j} \longrightarrow A^{i+j}\,,
			\quad x \otimes y \longmapsto xy\,,
		\end{equation*}
		such that $A = \bigoplus_{k \in \Z} A^{k}$.
		An element of $A^{k}$ is said \textbf{homogeneous of degree $k$};
		sometimes we will simply write $|x|$ instead of $k$,
		for a homogeneous element $x \in A^{k}$.
		
		A graded $\Lambda$-algebra is \textbf{graded-commutative} 
		(or \textbf{anti-}, sometimes called \textbf{super-}commutative),
		if it holds
		\begin{equation*}
			yx = (-1)^{kh} xy\,,
			\quad \text{for every } x \in A^{k}, y \in A^{h}\,.
		\end{equation*}
	\end{df}
	
	The concept of differential graded algebra $A$ is not new:
	it boils down to a cochain complex whose differential
	must satisfy some compatibility axiom with respect to the multiplication.
	
	\begin{df}
		A \textbf{differential} on a graded algebra $A$ is a
		$\Lambda$-linear endomorphism $d:A \to A$
		such that $d^{2} = 0$ and, for every $k \in \Z$,
		it holds $d(A^{k}) \subset A^{k+1}$;
		moreover, $d$ must satisfy the following \textbf{graded Liebniz rule}:
		\begin{equation}\label{graded-liebniz}
		 	d(xy) = (dx)y + (-1)^{k}x (dy)\,,
		 	\quad \text{for } x \in A^{k}\,.
		 \end{equation} 
		A \textbf{DG-algebra} $\Aa = (A,d)$
		is a graded $\Lambda$-algebra $A$ endowed with a differential $d$.
	\end{df}
	
	\begin{rmk}
		Notice that \eqref{graded-liebniz} implies that $d(1)=0$.
	\end{rmk}	
	
	The condition $d(A^{k}) \subset A^{k+1}$ in the above definition
	allows us to interpret a DG-algebra $\Aa = (A,d)$ as a
	sequence
	\begin{equation*}
		\begin{tikzcd}
			\dots \ar[r] & A^{k-1} \ar[r, "d"] \ar[rr, bend right=20, "0"']
			& A^{k} \ar[r, "d"] 
			& A^{k+1} \ar[r] & \dots
		\end{tikzcd}
	\end{equation*}
	thus, any DG-algebra is a cochain complex $\Aa = A^{\bullet}$
	of $\Lambda$-modules,
	whose coboundary map is the same $d$ at each level;
	this means we can compute its cohomology
	$H^{*}(\Aa)$. The interesting fact is that 
	this is not just a module:
	indeed, the relation \eqref{graded-liebniz} implies that
	the multiplication induced on $H^{*}(\Aa)$ 
	is well defined and moreover
	\begin{equation*}
		H^{k}(\Aa) \otimes H^{h}(\Aa) \longrightarrow H^{k+h}(\Aa)\,,
		\quad [x] \cdot [y] = [xy]
	\end{equation*}
	shows that the cohomology inherits a graded $\Lambda$-algebra 
	structure $H^{*}(\Aa) = \oplus_{k} H^{k}(\Aa)$.
	By endowing it with a trivial differential $d=0$,
	we conclude that the cohomology $H^{*}(\Aa)$ of a DG-algebra $\Aa$
	is again a DG-algebra.
	
	
	We define $\cat{DG}_{\Lambda}$ to be the category whose objects
	are DG-algebras over $\Lambda$, 
	and morphisms $f:\Aa \to \Bb$ between them
	given by homomorphisms of unital $\Lambda$-algebras
	\begin{equation*}
		f : A \longrightarrow B\,, \quad 
		\text{such that } f\left(1_{A}\right) = 1_{B}\,,
	\end{equation*}
	which are also maps of complexes, that is 
	$f(A^{k}) \subset B^{k}$ and $f \circ d = \delta \circ f$.
	One can check that $\cat{DG}_{\Lambda}$ is an abelian category
	(we might expect this because it is a category of modules).
	
	\begin{df}
		Given a DG-algebra $\Aa = (A,d)$, we define its \textbf{opposite algebra}
		$\Aa^{op} = (A^{op},d)$ as the DG-algebra whose elements and differential
		are the same of $\Aa$, but we consider a new multiplication
		\begin{equation*}
			a \cdot^{op} b := (-1)^{|a|\,|b|} ba\,,
		\end{equation*}
		where $ba$ denotes the usual multiplication in $\Aa$.
		Notice that $\Aa$ is graded-commutative if and only if $\Aa = \Aa^{op}$.
	\end{df}
	
	\begin{ex}
		Let $X$ be a smooth real manifold.
		The algebra $\Omega^{\bullet}(X)$
		of smooth differential forms endowed with 
		the \emph{exterior derivative} is a DG-algebra
		and $\Omega^{\bullet}$ determines a functor
		\begin{equation*}
			\Omega^{\bullet} : \cat{Man} \longrightarrow \cat{DG}_{\R}\,,
		\end{equation*}
		where $\cat{Man}$ is the category of smooth manifolds and smooth maps.
	\end{ex}
	
	\begin{ex}
		Taking cohomology defines a covariant functor 
		$H^{*}:\cat{DG}_{\Lambda} \to \cat{DG}_{\Lambda}$.
	\end{ex}
	
	\begin{ex}
		Given a finite dimensional $k$-vector space $V$, 
		define $T^{0} := k$ and for each $k \ge 1$ set
		\begin{equation*}
			T^{k}(V) := T^{\otimes k} 
			= \underbrace{V \otimes_{k} \dots \otimes_{k} V}_{k \text{ times}}\,.
		\end{equation*}
		Then $T(V) := \bigoplus_{k \ge 0} T^{k}(V)$ naturally 
		inherits an associative multiplication
		\begin{equation*}
			T^{k}(V) \otimes_{k} T^{h}(V) \longrightarrow T^{k+h}(V)\,;
		\end{equation*}
		from the canonical isomorphism $k \otimes_{k} V \simeq V$ 
		we deduce that $1 \in T^{0}(V)$ is the unit of the multiplication,
		hence $T(V)$ is a graded algebra over $k$,
		called \textbf{tensor algebra} of $V$.
		
		Let $\{e_{1}, \dots, e_{n}\}$ be a basis of $V$,
		and set $A^{-k}:=T^{k}(V)$.
		We can define a differential $d:T(V) \to T(V)$
		by setting on basis elements 
		\begin{equation*}
			d(e_{i}) = (-1)^{i}\,, \quad i \in \Set{1, 2, \dots, n}\,,
		\end{equation*}
		and then extending it componentwise to a map
		\begin{equation*}
			d : A^{-k} \longrightarrow A^{-k+1}\,, \quad
			d(e_{i_{1}} \otimes \dots \otimes e_{i_{k}})
			= \sum_{j=1}^{k} e_{i_1} \otimes \dots 
			\otimes d(e_{i_{j}}) \otimes \dots \otimes e_{i_{k}}\,.
		\end{equation*}
	\end{ex}
	
	We write $\iota_{\Aa} : \Lambda \to A$ to be the structure map
	$\iota(\lambda) := \lambda \cdot 1_{A}$.
	
	\begin{df}
		An \textbf{augmentation} on a DG-algebra $\Aa$
		is a morphism $\epsilon : \Aa \to \Lambda$
		of unital $\Lambda$-algebras such that $\epsilon \circ d = 0$ and
		$\epsilon \circ \iota_{\Aa} = \cat{1}_{\Lambda}$.
		Its kernel is a two-sided ideal of $\Aa$,
		called the \textbf{augmentation ideal} $\Aa^{+} := \ker \epsilon$.
	\end{df}
	
	Notice that the condition $\epsilon d = 0$ implies
	that there is a well defined aumentation on $H^*(\Aa)$.
	Since $\epsilon$ is a retraction on $\Lambda$,
	we may identify the ground ring $\Lambda$ with a subring of $A$;
	in particular, notice that $\Lambda$ is a direct summand of $A$
	by the \hyperref[split-lemma]{Split Lemma}.
	Thus, an augmentation is determined by its restriction to $A^{0}$.
	
	From now on, we will consider DG-algebras with terms of non-negative degree,
	that is $A^{k} = \cat{0}$ for $k<0$. In this case,
	the augmetation ideal coincides with the
	submodule of positive degree elements
	$\Aa^{+} = \bigoplus_{k \ge 0} A^{k}$.
	If moreover $\Aa$ is \textbf{connected},
	that is $A^{k} = \cat{0}$ for $k<0$ and $\iota_{\Aa}:\Lambda \to A^{0}$
	is an isomorphism, then $\Aa$ has a unique augmentation map
	given by $\iota_{\Lambda}^{-1}$.
	

	
	
	\begin{df}
		A DG-algebra $\Aa$ endowed with an augmentation $\epsilon$
		will be called \textbf{DGA-algebra} 
		(which stands for \textbf{differential augmented graded algebra}).
		Given two DGA-algebras 
		$\Aa = (A,d,\epsilon)$ and $\Aa' = (A',d',\epsilon')$,
		a \textbf{DGA-homomorphism} is a morphism
		of DG-algebras compatible with augmentations,
		that is $\epsilon' \circ f = \epsilon$.
		\begin{equation*}
			\begin{tikzcd}
				A \ar[rr, "f"] \ar[dr, "\epsilon"'] & & A' \ar[dl, "\epsilon'"] \\
				& \Lambda & \,.
			\end{tikzcd}
		\end{equation*}
	\end{df}
	
	We will write $\cat{DGA}_{\Lambda}$ for the category of
	DGA-algebras over $\Lambda$.
	Notice that a DGA-homomorphism induces a morphism $f*:H^*(\Aa) \to H^*(\Aa')$
	of DGA-algebras, hence cohomology defines a covariant functor
	$H^*:\cat{DGA}_{\Lambda} \to \cat{DGA}_{\Lambda}$.
	
	\begin{ex}
		Given a topological space $X$, let $S_{\bullet}(X)$ denote its
		singular simplicial complex. For each pair of spaces $X,Y$,
		one can define a chain map 
		$\sigma : S_{\bullet}(X) \otimes_{\Z} S_{\bullet}(Y) \to S_{\bullet}(X \times Y)$.
		Now, assume $X$ is a topological group, 
		with a continuous multiplication
		\begin{equation*}
			\mu: X \times X \longrightarrow X\,,
		\end{equation*}
		with neutral element $e \in X$;
		then the composition
		\begin{equation*}
			\begin{tikzcd}
				S_{\bullet}(X) \otimes_{\Z} S_{\bullet}(X) \ar[r, "\sigma"]
				& S_{\bullet}(X \times X) \ar[r, "\mu_{\bullet}"]
				& S_{\bullet}(X)
			\end{tikzcd}
		\end{equation*}
		defines an associative multiplication on $S_{\bullet}(X)$,
		whose neutral element is the $0$-simplex $e$,
		called \textbf{Pontrjagin product}.
		Moreover, there exists an augmentation $\epsilon : S(X) \to \Z$
		which sends each $0$-simplex (i.e. a point of $X$) to $1 \in \Z$,
		and vanishes on higher dimensional simplices;
		this makes $S_{\bullet}(X)$ into a DGA-algebra\footnote{In this case, $S_{\bullet}(X)$ is a
		DGA-algebra whose differential follows the \emph{homological} convention,
		while our definition is based on the cochain complex convention.}.
	\end{ex}
	
	\begin{prop}
		The category $\cat{DGA}_{\Lambda}$ is a %symmetric 
		monoidal category.
		\begin{proof}
			Consider any two DGA-algebras $\Aa = (A^{\bullet},d,\epsilon)$
			and $\Bb = (B^{\bullet}, \delta, \eta)$.
			We define their tensor product $\Aa \otimes \Bb$
			to be the cochain complex $A^{\bullet} \otimes B^{\bullet}$
			as defined in \hyperref[tensor-complex]{Example~\ref{tensor-complex}},
			i.e. the complex whose degree $n$ elements are
			\begin{equation*}
				\left( A^{\bullet} \otimes B^{\bullet} \right)^{n}
				= \bigoplus_{k \in \Z} A^{k} \otimes_{\Lambda} B^{n-k}\,,
			\end{equation*}
			equipped with the differential $D$ defined by
			\begin{equation*}
				D(a \otimes b) = da \otimes b + (-1)^{|a|}a \otimes \delta b\,.
				% \quad \text{for } a \in A^{k}\,.
			\end{equation*}
			We can define the multiplication
			\begin{equation*}
				(a \otimes b) \cdot (a' \otimes b')
				:= (-1)^{|a|\,|a'|}(aa') \otimes (bb')\,,
			\end{equation*}
			in such a way that $(A^{\bullet} \otimes B^{\bullet},D)$
			becomes a DG-algebra: indeed, it holds
			\begin{align*}
				D\big((a \otimes b) \cdot (a' \otimes b')\big)
				&= (-1)^{\deg(a)\deg(a')} D\big( (aa') \otimes (bb') ) \\
				&= (-1)^{\deg(a)\deg(a')} \big( d(aa') \otimes bb'
				+ (-1)^{\deg(aa')} aa' \otimes \delta(bb') \big) \\
				&= (-1)^{\deg(a)\deg(a')} 
				\Big( (da)a' \otimes bb' + (-1)^{\deg(a)}a(da') \otimes bb' \\
				&+ (-1)^{\deg(a) + \deg(a')} 
				\big(aa' \otimes (\delta b)b' 
				+ (-1)^{\deg(b)} aa' \otimes b(\delta b')\big)\Big) \\
				&= 
				\big( (-1)^{\deg(a')} ( da \otimes b %\cdot (a' \otimes b')
				+ (-1)^{\deg(a)} a \otimes \delta b) \cdot (a' \otimes b') \big) \\
				&+ \big((a \otimes b) \cdot (da' \otimes b' + a' \otimes \delta b' )\big) \\
				&= (-1)^{\deg(a')} \big( D(a \otimes b) \cdot (a' \otimes b')
				+ \big)\\
				&= D(a \otimes b) \cdot (a' \otimes b')
				+(-1)^{\deg(a \otimes b)} (a \otimes b) \cdot D(a' \otimes b')
			\end{align*}\todo{Please fix this.}
			so the differential satisfies the Liebniz rule \eqref{graded-liebniz}.
			Finally, the map $\alpha$ defined on every simple tensor 
			$a \otimes b$ by
			\begin{equation*}
				\alpha( a \otimes b ) := \epsilon(a)\eta(b) = 0\,,
			\end{equation*}
			is an augmentation on $\Aa \otimes \Bb$ because
			it clearly vanishes on elements of $(\Aa \otimes \Bb)^{0}
			= A^{0} \otimes B^{0}$, and
			by composing it with the differential one gets
			\begin{equation*}
				\alpha \circ D
				= (\epsilon \circ d) \eta
				\pm \epsilon (\eta \circ \delta)
				= 0\,.
			\end{equation*}
			This shows that $\Aa \otimes \Bb 
			= (A^{\bullet} \otimes B^{\bullet}, D, \alpha)$
			is a DGA-algebra.
			One can check that $\cat{DGA}_{\Lambda}$ equipped with $\otimes$
			becomes a monoidal category by
			an analgous argument as for $C^{\bullet}(\cat{Mod}_{\Lambda})$.
		\end{proof}
	\end{prop}
	

	
	
	
	
	
	
	