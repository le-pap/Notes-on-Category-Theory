\section{DG-modules}

	\begin{df}
		Let $\Aa = (A,d)$ be a DG-algebra.
		A \textbf{left DG-module} $\Mm$ over $\Aa$, or simply \textbf{$\Aa$-DG-module},
		is a cochain\footnote{Some authors, for instance Henri Cartan, use the homological indexing, which means they consider a differential of degree $-1$. These two concepts are basically the same: from our definition } 
		complex $\Mm = (M^{\bullet},d_{M})$ in $C^{\bullet}(\cat{Mod}_{A})$
		such that:
		\begin{itemize}
			\item the multiplication by an homogeneous element $a \in A^{k}$
			is a group homomorphism of degree $k$, that is
			\begin{equation*}
				A^{k} \otimes_{\Aa} M^{h} \longrightarrow M^{k+h}\,, \quad
				a \otimes m \longmapsto am\,;
			\end{equation*}
			
			\item the differential $d_{M}$ satisfies the following Liebniz rule:
			\begin{equation}\label{graded-mod-liebniz}
				d_{M}(am) = (da)m + (-1)^{|a|}a(d_{M}m)\,,
				\quad \text{for } a \in \Aa, m \in \Mm\,.
			\end{equation}
		\end{itemize}
		A \textbf{morphism of $\Aa$-DG-modules} $f:\Mm \to \Nn$
		is simply a cochain map $f:M^{\bullet} \to N^{\bullet}$
		of complexes of $A$-modules.
		
		One defines a \textbf{right DG-module} $\Mm$
		over $\Aa$ (in short a \textbf{DG-$\Aa$-module})
		in an analogous way, but the Liebniz rule for the
		right multiplication becomes
		\begin{equation*}
			d_{M}(ma) = (d_{M}m)a + (-1)^{|m|}m(da)\,,
			\quad \text{for } a \in \Aa, m \in \Mm\,.
		\end{equation*}
	\end{df}
	
	\begin{ex}
		Left DG-modules over a fixed DG-algebra $\Aa$, together with maps of complexes,
		form a category called ${}_{\Aa}\cat{DGMod}$. 
		If $\Aa = A^{0}$, then there is no differential structure
		on the ground ring, thus an $\Aa$-DG-modules is just a cochain complex of $A^{0}$-modules.
		In particular, ${}_{\Z}\cat{DGMod} = C^{\bullet}(\Ab)$.
	\end{ex}
	
	\begin{ex}
		Given any $\Aa$-DG-module $\Mm$, its cohomology $H^{*}(\Mm)$ is naturally
		a cochain complex (with trivial differential), which has a natural module structure
		over %the DG-algebra 
		$H^*(\Aa)$.
	\end{ex}
	
	\begin{df}
		Given a DG-module $\Mm$ over a DGA-algebra $\Aa=(A,d,\epsilon)$,
		an \textbf{augmentation} on $\Mm$ is a $\Lambda$-linear homomorphism
		$\epsilon_{\Mm}:\Mm \to \Lambda$ such that:
		\begin{rmnumerate}
			\item $\epsilon_{M} \circ d_{M} = 0$;
			\item vanishes in positive degree, i.e. if $|m|>0$, then $\epsilon_{M}(m) = 0$;
			\item it is compatible with augmentation on $\Aa$, that is
			for every $a \in \Aa$ and every $m \in \Mm$, 
			it holds $\epsilon_{\Mm}(am) = \epsilon(a)\epsilon_{\Mm}(m)$.
		\end{rmnumerate}
		An $\Aa$-DG-module $\Mm$ endowed with an augmentation is called \textbf{$\Aa$-DGA-module}.
	\end{df}	
	
	\begin{ex}
		Let $G$ be a topological group acting (on the left) on a topological space $X$.
		The action $G \times X \to X$ induces a morphism on singular complexes
		\begin{equation*}
			S_{\bullet}(G) \otimes_{\Z} S_{\bullet}(X) \longrightarrow S_{\bullet}(X)\,,
		\end{equation*}
		which makes $S_{\bullet}(X)$ into a left DGA-module over $S_{\bullet}(G)$.
	\end{ex}
	
	\begin{df}
		Consider a homomorphism $f:(\Aa,\epsilon) \to (\Bb, \eta)$ of DGA-algebras over $\Lambda$,
		an $\Aa$-DGA-module $\Mm$ and a $\Bb$-DGA-module $\Nn$.
		A \textbf{DGA-homomorphism compatible} with $f$ is a $\Lambda$-linear
		map of complexes $g:\Mm \to \Nn$ of degree $0$ such that: 
		\begin{rmnumerate}
			\item it is comptible with restriction of scalars, i.e. for every $a \in \Aa$
			and every $m \in \Mm$ it holds $g(am) = f(a)g(m)$;
			\item it preserves augmentations, that is $\eta_{\Nn} \circ g = \epsilon_{\Mm}$.
		\end{rmnumerate}
	\end{df}
	
	From now on we borrow all the terminology used for complexes of modules. 
	For example, we say that a $\Aa$-DG-module $\Mm$ is \textbf{acyclic} 
	if $H^{*}(\Mm) = \cat{0}$;
	we say that a morphism $f:\Mm \to \Nn$ of $\Aa$-DG-modules 
	is a \textbf{quasi-isomorphism} if it induces isomorphisms 
	$f^*:H^{*}(\Mm) \simeq H^{*}(\Nn)$, and so on and so forth.
	
	When a DG-module $\Mm$ is endowed with an augmentation,
	its structure becomes ``enough rigid'': in fact,
	whenever we have a base of homogeneous elements of $\Mm$
	(e.g. if the ground ring $\Lambda = k$ is a field),
	we can always define a DGA-homomorphism from $\Mm$
	to an acyclic module $\Nn \to \Lambda \to 0$; moreover, the quasi-isomorphism
	class of this map is \emph{unique}.
	The following result by Henri Cartan explains more precisely
	what it means.
	
	\begin{thm}
		Let $\Aa$ be DGA-algebra and $\Mm$ be a DGA-module over it, 
		with a free basis of homogeneous elements; similarly,
		consider $\Mm'$ to be a DGA-algebra over the DGA-algebra $\Aa'$,
		and assume $\Mm'$ has a homogoeneous free basis.
		Given a DGA-homomorphism $f:\Aa \to \Aa'$,
		if both $\Mm$ and $\Mm'$ are acyclic, then there exists
		a map $g:\Mm \to \Mm'$ compatible with $f$; moreover,
		if $f*:H_{*}(\Aa) \to H_{*}(\Aa')$ is an isomorphism,
		then also $g_{*}:H_{*}(\Mm) \to H_{*}(\Mm')$ is an isomorphism.
		\begin{proof}
			For details, look at \parencite{cartanDGA}.
			Notice that Cartan uses the homological convention.
			The proof shows that a map $g$ as above is defined
			on a basis $\{m_{i}\}$ of $\Mm$ by the formula
			\begin{equation*}
				g\left( \sum_{i} a_{i} m_{i} \right) 
				:= \sum_{i} (fa_{i}) m'_{i}\,,
			\end{equation*}
			where each homogeneous element $m_{i}$ of degree $k$
			is sent to an element $m'_{i} \in \Mm'$ of the same degree,
			which is built by induction on $k$ by following the rules:
			\begin{itemize}
				\item for each $m_{i} \in M_{0}$, consider $m'_{i} \in M'_{0}$
				such that $\epsilon_{\Mm}(m_{i}) = \epsilon_{\Mm'}(m'_{i})$;
				\item if $|m_{i}| \ge 1$, then pick $m'_{i}$ such that
				$d'm'_{i} = g(dm_{i})$.
			\end{itemize}
			This construction is obviously non-unique, but it can be
			shown that any two
			homomorphisms built this way are \emph{homotopic}.
		\end{proof}
	\end{thm}
	
	By considering the case $f=\cat{1}_{\Aa}$, one obtains the following
	\begin{cor}
		Any two acyclic free DGA-modules $\Mm$ and $\Mm'$ over a DGA-algebra $\Aa$
		are quasi-isomorphic.
	\end{cor}
	
	\begin{df}
		We define the \textbf{translation functor} 
		$[1] : {}_{\Aa}\cat{DGMod} \to {}_{\Aa}\cat{DGMod}$
		by sending any $\Aa$-DG-module $\Mm$ to the graded module
		\begin{equation*}
			\left( \Mm[1] \right)^{n} := M^{n+1}\,,
			\quad d_{M[1]} := -d_{M}\,,
		\end{equation*}
		in which the $A$-module structure on $M[1]$ is \textbf{twisted},
		i.e. we define the scalar multiplication as
		\begin{equation*}
			a \ast m := (-1)^{|a|} am\,,
		\end{equation*}
		where $am$ is the multiplication in $M$.
	\end{df}
	
	
	At this point we have enough structure to talk about \emph{triangles}:
	in fact, our next goal is to develop enough theory to be able
	to state and prove that the homotopy category $\Kk(\Aa)$
	is a triangulated category.
	
	
	\begin{df}
		Two morphisms $f,g: \Mm \to \Nn$ in ${}_{\Aa}\cat{DGMod}$
		are \textbf{homotopic} if there exists a homotopy
		$s:\Mm \to \Nn[-1]$ of $A$-modules (possibly \textbf{not} of $\Aa$-DG-modules)
		such that $$f-g=sd_{M} + d_{N}s\,,$$
		in which case we write $f \sim g$.
	\end{df}
	
	Since null-homotopic morphisms form a $2$-sided ideal in 
	$\Hom_{{}_{\Aa}\cat{DGMod}}(\Mm,\Nn)$, we may quotient
	by the equivalence relation $f \sim g$ given by homotopy
	to obtain the \textbf{homotopy category} $\Kk(\Aa)$,
	whose objects are $\Aa$-DG-modules and morphisms between them are
	\begin{equation*}
		\Hom_{\Kk(\Aa)}(\Mm,\Nn) := \Hom_{{}_{\Aa}\cat{DGMod}}(\Mm,\Nn)/\sim\,.
	\end{equation*}
	
	Given a morphism $f:\Mm \to \Nn$ in ${}_{\Aa}\cat{DGMod}$,
	we may build the \textbf{cone} of $f$ in the usual way, namely
	$\cat{C}(f) := N \oplus M[1]$ with differential $(d_{N} + f, -d_{M})$;
	this naturally inherits a structure of $\Aa$-DG-module,
	thus we may define a \textbf{strict triangle}
	\begin{equation*}
		\begin{tikzcd}
			\Mm \ar[r, "f"] & \Nn \ar[r] & \cat{C}(f) \ar[r] & \Mm{[1]}\,.
		\end{tikzcd}
	\end{equation*}
	An \textbf{exact triangle} in $\Kk(\Aa)$
	is a diagram $\Mm' \to \Nn' \to \Cc' \to \Mm'[1]$
	which is isomorphic (in he homotopy category) to a strict
	triangle as above: explicitly, there is a diagram
	\begin{equation*}
		\begin{tikzcd}
			\Mm' \ar[r] \ar[d, "\simeq"] & \Nn' \ar[r] \ar[d, "\simeq"] & \Cc' \ar[r] \ar[d, "\simeq"] & \Mm'{[1]} \ar[d, "\simeq"] \\
			\Mm \ar[r, "f"] & \Nn \ar[r] & \cat{C}(f) \ar[r] & \Mm{[1]}\,,
		\end{tikzcd}
	\end{equation*}
	which commutes up to homotopy, in which vertical maps are homotopy equivalences.
	
	As it happens for the homotopy category $\cat{K}(\Bb)$ of an abelian category $\Bb$,
	it turns out that $\Kk(\Aa)$ becomes a triangulated category 
	if we equip it with
	the translation functor $[1]$ and take the exact triangles as the family
	of distinguished triangles. In this category, quasi-isomorphisms of $\Aa$-DG-modules
	form a \emph{multiplicative system}, and hence we can 
	build the \textbf{derived category} $\Dd(\Aa)$ by formally inverting
	quasi-isomorphisms in $\Kk(\Aa)$: as it happens for $\cat{D}(\Bb)$,
	morphisms are pictured as \emph{roofs}, and $\Dd(\Aa)$ also inherits
	a triangulated structure.
	
	\begin{rmk}
		Even though $\Kk(\Aa)$, resp. $\Dd(\Aa)$, is called 
		the homotopy category of $\Aa$-DG-modules, resp. the derived category,
		the reader must be careful that these constructions are \emph{not}
		the usual ones described in 
		\hyperref[Derived-categories]{Section~\ref{Derived-categories}} 
		for abelian categories, 
		for $\Kk(\Aa)$ is not isomorphic to $\cat{K}({}_{\Aa}\cat{DGMod})$ in general
		(thus, we use a different symbol). Indeed, objects in $\Kk(\Aa)$ are
		DG-modules seen already as complexes, while objects in $\cat{K}({}_{\Aa}\cat{DGMod})$
		are complexes of DG-modules, which can be interpreted as bicomplexes.
		This explains why we should morally prove again 
		that $\Kk(\Aa)$ and $\Dd(\Aa)$ are triangulated,
		as it happens in \parencite[Part II]{bernstein-lunts}, 
		but in fact constructions are analogous for the classic homotopy and derived categories.
	\end{rmk}
	
	Let $\Aa=(A,d)$ be a DG-algebra.
	 Given two left DG-modules $\Mm$ and $\Nn$ over $\Aa$,
	we define the \textbf{internal hom} to be 
	the cochain complex $Hom^{\bullet}(\Mm,\Nn)$	of abelian groups
	\begin{equation*}
		Hom^{n}(\Mm,\Nn) := \Hom_{A}(M, N[n]) 
		= \Set{f:M \to N[n] | f \text{ morphism of %graded 
		} A\text{-modules}}\,,
	\end{equation*}
	where the differential $d$ is defined on $f \in Hom^{n}(\Mm,\Nn)$ by
	\begin{equation*}
		df := d_{\Nn} \circ f - (-1)^{n} f \circ d_{\Mm}\,.
	\end{equation*}
	We define the tensor product of $\Aa$-DG-modules in a similar fashion
	as for DG-algebras: given $\Mm$ and $\Nn$ left DG-modules over $\Aa$, 
	we define their \textbf{tensor product} $\Mm \otimes_{\Aa} \Nn$
	to be the total complex associated to $M^{\bullet} \otimes_{A} \Nn^{\bullet}$,
	that is the $A$-module $M \otimes_{A} N$ with the differential
	\begin{equation*}
		d(m \otimes n) := (d_{\Mm}m) \otimes n + (-1)^{|m|}m (d_{\Nn}n)\,.
	\end{equation*}
	
	\begin{ex}
		A $0$-cocycle of $Hom^{\bullet}(\Mm,\Nn)$ is a map of complexes:
		indeed, by definition $f:\Mm \to \Nn$ is such that $d_{\Nn}f - fd_{\Mm}$.
		Since $0$-coboundaries are null homotopic maps, one deduces that
		\begin{equation*}
			H^{0}\left(Hom^{\bullet}(\Mm,\Nn) \right)
			= \Hom_{\Kk(\Aa)}(\Mm,\Nn)\,.
		\end{equation*}
	\end{ex}
	
	If $\Aa$ is a graded-commutative DG-algebra, 
	then both $Hom^{\bullet}(\Mm,\Nn)$
	and $\Mm \otimes_{\Aa} \Nn$ have a natural $\Aa$-DG-module structure,
	in which scalar multiplications are defined for every $a \in \Aa$ by
	\begin{equation*}
		(a \cdot f) : m \mapsto af(m)\,, \quad
		a \cdot (m \otimes n) := (-1)^{|a|\,|m|} am \otimes n\,.
	\end{equation*}
	One can check that both $Hom^{\bullet}(\Nn,-)$ and $- \otimes_{\Aa} \Nn$ are
	endofunctors of ${}_{\Aa}\cat{DGMod}$ which send null-homotopic maps
	to null-homotopic maps, thus they descend to additive functors
	\begin{equation*}
		- \otimes_{\Aa} \Nn\,, Hom^{\bullet}(\Nn,-) : \Kk(\Aa) \longrightarrow \Kk(\Aa)\,,
	\end{equation*}
	which are in fact \emph{triangulated} functors.
	Moreover, given any three $\Aa$-DG-modules $\Mm,\Nn$ and $\Pp$,
	there exist functorial isomorphisms
	\begin{align*}
		(\Mm \otimes_{\Aa} \Nn) \otimes_{\Aa} \Pp &\simeq \Mm \otimes_{\Aa} (\Nn \otimes_{\Aa} \Pp)\,,\\
		Hom^{\bullet}(\Mm,Hom^{\bullet}(\Nn,\Pp)) &\simeq Hom^{\bullet}(M \otimes_{\Aa} \Nn,\Pp)\,\\
		\Hom_{{}_{\Aa}\cat{DGMod}}(\Mm,Hom^{\bullet}(\Nn,\Pp)) &\simeq \Hom_{{}_{\Aa}\cat{DGMod}}(M \otimes_{\Aa} \Nn,\Pp)\,,\\
		\Hom_{\Kk(\Aa)}(\Mm,Hom^{\bullet}(\Nn,\Pp)) &\simeq \Hom_{\Kk(\Aa)}(M \otimes_{\Aa} \Nn,\Pp)\,.
	\end{align*}
	One can sum up all these properties by saying that ${}_{\Aa}\cat{DGMod}$ is a
	\emph{symmetrical monoidal closed category}, where the monoidal structure
	is given by the tensor product $\otimes_{\Aa}$, with $\Aa$ as a neutral element,
	and the internal hom as its right adjoint. For details, see for instance
	\cite[\href{https://stacks.math.columbia.edu/tag/0FQ2}{Tag 0FQ2}]{stacksDGA}.
	
	As it happens for the classical notion of modules,
	whenever we have a homomorphsim of DG-algebras $f:\Aa \to \Bb$,
	we can give $\Bb$ a structure of $\Aa$-DG-module by setting the multiplication
	$a \cdot b := f(a)b$. Hence, we can relate the category of DG-modules over
	$\Aa$ and the ones over $\Bb$ via two functors:
	\begin{itemize}
		\item by \textbf{restriction of scalars}, which consists in considering
		any $\Bb$-DG-module as a left module over $\Aa$, via the induced
		multiplication defined above:
		\begin{equation*}
			f_{*}:{}_{\Bb}\cat{DGMod} \longrightarrow {}_{\Aa}\cat{DGMod}\,,
			\quad \Mm \longmapsto f_{*}\Mm\,;
		\end{equation*}
		
		\item by \textbf{extension of scalars}, given by the assignment
		\begin{equation*}
			f^{*}:{}_{\Aa}\cat{DGMod} \longrightarrow {}_{\Bb}\cat{DGMod}\,,
			\quad \Mm \longmapsto \Bb \otimes_{\Aa} \Mm\,.
		\end{equation*}
	\end{itemize}
	The two functors are adjoint to each other, 
	namely for every $\Aa$-DG-module $\Mm$
	and every $\Bb$-DG-module $\Nn$, 
	there are isomorphisms
	\begin{equation*}
		\Hom_{{}_{\Bb}\cat{DGMod}}(f^{*}\Mm,\Nn) \simeq \Hom_{{}_{\Aa}\cat{DGMod}}(\Mm,f_{*}\Nn)\,,
	\end{equation*}
	and one can check that they preserve homotopy equivalences, 
	thus induce adjoint functors between the homotopy categories.
	If both $\Aa$ and $\Bb$ are graded-commutative, then
	for every $\Mm, \Mm'$ DG-modules over $\Aa$ it holds
	\begin{equation*}
		\Bb \otimes_{\Aa} (\Mm \otimes_{\Aa} \Mm')
		\simeq (\Bb \otimes_{\Aa} \Mm) \otimes_{\Bb} (\Bb \otimes_{\Aa} \Mm')\,.
	\end{equation*}
	
	\begin{df}
		A left DG-module $\Pp$ over $\Aa$ is called \textbf{$\Kk$-projective}
		if one of the two following conditions holds:
		\begin{rmnumerate}
			\item $\Hom_{\Kk(\Aa)}(\Pp,-) = \Hom_{\Dd(\Aa)}(\Pp,-)$;
			\item for every acyclic $\Aa$-DG-module $\Cc$, 
			then the complex $Hom^{\bullet}(\Pp,\Cc)$ is acyclic.
		\end{rmnumerate}
	\end{df}
	
	The conditions i) and ii) are equivalent (see \parencite[Lemma~10.12.2.2]{bernstein-lunts}).
	Given any DG-module $\Mm$ over $\Aa$, there exists a canonical construction,
	called the \textbf{bar construction}, of a $\Kk$-projective module $B(\Mm)$
	together with a quasi-isomorphism $B(\Mm) \to \Mm$. 
	Thanks to this, whenever $\Aa$ is graded-commutative
	one may define the right derived functor
	\begin{equation*}
		RHom^{\bullet}(\Mm,-) : \Dd(\Aa) \longrightarrow \Dd(\Aa)\,,
		\quad \Nn \longmapsto RHom^{\bullet}(\Mm,\Nn) := Hom^{\bullet}(B(\Mm),\Nn)
	\end{equation*}
	and the left derived functor of the tensor product, i.e.
	\begin{equation*}
		- \overset{\cat{L}}{\otimes}_{\Aa} \Mm : \Dd(\Aa) \longrightarrow \Dd(\Aa)\,,
		\quad \Nn \longmapsto \Nn \overset{\cat{L}}{\otimes}_{\Aa} \Mm := \Nn \otimes_{\Aa} B(\Mm)\,,
	\end{equation*}
	thus, for any homomorphism $f:\Aa \to \Bb$ of DG-algebras,
	one can define extension of scalars between derived categories;
	this induces an adjunction
	\begin{equation*}
		\Hom_{\Dd(\Bb)}(f^{*}\Mm,\Nn) \simeq \Hom_{\Dd(\Aa)}(\Mm,f_{*}\Nn)\,.
	\end{equation*}
	
	\begin{thm}
		If $f:\Aa \to \Bb$ is a quasi-isomorphism of DG-algebras,
		then the extension of scalars induces an exact equivalence
		$f^*: \Dd(\Aa) \to \Dd(\Bb)$ 
		of triangulated categories.
		\begin{proof}
			The complete proof can be found in \parencite[Theorem~10.12.5.1]{bernstein-lunts}.
		\end{proof}
	\end{thm}
	
	
	
	
	
	
	
	
	
	
	
	
	
	
	
	
	