
\section{Derived functors}

Let $F : \Aa \to \Bb$ be an \emph{additive} functor 
between abelian categories; it naturally extends
to a unique functor between the associated categories of
cochain complexes $C^{\bullet}(\Aa) \to C^{\bullet}(\Bb)$
and hence, by dividing out by homotopy equivalence,
it induces a well-defined additive functor 
$\Tilde{F}:\cat{K}(\Aa) \to \cat{K}(\Bb)$:
indeed, if $s$ is a homotopy between two
cochain maps $f^{\bullet},g^{\bullet}:A^{\bullet} \to \Tilde{A}^{\bullet}$, 
then the identity
\begin{align*}
    \Tilde{F}(f^{\bullet}) - \Tilde{F}(g^{\bullet})
    &= \Tilde{F}(f^{\bullet} - g^{\bullet}) \\
    &= \Tilde{F}\left(sd_{A}+d_{\Tilde{A}}s\right) 
    = \Tilde{F}(s)\Tilde{F}\left(d_{A}\right) + \Tilde{F}\left(d_{\Tilde{A}}\right)\Tilde{F}(s)
    = \Tilde{F}(s)d_{F(A)} + d_{F(\Tilde{A})}F(s)
\end{align*}
shows that $\Tilde{F}(s)$ is a homotopy between
$\Tilde{F}(f^{\bullet})$ and $\Tilde{F}(g^{\bullet})$.
Moreover, we can check that 
$\Tilde{F}:\cat{K}(\Aa) \to \cat{K}(\Bb)$ is an
exact functor between triangulated categories:
    \begin{itemize}
        \item[(\textbf{EF1})] since the functor $\Tilde{F}$
        sends a complex $A^{\bullet} \in \cat{K}(\Aa)$ to
        the complex
        \begin{equation*}
            \begin{tikzcd}[column sep=large]
                \dots \ar[r]
                & F(A^{i-1}) \ar[r, "F(d_{A}^{i-1})"]
                & F(A^{i}) \ar[r, "F(d_{A}^{i})"]
                & F(A^{i+1}) \ar[r]
                & \dots
            \end{tikzcd}
        \end{equation*}
        it is clear that $\Tilde{F}$ commutes with the shift functor:
        \begin{equation*}
            \big( \Tilde{F}(A^{\bullet})[1] \big)^{n}
            = \Tilde{F}(A^{\bullet})^{n-1} 
            = F(A^{n-1})
            = F\big( (A^{\bullet}[1])^{n} \big)
            = \Tilde{F}(A^{\bullet}[1])^{n}\,;
        \end{equation*}

        \item[(\textbf{EF2})] recall that the image of a direct sum
        via an additive functor is the direct sum of the images:
        in particular, given a map 
        $f^{\bullet}:A^{\bullet} \to \Tilde{A}^{\bullet}$ in $\cat{K}(\Aa)$,
        then $\Tilde{F}\big(\cat{C}(f^{\bullet})\big) 
        \simeq \cat{C}\big(\Tilde{F}(f^{\bullet})\big)$,
        and hence any distinguished triangle of the form
        \begin{equation*}
            \begin{tikzcd}
                A^{\bullet} \ar[r, "f^{\bullet}"]
                & B^{\bullet} \ar[r]
                & \cat{C}(f^{\bullet}) \ar[r]
                & A^{\bullet}{[1]}
            \end{tikzcd}
        \end{equation*}
        is sent to a distinguished triangle
        \begin{equation*}
            \begin{tikzcd}
                \Tilde{F}(A^{\bullet}) \ar[r, "\Tilde{F}(f^{\bullet})"]
                & \Tilde{F}(B^{\bullet}) \ar[r]
                & \cat{C}\big(\Tilde{F}(f^{\bullet})\big) \ar[r]
                & \Tilde{F}(A^{\bullet}){[1]}\,;
            \end{tikzcd}
        \end{equation*}
        by definition of distinguished triangles in the homotopy
        category, this is enough to conclude that $\Tilde{F}$
        preserves all triangles.
    \end{itemize}

Now consider the following question:
given the diagram
\begin{equation*}
    \begin{tikzcd}
        \cat{K}^{*}(\Aa) \ar[r] \ar[d, "\Tilde{F}"'] & \cat{D}^{*}(\Aa) \\
        \cat{K}^{}(\Bb) \ar[r] & \cat{D}^{}(\Bb)
    \end{tikzcd}
\end{equation*}
there exists a triangulated functor $\cat{D}^{*}(\Aa) \to \cat{D}^{}(\Bb)$
making the diagram commute? When $F:\Aa \to \Bb$ is an exact functor
of abelian categories, then the answer is yes: indeed, $\Tilde{F}$
sends any acyclic complex $A^{\bullet}$ to an acyclic complex because,
for every $n \in \Z$, the short exact sequence
\begin{equation*}
    \begin{tikzcd}
        \cat{0} \ar[r]
        & \ker d_{A}^{n} \ar[r]
        & A^{n} \ar[r, "d_{A}^{n}"]
        & \ker d_{A}^{n+1} \ar[r]
        & \cat{0}
    \end{tikzcd}
\end{equation*}
is sent to the sequence
\begin{equation*}
    \begin{tikzcd}
        \cat{0} \ar[r]
        & \ker d_{F(A)}^{n} \ar[r]
        & \Tilde{F}(A^{\bullet})^{n} \ar[r, "d_{F(A)}^{n}"]
        & \ker d_{F(A)}^{n+1} \ar[r]
        & \cat{0}
    \end{tikzcd}
\end{equation*}
which is exact in $\Bb$, and hence $H^{n}(\Tilde{F}(A^{\bullet}) = \cat{0}$.
Thus, the hypothesis of the following lemma are satisfied:

\begin{lemma}\label{induced-derived-functor}
    Let $G:\cat{K}^{*}(\Aa) \to \cat{K}^{}(\Bb)$
    be an exact functor of triangulated categories.
    Then $G$ naturally induces a commutative diagram
        \begin{equation*}
            \begin{tikzcd}
                \cat{K}^{*}(\Aa) \ar[r, "G"] \ar[d]
                & \cat{K}^{}(\Bb) \ar[d] \\
                \cat{D}^{*}(\Aa) \ar[r] & \cat{D}^{}(\Bb)
            \end{tikzcd}
        \end{equation*}
        if one of the two following conditions is satisfied:
        \begin{rmnumerate}
            \item the image of a qis under $G$ is a qis;
            \item the functor $G$ sends acyclic complexes to acyclic complexes.
        \end{rmnumerate}
        \begin{proof}
            If condition (i) holds, then the composition 
            $\cat{K}^{*}(\Aa) \to \cat{K}^{}(\Bb) \to \cat{D}^{}(\Bb)$
            sends each qis to an isomorphism, hence by the universal
            property of the derived category of $\Aa$,
            it factors through $\cat{D}^{*}(\Aa)$ as in the diagram above.

            Assume now that condition (ii) holds true and consider
            a qis $f^{\bullet} : A^{\bullet} \to B^{\bullet}$;
            the cone $\cat{C}(f^{\bullet})$ is acyclic,
            so by hypothesis $\cat{C}\big(G(f^{\bullet})\big)$ is acyclic too,
            which implies that $G(f^{\bullet})$ is a quasi-isomorphism.
            Thus, $G$ sends quasi-isomorphisms to quasi-isomorphisms,
            and we conclude by part (i).
        \end{proof}
\end{lemma}

\begin{rmk}
    In the Lemma above, the functor $G$ need not 
    come from a functor between the abelian categories!
\end{rmk}

If $F$ is not exact, the image of an acyclic complex in $A$, 
i.e. one that becomes trivial in $\cat{D}(\Aa)$, is not, 
in general, acyclic. Thus, the naive extension of $F$ 
to a functor between the derived categories 
$\cat{D}(\Aa) \to \cat{D}(\Bb)$ does not make sense
and in this case a more complicated construction is needed 
in order to induce a natural functor between the derived categories. 
The new functor, called the \emph{derived functor}, 
will not produce a commutative diagram 
as in \hyperref[induced-derived-functor]{Lemma~\ref*{induced-derived-functor}}, 
but it has the advantage to encode more information 
even when applied to an object in the abelian category. 
Roughly, it explains why the original functor fails to be exact.
In order to ensure existence of the derived functor, 
we will always have to assume some kind of exactness:
for a \textbf{left exact} functor $F : \Aa \to \Bb$,
one constructs the \textbf{right derived functor}
\begin{equation*}
    RF : \cat{D}^{+}(\Aa) \longrightarrow \cat{D}^{}(\Bb)\,,
\end{equation*}
and for a \textbf{right exact} functor $G:\Aa \longrightarrow \Bb$
one constructs the \textbf{left derived} functor
\begin{equation*}
    LG : \cat{D}^{-}(\Aa) \longrightarrow \cat{D}^{}(\Bb)\,.
\end{equation*}
They can be defined by a universal property:

\begin{df}
    Let $F:\cat{K}^{+}(\Aa) \to \cat{K}^{}(\Bb)$ be a triangulated functor.
    A \textbf{right derived functor} for $F$ is a triangulated functor
    \begin{equation*}
        RF : \cat{D}^{+}(\Aa) \longrightarrow \cat{D}^{}(\Bb)\,,
    \end{equation*}
    together with a morphism of functors
        $\epsilon : Q \circ F \to RF \circ Q$ such that,
        for any pair $(\Phi,\eta)$ of a triangulated functor
        $\Phi:\cat{D}^{+}(\Aa) \to \cat{D}^{+}{(\Bb)}$ and natural
        transformation $\eta : Q \circ F \to \Phi \circ Q$,
        there exists a unique morphism of functors
        $\alpha : RF \to \Phi$ such that
        $\eta = \alpha_{Q(-)} \circ \epsilon$;
        more explicitly, for every $A^{\bullet} \in \cat{K}^{+}(\Aa)$,
        in $\cat{D}^{}(\Bb)$ we have the following commutative diagram:
        \begin{equation*}
            \begin{tikzcd}
                Q \circ F(A^{\bullet}) \ar[rr, "\eta_{A^{\bullet}}"]
                \ar[dr, "\epsilon_{A^{\bullet}}"']
                & & \Phi \circ Q(A^{\bullet}) \\
                & RF \circ Q (A^{\bullet}) \ar[ur, "\alpha_{Q(A^{\bullet})}"'] & \,.
            \end{tikzcd}
        \end{equation*}

    Similarly, if $G:\cat{K}^{-}(\Aa) \to \cat{K}^{-}(\Bb)$ is a triangulated functor,
    a \textbf{left derived functor} for $G$ is a triangulated functor
    \begin{equation*}
        LG : \cat{D}^{-}(\Aa) \longrightarrow \cat{D}^{-}(\Bb)\,,
    \end{equation*}
    together with a morphism of functors
        $\epsilon : LG \circ Q \to Q \circ F$ such that,
        for any pair $(\Phi,\eta)$ of a triangulated functor
        $\Phi:\cat{D}^{-}(\Aa) \to \cat{D}^{-}{(\Bb)}$ and natural
        transformation $\eta : \Phi \circ Q \to Q \circ G$,
        there exists a unique morphism of functors
        $\beta : \Phi \to LG$ such that
        $\eta = \epsilon \circ \beta_{Q(-)}$;
        more explicitly, for every $A^{\bullet} \in \cat{K}^{-}(\Aa)$,
        in $\cat{D}^{-}(\Bb)$ we have the following commutative diagram:
        \begin{equation*}
            \begin{tikzcd}
                \Phi \circ Q(A^{\bullet}) \ar[rr, "\eta_{A^{\bullet}}"]
                \ar[dr, "\beta_{Q(A^{\bullet})}"']
                & & Q \circ G(A^{\bullet}) \\
                & LG \circ Q (A^{\bullet}) \ar[ur, "\epsilon_{A^{\bullet}}"'] & \,.
            \end{tikzcd}
        \end{equation*}
\end{df}

Of course, when $LG$ or $RF$ exists, 
it is unique up to unique isomorphism,
for it is defined by a universal property.
A sufficient condition for derived functors
to exist is the existence of enough injectives
and projectives.

\begin{prop}\label{derived-functor}
    If $F:\cat{K}^{+}(\Aa) \to \cat{K}^{}(\Bb)$ is
    a triangulated functor and $\Aa$ has enough
    injective objects, then the right derived functor
    $RF$ exists.

    Similarly, if $\Aa$ has enough projectives
    and $G:\cat{K}^{-}(\Aa) \to \cat{K}^{}(\Bb)$
    is triangulated, then $LG$ exists.
    \begin{proof}
        We do the case of the right derived functor:
        if $\Ii_{\Aa}$ is the full subcategory of injective objects of $\Aa$, 
        then consider $\psi:\cat{D}^{+}(\Aa) \to \cat{K}^{+}(\Ii_{\Aa})$
        a quasi-inverse of the equivalence stated in 
        \hyperref[inj-equivalence]{Theorem~\ref*{inj-equivalence}}
        and define $RF$ to be the composition
        $RF := Q \circ F\vert_{\cat{K}^{+}(\Ii_{\Aa})} \circ \psi$,
        i.e.
        \begin{equation*}
            \begin{tikzcd}
                \cat{K}^{+}(\Ii_{\Aa}) \ar[r, hook] \ar[d, "Q"]
                & \cat{K}^{+}(\Aa) \ar[r, "F"]
                & \cat{K}^{}(\Bb) \ar[d, "Q"] \\
                \cat{D}^{+}(\Aa) \ar[u, "\psi", bend left] 
                \ar[rr, dashed, "RF"] & & \cat{D}^{}(\Bb)\,.
            \end{tikzcd}
        \end{equation*}

        To define $\epsilon: Q \circ F \to RF \circ Q$
        on objects, let $A^{\bullet}$ be a complex bounded below
        and consider a qis $A^{\bullet} \to I^{\bullet}$
        in $\cat{K}^{+}(\Aa)$, 
        where $I^{\bullet} \in \cat{K}^{+}(\Ii_{\Aa})$,
        which exists because of the equivalence 
        $\cat{D}^{+}(\Aa) \simeq \cat{K}^{+}(\Ii_{\Aa})$.
        By applying $Q$ to the induced morphism 
        $F(A^{\bullet}) \to F(I^{\bullet})$, one gets a morphism
        \begin{equation*}
            \epsilon_{A^{\bullet}} :
            Q \circ F(A^{\bullet}) 
            \longrightarrow Q \circ F(I^{\bullet}) 
            \simeq Q \circ F \circ \psi \big( Q(I^{\bullet}) \big)
            \simeq RF \circ Q (A^{\bullet})\,,
        \end{equation*}
        where we used that $Q(I^{\bullet}) \simeq Q(A^{\bullet})$.
        One can check that $\epsilon$ is natural because,
        given any map $f^{\bullet} : A^{\bullet} \to B^{\bullet}$
        and quasi-isomorphisms $A^{\bullet} \to I^{\bullet}$
        and $B^{\bullet} \to J^{\bullet}$, there is a unique
        (up to homotopy) induced map $I^{\bullet} \to J^{\bullet}$
        in $\cat{K}^{+}(\Ii_{\Aa})$, 
        making the following diagram in $\cat{K}^{+}(\Aa)$ commute:
        \begin{equation*}
            \begin{tikzcd}
                A^{\bullet} \ar[r, "\textrm{qis}"] \ar[d, "f^{\bullet}"']
                & I^{\bullet} \ar[d, dashed, "\exists !"] \\
                B^{\bullet} \ar[r, "\textrm{qis}"] & J^{\bullet}\,.
            \end{tikzcd}
        \end{equation*}

        Consider now any $(\Phi,\eta)$ as in the definition of derived functor,
        and we show that $\eta$ factors through $\epsilon$:
        as above, let $A^{\bullet} \simeq I^{\bullet}$, where $I^{\bullet}$.
    \end{proof}
\end{prop}

The above result actually holds in a more general setting:
given a functor $F$ between homotopy categories,
the right and the left derived functors exist whenever
we find a class of objects adapted to the functor.

\begin{df}
    Given a triangulated functor $F:\cat{K}^{*}(\Aa) \to \cat{K}^{}(\Bb)$,
    a triangulated subcategory $\Kk_{F} \subset \cat{K}^{*}(\Aa)$
    is \textbf{$F$-adapted} if it satisfies the following two conditions:
    \begin{rmnumerate}
        \item if $A^{\bullet} \in \Kk_{F}$ is acyclic,
        then $F(A^{\bullet})$ is acyclic;

        \item any $A^{\bullet} \in \cat{K}^{*}(\Aa)$ is quasi-isomophic
        to a complex in $\Kk_{F}$.
    \end{rmnumerate}
\end{df}

We can define the notion of adapted class already
on the level of the abelian category $\Aa$.

\begin{df}
    Let $F:\Aa \to \Bb$ be a left (resp. right) exact functor between abelian categories.
    A class of objects $\Ii_{F} \subset \Aa$ is \textbf{$F$-adapted}
    if the following conditions hold true:
    \begin{rmnumerate}
        \item the class $\Ii_{F}$ is closed under finite sums, 
        i.e. given $A, B \in \Ii_{F}$, then $A \oplus B \in \Ii_{F}$;
        \item if $A^{\bullet} \in \cat{K}^{+}(\Ii_{F})$ 
        (resp. $\cat{K}^{-}(\Ii_{F})$) is acyclic,
        then $F(A^{\bullet})$ is acyclic too;
        \item any object of $\Aa$ can be embedded into an
        object of $\Ii_{F}$.
    \end{rmnumerate}
\end{df}

One can prove that, whenever $\Ii_{F}$ is an adapted class for a
left exact functor $F$, then the localization 
of $\cat{K}^{+}(\Ii_{F})$ by the class of quasi-isomorphisms
is equivalent to the derived category $\cat{D}^{+}(\Aa)$;
for a complete description of this fact, 
check \parencite[Proposition~III.6.4]{gelfand}:
the idea is that $F$ sends qis of complexes in $\Ii_{F}$
to qis by condition (ii), while condition (iii) ensures
that we can identify any object $A \in \Aa$ with an
object of $\cat{K}^{+}(\Ii_{F})$, as it happens
with injetive resolutions.
In fact, this is not by chance: whenever $\Aa$ has enough
injectives, the class $\Ii_{\Aa}$ spanned by injective objects
is $F$-adapted for \emph{any} left exact functors, indeed
we know that $I \oplus J$ is injective whenever $I, J \in \Ii$;
left exactness guarantees (ii) and (iii) is the definition of
injective resolution.

Whenever $\Ii_{F} \subset \Aa$ is an $F$-adapted class 
with respect to a left exact functor $F:\Aa \to \Bb$,
one can prove that the full subcategory
$\cat{K}^{+}(\Ii_{F})$ is an adapted
with respect to the induced homotopy functor
$\Tilde{F}:\cat{K}^{+}(\Aa) \to \cat{K}(\Bb)$,
thus there exists the right derived functor,
that we will write as
\begin{equation*}
    RF : \cat{D}^{+}(\Aa) \longrightarrow \cat{D}(\Bb)\,.
\end{equation*}
An analogous statement is true for 
right exact functors and the class of projective objects;
for a complete treatment, see \parencite[Theorem~III.6.8]{gelfand}.
By assuming these general constructions,
we can deduce the following

\begin{cor}
    Let $F:\Aa \to \Bb$ be an additive functor between abelian categories.
    \begin{itemize}
        \item If $\Aa$ has enough injective objects and $F$ is left exact,
        then the right derived functor $RF: \cat{D}^{+}(\Aa) \to \cat{D}(\Bb)$ exists.
        \item If $\Aa$ has enough projective objects and $F$ is right exact,
        then the left derived functor $LF: \cat{D}^{-}(\Aa) \to \cat{D}(\Bb)$ exists.
    \end{itemize}
\end{cor}

\begin{df}
    Whenever the right derived functor $RF$ of a left exact functor $F:\Aa \to \Bb$ exists,
    for every $i \in \Z$ we define
    \begin{equation*}
        R^{i}F : \cat{D}^{+}(\Aa) \longrightarrow \Bb\,,
        \qquad A^{\bullet} \longmapsto H^{i}\big(RF(A^{\bullet})\big)\,.
    \end{equation*}
    By precomposing with the canonical embedding $\Aa \subset \cat{D}^{+}(\Aa)$,
    we can define, for every $i \in \Z$, the \textbf{$i$-th higher derived functor}
    $R^{i}F : \Aa \longrightarrow \Bb$.
\end{df}

The proof of \hyperref[derived-functor]{Proposition~\ref*{derived-functor}}
shows a way to compute the higher derived functors of $F$: indeed,
whenever $\Aa$ has enough injectives, for any $A \in \Aa$
we can pick an injective resolution $\cat{0} \to A \to I^{\bullet}$ and,
by the the natural isomorphism $\epsilon : Q \circ F \simeq RF \circ Q$,
we have 
\begin{equation*}
    RF^{i}(A) = H^{i}\big( RF(A[0]) \big) 
    %\simeq H^{i}\big(Q(I^{\bullet})\big) 
    \simeq H^{i}\big(F(I^{\bullet})\big)\,.
\end{equation*}
It follows that, if $i<0$, the higher derived functors are $RF^{i}=0$
and $R^{0}F(A) \simeq F(A)$ by left exactness:
\begin{equation*}
    R^{0}F(A) = H^{0}\big(F(I^{\bullet})\big) 
    = \ker\big(F(I^{0}) \to F(I^{1})\big) \simeq F(A)
\end{equation*}

\begin{cor}
    Let $F : \Aa \to \Bb$ be a left exact functor,
    where $\Aa$ has enough injectives. Then every short exact sequence in $\Aa$
    \begin{equation*}
        \begin{tikzcd}
            \cat{0} \ar[r]
            & A \ar[r] 
            & B \ar[r]
            & C \ar[r]
            & \cat{0}
        \end{tikzcd}
    \end{equation*}
    gives rise in $\Bb$ to the long exact sequence
    \begin{equation*}
        \begin{tikzcd}[column sep=small]
            \cat{0} \ar[r]
            & F(A) \ar[r] 
            & F(B) \ar[r]
            & F(C) \ar[r]
            & R^{1}F(A) \ar[r]
            & R^{1}F(B) \ar[r]
            & \dots \\
            \dots \ar[r]
            & R^{i}F(A) \ar[r] 
            & R^{i}F(B) \ar[r]
            & R^{i}F(C) \ar[r]
            & R^{i+1}F(A) \ar[r]
            & R^{i+1}F(B) \ar[r]
            & \dots
        \end{tikzcd}
    \end{equation*}
    \begin{proof}
        According to \hyperref[SES-TRI]{Exercise~\ref*{SES-TRI}},
        any short exact sequence in $\Aa$ gives a distinguished triangle
        $A \to B \to C \to A[1]$ in $\cat{D}^{+}(\Aa)$.
        Since $RF$ is triangulated, the triangle 
        $RF(A) \to RF(B) \to RF(C) \to RF(A)[1]$ 
        is distinguished in $\cat{D}^{+}(\Bb)$,
        thus it induces a LECS by \hyperref[derived-LECS]{Exercise~\ref*{derived-LECS}},
        which is the desired long exact sequence in $\Bb$.
    \end{proof}
\end{cor}

\begin{exercise}
    Let $F$ be a left exact functor and $\Ii_{F}$ an $F$-adapted class in $\Aa$.
    We say that $A \in \Aa$ is \textbf{$F$-acyclic} if $R^{i}F(A) \simeq \cat{0}$
    for all $i \ne 0$. Show that we obtain an $F$-adapted class by
    enlarging $\Ii_{F}$ by all $F$-acyclic objects.
    \begin{proof}
        It is enough to check the three conditions in the definition of 
        $F$-adapted class:
        \begin{rmnumerate}
            \item by additivity of the functors $R^{i}F$, it holds
            \begin{equation*}
                R^{i}F(A \oplus B) \simeq R^{i}F(A) \oplus R^{i}F(B)\,,
            \end{equation*}
            thus the sum of finitely many $F$-acyclic objects is still $F$-acyclic.
            Nevertheless, the sum $I \oplus A$ of some $I \in \Ii_{F}$ 
            and $A$ an $F$-acyclic object needs not be neither in $\Ii_{F}$,
            nor $F$-acyclic. Thus, when we say ``enlarge'',
            we mean we need to consider also direct sums of these objects;

            \item consider 

            \item any objects of $\Aa$ can be embedded into some object of $\Ii_{F}$,
            so it still remains true if we enlarge the class.
        \end{rmnumerate}
    \end{proof}
\end{exercise}

\begin{ex}[\textbf{Kernel}]
    Let $\Aa$ be an abelian category.
    Let $\Aa^{\{\ast \to \ast\}}$ be the category whose
    objects are maps $f:A \to A'$ in $\Aa$;
    a morphism $\Phi$ from $f:A \to A'$ to $g:B \to B'$
    is a pair $(\phi,\phi')$ of arrows $\phi:A \to B$
    and $\phi':A' \to B'$ such that the following square commutes:
    \begin{equation*}
        \begin{tikzcd}
            A \ar[r, "\phi"] \ar[d,"f"'] & B \ar[d, "g"]\\
            A' \ar[r, "\phi'"] & B'\,.
        \end{tikzcd}
    \end{equation*}
    Given any two morphisms $\Phi = (\phi,\phi') : f \to g$
    and $\Psi = (\psi,\psi'):g \to h$, set $\Psi \circ \Phi$
    to be the componentwise composition, i.e.
    $$\Psi \circ \Phi := (\psi \circ \phi,\psi' \circ \phi')\,.$$
    One can check this composition is associative,
    making $\Aa^{\{\ast \to \ast\}}$ into a category.
    Moreover, $\Aa^{\{\ast \to \ast\}}$ is abelian:
    \begin{itemize}
        \item[(\textbf{A1})] if we write $\cat{0}$
        for the zero object of $\Aa$, then the zero morphism
        $0: \cat{0} \to \cat{0}$ is the zero object of 
        $\Aa^{\{\ast \to \ast\}}$;

        \item[(\textbf{A2})] the existence of finite products
        and coproducts in $\Aa^{\{\ast \to \ast\}}$ 
        follows by their existence in $\Aa$: 
        indeed, given $f:A \to A'$ and $g:B \to B'$,
        their product is given by $f \oplus g : A \oplus B \to A' \oplus B'$
        because, for any object $t:C \to C'$ and 
        any pair of maps $t \to f, t \to g$, we have the factorization
        \begin{equation*}
            \begin{tikzcd}
                   & C \arrow[ld] \arrow[rrd] \arrow[dd, "t"', near end] \arrow[rd, "\exists !"', dashed] &                                                                  &                   \\
A \arrow[dd, "f"'] &                                                                            & A \oplus B \arrow[ll, crossing over] \arrow[r]  & B \arrow[dd, "g"] \\
                   & C' \arrow[ld] \arrow[rrd] \arrow[rd, "\exists !"', dashed]                 &                                                                  &                   \\
A'                 &                                                                            & A' \oplus B' \arrow[ll] \arrow[r] \arrow[from=uu, "\exists !"', dashed, crossing over]                               & B'         \,,          
\end{tikzcd}
        \end{equation*}
        which can be rewritten as the following commutative diagram
        in $\Aa^{\{\ast \to \ast\}}$
        \begin{equation*}
            \begin{tikzcd}
                & t \ar[d, dashed, "\exists !"'] \ar[dr] \ar[dl] & \\
                f & f \oplus g \ar[l] \ar[r] & g\,.
            \end{tikzcd}
        \end{equation*}
        The coproduct of $f$ and $g$ is given again by $f \otimes g$
        because $A \times B \simeq A \amalg B$ in $\Aa$, which is indeed the direct sum;

        \item[(\textbf{A3})] we already know that $\Aa$ has kernels, 
        and hence for any $\Phi:f \to g$ in 
        $\Aa^{\{\ast \to \ast\}}$ we have
        \begin{equation*}
            \begin{tikzcd}
                \ker \phi \ar[r, "i", hook] \ar[d, "\exists !"', "k", dashed]
                & A \ar[r, "\phi"] \ar[d, "f"] 
                & B \ar[d, "g"] \\
                \ker \phi' \ar[r, hook]
                & A' \ar[r, "\phi'"]
                & B'\,,
            \end{tikzcd}
        \end{equation*}
        where the vertical arrow $k$ is the unique map
        described by the universal property of $\ker \phi'$,
        which exists by $\phi' \circ f \circ i 
        = g \circ \phi \circ i = 0$.
        We claim that $k = \ker \Phi$: given any $t:C \to C'$
        and $\Psi:t \to f$ such that $\Phi \circ \Psi = 0$,
        then in $\Aa$ we have the commutative diagram
        \begin{equation*}
            \begin{tikzcd}
C \arrow[rrd, shorten=1mm] \arrow[rrrd, "0", shorten=3mm] \arrow[rd, "\exists !"', dashed] \arrow[dd, "t"'] &                                            &                                       &                   \\
                                                                                  & \ker \phi \arrow[r, hook]  & A \arrow[r, "\phi"']  & B \arrow[dd, "g"] \\
C' \arrow[rrd, shorten=1mm] \arrow[rrrd, "0", shorten=3mm] \arrow[rd, "\exists !"', dashed]                 &                                            &                                       &                   \\
                                                                                  & \ker \phi' \arrow[from=uu, "k"', crossing over] \arrow[r, hook]                 & A' \arrow[from=uu, "f"', crossing over]\arrow[r, "\phi'"']                & B'     \,,          
\end{tikzcd}
        \end{equation*}
        which can be written in $\Aa^{\{\ast \to \ast\}}$ as the commutative diagram
        \begin{equation*}
            \begin{tikzcd}
                t \ar[rd, "\Psi"'] \ar[rrd, "0"] \ar[d, dashed] & & \\
                k \ar[r] & f \ar[r, "\Phi"'] & g\,,
            \end{tikzcd}
        \end{equation*}
        which proves $k = \ker \Phi$. 
        Similarly, one shows that $\Coker \Phi$ is given by the induced map
        on cokernels;
        
        \item[(\textbf{A4})] consider $\Phi = (\phi,\phi') : f \to g$ 
        in $\Aa^{\{\ast \to \ast\}}$. Since axiom \hyperref[A4]{(\textbf{A4})} 
        holds in $\Aa$, then $\operatorname{coim} \phi \simeq \operatorname{im} \phi$
        and $\operatorname{coim} \phi' \simeq \operatorname{im} \phi'$, which means
        that $\operatorname{coim} \Phi \simeq \operatorname{im} \Phi$ beacuse
        isomorphisms in $\Aa^{\{\ast \to \ast\}}$ are given by pairs of isomorphisms.
    \end{itemize}

    We now define the functor
    \begin{equation*}
        \ker: \Aa^{\{\ast \to \ast\}} \longrightarrow \Aa\,,
        \qquad \big( f \to g \big) 
        \longmapsto \big( \ker f \to \ker g \big)\,.
    \end{equation*}
    This is an additive functor between abelian categories, 
    and we claim it is left exact: given a short exact sequence 
    $0 \to f \to g \to h \to 0$ in $\Aa^{\{\ast \to \ast\}}$,
    in $\Aa$ we get a commutative diagram of the form
    \begin{equation*}
        \begin{tikzcd}
            %\cat{0} \ar[r]
            & \ker f \ar[d, hook] \ar[r, dashed, "i"]
            & \ker g \ar[d, hook] \ar[r, dashed, "j"]
            & \ker h \ar[d, hook] 
            & \\
            \cat{0} \ar[r]
            & A \ar[d, "f"] \ar[r]
            & B \ar[d, "g"] \ar[r]
            & C \ar[d, "h"] \ar[r]
            & \cat{0} \\
            \cat{0} \ar[r]
            & A' \ar[r]
            & B' \ar[r]
            & C' \ar[r]
            & \cat{0} \,.
        \end{tikzcd}
    \end{equation*}
    Notice that $i:\ker f \to \ker g$ is a monomorphism, 
    for given $t:W \to \ker f$ such that $i \circ t = 0$,
    then the composition $W \to A \to B$ is zero again,
    so that $t$ factors through $\ker\big(A \to B) \simeq \cat{0}$.
    Thus $\ker i \simeq \cat{0}$ and this shows $\ker$ is exact on the left.
    To show exactness in $\ker g$, noticce that 
    the exactness of the middle row implies $j \circ i = 0$,
    so that we have an induced map $\operatorname{im} i \to \ker j$.
    Conversely, we can find a map $\ker j \to \operatorname{im} i$
    which will be the inverse of the above 
    $\ker j \to \operatorname{im} i$
    because of the uniqueness guaranteed by the universal property.
    To get the desired map, one sees that the composition
    \begin{equation*}
        \ker j \to \ker g \to B \xhookrightarrow{\psi} C
    \end{equation*}
    is the zero map, and hence we get a factorization through
    $\ker j \to \ker \phi$; by exactness, 
    $\operatorname{im} \phi \simeq \ker \psi$, and by composing
    $\ker j \to \operatorname{im} \phi \to \operatorname{im}\phi'$
    we get the zero morphism (because it commutes with $\ker j \to \ker g \to B \to B'$), and hence we can lift $\ker j \to \operatorname{im} i$.
    \begin{equation*}
\begin{tikzcd}
                                                         &   \operatorname{im} i                                                               & \ker j \arrow[rd, hook] \arrow[rrrdd, "0", bend left] \arrow[ddd, dashed, crossing over] \ar[l, "\exists !"', dashed] &                                      &  &                   \\
\ker f \arrow[d, hook] \arrow[rrr, "i"] \ar[ru, dashed]                 &                                                                 &                                                                           & \ker g \arrow[d, hook]               &  &                   \\
A \arrow[rrr, "\phi"] \arrow[rd, dashed] \arrow[dd, "f"'] &                                                                 &                                                                           & B \arrow[rr, "\psi"] \arrow[dd, "g"] &  & C \arrow[dd, "h"] \\
                                                         & \operatorname{im}\phi \ar[from=uuu, crossing over] \arrow[r, no head, equals] & \ker \psi \arrow[ru, hook] \arrow[from=uuu, dashed, crossing over]                                              &                                      &  &                   \\
A' \arrow[rrr, "\phi'"]  \arrow[rd, dashed]               &                                                                 &                                                                           & B' \arrow[rr]                        &  & C'                \\
                                                         & \operatorname{im}\phi' \arrow[rru, hook]   \arrow[from=uu, crossing over]                     &                                                                           &                                      &  &        \,.          
\end{tikzcd}
    \end{equation*}

    Now that we know $\ker$ is a left exact functor,
    assume $\Aa$ has enough injective objects; it follows that
    $\Aa^{\{\ast \to \ast\}}$ has enough injectives too:
    indeed, a map $I \to J$ between injective objects in $\Aa$
    is injective as an object of $\Aa^{\{\ast \to \ast\}}$,
    and by the dual version of \hyperref[res-htp]{Proposition~\ref*{res-htp}}
    we can always find an injective resolution
        \begin{equation}\label{ladder-ker}
            \begin{tikzcd}
                \cat{0} \ar[r]
                & A \ar[r] \ar[d, "f"]
                & I^{0} \ar[r] \ar[d, "j^{0}"] 
                & I^{1} \ar[r] \ar[d, "j^{1}"]
                & I^{2} \ar[r] \ar[d, "j^{2}"]
                & \dots \\
                \cat{0} \ar[r]
                & A' \ar[r] 
                & J^{0} \ar[r]  
                & J^{1} \ar[r] 
                & J^{2} \ar[r] 
                & \dots
            \end{tikzcd}
        \end{equation}
    for any object $f \in \Aa^{\{\ast \to \ast\}}$.
    This means that we can compute $\ker$'s higher right derived functors:
    given any $f : A \to A'$, 
    we already know that $R^{0}\ker f \simeq \ker f$,
    so now consider an injective resolution $f \to j^{\bullet}$
    as in \eqref{ladder-ker}.\todo{Finish this exercise.}
\end{ex}
