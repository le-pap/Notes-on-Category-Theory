
\section{The derived category of an abelian category}

We begin by stating the existence of the derived category 
as a theorem, and explain the technical features, 
necessary for any calculation, later on.
In the sequel, we will mostly be interested in 
the derived category of the abelian category of 
(coherent) sheaves or of modules over a ring. 

Let $\Aa$ be an abelian category.
Recall that in the chapter 
of \hyperref[AbelianCategories]{Abelian Categories},
we have defined the category $C^{\bullet}(\Aa)$
of cochain complexes and cohomology functors $H^{n}$
over it.

\begin{df}
    A morphism of complexes
    $f^{\bullet} : A^{\bullet} \to B^{\bullet}$
    is a \textbf{quasi-isomorphism} (shortened \textbf{qis})
    if, for all $n \in \NN$, the induced map
    \begin{equation*}
        H^{n}(f^{\bullet}) : H^{n}(A^{\bullet}) \xrightarrow[]{\sim} H^{n}(B^{\bullet})
    \end{equation*}
    is an isomorphism.
\end{df}

The central idea for the definition of the derived category is this: 
quasi-isomorphic complexes should become isomorphic objects 
in the derived category. 
We shall begin our discussion with the following existence theorem.

\begin{thmdef}
    Given an abelian category $\Aa$,
    there exists a category $\cat{D}(\Aa)$,
    called the \textbf{derived category} of $\Aa$,
    and a functor
    \begin{equation*}
        Q : C^{\bullet}(\Aa) \longrightarrow \cat{D}(\Aa)
    \end{equation*}
    that satisfy the following two conditions:
    \begin{rmnumerate}
        \item if $f^{\bullet}$ is a qis, 
        then $Q(f^{\bullet})$ is an isomorphism;

        \item \textbf{universal property}:
        if a functor $F:C^{\bullet}(\Aa) \to \Dd$
        satisfies \emph{(i)}, then it factorizes uniquely through $Q$,
        i.e. there exists a unique (up to isomorphism) functor 
        $G:\cat{D}(\Aa) \to \Dd$ such that $F \simeq GQ$:
        \begin{equation*}
            \begin{tikzcd}
                C^{\bullet}(\Aa) \ar[rr, "Q"] \ar[dr, "F"']
                && \cat{D}(\Aa) \ar[dl, "G", "\exists !"',dashed]\\
                &\Dd & \,.
            \end{tikzcd}
        \end{equation*}
    \end{rmnumerate}
    \begin{proof}[Construction]       
In order to be able to work with the derived category, 
we have to understand which objects become isomorphic under 
$Q : C^{\bullet}(A) \to \cat{D}(A)$ and, more complicated, 
how to represent morphisms in the derived category. 
Explaining this, will at the same time 
provide a proof the theorem. 
Recall that $\cat{K}(\Aa)$ denotes the homotopy category of complexes.

        First, we set the objects of the derived category to be
        cochain complexes, so
            \begin{equation*}
                \textrm{Obj}(\cat{D}(\Aa)) := \textrm{Obj}(\cat{K}(\Aa)) 
                = \textrm{Obj}(C^{\bullet}(\Aa))\,.
            \end{equation*}

        Now we describe how morphisms should behave.
        Since the derived category is built in such
        a way that quasi-isomorphisms become isomorphisms,
        if $C^{\bullet} \to A^{\bullet}$ is an isomorphism,
        then any morphism of complexes $C^{\bullet} \to B^{\bullet}$
        will have to count as a morphism $A^{\bullet} \to B^{\bullet}$
        in $\cat{D}(\Aa)$. Thus, given two complexes $A^{\bullet}, B^{\bullet}$,
        a representative of a morphism 
        $A^{\bullet} \to B^{\bullet}$ in $\cat{D}(\Aa)$
        is given by a diagram
        \begin{equation*}
            \begin{tikzcd}
                & C^{\bullet} \ar[dl, "\textrm{qis}"] \ar[dr] & \\
                A^{\bullet} & & B^{\bullet}\,,
            \end{tikzcd}
        \end{equation*}
        where $C^{\bullet} \to A^{\bullet}$ is a quasi-isomorphism;
        we will call such a diagram a \textbf{roof} 
        over $A^{\bullet}$ and $B^{\bullet}$.
        Two roofs over $A^{\bullet}$ and $B^{\bullet}$ are
        said to be \textbf{equivalent} if they are 
        dominated by a third roof in $\cat{K}(\Aa)$, 
        i.e. there is a diagram
        \begin{equation*}
            \begin{tikzcd}
                & & C^{\bullet} \arrow[ld, "\textrm{qis}"'] \arrow[rd] 
                \arrow[lldd, "\textrm{qis}"', dashed, bend right=50, shift right] 
                & & \\
                & C_1^{\bullet} \arrow[ld, "\textrm{qis}"'] \arrow[rrrd] 
                & & C_2^{\bullet} \arrow[llld, "\textrm{qis}", crossing over] \arrow[rd] 
                & \\
                A^{\bullet} 
                &  &  &  & B^{\bullet}
            \end{tikzcd}
        \end{equation*}
        which commutes in $\cat{K}(\Aa)$, i.e. compositions are homotopy equivalent.
        This property defines, in fact, an \emph{equivalence relation} on roofs:
        \begin{itemize}
            \item \textbf{reflextivity}: a roof is equivalent to itself because
            we have the diagram
            \begin{equation*}
            \begin{tikzcd}
                & & C^{\bullet} \arrow[ld, equals] \arrow[rd, equals] 
                & & \\
                & C^{\bullet} \arrow[ld, "\textrm{qis}"'] \arrow[rrrd] 
                & & C^{\bullet} \arrow[llld, "\textrm{qis}", crossing over] \arrow[rd] 
                & \\
                A^{\bullet} 
                &  &  &  & B^{\bullet}\,;
            \end{tikzcd}
            \end{equation*}

            \item \textbf{symmetry}: given an equivalence of roofs
            \begin{equation*}
            \begin{tikzcd}
                & & C^{\bullet} \arrow[ld, "\textrm{qis}"'] \arrow[rd] 
                \arrow[lldd, "\textrm{qis}"', dashed, bend right=50, shift right] 
                & & \\
                & C_1^{\bullet} \arrow[ld, "\textrm{qis}"'] \arrow[rrrd] 
                & & C_2^{\bullet} \arrow[llld, "\textrm{qis}", crossing over] \arrow[rd] 
                & \\
                A^{\bullet} 
                &  &  &  & B^{\bullet}\,,
            \end{tikzcd}
            \end{equation*}
            we notice that the map $C^{\bullet} \to C^{\bullet}_{2}$ is, in fact, a qis:
            indeed, by passing to cohomology objects, in $\Aa$ we have
            a commutative diagram
            \begin{equation*}
                \begin{tikzcd}
                    H^*(C^{\bullet}) \ar[rr, "\simeq"] \ar[dr] 
                    & & H^*(A^{\bullet}) \\
                    & H^*(C_{2}^{\bullet}) \ar[ur, "\simeq"] & \,,
                \end{tikzcd}
            \end{equation*}
            implying that $H^*(C^{\bullet}) \simeq H^*(C_{2}^{\bullet})$.
            Thus, the ``equivalence'' diagram is symmetric;

            \item \textbf{transitivity}: given two diagrams
            \begin{equation*}
            \begin{tikzcd}[column sep=small]
                & & C^{\bullet} \arrow[ld, "\textrm{qis}"'] \arrow[rd] 
                & & &
                & & D^{\bullet} \arrow[ld, "\textrm{qis}"'] \arrow[rd] 
                & &\\
                & C_1^{\bullet} \arrow[ld, "\textrm{qis}"'] \arrow[rrrd] 
                & & C_2^{\bullet} \arrow[llld, "\textrm{qis}", crossing over] \arrow[rd] 
                & &
                & C_2^{\bullet} \arrow[ld, "\textrm{qis}"'] \arrow[rrrd] 
                & & C_3^{\bullet} \arrow[llld, "\textrm{qis}", crossing over] \arrow[rd] 
                &\\
                A^{\bullet} 
                &  &  &  & B^{\bullet} \,, &
                A^{\bullet} 
                &  &  &  & B^{\bullet}\,,
            \end{tikzcd}
            \end{equation*}
            by \hyperref[roof-comp]{Proposition~\ref*{roof-comp}}
            we can build a diagram
            \begin{equation*}
            \begin{tikzcd}
                && C_{0}^{\bullet} \ar[dl, "\textrm{qis}"'] \ar[dr] && \\
                & C^{\bullet} \ar[dl, "\textrm{qis}"'] \ar[dr]
                && D^{\bullet} \ar[dl, "\textrm{qis}"'] \ar[dr] & \\
                C_{1}^{\bullet} & & C_{2}^{\bullet} & & C_{3}^{\bullet}
            \end{tikzcd}
            \end{equation*}
            which commutes up to homotopy; thus, we obtain the diagram
            \begin{equation*}
            \begin{tikzcd}
                & & C_{0}^{\bullet} \arrow[ld, "\textrm{qis}"'] \arrow[rd]  
                & & \\
                & C_1^{\bullet} \arrow[ld, "\textrm{qis}"'] \arrow[rrrd] 
                & & C_3^{\bullet} \arrow[llld, "\textrm{qis}", crossing over] \arrow[rd] 
                & \\
                A^{\bullet} 
                &  &  &  & B^{\bullet}\,,
            \end{tikzcd}
            \end{equation*}
            which shows that the roof dominated by $C_{1}^{\bullet}$
            is equivalent to the roof dominated by $C^{\bullet}_{3}$.
        \end{itemize}

        Finally, we can define $\Hom_{\cat{D}(\Aa)}(A^{\bullet}, B^{\bullet})$:
        a morphism $A^{\bullet} \to B^{\bullet}$ in the derived category
        is the equivalence class of a roof 
        over $A^{\bullet}$ and $B^{\bullet}$. Given two morphisms
        $A^{\bullet} \to B^{\bullet}$ 
        and $B^{\bullet} \to C^{\bullet}$ in $\cat{D}(\Aa)$,
        we describe the composition in the derived category:
        taken two representatives
        \begin{equation*}
            \begin{tikzcd}
                & C_{1}^{\bullet} \ar[dl, "\textrm{qis}"'] \ar[dr]
                && C_{2}^{\bullet} \ar[dl, "\textrm{qis}"'] \ar[dr] & \\
                A^{\bullet} & & B^{\bullet} & & C^{\bullet}\,,
            \end{tikzcd}
        \end{equation*}
        the composition $A^{\bullet} \to B^{\bullet} \to C^{\bullet}$
        is the equivalence class of a roof on top of the other two,
        in such a way that the following
        \begin{equation*}
            \begin{tikzcd}
                && C_{0}^{\bullet} \ar[dl, "\textrm{qis}"'] \ar[dr] && \\
                & C_{1}^{\bullet} \ar[dl, "\textrm{qis}"'] \ar[dr]
                && C_{2}^{\bullet} \ar[dl, "\textrm{qis}"'] \ar[dr] & \\
                A^{\bullet} & & B^{\bullet} & & C^{\bullet}
            \end{tikzcd}
        \end{equation*}
        is a commutative diagram in $\cat{K}(\Aa)$: 
        \hyperref[roof-comp]{Proposition~\ref*{roof-comp}}
        ensures that such a diagram always exists; 
        moreover, the equivalence class of the
        appearing roof is unique: indeed, 
        for two choices $C_{0}^{\bullet}$
        and $D_{0}^{\bullet}$ on the top of the above diagram,
        one can use \hyperref[roof-comp]{Proposition~\ref*{roof-comp}}
        again to show that the two choices are equivalent.
        Thus, the composition of two morphisms in $\cat{D}(\Aa)$
        is well defined and one can show it is associative,
        so that $\cat{D}(\Aa)$ defines a category.

        Notice that, for any given qis
        $f^{\bullet}:A^{\bullet} \to B^{\bullet}$,
        its image $Q(f^{\bullet})$ is an isomorphism
        in the category $\cat{D}(\Aa)$ just described:
        indeed, it can be represented by the roof
        $A^{\bullet} = A^{\bullet} \to B^{\bullet}$,
        whose inverse is the equivalence class of
        \begin{equation*}
            \begin{tikzcd}
                & A^{\bullet} \ar[dr, equals] \ar[dl, "f^{\bullet}"'] & \\
                B^{\bullet} & & A^{\bullet}\,,
            \end{tikzcd}
        \end{equation*}
        as we can see from the diagrams:
        \begin{equation*}
            \begin{tikzcd}[column sep=small]
                && A^{\bullet} \ar[dl, equals] \ar[dr, equals] && &
                && A^{\bullet} \ar[dl, equals] \ar[dr, equals] && \\
                & A^{\bullet} \ar[dl, equals] \ar[dr, "f^{\bullet}"]
                && A^{\bullet} \ar[dr, equals] \ar[dl, "f^{\bullet}"'] & &
                & A^{\bullet} \ar[dr, equals] \ar[dl, "f^{\bullet}"']
                && A^{\bullet} \ar[dl, equals] \ar[dr, "f^{\bullet}"] &\\
                A^{\bullet} & & B^{\bullet} & & A^{\bullet} \,, &
                B^{\bullet} & & A^{\bullet} & & B^{\bullet}\,,
            \end{tikzcd}
        \end{equation*}
        where the second composition is indeed in the equivalence class of the
        identity of $B^{\bullet}$ because
        \begin{equation*}
            \begin{tikzcd}
                & & A^{\bullet} \arrow[ld, equals] \arrow[rd, "f^{\bullet}"]  
                & & \\
                & A^{\bullet} \arrow[ld, "f^{\bullet}"'] \arrow[rrrd, "f^{\bullet}"'] 
                & & B^{\bullet} \arrow[llld, equals, crossing over] \arrow[rd, equals] 
                & \\
                B^{\bullet} 
                &  &  &  & B^{\bullet}\,.
            \end{tikzcd}
        \end{equation*}
        Hence, one can easily verify the \emph{\textbf{universal property}}
        holds by setting $G(f:A^{\bullet} \to B^{\bullet}) 
        := \left(F(f^{\bullet}):F(A^{\bullet}) \to F(B^{\bullet}) \right)$,
        for any representative $f^{\bullet}$ of the morphism $f$.
    \end{proof}
\end{thmdef}

\begin{rmk}
    Under the functor $Q$, we identify objects 
    in $\cat{D}(\Aa)$ with objects in $C^{\bullet}(\Aa)$:
    hence, we may speak of complexes 
    $A^{\bullet}, B^{\bullet}, ... \in \cat{D}(\Aa)$.
    As a consequence, the cohomology objects $H^{n}(A^{\bullet})$
    of $A^{\bullet}\in \cat{D}(\Aa)$ are well-defined
    objects of the abelian category $\Aa$ because
    of the universal property of the derived category.
    Moreover, given two homotopic maps
    $f^{\bullet}, g^{\bullet}:A^{\bullet} \to B^{\bullet}$,
    the diagram
    \begin{equation*}
            \begin{tikzcd}
                & & A^{\bullet} \arrow[ld, equals] \arrow[rd, equals]  
                & & \\
                & A^{\bullet} \arrow[ld, equals] \arrow[rrrd, "f^{\bullet}"'] 
                & & A^{\bullet} \arrow[llld, equals, crossing over] \arrow[rd, "g^{\bullet}"] 
                & \\
                A^{\bullet} 
                &  &  &  & B^{\bullet}\,.
            \end{tikzcd}
    \end{equation*}
    commutes in $\cat{K}(\Aa)$, so we deduce that 
    $Q(f^{\bullet}) = Q(g^{\bullet})$.
    By the \textbf{universal property of the homotopy category},
    it follows that there exists a unique factorization
    \begin{equation*}
        \begin{tikzcd}
            C^{\bullet}(\Aa) \ar[rr] \ar[dr, "Q"'] 
            & & \cat{K}(\Aa) \ar[dl] \\
            & \cat{D}(\Aa) & \,.
        \end{tikzcd}
    \end{equation*}
    This means that, for every $n \in \Z$, 
    the $n$-th cohomology $H^{n}$ is a well-defined functor
    on the derived category.
\end{rmk}

\begin{rmk}
    Viewing any object $A \in \Aa$ as a complex 
    concentrated in degree zero yields an equivalence 
    between $\Aa$ and the full subcategory of $\cat{D}(A)$
    that consists of all complexes $B^{\bullet}$ 
    with $H^{n}(B^{\bullet}) = \cat{0}$, for $n \ne 0$.
\end{rmk}

\begin{thm}
    The inclusion $\Aa \hookrightarrow \cat{D}(\Aa)$
    which sends an object $A$ to the complex
    concentrated in degree zero $A{[0]}$ 
    yields an equivalence with the full subcategory 
    $\cat{D}^{0}(\Aa)$ of $\cat{D}(\Aa)$ formed by
    complexes $B^{\bullet}$ such that 
    $H^{n}(B^{\bullet}) \simeq \cat{0}$,
    for $n \ne 0$.
    \begin{proof}
        \textcolor{red}{Guarda Manin - III.5.Prop2}.
    \end{proof}
\end{thm}

\begin{exercise!}\label{derived-trivial}
    Show that $A^{\bullet}$ is isomorphic to $\cat{0}$
    in $\cat{D}(\Aa)$ if and only if 
    $H^{n}(A^{\bullet}) \simeq \cat{0}$ in $\Aa$, 
    for every $n \in \Z$.
    On the other hand, find an example of a complex morphism
    $f^{\bullet} : A^{\bullet} \to B^{\bullet}$ which is
    trivial in cohomology, but $Q(f^{\bullet}) \ne 0$.
    \begin{proof}[Solution]
        We have already noticed that cohomology
        is a well defined functor on the derived category,
        so if $A^{\bullet} \simeq \cat{0}$ 
        in $\cat{D}(\Aa)$, then $H^*(A^{\bullet})$
        is trivial. 
        Conversely, assume $A^{\bullet}$ has trivial cohomology:
        then the zero morphism $A^{\bullet} \to \cat{0}$
        is a qis, hence an isomorphism in $\cat{D}(\Aa)$.

        As an example of a non-trivial morphism,
        with trivial cohomology, consider the following
        commutative diagram in $\Aa = \Mod_{\Z}$:
        \begin{equation*}
            \begin{tikzcd}
                A^{\bullet}\,: \ar[d, "f^{\bullet}"']
                & \cat{0} \ar[r] 
                & \Z/4\Z \ar[r, "\cdot 2"] \ar[d]
                & \Z/4\Z \ar[r, "\cdot 2"] \ar[d, "q"]
                & \Z/4\Z \ar[r] \ar[d]
                & \cat{0} \\
                B^{\bullet} \,:
                & \cat{0} \ar[r]
                & \cat{0} \ar[r]
                & \Z/2\Z \ar[r]
                & \cat{0} \ar[r]
                & \cat{0} 
            \end{tikzcd}\,;
        \end{equation*}
        since $H^0(A^{\bullet}) \simeq \cat{0}$,
        then it is clear that $f^*=0$.
        Nevertheless, $f^{\bullet}$ in not null-homotopic:
        indeed, if there existed a homotopy $h$, 
        then for each $x \in \Z/4\Z$ we would have
        \begin{equation*}
            q(x) = h(2x) = 2h(x) = 0\,,
        \end{equation*}
        which is a contradiction. In particular,
        this means that in $\cat{K}(\Mod_{\Z})$
        there is no commutative diagram of the
        form
        \begin{equation*}
            \begin{tikzcd}
                & & C_{0}^{\bullet} \arrow[ld, "\mathrm{qis}"'] \arrow[rd]  
                & & \\
                & A^{\bullet} \arrow[ld, equals] \arrow[rrrd, "f^{\bullet}"'] 
                & & C^{\bullet} \arrow[llld, "\mathrm{qis}", crossing over] \arrow[rd, "0"] 
                & \\
                A^{\bullet} 
                &  &  &  & B^{\bullet}\,,
            \end{tikzcd}
        \end{equation*}
        so $Q(f^{\bullet}) \ne 0$. \textcolor{red}{Why? Guarda esercizi Danilo.}
    \end{proof}
\end{exercise!}

\begin{exercise}
    \textcolor{red}{Check that the derived category $\cat{D}(\Aa)$
    is additive.}
\end{exercise}

Thus, if $\Aa$ is an abelian category, 
its derived category $\cat{D}(\Aa)$ is additive, 
but in general not abelian;
nevertheless, it is triangulated.

\begin{prop}
    The category $\cat{D}(\Aa)$ is triangulated and the canonical functor
    \begin{equation*}
        \cat{K}(\Aa) \longrightarrow \cat{D}(\Aa)
    \end{equation*}
    is an exact functor of triangulated categories.
    \begin{proof}
        \textcolor{red}{Check Gelfand-Manin IV.2}
    \end{proof}
\end{prop}

\begin{exercise!}\label{SES-TRI}
    Suppose
    \begin{center}
        \begin{tikzcd}
            0 \ar[r]
            & A \ar[r, "f"]
            & B \ar[r, "g"]
            & C \ar[r]
            & 0
        \end{tikzcd}
    \end{center}
        is a short exact sequence in an abelian category $\Aa$.
        Show that under the full embedding into the
        homotopy category $\Aa \hookrightarrow \cat{K}(\Aa)$
        (or the one in the derived category $\Aa \hookrightarrow \cat{D}(\Aa)$),
        this becomes a distinguished triangle 
    \begin{center}
        \begin{tikzcd}
            A \ar[r]
            & B \ar[r]
            & C \ar[r, "\delta"]
            & A{[1]}\,,
        \end{tikzcd}
    \end{center}
        where $\delta$ is the composition of the inverse
        of the qis $\cat{C}(f) \to C$ with the projection $\cat{C}(f) \to A[1]$.
        
        Conversely, if $A,B, C \in \Aa$ form a distinguished triangle 
        \begin{center}
        \begin{tikzcd}
            A \ar[r]
            & B \ar[r]
            & C \ar[r]
            & A{[1]}\,,
        \end{tikzcd}
        \end{center}
        then $0 \to A \to B \to C \to 0$ is a short exact sequence in $\Aa$.
    \begin{proof}[Solution]
        Notice that the cone $\cat{C}(f)$ is given by the complex
        \begin{equation*}
            \begin{tikzcd}
                0 \ar[r] & A \ar[r, "f"] & B \ar[r] & 0\,,
            \end{tikzcd}
        \end{equation*}
        thus $H^{-1}\left(\cat{C}(f)\right) = \ker f$ 
        and $H^{0}\left(\cat{C}(f)\right) = \Coker f$.
        Since the short sequence is exact, it follows that
        \begin{equation*}
            H^{-1}\left(\cat{C}(f)\right) = 0\,, \quad
            H^{0}\left(\cat{C}(f)\right) \simeq C\,,
        \end{equation*}
        hence the diagram
        \begin{equation*}
            \begin{tikzcd}
                0 \ar[r] & A \ar[r, "f"] \ar[d] & B \ar[r] \ar[d, "g"] & 0 \\
                0 \ar[r] & 0 \ar[r] & C \ar[r] & 0
            \end{tikzcd}
        \end{equation*}
        defines a qis $\cat{C}(f) \to C$. 
        Then, by passing in $\cat{K}(\Aa)$ (or equivalently in $\cat{D}(\Aa)$),
        we end up with the commutative diagram
        \begin{equation*}
            \begin{tikzcd}
                A \ar[r, "f"] \ar[d, equals] 
                & B \ar[r] \ar[d, equals] 
                & C \ar[r] \ar[d]
                & A{[1]} \ar[d, equals] \\
                A \ar[r, "f"]
                & B \ar[r]
                & \cat{C}(f) \ar[r] 
                & A{[1]}\,,
            \end{tikzcd}
        \end{equation*}
        which is an isomorphism of triangles by the 
        \hyperref[5lemma]{5-lemma~\ref*{5lemma}}.

        Conversely, if three objects $A,B,C \in \Aa$ form
        a distinguished triangle $A \to B \to C \to A{[1]}$,
        then the induced \hyperref[LECS]{LECS~\ref*{LECS}}
        is a short exact sequence in $\Aa$.
    \end{proof}
\end{exercise!}

\begin{exercise!}\label{derived-LECS}
    Suppose $A^{\bullet} \to B^{\bullet} \to C^{\bullet} \to A^{\bullet}[1]$
    is a distinguished triangle in the derived category $\cat{D}(\Aa)$.
    Show that it naturally induces a long exact sequence
    \begin{center}
        \begin{tikzcd}
            \dots \ar[r]
            & H^{i}(A^{\bullet}) \ar[r]
            & H^{i}(B^{\bullet}) \ar[r]
            & H^{i}(C^{\bullet}) \ar[r]
            & H^{i+1}(A^{\bullet}) \ar[r]
            & \dots
        \end{tikzcd}
    \end{center}
    \begin{proof}[Solution]
        \textcolor{red}{LALALA.}
    \end{proof}
\end{exercise!}


By definition, complexes in the categories 
$\cat{K}(A)$ and $\cat{D}(A)$ are unbounded, 
but often it is more convenient to work with bounded ones,
especially in the algebro-geometric context.

\begin{df}
    Given an abelian category $\Aa$,
    let $C^{*}(\Aa)$, with $* \in \Set{+,-,b}$, 
    be the category of complexes $A^{\bullet}$
    with $A^{n} = 0$ for $n << 0, n >> 0$,
    respectively $|n| >> 0$.
\end{df}

By dividing out first by homotopy equivalence 
and then by qis one obtains
the categories $\cat{K}^{*}(\Aa)$ and $\cat{D}^{*}(\Aa)$,
with $* \in \Set{+,-,b}$. 
Let us consider the natural functors 
$\cat{D}^{*}(A) \to \cat{D}(A)$ 
given by just forgetting the boundedness condition.

\begin{prop}
    The natural functor $\cat{D}^+(\Aa) \hookrightarrow \cat{D}(\Aa)$,
    defines an equivalence of $\cat{D}^{+}(\Aa)$ with
    the full triangulated subcategory of all 
    complexes $A^{\bullet} \in \cat{D}(\Aa)$ with
    $A^{n}=0$ for $n << 0$.
    Analogous statements hold true for $\cat{D}^{-}(\Aa)$
    and $\cat{D}^{b}(\Aa)$.
    \begin{proof}
        \textcolor{red}{KASHIWARA SHAPIRA.}
    \end{proof}
\end{prop}

\begin{exercise!}\label{null-terms}
    Let $A^{\bullet}$ be a complex with $H^{n}(A^{\bullet}) = 0$, for $n > m$.
    Show that $A^{\bullet}$ is quasi-isomorphic 
    (and hence isomorphic as an object in $\cat{D}(\Aa)$)
    to a complex $B^{\bullet}$, with $B^{n}=0$ for $n > m$.
    \begin{proof}[Solution]
        \textcolor{red}{AAAAAAAH}
    \end{proof}
\end{exercise!}

\begin{exercise}
    Let $A^{\bullet}$ be a complex with 
    $m := \Set{n \in \Z| H^{n}(A^{\bullet}) \ne 0} < \infty$.
    Show there exists a morphism
    \begin{equation*}
        \phi : A^{\bullet} \longrightarrow H^{m}(A^{\bullet})[-m]
    \end{equation*}
    in $\cat{D}(\Aa)$ such that 
    $H^{m}(\phi) : H^{m}(A^{\bullet}) \to H^{m}(A^{\bullet})$
    is the identity.
    \begin{proof}[Solution]
        \textcolor{red}{AAAAAAAAAA}
    \end{proof}
\end{exercise}

\begin{exercise}
    Suppose $H^{n}(A^{\bullet}) = 0$ for $n < n_{0}$.
    Show there exists a distinguished triangle
    \begin{equation*}
        \begin{tikzcd}
            H^{n_{0}}(A^{\bullet}){[-n_{0}]} \ar[r]
            & A^{\bullet} \ar[r, "\phi"]
            & B^{\bullet} \ar[r]
            & H^{n_{0}}(A^{\bullet}){[1-n_{0}]}
        \end{tikzcd}
    \end{equation*}
    in $\cat{D}(\Aa)$ with $H^{n}(B^{\bullet})=0$ 
    for $n \le n_{0}$ and $\phi$ inducing isomorphisms
    $H^{n}(A^{\bullet}) \simeq H^{n}(B^{\bullet})$ for $n > n_{0}$.
    \begin{proof}[Solution]
        \textcolor{red}{AAAAAAAAA.}
    \end{proof}
\end{exercise}

\textcolor{red}{Splitting and derived categories in Gelfand Manin.}