%%% Plan of the thesis

The idea is to go through the article 
\emph{Braid groups actions on derived categories of coherent sheaves},
by Siedel and Thomas.

The article starts with a sections containing 
motivations on the theory: the authors give a survey on 
the main fields where the derived category $\cat{D}^{b}(X)$
appears. In particular, symplectic geometry
and mirror symmetry are the main interest of the paper.

Section 2 introduces braid group actions on a category.
For a detailed description, check \textbf{Deligne}'s article
``\emph{Action du group de tresses sur une catégorie}''.
In Section 2.b spherical objects and twist functors are
introduced in a pure functorial way, i.e. using only
monoidal abelian categories.
By introducing enough hypothesis on our category,
we can study braid relations in a pure abstract way:
Section 2.c is devoted to the contruction of braid relations
between twist functors induced by spherical objects.
Here, \textbf{Proposition 2.12} and \textbf{Proposition 2.13}
show that there exists a braid group action an suitable
category $\Kk$. The main result is stated in \textbf{Theorem 2.18}.

\begin{thm**}
	Suppose that $n \ge 2$. Then the homomorphism 
	\begin{equation*}
	 	\rho : B_{m+1} \longrightarrow \textrm{Autoeq}(\cat{D}^{b}(X))
	 \end{equation*} 
	is injective, and in fact the following stronger statement holds: 
	if $g \in B_{m+1}$ is not the identity element, 
	then $\rho(g)(E_{i}) \ne E_{i}$ for some $i \in \{1,...,m\}$.
\end{thm**}

The proof of the faithfulness of this action
is the whole Section 4.
The machinery involved is mainly based on
the theory of \textbf{differential graded algebras} (dgas).
A detailed survey of dgas can be found in
\textbf{Bernstein} and \textbf{Lunts}'s
``\emph{Equivariant sheaves and functors}''.

In the context of the derived category of a dga,
standard twist functors are introduced.
These twists commute with equivalences induced 
by qis of dgas.
\textbf{Lemma 4.3} relates the standard twist functors $t$
of dgas with the twist functors $T_{E}$ in our favorite triangulated category.

We then try to study the twist functors $T_{E_{i}}$
by analizing the dga $end(E)$. We will focus
on a particular class of dgas, 
namely \textbf{intrinsically formal} graded algebras:
these objects are determined by their cohomology.
This property can be studied in terms
of the vanishing of some \textbf{Hochschild cohomology} modules,
as stated in \textbf{Theorem 4.7}.
A simple intoduction to this topic can be found
in the ``Istituzioni di Algebra'' notes by \textbf{Tamas}.

The proof of this fact is by no mean trivial.
The authors develop a theory of \textbf{$A_{\infty}$-morphisms},
because they induce morphisms of dgas.

In Section 4.c the algebras $A_{m,n}$ are introduced
as the path algebras of a suitable \textbf{quiver}.
I should check for a little survey on this topic.
This section aims to explain how to translate the
problem on twist funtors with the language of 
dgas, indeed $A_{m,n}$ are isomorphic
to some $end(E)$.

Standard twist functors $t_{i}:\cat{D}(A_{m,n}) \to \cat{D}(A_{m,n})$
are exact equivalences (\textbf{Lemma 4.11}) that
satisfy the braid relations.

Finally, the faithfullness of the braid action
is proved in Section 4.d 
for the derived categories of the algebras $A_{m,n}$.
To achieve this, one studies the twist functor
associated to the generator of the center of $B_{m+1}$.

\textbf{The authors conjecture that $(t_{1}t_{2} \dots t_{m})^{m+1}$
is isomorphic to the translation $[2m-(m+1)n]$.}

The proof of the faithfulness relies on topology:
in particular, \textbf{mapping class groups} 
of the punctured disc are involved. It is well known
their link with the braid groups,
and \textbf{Lemma 4.16} explains the link
between some diffeomorphisms
and the generator of the center of $B_{m+1}$.


\textbf{Geometric intersection numbers}
are defined; one can check
``\emph{Quivers, Floer cohomology, and braid group actions}''
by \textbf{Khovanov} and \textbf{Seidel}
for an insight. 
In fact, we can find braid group actions 
already in this paper;
Siedel and Thomas make a comparison
with their definition $A_{m,n}$ .

Strictly speaking, Section 4.d is a comparison
with the article by Khovanov and Seidel,
by the end of which the faithfulness
is proved in the category of some dga.

The final section is dedicated
to the computation of Hochschild cohomology
of $A_{m,n}$, so that its intrinsic formality is proved.
This fact is essential in the proof of \textbf{Theorem 2.18},
together with the faithfulness described above.


\section{Study plan}

	I'm new to the following topics:
	\begin{itemize}
		\item derived category of a scheme;
		\item Fourier-Moukai Transforms;
		\item group actions on a category;
		\item DG-algebras;
		\item Hochschild cohomology and its applications;
		\item quivers and representations;
		\item mapping class group.
	\end{itemize}
	Thus, I'll need some study!
	
\section{Structure idea}

	Here I write a possible structure of the strategy:
	\begin{itemize}
		\item introduction, which will be written at the very end,
		possibly with motivations;
		
		\item recollection of some properties of derived categories;
		
		\item a survey about DG-algebras;
		
		\item something about quivers and their path algebras;
		
		\item we start to set the main tools:
		definition of spherical objects in a triangulated category,
		and definition of twists;
		
		\item definition of group action over a category.
		Braids acting on a category and faithfulness: discussion and proof,
		assuming \textbf{Theorems 4.13} e \textbf{4.21}, whose proofs
		are postponed;
		
		\item technicalities about dgas, intrinsic formality 
		and geometric intersections;
		
		\item applications to the algebro-geometric context
		(definition of FMT is needed) and examples;
		recent develpments.
	\end{itemize}



















